\renewcommand{\abstractname}{Abstract}

\begin{abstract}
A memory model defines semantics of 
concurrent programs operating on a shared memory.
The most well-known and intuitive memory model, sequential consistency,
is too \emph{strong} for modern languages as it forbids
many outcomes observable on modern hardware as a result of
compiler and CPU optimizations. This gave rise to so-called
\emph{weak} or \emph{relaxed} memory models.
In recent years dozens of (weak) memory models for 
programming languages were proposed
making different compromises with respect 
to programmability and the optimization potential.
A goal of this paper is to survey and classify these models as well as
to provide practical recommendations for language and compiler designers
regarding a choice of a memory model.

To achieve this we picked over 2000 research items from Google Scholar
with keywords ``Relaxed Memory Models'', ``Weak Memory Models'',
and ``Weak Memory Consistency''. Then, we narrowed down this list to 40 papers
having as a contribution a programming language memory model
(as oppose to hardware memory models).
We divide these models to six main classes and analyze their properties and limitations.
We conclude with a discussion on how a choice of a memory model affected by
desired features of a language and suggest several possible directions for
researh in the field of weak memory models.
\end{abstract}
