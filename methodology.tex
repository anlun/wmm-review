\section{Methodology}
\label{sec:methodology}

The main purpose of our study was to answer the following research question:

\begin{itemize}
  \item What are the trade-offs in the design of a memory model for a programming language?
\end{itemize}

As we have already briefly mentioned, stronger memory models 
give more guarantees to the programmers, while weaker models 
provide more optimization opportunities. 
Thus our research question can be further refined into the following one:

\begin{itemize}
  \item How the formal reasoning guarantees provided by the memory model 
    to the end-user of a programming language narrow  
    optimization opportunities for a language implementation?
\end{itemize}

To answer this question, we consulted the existing research studies 
in the field of programming language memory models.
Our goal was to identify existing proposed memory models, 
study their properties, and classify them.
In particular, with respect to each model we wanted to answer the following questions:

\begin{itemize}
  
  \item Can the memory model be efficiently implemented on modern hardware? 

  \item What program transformations and optimizations are correct in the model? 

  \item What kind of guarantees for formal reasoning about the behavior 
    of programs does the memory model provide?
  
\end{itemize}

In order to select memory models for our study, we performed the following procedure.
On the \emph{first stage}, we have manually selected 10  
peer-reviewed research papers~\cite{
Manson-al:POPL05,
Batty-al:POPL11,
Lahav-al:PLDI17,
Dolan-al:PLDI18,
Watt-al:PLDI2020,
Jeffrey-Riely:LICS16,
PichonPharabod-Sewell:POPL16,
Kang-al:POPL17,
Chakraborty-Vafeiadis:POPL19,
Paviotti-al:ESOP20
} 
whose main contribution was a proposal of a new PL weak memory model.
We then took the list of keywords from these papers. 
From this list of keywords we manually excluded those 
that were either too broad or too narrow.
As a result we get three keyword phrases:
\begin{itemize}
  \item Relaxed Memory Models
  \item Weak Memory Models
  \item Weak Memory Consistency
\end{itemize}
 
On the \emph{second stage} we used these phrases as a search queries. 
Using Google Scholar\footnote{https://scholar.google.com/} as a search engine, 
we obtained a list of 2000 research items. 
As a sanity check, we verified that each of the 10 initially selected papers 
was contained in the selection. 

On the \emph{third stage} we removed from the selection duplicates and non-peer-reviewed papers. 
Also we decided to remove technical reports, theses, 
non-English publications, as well as short papers (up to 4 page long).
After this stage only the 1077 research items have left.

Next, on the \emph{fourth stage}, we further filtered the selection 
by consulting titles and abstracts of the papers. 
We included only the papers which are directly related to the 
topic of PL memory models and whose main focus is memory models themselves,
as opposed to papers that only use established results about PL memory models,
or papers related to adjacent topics. 
In particular, we filtered out papers related to the following adjacent topics:
\begin{itemize}
  \item memory models of hardware, heterogeneous systems, and distributed systems;
  \item semantics of transactions and persistency;
  \item verification techniques for weak memory.
\end{itemize}
As a result we got 137 research items.

Finally, on the \emph{fifth stage}, we carefully examined the remaining articles.
In our final selection, we have included only those whose contribution was claimed to include:
\begin{itemize}
  \item a new PL memory model;
  \item a study of an existing PL memory model.
  \item a refinement of an existing PL memory model;
\end{itemize}
After this stage, only the 40 papers have left.

