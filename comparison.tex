\section{Comparison}
\label{sec:comparison}

We performed a comparison of memory models 
found via a search procedure described in \cref{sec:methodology} 
by a criteria given in \cref{sec:background}. 
Based on this comparison, we identified six classes of models:
sequentially consistent models, %~\ref{sec:analysis:seqcst}, 
models with total or partial order on stores, %~\ref{sec:analysis:tso}, 
program order preserving models, %~\ref{sec:analysis:porf}, 
syntactic dependency preserving models, %~\ref{sec:analysis:deprf}, 
semantic dependency preserving models, %~\ref{sec:analysis:sdeprf}, 
and models with out of thin-air values. %~\ref{sec:analysis:oota}.
The models from the same class have similar compilation mappings, 
set of sound program transformations, and provided reasoning guarantees.
We first present the result of our comparison on a per-class basis 
(see \cref{table:cmp-cls} and \cref{sec:analysis}), and 
then give a more detailed comparison with respect to individual models
(see \cref{table:cmp-mms} and \cref{sec:catalog}).
Thus we have an opportunity to first discuss the common principles
behind the PL relaxed memory models in general, and then dive deeper 
into details of each particular model. 

In both \cref{table:cmp-cls} and \cref{table:cmp-mms} we order
the classes by the weakness of the memory models they consist of.  
The strongest models are located at the top rows of the table, 
while the weakest are at the bottom.  

Columns of both tables correspond to the properties 
of the models discussed in~\cref{sec:background}.
Each row of the \cref{table:cmp-cls} corresponds to a class of memory models.
The numbers in the cells of the table denote the number of models in this 
class possessing the corresponding property relative to a total number 
of models in this class. \todo{Give an example once table is ready}.
For binary classified properties if at least one of the models 
in the given class possess the property,
then we additionally mark it by \cmark, 
if none of the models posses the property, we mark it by \xmark.    
In table \cref{table:cmp-mms} each row corresponds to 
a specific memory model, denoted by its abbreviation, 
and thus all cells describe just the properties 
of this particular model. 

As for the properties themselves, we split them into several subgroups.

The first group is devoted to optimality of compilation mappings
to target hardware architectures. In order to be concise, 
we chose the binary classification of optimality, 
that is, we classify the compilation scheme as either optimal or not,
in the following sense.
We chose the weakest possible access mode supported by the model
and consider the compilation scheme for the memory accesses annotated by this mode. 
For the memory models that treat racy non-atomic accesses
as undefined behavior, we consider the compilation mapping
for most relaxed access types the model provides.
This is because the catch-fire semantics for racy non-atomics 
trivially permits the most optimal compilation mappings (see~\cref{sec:background:ub}).
We say that compilation scheme is \emph{optimal} if the 
accesses annotated by the most relaxed mode 
can be compiled just as plain load and store instructions 
of the given hardware architecture (\ie without memory fences or extra dependencies). 

The second group is dedicated to soundness of various program transformations. 
The classification is also binary: a transformation is either sound or unsound 
in the given memory model (in a sense stated in~\cref{sec:background:trans}).
Again, to be concise, we do not consider all the combinations 
of program tranformations and memory access modes. 
Instead, we consider the weakest possible accesses which have fully defined semantics. 
We further split the transformations into global and local as in~\cref{sec:background:trans}.

The third group corresponds to reasoning principles guaranteed by the model. 
It includes the following properties. What kind of DRF guarantee the model provides.
(we distinguish the internal, external and local DRF theorems, see~\cref{sec:background:drf}).
Whether the model has undefined behaviors (see~\cref{sec:background:ub}).
Whether the model permits out-of-thin-air values (see~\cref{sec:background:oota}).

Finally, the last group enumerates the list of memory access modes 
and fences supported by the model, as well as whether the model 
supports read-modify-write operations, locks, and mixed-size accesses.