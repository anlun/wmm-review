\section{Related Work}
\label{sec:related}

Weak memory models can be partitioned into two subclasses: 
models for hardware architectures and models for programming languages. 
To this day, the hardware weak memory models are relatively well-studied and understood.
A large amount of efforts was put to formally specify models of various 
hardware architectures%
\cite{Chong-ASPLOS08, Alglave-DAMP09, Sewell-al:CACM10, Sarkar-al:PLDI11}.
Alglave \etal~\cite{Alglave-al:TOPLAS14} summarized different studies in this area 
and provided a comprehensive overview of existing models.  
Besides that, they also proposed a general framework for specification, 
testing and verification of hardware memory models.
More recent hardware models, \eg \cite{Pulte-al:POPL18}, 
also adhere to this framework.   

In the context of programming language memory models the situation is more complex. 
There were various memory models proposed for different programming languages, \eg
for \Java~\cite{Manson-al:POPL05, Bender-Palsberg:OOPSLA19}, \CPP~\cite{Batty-al:POPL11}, 
\LLVM~\cite{Chakraborty-Vafeiadis:CGO17}, \JS~\cite{Watt-al:PLDI2020}, 
\OCaml~\cite{Manson-al:POPL05}, \Haskell~\cite{Vollmer-al:PPoPP17}, \etc. 
However, we are unaware of any survey or detailed comparison of these memory models.
%The lack of such study motivated our work.

