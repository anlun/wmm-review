\makeatletter
\providecommand{\@thefnmark}{\the\@fnmark}
\makeatother

\makeatletter
%\let\orgdescriptionlabel\descriptionlabel
%\renewcommand*{\descriptionlabel}[1]{%
%  \let\orglabel\label
%  \let\label\@gobble
%  \phantomsection
%  \edef\@currentlabel{#1}%
%  %\edef\@currentlabelname{#1}%
%  \let\label\orglabel
%  \orgdescriptionlabel{#1}%
%}
\makeatother

\definecolor{StringRed}{rgb}{.637,0.082,0.082}
\definecolor{CommentGreen}{rgb}{0.0,0.55,0.3}
\definecolor{KeywordBlue}{rgb}{0.0,0.3,0.55}
\definecolor{LinkColor}{rgb}{0.55,0.0,0.3}
\definecolor{CiteColor}{rgb}{0.55,0.0,0.3}
\definecolor{HighlightColor}{rgb}{0.0,0.0,0.0}


\definecolor{grey}{rgb}{0.5,0.5,0.5}
\definecolor{red}{rgb}{1,0,0}
\definecolor{darkgreen}{rgb}{0.0,0.7,0.0}

\hypersetup{%
  linktocpage=true, pdfstartview=FitV,
  breaklinks=true, pageanchor=true, pdfpagemode=UseOutlines,
  plainpages=false, bookmarksnumbered, bookmarksopen=true, bookmarksopenlevel=3,
  hypertexnames=true, pdfhighlight=/O,
  colorlinks=true,linkcolor=LinkColor,citecolor=CiteColor,
  urlcolor=LinkColor
}

\newcommand\dhxrightarrow[2][]{%
\xrightarrow[{#1}]{{#2}}\!\!\!\!\!\rightarrow %
}

\newcommand{\itemref}[1]{[{\nameref{#1}}]}

\newcommand{\TODO}[1]{\emph{\color{red} \textbf{TODO: #1}}}
%% \newcommand{\black}[1]{{\color{black}{#1}}}
%% \newcommand{\red}[1]{{\color{red}{#1}}}
%% \newcommand{\blue}[1]{{\color{blue}{#1}}}
%% \newcommand{\purple}[1]{{\color{purple}{#1}}}

%% \newcommand{\derek}[1]{{\color{purple}{\bf{D: #1}}}}
%% \newcommand{\jeehoon}[1]{{\color{magenta!60!black}{\bf{J: #1}}}}
%% \newcommand{\ori}[1]{{\color{blue}{\bf{O: #1}}}}
%% \newcommand{\viktor}[1]{{\color{green!60!black}{\bf{V: #1}}}}
%% \newcommand{\gil}[1]{{\color{red}{\bf{G: #1}}}}

%% \renewcommand{\derek}[1]{}
%% \renewcommand{\jeehoon}[1]{}
%% \renewcommand{\ori}[1]{}
%% \renewcommand{\viktor}[1]{}
%% \renewcommand{\gil}[1]{}

%% \newcommand{\ie}{\emph{i.e.,} }
%% \newcommand{\eg}{\emph{e.g.,} }
%% \newcommand{\etal}{\emph{et al.}}
%% \newcommand{\hide}[1]{}

%% \newcommand{\nocomment}[1]{{\color{red}~~\texttt{/\!\!/}\ensuremath{{\neq}}\,{#1}}}
%% \newcommand{\comment}[1]{{\color{teal}~\texttt{/\!\!/}\,{#1}}}
%% \newcommand{\inarr}[1]{\begin{array}{@{}l@{}}#1\end{array}}
%% \newcommand{\inarrII}[2]{\begin{array}{@{}l@{~~}||@{~~}l@{}}\inarr{#1}&\inarr{#2}\end{array}}
%% \newcommand{\inarrIId}[2]{\begin{array}{@{}l@{~}||@{~}l@{}}\inarr{#1}&\inarr{#2}\end{array}}
%% \newcommand{\inarrIII}[3]{\begin{array}{@{}l@{~~}||@{~~}l@{~~}||@{~~}l@{}}\inarr{#1}&\inarr{#2}&\inarr{#3}\end{array}}
%% \newcommand{\inarrIIId}[3]{\begin{array}{@{}l@{~}||@{~}l@{~}||@{~}l@{}}\inarr{#1}&\inarr{#2}&\inarr{#3}\end{array}}
%% \newcommand{\inarrIV}[4]{\begin{array}{@{}l@{~~}||@{~~}l@{~~}||@{~~}l@{~~}||@{~~}l@{}}\inarr{#1}&\inarr{#2}&\inarr{#3}&\inarr{#4}\end{array}}
%% \newcommand{\inpar}[1]{\left(\inarr{#1}\right)}
%% \newcommand{\inparII}[2]{\begin{array}{@{}l@{~~}||@{~~}l@{}}\inarr{#1}&\inarr{#2}\end{array}}
%% \newcommand{\inparIII}[3]{\begin{array}{@{}l@{~~}||@{~~}l@{~~}||@{~~}l@{}}\inarr{#1}&\inarr{#2}&\inarr{#3}\end{array}}
%% \newcommand{\inparIV}[4]{\begin{array}{@{}l@{~~}||@{~~}l@{~~}||@{~~}l@{~~}||@{~~}l@{}}\inarr{#1}&\inarr{#2}&\inarr{#3}&\inarr{#4}\end{array}}

%% \newcommand{\inparaII}[2]{\begin{array}{@{}l@{\qquad\qquad}l@{}}\inarr{#1}&\inarr{#2}\end{array}}



%% \theoremstyle{definition}
%% \newtheorem{theorem}{Theorem}
%% \newtheorem{lemma}[theorem]{Lemma}
%% \newtheorem{corollary}[theorem]{Corollary}
%% \newtheorem{proposition}[theorem]{Proposition}
%% \newtheorem{claim}{Claim}
%% \newtheorem{conjecture}{Conjecture}
%% \newtheorem{definition}{Definition}
%% \newtheorem{notation}{Notation}
%% \newtheorem*{claim*}{Claim}
%% \newtheorem*{definition*}{Definition}
%% \newtheorem{example}{Example}
%% \newtheorem{remark}{Remark}
%% \newtheorem*{notation*}{Notation}

\crefformat{section}{#2\S{}#1#3}
\Crefname{section}{Section}{Section}
\Crefformat{section}{Section #2#1#3}

\Crefname{figure}{\text{Figure}}{\text{Figures}}
\crefname{corollary}{\text{Corollary}}{\text{corollaries}}
\Crefname{corollary}{\text{Corollary}}{\text{Corollaries}}
\crefname{lemma}{\text{Lemma}}{\text{Lemmas}}
\Crefname{lemma}{\text{Lemma}}{\text{Lemmas}}
\crefname{theorem}{\text{Theorem}}{\text{Theorems}}
\Crefname{theorem}{\text{Theorem}}{\text{Theorems}}
\crefname{proposition}{\text{Prop.}}{\text{Propositions}}
\Crefname{proposition}{\text{Proposition}}{\text{Propositions}}
\crefname{definition}{\text{Def.}}{\text{Definitions}}
\Crefname{definition}{\text{Definition}}{\text{Definitions}}

\newcommand{\setof}[1]{\{\, #1 \,\}}
\newcommand{\setofz}[1]{\{ #1 \}}
\newcommand{\suchthat}{\;|\;}
\newcommand{\txtsub}[2]{{#1}_{\text{#2}}}

\DeclareMathOperator{\disj}{{\#}}
%% \newcommand{\textdom}[1]{\mathsf{#1}}
%% \newcommand{\textcode}[1]{#1}
%% \newcommand{\kw}[1]{\textbf{\textcode{#1}}}
%% \newcommand{\skipc}{\kw{skip}}
%% \newcommand{\ite}[3]{\kw{if}\;#1\:\kw{then}\;#2\;\kw{else}\;#3}
%% \newcommand{\itne}[2]{\kw{if}\;#1\:\kw{then}\;#2}
%% \newcommand{\while}[2]{\kw{while}\;#1\;\kw{do}\;#2}
%% \newcommand{\ALT}{\;|\;}
%\newcommand{\print}{\kw{print }}
%\newcommand{\assert}[1]{\kw{assert}\;#1}
\newcommand{\fencec}{\kw{fence()}}
\newcommand{\fencesc}{\kw{fence-sc}}
\newcommand{\fencerel}{\kw{fence-rel}}
\newcommand{\fenceacq}{\kw{fence-acq}}
\newcommand{\fencerelacq}{\kw{fence-ra}}
\newcommand{\relw}{\kw{rel}}
\newcommand{\acqr}{\kw{acq}}
\newcommand{\sca}{\kw{sc}}
\newcommand{\cas}[4]{\textrm{CAS}({#1},{#2},{#3},{#4})}
\newcommand{\wait}[1]{\kw{wait}\;#1}
\newcommand{\when}[2]{\kw{when}\;#1\;\kw{do}\;#2}
\newcommand{\lock}[1]{\kw{lock}({#1})}
\newcommand{\unlock}[1]{\kw{unlock}({#1})}
%\newcommand{\comment}[1]{\color{CommentColor}{\quad\texttt{/*}\textit{ #1 }\texttt{*/}}}
\newcommand{\mfence}{\texttt{MFENCE}}
\newcommand{\fai}[1]{\textrm{FAA}({#1})}



%% \newcommand{\set}[1]{\{{#1}\}}
%% \newcommand{\size}[1]{|{#1}|}
%% \newcommand{\sem}[1]{\llbracket #1 \rrbracket}
%% \newcommand{\pfn}{\rightharpoonup}
%% \newcommand{\nin}{\not\in}
%% \newcommand{\suq}{\subseteq}
%% \newcommand{\N}{{\mathbb{N}}}
%% \newcommand{\tup}[1]{{\langle{#1}\rangle}}
%% \newcommand{\block}[1]{\langle {#1}\rangle}
%% \newcommand{\dom}[1]{\textit{dom}{({#1})}}
%% \newcommand{\codom}[1]{\textit{codom}{({#1})}}
%% \newcommand{\fv}[1]{\textit{fv}{[{#1}]}}
%% \newcommand{\til}{,\ldots,}
%% \newcommand{\cuptil}{\cup\ldots\cup}
%% \newcommand{\uplustil}{\uplus\ldots\uplus}
%% \renewcommand*{\mathellipsis}{\mathinner{{\ldotp}{\ldotp}{\ldotp}}}
%% \newcommand{\powerset}[1]{\mathcal{P}({#1})}
%% \newcommand{\finpowerset}[1]{\mathcal{P}_{<\omega}({#1})}
%% \newcommand{\defeq}{\mathrel{\stackrel{\mathsf{def}}{=}}}
%% \newcommand{\ttt}{\texttt}
%% \newcommand{\Suc}{\mathsf{Successor}}
%% \newcommand{\false}{\mathsf{False}}
%% \newcommand{\true}{\mathsf{True}}

%% \newcommand{\id}{{id}}
%% \newcommand{\typ}{{typ}}
%% \newcommand{\lab}{{lab}}
%% \newcommand{\loc}{{loc}}
%% \newcommand{\tid}{{tid}}
%% \newcommand{\val}{{val}}
%% \newcommand{\ord}{{ord}}
%% \newcommand{\upd}{{upd}}

%% \newcommand{\rlab}{\texttt{R}}
%% \newcommand{\wlab}{\texttt{W}}
%% \newcommand{\ulab}{\texttt{U}}
%% \newcommand{\slab}{\texttt{S}}
%% \newcommand{\tlab}{\texttt{T}}
%% \newcommand{\flab}{\texttt{F}}
%% \newcommand{\xlab}{\texttt{X}}
%% \newcommand{\ylab}{\texttt{Y}}


\newcommand{\relo}{{\texttt{rel}}}
\newcommand{\acqo}{{\texttt{acq}}}
%% \newcommand{\sco}{{\texttt{sc}}}
%% \newcommand{\na}{\texttt{na}}
%% \newcommand{\pln}{\texttt{pln}}
%% \newcommand{\atm}{\texttt{atm}}
\newcommand{\ra}{\texttt{ra}}
%% \newcommand{\nf}{\texttt{nf}}
%% \newcommand{\rlx}{\texttt{rlx}}
%% %\newcommand{\unord}{\texttt{uno}}
%% \newcommand{\relacqo}{{\texttt{relacq}}}

%% \newcommand{\Typ}{{\textdom{Typ}}}
%% \newcommand{\Lab}{{\textdom{Lab}}}
%% \newcommand{\Loc}{{\textdom{Loc}}}
%% \newcommand{\Val}{{\textdom{Val}}}
\newcommand{\AVal}{{\textdom{G}}}
%% \newcommand{\Tid}{{\textdom{Tid}}}
%% \newcommand{\Ord}{{\textdom{Ord}}}
%% \newcommand{\Reg}{{\textdom{Reg}}}
%% \newcommand{\Time}{{\textdom{Time}}}
%% \newcommand{\Timemap}{{\textdom{Timemap}}}
%% \newcommand{\Id}{{\textdom{Id}}}

\newcommand{\step}{\longrightarrow}
\newcommand{\astep}[1]{\xrightarrow{#1}}
\newcommand{\bstep}[1]{\xRightarrow{#1}}
\newcommand{\cstep}[2]{\xrightarrow[{#1}]{{#2}}}
\newcommand{\dstep}[2]{\dhxrightarrow[{#1}]{{#2}}}
\newcommand{\Pfin}{\txtsub{P}{final}}
\newcommand{\fin}{\text{final}}
\newcommand{\rulename}[1]{{\textsc{({#1})}}}
\newcommand{\onestep}[2]{{\textsc{({#1}}{({#2})}\textsc{)}}}

\newcommand{\ts}[1]{\mbox{\small#1}}
\newcommand{\rlxmsg}[3]{\tup{#1\mathbin{:}#2\text{\small@}#3}}
\newcommand{\updmsg}[4]{\tup{#1\mathbin{:}#2\text{\small@}(#3,#4]}}
%% \newcommand{\msg}[5]{\tup{#1\mathbin{:}#2\text{\small@}(#3,#4],#5}}
%\newcommand{\updmsg}[4]{\tup{#1\mathbin{:}#2\text{\small@}(#3,#4]}}
%\newcommand{\msg}[5]{\tup{#1\mathbin{:}#2\text{\small@}(#3,#4],#5}}
%\newcommand{\updmsg}[4]{\tup{#1\mathbin{:}#2\text{\small@}#3\text{-}#4}}
%\newcommand{\msg}[5]{\tup{#1\mathbin{:}#2\text{\small@}#3\text{-}#4,#5}}
\newcommand{\vtup}[2]{#1\text{\small@}#2}

%% \newcommand{\Msg}{{\textdom{Msg}}}
%% \newcommand{\MsgId}{{\textdom{MsgId}}}
%% \newcommand{\Snapshot}{{\textdom{Snapshot}}}
%% \newcommand{\View}{{\textdom{View}}}
%% \newcommand{\Cell}{{\textdom{Cell}}}
%% \newcommand{\LState}{{\textdom{Local-State}}}
%% \newcommand{\Commit}{{\textdom{Commit}}}
%% \newcommand{\Memory}{{\textdom{Memory}}}
%% \newcommand{\Thread}{{\textdom{Thread}}}

\newcommand{\tmin}{\in}
\newcommand{\tmle}{\leqslant}
\newcommand{\tcom}{{\cal V}}
%\newcommand{\tcom}{V}
\newcommand{\gsco}{{\cal S}}
%\newcommand{\gsco}{F}
%\newcommand{\mem}{{\cal M}}
\newcommand{\mem}{M}
\newcommand{\scmap}{S}
\newcommand{\msgid}{i}
\newcommand{\gts}{\mathcal{T\!S}}
\newcommand{\lts}{\mathit{TS}}
\newcommand{\tconf}{{\mathbf{TC}}}
\newcommand{\mconf}{{\mathbf{MS}}}
\newcommand{\lstate}{\texttt{st}}
\newcommand{\lview}{\texttt{view}}
\newcommand{\lprmem}{\texttt{prm}}
\newcommand{\lmem}{\txtsub{\mem}{p}}
%\newcommand{\lprom}{{\cal P}}
\newcommand{\lprom}{P}
\newcommand{\limitedmem}{{\cal L}}
\newcommand{\ogts}{\txtsub{\gts}{fut}}
\newcommand{\ogsco}{\txtsub{\gsco}{future}}
\newcommand{\olts}{\txtsub{\lts}{fut}}
\newcommand{\omem}{\txtsub{\mem}{future}}
\newcommand{\omsc}{\txtsub{\msc}{future}}
\newcommand{\lr}{{\text{r}}}
\newcommand{\lw}{{\text{w}}}
\newcommand{\cur}{cur}
%% \newcommand{\acq}{acq}
%% \newcommand{\rel}{rel}
%% \newcommand{\lsc}{sc}
%% \newcommand{\lscm}{s\!f}

\newcommand{\suc}{{succ}}
\newcommand{\lsuc}{\texttt{succ}}

\newcommand{\locprm}{P}
\newcommand{\llocprm}{\texttt{promise}}
\newcommand{\SUBS}{\mathit{sub}}
\newcommand{\lSUBS}{\texttt{sub}}

% some C11 notions

\newcommand{\SB}{\mathit{sb}}
\newcommand{\RF}{\mathit{rf}{}}
\newcommand{\FR}{\mathit{fr}{}}
\newcommand{\FRx}{\mathit{fr_x}}
\newcommand{\FRy}{\mathit{fr_y}}
\newcommand{\PO}{\mathit{po}}
\newcommand{\AT}{\mathit{at}}
\newcommand{\RMW}{\mathit{rmw}}
\newcommand{\POAA}{\mathit{po\text{-}aa}}
\newcommand{\MO}{\mathit{mo}}
\newcommand{\MOx}{\mathit{mo_x}}
\newcommand{\MOy}{\mathit{mo_y}}
\newcommand{\MOf}{\mathit{mo_f}}
\newcommand{\SC}{\mathit{sc}}
\newcommand{\CO}{\mathit{co}}
\newcommand{\ICO}{\mathit{ico}}
\newcommand{\HB}{\mathit{hb}}
\newcommand{\isW}{\mathsf{W}}
\newcommand{\isR}{\mathsf{R}}
\newcommand{\TSOr}{\mathit{tso}}
\newcommand{\WB}{\mathit{wb}}
\newcommand{\PORF}{(\PO \cup \RF)^+}
\newcommand{\IPO}{\mathit{ipo}}
\newcommand{\DEPS}{\mathit{deps}}
\newcommand{\valr}{{val_\lr}}
\newcommand{\valw}{{val_\lw}}
\newcommand{\ordr}{{ord_\lr}}
\newcommand{\ordw}{{ord_\lw}}
\newcommand{\labf}{{lab}}
\newcommand{\E}{{\cal E}}
\newcommand{\ER}{{\cal R}}
\newcommand{\EW}{{\cal W}}
\newcommand{\EU}{{\cal U}}
\newcommand{\ES}{{\cal S}}
\newcommand{\EF}{{\cal F}}
\newcommand{\AFr}{{\cal F}_\relo}
\newcommand{\AFa}{{\cal F}_\acqo}
\newcommand{\AF}{{\cal F}}

\newcommand{\prog}{\mathcal{P}}
\newcommand{\progb}{\mathcal{Q}}

%% \newcommand{\lE}{{\texttt{E}}}
%% \newcommand{\lM}{{\texttt{M}}}
%% \newcommand{\lS}{{\texttt{S}}}
%% \newcommand{\lR}{{\texttt{R}}}
%% \newcommand{\lW}{{\texttt{W}}}
%% \newcommand{\lU}{{\texttt{U}}}
%% \newcommand{\lF}{{\texttt{F}}}
%% \newcommand{\lSB}{{\color{gray!40!black} \texttt{sb}}}
%% \newcommand{\lRF}{{\color{green!60!black} \texttt{rf}}}
%% \newcommand{\lSYN}{{\color{red!60!black} \texttt{sw}}}
%% \newcommand{\lPO}{{\color{gray} \texttt{po}}}
%% \newcommand{\lHB}{{\color{blue} \texttt{hb}}}
%% \newcommand{\lWB}{\texttt{wb}}
%% \newcommand{\lRLX}{{\color{red} \texttt{rlx}}}
%% \newcommand{\lACQ}{{\color{blue} \texttt{acq}}}
%% \newcommand{\lAT}{{\color{brown} \texttt{at}}}
%% \newcommand{\lRMW}{{\color{brown!90!black} \texttt{rmw}}}
%% \newcommand{\lMO}{{\color{orange} \texttt{mo}}}
%% \newcommand{\lMOx}{{\color{orange} \texttt{mo}}_x}
%% \newcommand{\lMOy}{{\color{orange} \texttt{mo}}_y}
%% \newcommand{\lRSEQ}{{\color{purple!60!black} \texttt{rs}}}
%% \newcommand{\lSC}{{\color{violet!60!black} \texttt{sc}}}
%% \newcommand{\lFR}{{\color{purple} \texttt{fr}}}
%% \newcommand{\lREL}{{\color{olive} \texttt{rel}}}
%% \newcommand{\lCONFLICT}{{\color{olive} \texttt{conflict}}}
%% \newcommand{\lRACE}{{\color{olive} \texttt{race}}}
%% \newcommand{\lNARACE}{{\color{olive} \texttt{na-race}}}
%% \newcommand{\lURR}{{\color{purple} \texttt{urr}}}
%% \newcommand{\lRWR}{{\color{purple} \texttt{rwr}}}
%% \newcommand{\lSCR}{{\color{purple} \texttt{scr}}}

\newcommand{\Einit}{{\texttt{E}_{\texttt{init}}}}
\newcommand{\SBinit}{{\texttt{SB}_{\texttt{init}}}}

%\newcommand{\lSCREL}{{\color{violet} \texttt{screl}}}
\newcommand{\LOC}{\mathit{loc}}

\newcommand{\lloc}{{\texttt{loc}}}
\newcommand{\lid}{{\texttt{id}}}
\newcommand{\ltyp}{{\texttt{typ}}}
\newcommand{\lupd}{{\texttt{upd}}}

%% blue, brown, cyan, darkgray, gray, green, lime, magenta, olive, orange, pink, purple, red, teal, violet, yellow.

\newcommand{\lcur}{{\texttt{cur}}}
\newcommand{\lrel}{{\texttt{rel}}}
\newcommand{\lacq}{{\texttt{acq}}}
\newcommand{\srlx}{{\texttt{srlx}}}
\newcommand{\lacqrel}{{\texttt{acqrel}}}
\newcommand{\lsco}{{\texttt{sc}}}
\newcommand{\lscmo}{{\texttt{sf}}}
\newcommand{\lread}{{\texttt{rd}}}
\newcommand{\lwrite}{{\texttt{wr}}}
\newcommand{\lrw}{{\texttt{rlx}}}
\newcommand{\lur}{{\texttt{pln}}}
\newcommand{\lall}{\ast}
\newcommand{\lrdwr}{\setofz{\lread,\lwrite}}
\newcommand{\lval}{{\texttt{val}}}
\newcommand{\lfrom}{{\texttt{from}}}
\newcommand{\lto}{{\texttt{to}}}
\newcommand{\ltime}{{\texttt{t}}}
%\newcommand{\lsnapshot}{{\texttt{rel}}}
%% \newcommand{\insertadd}{\hookleftarrow^{\!\!\!\!\!\!\rotatebox{0}{\scriptsize$\uplus$}\;}}
%% \newcommand{\insertsplit}{\hookleftarrow^{\!\!\!\!\!\rotatebox{90}{\tiny$\boxminus$}\;}}
%% \newcommand{\insertupdate}{\hookleftarrow^{\!\!\!\!\!\!\rotatebox{0}{\scriptsize$\downarrow$}\;}}
\newcommand{\insertadd}{\hookleftarrow^{\!\!\!\!\!\textsc{a}\;}}
\newcommand{\insertsplit}{\hookleftarrow^{\!\!\!\!\!\textsc{s}\;}}
\newcommand{\insertupdate}{\hookleftarrow^{\!\!\!\!\!\textsc{u}\;}}
\newcommand{\insertsc}{\hookleftarrow^{\!\!\!\!\!\textsc{c}\;}}

\newcommand{\joinupdate}{:=\!\sqcup\;}

\newcommand{\mrel}{R}
\newcommand{\lmrel}{\texttt{view}}
\newcommand{\msc}{S}

\newcommand{\I}{{\cal I}}

%% \newcommand{\rst}[1]{|_{#1}}
%% \newcommand{\imm}[1]{{#1}{\rst{\text{imm}}}}

%% \newcommand{\add}[1]{\mathsf{Add}({#1})}
%% \newcommand{\addatomic}[1]{\textsf{AddRMW}({#1})}
%% \newcommand{\lX}{{\texttt{X}}}


\newcommand{\iteb}[2]{({#1}\mathsf{\,{?}\,}{#2})}
\newcommand{\rlxcur}{V}
\newcommand{\lrlxcur}{\texttt{V}}
%% \newcommand{\view}{V}

\newcommand{\sbul}{\raisebox{.3ex}{\tiny$\bullet$\;\;}}


\newcommand{\syscallt}{{\text{SysCall}(v)}}
\newcommand{\silentt}{{\text{Silent}}}

% \SetSymbolFont{stmry}{bold}{U}{stmry}{m}{n} % this fixes warnings when \boldsymbol is used with stmaryrd included

% \extrarowheight=\jot	% else, arrays are scrunched compared to, say, aligned
% \newcolumntype{.}{@{}}
% % Array {rMcMl} modifies array {rcl}, putting mathrel-style spacing
% % around the centered column. (We used this, for example, in laying
% % out some of Iris' axioms. Generally, aligned is simpler but aligned
% % does not work in mathpar because \\ inherits mathpar's 2em vskip.)
% % The capital M stands for THICKMuskip. The smaller medmuskip would be
% % right for mathbin-style spacing.
% \newcolumntype{M}{@{\mskip\thickmuskip}}

\definecolor{StringRed}{rgb}{.637,0.082,0.082}
\definecolor{CommentGreen}{rgb}{0.0,0.55,0.3}
\definecolor{KeywordBlue}{rgb}{0.0,0.3,0.55}
\definecolor{LinkColor}{rgb}{0.55,0.0,0.3}
\definecolor{CiteColor}{rgb}{0.55,0.0,0.3}
\definecolor{HighlightColor}{rgb}{0.0,0.0,0.0}


\definecolor{grey}{rgb}{0.5,0.5,0.5}
\definecolor{red}{rgb}{1,0,0}
\definecolor{darkgreen}{rgb}{0.0,0.7,0.0}

\hypersetup{%
  linktocpage=true, pdfstartview=FitV,
  breaklinks=true, pageanchor=true, pdfpagemode=UseOutlines,
  plainpages=false, bookmarksnumbered, bookmarksopen=true, bookmarksopenlevel=3,
  hypertexnames=true, pdfhighlight=/O,
%  colorlinks=true,linkcolor=LinkColor,citecolor=CiteColor,
%  urlcolor=LinkColor
}


\newcommand{\compile}[1]{{\llparenthesis{#1}\rrparenthesis}}%{{\mathtt{compile}({#1})}}

%% \newcommand{\squishlist}[1][$\tinybullet$]{
%%  \begin{list}{#1}
%%   { \setlength{\itemsep}{0pt}
%%      \setlength{\parsep}{0pt}
%%      \setlength{\topsep}{1pt}
%%      \setlength{\partopsep}{0pt}
%%      \setlength{\leftmargin}{1.2em}
%%      \setlength{\labelwidth}{0.5em}
%%      \setlength{\labelsep}{0.4em} } }
%% \newcommand{\squishend}{
%%   \end{list}  }


\newcommand{\tinybullet}{\raisebox{0.25ex}{\tiny$\bullet$}}

\definecolor{AssertColor}{rgb}{0,0,0.5}
\newcommand{\asrt}[1]{
  {\color{AssertColor}\ensuremath{
  \left\{
    \begin{array}{@{}l@{}}
      #1
    \end{array}
  \right\}}}
}

%%% Local Variables:
%%% mode: latex
%%% TeX-master: "main"
%%% End:
