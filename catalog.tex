\section{A Catalog of Memory Models}
\label{sec:catalog}

\newpage
\onecolumn

\begin{landscape}

\begin{table*}
\scriptsize
\begin{tabular}{|c|l|c|c|c|c|c|c|c|c|c|c|c|c|c|c|c|c|c|c|c|c|c|c|c|c|c|c|c|c|c|c|c|c|c|c|} % 35
 \hline

                                                      &
 \multirow{3}{*}{Model}                               & 
 \multicolumn{ 4}{c|}{\multirow{2}{*}{Compilation}}   &
 \multicolumn{17}{c|}{Transformations}                &
 \multicolumn{ 4}{c|}{\multirow{2}{*}{Reasoning}}     &
 \multicolumn{ 9}{c|}{\multirow{2}{*}{Features}}      \\ 

 \cline{7-23}
                             &
                             &
 \multicolumn{4}{c|}{}       &
 \multicolumn{14}{c|}{Local} &
 \multicolumn{3}{c|}{Global} &
 \multicolumn{4}{c|}{}       &
 \multicolumn{9}{c|}{}       \\ 
 
 \hline
                                     &
                                     &
 \rotatebox[origin=c]{270}{x86}      & 
 \rotatebox[origin=c]{270}{Power}    & 
 \rotatebox[origin=c]{270}{ARMv7}    & 
 \rotatebox[origin=c]{270}{ARMv8}    & 
 
 \multicolumn{4}{c|}{\rotatebox[origin=c]{270}{RI}}   &
 \multicolumn{4}{c|}{\rotatebox[origin=c]{270}{RE}}   &

 \rotatebox[origin=c]{270}{ILE}    &
 \rotatebox[origin=c]{270}{SLI}    &
 \rotatebox[origin=c]{270}{RM}     &
 \rotatebox[origin=c]{270}{S}      &
 \rotatebox[origin=c]{270}{TP}     &
 \rotatebox[origin=c]{270}{CSE}    &
 \rotatebox[origin=c]{270}{RP}     &
 \rotatebox[origin=c]{270}{VR}     &
 \rotatebox[origin=c]{270}{TI}     &
 
 \rotatebox[origin=c]{270}{DRF}                                &
 \rotatebox[origin=c]{270}{COH}                                &
 \rotatebox[origin=c]{270}{no-UB}                              &
 % \rotatebox[origin=c]{270}{\makecell{$\lPO\lRF$}}              & 
 \rotatebox[origin=c]{270}{no-OOTA}                            &                              

 \rotatebox[origin=c]{270}{NA}                      &
 \rotatebox[origin=c]{270}{RLX}                     &
 \rotatebox[origin=c]{270}{RA}                      &
 \rotatebox[origin=c]{270}{SC}                      &
 \rotatebox[origin=c]{270}{F-RA}                    &
 \rotatebox[origin=c]{270}{F-SC}                    &
 \rotatebox[origin=c]{270}{RMW}                     &
 \rotatebox[origin=c]{270}{Lock}                    &
 \rotatebox[origin=c]{270}{\makecell{Mix.Sz.}}      \\ 

 \hline
  & & & & & &
 \rotatebox[origin=c]{270}{SL}       &
 \rotatebox[origin=c]{270}{SS}       &
 \rotatebox[origin=c]{270}{LL}       &
 \rotatebox[origin=c]{270}{LS}       &
 \rotatebox[origin=c]{270}{SL}       &
 \rotatebox[origin=c]{270}{SS}       &
 \rotatebox[origin=c]{270}{LL}       &
 \rotatebox[origin=c]{270}{LS}       &
 & & & & & & & & & & & & & & & & & & & & & \\
  
 \Xhline{2\arrayrulewidth}
 
 \multirow{3}{*}{\rotatebox[origin=c]{270}{\makecell{SeqCst}}}   

 & SC 
     &             
     % compilation 
     \badcell & \badcell & \badcell & \badcell & 
     % reorderings
     \badcell & \badcell & \badcell & \badcell &
     % eliminations
     \okcell & \okcell & \okcell & \okcell &
     % irrelevant load elim & speculative load intro 
     \okcell & \okcell &
     % roach motel 
     \okcell & 
     % strength.
     \unkwcell &
     % trace preserving 
     \okcell &
     % common subexpression elimination
     \badcell &
     % global: register promotion, value range, thread inlining 
     \okcell & \unkwcell & \okcell &
     % DRF, COH, no-UB, no-OOTA
     \ldrf & \okcell & \okcell & \okcell &
     % NA, RLX, RA, SC 
     \badcell & \badcell & \badcell & \okcell & 
     % F-RA, F-SC
     \badcell & \badcell & 
     % RMW, Lock
     \badcell & \badcell & 
     % Mix.sz.
     \badcell 
     \\ \cline{2-36}

 & SC-Hs 
     &
     % compilation 
     \badcell & \badcell & \badcell & \badcell & 
     % reorderings
     \badcell & \badcell & \badcell & \badcell & 
     % eliminations
     \okcell & \okcell & \okcell & \okcell &
     % irrelevant load elim & speculative load intro 
     \okcell & \okcell &
     % roach motel 
     \okcell & 
     % strength.
     \unkwcell &
     % trace preserving 
     \okcell &
     % common subexpression elimination
     \badcell &
     % global: register promotion, value range, thread inlining 
     \okcell & \unkwcell & \okcell & 
     % DRF, COH, no-UB, no-OOTA
     \ldrf & \okcell & \okcell & \okcell &
     % NA, RLX, RA, SC 
     \okcell & \badcell & \badcell & \okcell & 
     % F-RA, F-SC
     \badcell & \badcell & 
     % RMW, Lock
     \okcell & \badcell &
     % Mix.sz. 
     \badcell 
     \\ \cline{2-36}

 & DRFx
     &           
     % compilation 
     \badcell & \badcell & \badcell & \badcell & 
     % reorderings
     \badcell & \badcell & \badcell & \badcell & 
     % eliminations
     \okcell & \okcell & \okcell & \okcell &
     % irrelevant load elim & speculative load intro 
     \okcell & \badcell &
     % roach motel 
     \okcell & 
     % strength.
     \unkwcell &
     % trace preserving 
     \okcell &
     % common subexpression elimination
     \badcell &
     % global: register promotion, value range, thread inlining 
     \okcell & \unkwcell & \okcell &                                              
     % DRF, COH, no-UB, no-OOTA
     \ldrf & \okcell & \warncell & \okcell &
     % NA, RLX, RA, SC 
     \okcell & \badcell & \badcell & \okcell & 
     % F-RA, F-SC
     \badcell & \badcell & 
     % RMW, Lock
     \badcell & \badcell & 
     % Mix.sz.
     \badcell 
     \\ \Xhline{2\arrayrulewidth}

 \multirow{2}{*}{\rotatebox[origin=c]{270}{\makecell{TSO\\PSO}}}   

 & BMM
     &
     % compilation 
     \okcell & \badcell & \badcell & \badcell & 
     % reorderings
     \okcell & \badcell & \badcell & \badcell & 
     % eliminations
     \okcell & \okcell & \okcell & \badcell &  
     % irrelevant load elim & speculative load intro 
     \okcell & \okcell &
     % roach motel 
     \badcell & 
     % strength.
     \unkwcell &
     % trace preserving 
     \okcell &
     % common subexpression elimination
     \badcell &
     % global: register promotion, value range, thread inlining 
     \unkwcell & \unkwcell & \badcell &
     % DRF, COH, no-UB, no-OOTA
     \edrf & \okcell & \okcell & \okcell &
     % NA, RLX, RA, SC 
     \okcell & \badcell & \badcell & \okcell & 
     % F-RA, F-SC
     \badcell & \badcell & 
     % RMW, Lock
     \badcell & \okcell & 
     % Mix.sz.
     \badcell 
     \\ \cline{2-36}

 & RMMOA
     &
     % compilation 
     \okcell & \badcell & \badcell & \badcell & 
     % reorderings
     \okcell & \okcell & \badcell & \badcell & 
     % eliminations
     \unkwcell & \unkwcell & \unkwcell & \unkwcell &  
     % irrelevant load elim & speculative load intro 
     \unkwcell & \unkwcell &
     % roach motel 
     \unkwcell & 
     % strength.
     \unkwcell &
     % trace preserving 
     \unkwcell &
     % common subexpression elimination
     \unkwcell &
     % global: register promotion, value range, thread inlining 
     \unkwcell & \unkwcell & \unkwcell &
     % DRF, COH, no-UB, no-OOTA
     \edrf & \unkwcell & \okcell & \okcell &
     % NA, RLX, RA, SC 
     \okcell & \badcell & \badcell & \badcell & 
     % F-RA, F-SC
     \badcell & \badcell & 
     % RMW, Lock
     \badcell & \okcell & 
     % Mix.sz.
     \badcell 
     \\ \Xhline{2\arrayrulewidth}

 \multirow{5}{*}{\rotatebox[origin=c]{270}{\makecell{$\lPO\lRF$\\acyc}}}   

 & RC11
     &
     % compilation 
     \okcell & \badcell & \badcell & \badcell &  
     % reorderings
     \okcell & \okcell & \okcell & \badcell & 
     % eliminations
     \okcell & \okcell & \okcell & \badcell & 
     % irrelevant load elim & speculative load intro 
     \unkwcell & \badcell &
     % roach motel 
     \okcell & 
     % strength.
     \okcell &
     % trace preserving 
     \okcell &
     % common subexpression elimination
     \badcell &
     % global: register promotion, value range, thread inlining 
     \okcell & \unkwcell & \okcell &                                              
     % DRF, COH, no-UB, no-OOTA
     \edrf & \okcell & \warncell & \okcell &
     % NA, RLX, RA, SC 
     \okcell & \okcell & \okcell & \okcell & 
     % F-RA, F-SC
     \okcell & \okcell & 
     % RMW, Lock
     \okcell & \badcell & 
     % Mix.sz.
     \badcell 
     \\ \cline{2-36}

 & ORC11
     &
     % compilation 
     \okcell & \badcell & \badcell & \badcell &  
     % reorderings
     \okcell & \okcell & \okcell & \badcell & 
     % eliminations
     \okcell & \okcell & \okcell & \badcell & 
     % irrelevant load elim & speculative load intro 
     \unkwcell & \badcell &
     % roach motel 
     \okcell & 
     % strength.
     \okcell &
     % trace preserving 
     \okcell &
     % common subexpression elimination
     \badcell &
     % global: register promotion, value range, thread inlining 
     \okcell & \unkwcell & \okcell &
     % DRF, COH, no-UB, no-OOTA
     \edrf & \okcell & \warncell & \okcell &
     % NA, RLX, RA, SC 
     \okcell & \okcell & \okcell & \badcell & 
     % F-RA, F-SC
     \okcell & \badcell & 
     % RMW, Lock
     \okcell & \badcell & 
     % Mix.sz.
     \badcell 
     \\ \cline{2-36}

 & RAR
     & 
     % compilation 
     \okcell & \badcell & \badcell & \badcell &  
     % reorderings
     \okcell & \okcell & \okcell & \badcell & 
     % eliminations
     \okcell & \okcell & \okcell & \badcell & 
     % irrelevant load elim & speculative load intro 
     \unkwcell & \badcell &
     % roach motel 
     \okcell & 
     % strength.
     \okcell &
     % trace preserving 
     \okcell &
     % common subexpression elimination
     \badcell &
     % global: register promotion, value range, thread inlining 
     \okcell & \unkwcell & \okcell &
     % DRF, COH, no-UB, no-OOTA
     \edrf & \okcell & \okcell & \okcell &
     % NA, RLX, RA, SC 
     \badcell & \okcell & \okcell & \badcell & 
     % F-RA, F-SC
     \badcell & \badcell & 
     % RMW, Lock
     \okcell & \badcell & 
     % Mix.sz.
     \badcell 
     \\ \cline{2-36}

 & OCMM
     & 
     % compilation 
     \okcell & \badcell & \badcell & \badcell &  
     % reorderings
     \okcell & \okcell & \okcell & \badcell & 
     % eliminations
     \okcell & \okcell & \okcell & \badcell & 
     % irrelevant load elim & speculative load intro  
     \unkwcell & \unkwcell &
     % roach motel 
     \unkwcell & 
     % strength.
     \unkwcell &
     % trace preserving 
     \okcell &
     % common subexpression elimination
     \okcell &
     % global: register promotion, value range, thread inlining 
     \unkwcell & \unkwcell & \unkwcell &
     % DRF, COH, no-UB, no-OOTA
     \ldrf & \warncell & \okcell & \okcell &
     % NA, RLX, RA, SC 
     \okcell & \badcell & \badcell & \okcell & 
     % F-RA, F-SC
     \badcell & \badcell & 
     % RMW, Lock
     \okcell & \badcell & 
     % Mix.sz.
     \badcell 
     \\ \cline{2-36}

 & JAM
     & 
     % compilation 
     \okcell & \badcell & \badcell & \badcell &  
     % reorderings
     \unkwcell & \unkwcell & \unkwcell & \unkwcell &  
     % eliminations
     \unkwcell & \unkwcell & \unkwcell & \unkwcell & 
     % irrelevant load elim & speculative load intro  
     \unkwcell & \unkwcell &
     % roach motel 
     \unkwcell & 
     % strength.
     \okcell &
     % trace preserving 
     \unkwcell &
     % common subexpression elimination
     \badcell &
     % global: register promotion, value range, thread inlining 
     \unkwcell & \unkwcell & \unkwcell &
     % DRF, COH, no-UB, no-OOTA
     \edrf & \okcell & \okcell & \okcell &
     % NA, RLX, RA, SC 
     \okcell & \okcell & \okcell & \okcell & 
     % F-RA, F-SC
     \okcell & \okcell & 
     % RMW, Lock
     \okcell & \badcell & 
     % Mix.sz.
     \badcell 
     \\ \cline{2-36}

 & CRC
     &
     % compilation 
     \okcell & \badcell & \badcell & \badcell &  
     % reorderings
     \okcell & \badcell & \badcell & \badcell &  
     % eliminations
     \okcell & \okcell & \okcell & \badcell &  
     % irrelevant load elim & speculative load intro  
     \badcell & \badcell &
     % roach motel 
     \unkwcell & 
     % strength.
     \unkwcell &
     % trace preserving 
     \unkwcell &
     % common subexpression elimination
     \badcell &
     % global: register promotion, value range, thread inlining 
     \unkwcell & \unkwcell & \unkwcell &
     % DRF, COH, no-UB, no-OOTA
     \edrf & \okcell & \warncell & \okcell &
     % NA, RLX, RA, SC 
     \okcell & \badcell & \okcell & \badcell & 
     % F-RA, F-SC
     \badcell & \okcell & 
     % RMW, Lock
     \okcell & \badcell & 
     % Mix.sz.
     \badcell 
     \\ \Xhline{2\arrayrulewidth}

 \multirow{2}{*}{\rotatebox[origin=c]{270}{\makecell{$\lPPO\lRF$\\acyc}}}   

 & LKMM
     &           
     % compilation 
     \okcell & \okcell & \okcell & \okcell &  
     % reorderings
     \okcell & \okcell & \okcell & \okcell &
     % eliminations
     \unkwcell & \unkwcell & \unkwcell & \unkwcell &  
     % irrelevant load elim & speculative load intro 
     \unkwcell & \unkwcell &
     % roach motel 
     \unkwcell & 
     % strength.
     \unkwcell &
     % trace preserving 
     \badcell &
     % common subexpression elimination
     \badcell &
     % global: register promotion, value range, thread inlining 
     \unkwcell & \unkwcell & \unkwcell &
     % DRF, COH, no-UB, no-OOTA
     \edrf & \okcell & \okcell & \okcell &
     % NA, RLX, RA, SC 
     \badcell & \okcell & \okcell & \badcell & 
     % F-RA, F-SC
     \okcell & \okcell & 
     % RMW, Lock
     \okcell & \badcell & 
     % Mix.sz.
     \badcell 
     \\ \cline{2-36}

 & OHMM
     &
     % compilation 
     \unkwcell & \unkwcell & \unkwcell & \unkwcell &
     % reorderings
     \okcell & \okcell & \okcell & \okcell &
     % eliminations
     \okcell & \okcell & \okcell & \okcell &
     % irrelevant load elim & speculative load intro 
     \okcell & \okcell &
     % roach motel 
     \okcell & 
     % strength.
     \unkwcell &
     % trace preserving 
     \badcell &
     % common subexpression elimination
     \unkwcell &
     % global: register promotion, value range, thread inlining 
     \unkwcell & \unkwcell & \unkwcell &         
     % DRF, COH, no-UB, no-OOTA
     \edrf & \unkwcell & \okcell & \okcell &
     % NA, RLX, RA, SC 
     \okcell & \badcell & \badcell & \okcell & 
     % F-RA, F-SC
     \badcell & \badcell & 
     % RMW, Lock
     \badcell & \okcell & 
     % Mix.sz.
     \badcell 
     \\ \Xhline{2\arrayrulewidth}

 \multirow{7}{*}{\rotatebox[origin=c]{270}{\makecell{no-OOTA}}}   

 & JMM
     &            
     % compilation 
     & & & &
     % reorderings
     \okcell & \okcell & \okcell & \okcell &
     % eliminations
     \okcell & \okcell & \badcell & \badcell &
     % irrelevant load elim & speculative load intro  
     \okcell & \badcell &
     % roach motel 
     \badcell & 
     % strength.
     \unkwcell &
     % trace preserving 
     \okcell &
     % common subexpression elimination
     \badcell &
     % global: register promotion, value range, thread inlining 
     \unkwcell & \unkwcell & \badcell &
     % DRF, COH, no-UB, no-OOTA
     \edrf & \warncell & \okcell & \okcell &
     % NA, RLX, RA, SC 
     \okcell & \badcell & \badcell & \okcell & 
     % F-RA, F-SC
     \badcell & \badcell & 
     % RMW, Lock
     \okcell & \okcell & 
     % Mix.sz.
     \badcell 
     \\ \cline{2-36}

 & Prm
     &
     % compilation 
     \okcell & \okcell & \okcell & \okcell &  
     % reorderings
     \okcell & \okcell & \okcell & \okcell &
     % eliminations
     \okcell & \okcell & \okcell & \okcell &  
     % irrelevant load elim & speculative load intro  
     \okcell & \okcell &
     % roach motel 
     \okcell & 
     % strength.
     \okcell &
     % trace preserving 
     \okcell &
     % common subexpression elimination
     \okcell &
     % global: register promotion, value range, thread inlining 
     \okcell & \okcell & \badcell &
     % DRF, COH, no-UB, no-OOTA
     \edrf & \okcell & \okcell & \okcell &
     % NA, RLX, RA, SC 
     \okcell & \okcell & \okcell & \badcell & 
     % F-RA, F-SC
     \okcell & \okcell & 
     % RMW, Lock
     \okcell & \okcell & 
     % Mix.sz.
     \badcell 
     \\ \cline{2-36}

 & Wkmo
     &
     % compilation 
     \okcell & \okcell & \okcell & \okcell &
     % reorderings
     \okcell & \okcell & \okcell & \okcell &
     % eliminations
     \okcell & \okcell & \okcell & \okcell &  
     % irrelevant load elim & speculative load intro  
     \unkwcell & \okcell &
     % roach motel 
     \badcell & 
     % strength.
     \okcell &
     % trace preserving 
     \unkwcell &
     % common subexpression elimination
     \badcell &
     % global: register promotion, value range, thread inlining 
     \unkwcell & \unkwcell & \badcell &
     % DRF, COH, no-UB, no-OOTA
     \edrf & \okcell & \okcell & \okcell &
     % NA, RLX, RA, SC 
     \okcell & \okcell & \okcell & \okcell & 
     % F-RA, F-SC
     \okcell & \okcell & 
     % RMW, Lock
     \okcell & \badcell & 
     % Mix.sz.
     \badcell 
     \\ \cline{2-36}

 & MRD
     &
     % compilation 
     \okcell & \okcell & \okcell & \okcell &
     % reorderings
     \unkwcell & \unkwcell & \unkwcell & \unkwcell &
     % eliminations
     \unkwcell & \unkwcell & \unkwcell & \unkwcell &
     % irrelevant load elim & speculative load intro 
     \unkwcell & \unkwcell &
     % roach motel 
     \unkwcell & 
     % strength.
     \unkwcell &
     % trace preserving 
     \unkwcell &
     % common subexpression elimination
     \unkwcell &
     % global: register promotion, value range, thread inlining 
     \unkwcell & \unkwcell & \unkwcell &                                              
     % DRF, COH, no-UB, no-OOTA
     \edrf & \okcell & \okcell & \okcell &
     % NA, RLX, RA, SC 
     \badcell & \okcell & \badcell & \badcell & 
     % F-RA, F-SC
     \badcell & \badcell & 
     % RMW, Lock
     \badcell & \okcell & 
     % Mix.sz.
     \badcell 
     \\ \cline{2-36}

 & P-P/S
     &
     % compilation 
     \okcell & \okcell & \badcell & \badcell &
     % reorderings 
     \okcell & \okcell & \okcell & \okcell &
     % eliminations
     \okcell & \okcell & \okcell & \badcell &  
     % irrelevant load elim & speculative load intro  
     \okcell & \badcell &
     % roach motel 
     \okcell & 
     % strength.
     \badcell &
     % trace preserving 
     \unkwcell &
     % common subexpression elimination
     \badcell &
     % global: register promotion, value range, thread inlining 
     \unkwcell & \okcell & \badcell &
     % DRF, COH, no-UB, no-OOTA
     \unkwcell & \okcell & \warncell & \okcell &
     % NA, RLX, RA, SC 
     \okcell & \okcell & \okcell & \badcell & 
     % F-RA, F-SC
     \badcell & \badcell & 
     % RMW, Lock
     \okcell & \okcell & 
     % Mix.sz.
     \badcell 
     \\ \cline{2-36}

 & J/R
     &
     % compilation 
     \unkwcell & \badcell & \badcell & \badcell &
     % reorderings
     \unkwcell & \unkwcell & \badcell & \unkwcell &
     % eliminations
     \unkwcell & \unkwcell & \badcell & \badcell &  
     % irrelevant load elim & speculative load intro 
     \unkwcell & \unkwcell &
     % roach motel 
     \unkwcell & 
     % strength.
     \unkwcell &
     % trace preserving 
     \unkwcell &
     % common subexpression elimination
     \unkwcell &
     % global: register promotion, value range, thread inlining 
     \unkwcell & \unkwcell & \unkwcell &                                              
     % DRF, COH, no-UB, no-OOTA
     \edrf & \warncell & \okcell & \okcell &
     % NA, RLX, RA, SC 
     \badcell & \okcell & \badcell & \badcell & 
     % F-RA, F-SC
     \badcell & \badcell & 
     % RMW, Lock
     \okcell & \okcell & 
     % Mix.sz.
     \badcell 
     \\ \cline{2-36}

 & GOS
     &
     % compilation 
     \unkwcell & \unkwcell & \unkwcell & \unkwcell &
     % reorderings 
     \unkwcell & \unkwcell & \unkwcell & \unkwcell &
     % eliminations 
     \unkwcell & \unkwcell & \unkwcell & \unkwcell &
     % irrelevant load elim & speculative load intro 
     \unkwcell & \unkwcell &
     % roach motel 
     \unkwcell & 
     % strength.
     \unkwcell &
     % trace preserving 
     \unkwcell &
     % common subexpression elimination
     \unkwcell &
     % global: register promotion, value range, thread inlining 
     \unkwcell & \unkwcell & \unkwcell &                                              
     % DRF, COH, no-UB, no-OOTA
     \edrf & \unkwcell & \okcell & \okcell &
     % NA, RLX, RA, SC 
     \okcell & \badcell & \badcell & \badcell & 
     % F-RA, F-SC
     \badcell & \badcell & 
     % RMW, Lock
     \badcell & \okcell & 
     % Mix.sz.
     \badcell 
     \\ \Xhline{2\arrayrulewidth}

 \multirow{5}{*}{\rotatebox[origin=c]{270}{\makecell{OOTA}}}   

 & C11
     &            
     % compilation 
     \okcell & \okcell & \okcell & \okcell &
     % reorderings
     \okcell & \okcell & \okcell & \okcell &
     % eliminations
     \okcell & \okcell & \okcell & \badcell &  
     % irrelevant load elim & speculative load intro 
     \unkwcell & \badcell &
     % roach motel 
     \badcell & 
     % strength.
     \badcell &
     % trace preserving 
     \okcell &
     % common subexpression elimination
     \badcell &
     % global: register promotion, value range, thread inlining 
     \unkwcell & \unkwcell & \badcell &
     % DRF, COH, no-UB, no-OOTA
     \idrf & \okcell & \warncell & \badcell &
     % NA, RLX, RA, SC 
     \okcell & \okcell & \okcell & \okcell & 
     % F-RA, F-SC
     \okcell & \okcell & 
     % RMW, Lock
     \okcell & \okcell & 
     % Mix.sz.
     \okcell 
     \\ \cline{2-36}

 & JSMM
     &
     % compilation 
     \okcell & \okcell & \okcell & \okcell &
     % reorderings
     \unkwcell & \unkwcell & \unkwcell & \unkwcell &
     % eliminations
     \unkwcell & \unkwcell & \unkwcell & \unkwcell &
     % irrelevant load elim & speculative load intro 
     \unkwcell & \unkwcell &
     % roach motel 
     \unkwcell & 
     % strength.
     \unkwcell &
     % trace preserving 
     \unkwcell &
     % common subexpression elimination
     \unkwcell &
     % global: register promotion, value range, thread inlining 
     \unkwcell & \unkwcell & \unkwcell &
     % DRF, COH, no-UB, no-OOTA
     \idrf & & \okcell & \badcell &
     % NA, RLX, RA, SC 
     \okcell & \badcell & \badcell & \okcell & 
     % F-RA, F-SC
     \badcell & \badcell & 
     % RMW, Lock
     \okcell & \okcell & 
     % Mix.sz.
     \okcell 
     \\ \cline{2-36}

 & RMC
     &
     % compilation 
     \okcell & \okcell & \okcell & \okcell &
     % reorderings
     \unkwcell & \unkwcell & \unkwcell & \unkwcell &
     % eliminations
     \unkwcell & \unkwcell & \unkwcell & \unkwcell &
     % irrelevant load elim & speculative load intro 
     \unkwcell & \unkwcell &
     % roach motel 
     \unkwcell & 
     % strength.
     \unkwcell &
     % trace preserving 
     \unkwcell &
     % common subexpression elimination
     \unkwcell &
     % global: register promotion, value range, thread inlining 
     \unkwcell & \unkwcell & \unkwcell &
     % DRF, COH, no-UB, no-OOTA
     \idrf & \okcell & \okcell & \badcell &
     % NA, RLX, RA, SC 
     \badcell & \okcell & \okcell & \okcell & 
     % F-RA, F-SC
     \okcell & \okcell & 
     % RMW, Lock
     \okcell & \badcell & 
     % Mix.sz.
     \badcell 
     \\ \cline{2-36}


 & RAO
     &
     % compilation 
     \unkwcell & \unkwcell & \unkwcell & \unkwcell &
     % reorderings
     \unkwcell & \unkwcell & \unkwcell & \unkwcell &
     % eliminations
     \unkwcell & \unkwcell & \unkwcell & \unkwcell &
     % irrelevant load elim & speculative load intro 
     \unkwcell & \unkwcell &
     % roach motel 
     \unkwcell & 
     % strength.
     \unkwcell &
     % trace preserving 
     \unkwcell &
     % common subexpression elimination
     \unkwcell &
     % global: register promotion, value range, thread inlining 
     \unkwcell & \unkwcell & \unkwcell & 
     % DRF, COH, no-UB, no-OOTA
      & \warncell & \okcell & \badcell &
     % NA, RLX, RA, SC 
     \okcell & \badcell & \badcell & \okcell & 
     % F-RA, F-SC
     \badcell & \badcell & 
     % RMW, Lock
     \badcell & \badcell & 
     % Mix.sz.
     \badcell 
     \\ \cline{2-36}

 & TSC
     % compilation 
     & \unkwcell & \unkwcell & \unkwcell & \unkwcell &
     % reorderings
     \unkwcell & \unkwcell & \unkwcell & \unkwcell &
     % eliminations
     \unkwcell & \unkwcell & \unkwcell & \unkwcell &
     % irrelevant load elim & speculative load intro 
     \unkwcell & \unkwcell &
     % roach motel 
     \unkwcell & 
     % strength.
     \unkwcell &
     % trace preserving 
     \unkwcell &
     % common subexpression elimination
     \unkwcell &
     % global: register promotion, value range, thread inlining 
     \unkwcell & \unkwcell & \unkwcell & 
     % DRF, COH, no-UB, no-OOTA
     \idrf & \unkwcell & \okcell & \badcell &
     % NA, RLX, RA, SC 
     \okcell & \badcell & \badcell & \okcell & 
     % F-RA, F-SC
     \badcell & \badcell & 
     % RMW, Lock
     \badcell & \okcell & 
     % Mix.sz.
     \badcell 
     \\ \Xhline{2\arrayrulewidth}


\end{tabular}
\caption{
  % \textit{T.P.} --- trace preserving.
  % \textit{R.I.} --- reordering of independent instructions.
  % \textit{R.E.} --- redundunt load/store elimination.
  % \textit{I.L.E.} --- irrelevant load elimination.
  % \textit{S.L.I.} --- speculative load introduction.
  % \textit{S.} --- strengthening.
  % \textit{R.M.} --- roach motel reordering.
  % \textit{R.P.} --- register promotion.
  % \textit{V.R.} --- value range analysis based optimizations.
  % \textit{T.I.} --- thread inlining (sequentialization).
  % \textit{Int.} --- internal.
  % \textit{Ext.} --- external.
  % \textit{Loc.} --- local.
  % \textit{UB} --- undefined behavior.
  % \textit{OOTA} --- out-of-thin air values.
  % \textit{Mix.Sz.} --- mixed-size accesses.
}
\label{table:cmp-mms}
\end{table*}

\end{landscape}

\twocolumn



\begin{table}[t]

\newcommand{\rotateAngle}{270}
\newcommand{\lastcol}{11}

\def\arraystretch{1}
\setlength\tabcolsep{2pt}

\begin{tabular}{|c|l|c|c|c|c|c|c|c|c|c|} 

  \hline

  \multirow{2}{*}{Class}          &
  \multirow{2}{*}{Model}          &
  \multicolumn{ 9}{c|}{Features}   
  \\ 

  \cline{3-\lastcol}

                                                            &
                                                            &
  \rotatebox[origin=c]{\rotateAngle}{NA}                    &
  \rotatebox[origin=c]{\rotateAngle}{RLX}                   &
  \rotatebox[origin=c]{\rotateAngle}{RA}                    &
  \rotatebox[origin=c]{\rotateAngle}{SC}                    &
  \rotatebox[origin=c]{\rotateAngle}{F-RA}                  &
  \rotatebox[origin=c]{\rotateAngle}{F-SC}                  &
  \rotatebox[origin=c]{\rotateAngle}{RMW}                   &
  \rotatebox[origin=c]{\rotateAngle}{LK}                    &
  \rotatebox[origin=c]{\rotateAngle}{MIX}      
  \\[9pt] 

  \Xhline{2\arrayrulewidth}

  \multirow{4}{*}{\clsSC}

  & \EtESC
     & 
     % NA, RLX, RA, SC 
     \badcell & \badcell & \badcell & \okcell & 
     % F-RA, F-SC
     \badcell & \badcell & 
     % RMW, LK
     \badcell & \badcell & 
     % MIX
     \badcell 
     \\ \cline{2-\lastcol}

  & \VbD
     & 
     % NA, RLX, RA, SC 
     \okcell & \badcell & \badcell & \okcell & 
     % F-RA, F-SC
     \badcell & \badcell & 
     % RMW, LK
     \okcell & \okcell & 
     % MIX
     \badcell 
     \\ \cline{2-\lastcol}

  & \SCHs 
     &
     % NA, RLX, RA, SC 
     \okcell & \badcell & \badcell & \okcell & 
     % F-RA, F-SC
     \badcell & \badcell & 
     % RMW, LK
     \okcell & \badcell &
     % MIX 
     \badcell 
     \\ \cline{2-\lastcol}

  & \DRFx
     &           
     % NA, RLX, RA, SC 
     \okcell & \badcell & \badcell & \okcell & 
     % F-RA, F-SC
     \badcell & \badcell & 
     % RMW, LK
     \badcell & \badcell & 
     % MIX
     \badcell 
     \\ \Xhline{2\arrayrulewidth}

  \multirow{2}{*}{\clsTSO}

  & \BMM
     &
     % NA, RLX, RA, SC 
     \okcell & \badcell & \badcell & \okcell & 
     % F-RA, F-SC
     \badcell & \badcell & 
     % RMW, Lock
     \badcell & \okcell & 
     % Mix.sz.
     \badcell 
     \\ \cline{2-\lastcol}

  & \RMMOA
     &
     % NA, RLX, RA, SC 
     \okcell & \badcell & \badcell & \badcell & 
     % F-RA, F-SC
     \badcell & \badcell & 
     % RMW, Lock
     \badcell & \okcell & 
     % Mix.sz.
     \badcell 
     \\ \Xhline{2\arrayrulewidth}

  \multirow{6}{*}{\clsPO}   

  & \RCMM
     &
     % NA, RLX, RA, SC 
     \okcell & \okcell & \okcell & \okcell & 
     % F-RA, F-SC
     \okcell & \okcell & 
     % RMW, Lock
     \okcell & \badcell & 
     % Mix.sz.
     \badcell 
     \\ \cline{2-\lastcol}

  & \ORCMM
     &
     % NA, RLX, RA, SC 
     \okcell & \okcell & \okcell & \badcell & 
     % F-RA, F-SC
     \okcell & \badcell & 
     % RMW, Lock
     \okcell & \badcell & 
     % Mix.sz.
     \badcell 
     \\ \cline{2-\lastcol}

  & \RAR
     & 
     % NA, RLX, RA, SC 
     \badcell & \okcell & \okcell & \badcell & 
     % F-RA, F-SC
     \badcell & \badcell & 
     % RMW, Lock
     \okcell & \badcell & 
     % Mix.sz.
     \badcell 
     \\ \cline{2-\lastcol}

  & \CRC
     &
     % NA, RLX, RA, SC 
     \okcell & \badcell & \okcell & \badcell & 
     % F-RA, F-SC
     \badcell & \okcell & 
     % RMW, Lock
     \okcell & \badcell & 
     % Mix.sz.
     \badcell 
     \\ \cline{2-\lastcol}

  & \OCMM
     & 
     % NA, RLX, RA, SC 
     \okcell & \badcell & \badcell & \okcell & 
     % F-RA, F-SC
     \badcell & \badcell & 
     % RMW, Lock
     \okcell & \badcell & 
     % Mix.sz.
     \badcell 
     \\ \cline{2-\lastcol}

  & \JAM
     &
     % NA, RLX, RA, SC 
     \okcell & \okcell & \okcell & \okcell & 
     % F-RA, F-SC
     \okcell & \okcell & 
     % RMW, Lock
     \okcell & \badcell & 
     % Mix.sz.
     \badcell 
     \\ \Xhline{2\arrayrulewidth}

  \multirow{2}{*}{\clsSyDEP}   

  & \LKMM
     &           
     % NA, RLX, RA, SC 
     \badcell & \okcell & \okcell & \badcell & 
     % F-RA, F-SC
     \okcell & \okcell & 
     % RMW, Lock
     \okcell & \badcell & 
     % Mix.sz.
     \badcell 
     \\ \cline{2-\lastcol}

  & \OHMM
     &
     % NA, RLX, RA, SC 
     \okcell & \badcell & \badcell & \okcell & 
     % F-RA, F-SC
     \badcell & \badcell & 
     % RMW, Lock
     \badcell & \okcell & 
     % Mix.sz.
     \badcell 
     \\ \Xhline{2\arrayrulewidth}

  \multirow{7}{*}{\clsSemDEP}   

  & \JMM
     &            
     % NA, RLX, RA, SC 
     \okcell & \badcell & \badcell & \okcell & 
     % F-RA, F-SC
     \badcell & \badcell & 
     % RMW, Lock
     \okcell & \okcell & 
     % Mix.sz.
     \badcell 
     \\ \cline{2-\lastcol}

  & \PRM
     &
     % NA, RLX, RA, SC 
     \okcell & \okcell & \okcell & \badcell & 
     % F-RA, F-SC
     \okcell & \okcell & 
     % RMW, Lock
     \okcell & \okcell & 
     % Mix.sz.
     \badcell 
     \\ \cline{2-\lastcol}

  & \WMO
     &
     % NA, RLX, RA, SC 
     \okcell & \okcell & \okcell & \okcell & 
     % F-RA, F-SC
     \okcell & \okcell & 
     % RMW, Lock
     \okcell & \badcell & 
     % Mix.sz.
     \badcell 
     \\ \cline{2-\lastcol}

  & \CSRA
     &
     % NA, RLX, RA, SC 
     \okcell & \okcell & \okcell & \badcell & 
     % F-RA, F-SC
     \badcell & \badcell & 
     % RMW, Lock
     \okcell & \okcell & 
     % Mix.sz.
     \badcell 
     \\ \cline{2-\lastcol}

  & \WJES
     &
     % NA, RLX, RA, SC 
     \badcell & \okcell & \badcell & \badcell & 
     % F-RA, F-SC
     \badcell & \badcell & 
     % RMW, Lock
     \okcell & \okcell & 
     % Mix.sz.
     \badcell 
     \\ \cline{2-\lastcol}

  & \MRD
     &
     % NA, RLX, RA, SC 
     \badcell & \okcell & \badcell & \badcell & 
     % F-RA, F-SC
     \badcell & \badcell & 
     % RMW, Lock
     \badcell & \okcell & 
     % Mix.sz.
     \badcell 
     \\ \cline{2-\lastcol}

  & \GOS
     &
     % NA, RLX, RA, SC 
     \okcell & \badcell & \badcell & \badcell & 
     % F-RA, F-SC
     \badcell & \badcell & 
     % RMW, Lock
     \badcell & \okcell & 
     % Mix.sz.
     \badcell 
     \\ \Xhline{2\arrayrulewidth}

  \multirow{5}{*}{\clsOOTA}   

  & \CMM
     &            
     % NA, RLX, RA, SC 
     \okcell & \okcell & \okcell & \okcell & 
     % F-RA, F-SC
     \okcell & \okcell & 
     % RMW, Lock
     \okcell & \okcell & 
     % Mix.sz.
     \okcell 
     \\ \cline{2-\lastcol}

  & \JSMM
     &
     % NA, RLX, RA, SC 
     \okcell & \badcell & \badcell & \okcell & 
     % F-RA, F-SC
     \badcell & \badcell & 
     % RMW, Lock
     \okcell & \okcell & 
     % Mix.sz.
     \okcell 
     \\ \cline{2-\lastcol}

  & \RMC
     &
     % NA, RLX, RA, SC 
     \badcell & \okcell & \okcell & \okcell & 
     % F-RA, F-SC
     \okcell & \okcell & 
     % RMW, Lock
     \okcell & \badcell & 
     % Mix.sz.
     \badcell 
     \\ \cline{2-\lastcol}


  & \RAO
     &
     % NA, RLX, RA, SC 
     \okcell & \badcell & \badcell & \okcell & 
     % F-RA, F-SC
     \badcell & \badcell & 
     % RMW, Lock
     \badcell & \badcell & 
     % Mix.sz.
     \badcell 
     \\ \cline{2-\lastcol}

  & \TSC
     &
     % NA, RLX, RA, SC 
     \okcell & \badcell & \badcell & \okcell & 
     % F-RA, F-SC
     \badcell & \badcell & 
     % RMW, Lock
     \badcell & \okcell & 
     % Mix.sz.
     \badcell 
     \\ \Xhline{2\arrayrulewidth}

\end{tabular}

\iftoggle{langeng}{%
  \caption{Features supported by memory models}
}{%
  \caption{Поддержка примитивов моделями памяти}
}


\label{table:features}

\end{table}

\subsection{Sequential Consistency}
\label{sec:catalog:sc}

We start with a description of several attempts 
to adopt sequential consistency as memory model for 
existing languages and runtimes. 
Most of the proposed solutions share similar properties, 
and thus in \cref{table:cmp-mms} we unite them 
in a single row under the name \SC. 
The only exception is \DRFx model which implements
catch-fire semantics for racy programs 
and thus has a slightly different properties. 

\paragraph{End-to-end Sequential Consistency}

Marino et al~\cite{Marino-al:PLDI11, Singh-al:ISCA12} 
examined the performance penalties to ensure end-to-end SC
enforced by (1) modified SC-preserving version 
of LLVM compiler infrastructure and 
(2) a modified version of x86-TSO hardware. 
To mitigate the induced overhead the authors 
utilized the observation that hardware need to 
enforce SC only for memory accesses to shared mutable variables. 
To classify the memory regions as either thread-local,
shared immutable, or shared mutable they have used 
a combination of static compiler analysis and 
dynamic analysis powered by modified hardware. 
They evaluated their approach on a number of benchmarks
and reported performance overhead of 6.2\% on average 
and ~17\% in maximum, compared to stock LLVM compiler 
and regular x86 hardware. 

\paragraph{Volatile-by-default}

Liu~\etal~\cite{Liu-al:OOPSLA17, Liu-al:PLDI19} studied 
sequential consistency in the context of \Java.  
They proposed a \emph{volatile-by-default} semantics,
where each memory access is treated as volatile 
(\ie sequentailly consistent) by default. 
The experiments showed a considerable performance penalty:
28\% slowdown on average with 81\% in maximum on x86 hardware,
and 57\% slowdown on average with 157\% in maximum on ARMv8 hardware.
The authors tried to mitigate performance overhead and presented  
a novel optimization techinique for language-level SC
compatible with \emph{just-in-time} compilation. 
They propose to treat each object as thread-local speculatively 
and compile memory accesses without fences. 
If later during the execution concurrent accesses to the object  
are detected, the code is recompiled to place necessary fences.
A modified version of \JVM which implements the technique
described above was reported to incur 37\% slowdon on average 
with 73\% in maximum on ARMv8 hardware.

\paragraph{SC-Haskell}

SC-Haskell memory model~\cite{Vollmer-al:PPoPP17}
were inspired by the similar idea of separation
between the thread-local and shared mutable memory. 
To safely distinguish between the two 
the authors utilized the powerful strong type system of Haskell. 
The consequence of this approach is that the 
programmers need to follow a stricter discipline 
in order to please the type checker. 
The authors modified the GHC to preserve SC 
and then run 1,279 benchmarks on x86-64 hardware
to measure the performance penalties.
They reported 0.4\% geometric mean slowdown,
and noticed that only 12 benchmarks experienced 
slowdown greater than 10\%.

\paragraph{\DRFx}

The \DRFx~\cite{Marino-al:PLDI10} is another 
SC preserving memory model. In this model
the runtime system is guaranteed to raise 
an exception if the program has data-races, 
and yeild only sequantially consistent outcomes otherwise.
In order to make the runtime data-race detection feasible 
in practice, the authors propose several modifications 
to existing hardware.

The authors claim that any sequentially valid optimization 
(\eg instruction reorderings or common subexpression elimination),
is sound in DRFx model, the only limitation is that
these transformations can only be performed
withing the bounds of compiler-designated program regions.
Besides that any transformation that introduces 
speculative reads or writes is unsound,
since speculative optimizations can bring
data-races into otherwise race-free programs.

Note that in \cref{table:cmp-mms} we still do not consider 
many of the transformations that claimed to be sound by the authors
as actually being sound because of our convention described in \cref{sec:comparison}.
We consider transformations sound only if they are 
sound in a fully-defined semantics. 
The \DRFx model does not meet this criterion as 
it provides catch-fire semantics.

The expected performance overhead of the model 
is reported to be 3.25\% on average
assuming the efficient implementation 
of data-race detection in hardware. 
(compared to stock compiler and x86 hardware). 

\subsection{Total and Partial Store Order}

In this section we consider PL memory models 
inspired by \TSO and \PSO.  

\paragraph{Buffered Memory Model}

Demange et al.~\cite{Demange-al:POPL13} presented 
the \emph{Buffered Memory Model} (or BMM in short)
as a candidate model for Java language.
Their motivation, however, stemmed not from the desire 
to fully replace the Java Memory Model, but rather 
from the goal to build a verified version of 
Java Virtual Machine (akin to CompCertTSO project~\cite{Sevcik-al:JACM13}).
A more simple yet pragmatic memory \TSO model 
was considered as a first step to achive this goal. 

The authors proved soundness of several program transformations
(including store/load reordering, speculative load introduction,
and several others (see \cref{table:cmp-mms})
and the external DRF-SC theorem. 
They also modified existing open-source implementation of 
JVM~\cite{Pizlo-al:ECCS10} to preserve BMM and 
reported only~1\% average overhead 
compared to original version of the virtual machine. 
Again, the authors used only x86 hardware in their 
experiments, and the performance penalties 
are expected to be more significant on weaker hardware.   

\paragraph{Relaxed Memory Models: an Operational Approach}

Boudol and Petri~\cite{Boudol-Petri:POPL09} proposed 
an approach to formal semantics of relaxed memory models 
based on the abstract machine with the main memory 
and the hierarchial structure of store buffers 
with stores to different locations possibly 
propagating to the main memory out-of-order
(similarly to \PSO model).
The authors present a proof of DRF-SC theorem,
but do not provide an extensive study 
of soundness os program transformations.

\subsection{Out of Thin-Air Values}

Next we discuss memory models admitting thin-air values. 

\paragraph{C11}

The most notable member of the OOTA class is the \CMM model~\cite{Batty-al:POPL11}.
The main purpose of the \CMM model was to adhere to the fundamental principle of \CPP, 
\ie to provide so-called zero-cost abstraction. 
In other words, the memory model was meant to provide 
efficient compilation mappings and support as many transformation as possible.
It was later shown that the formal model partially fails in achieving these goals.

Vafeiadis~\etal~\cite{Vafeiadis-al:POPL15} have shown that several program transformation 
(load/store elimination, strengthening, roach motel reorderings, sequentialisation) 
that deemed to be correct are actually unsound according to the formal model.
They proposed several local fixes to the model which 
partly repair soundness of transformations and improve 
its meta-theoretical properties. 

Batty~\etal~\cite{Batty-al:ESOP15} have shown that 
the model also fails to provide external DRF guarantee, 
and that it is ultimately not possible to provide this guarantee
at all within the style of the \CMM formal semantics,
only the internal DRF can be achieved. 

A lot of work~\cite{Batty-al:POPL11, Sarkar-al:PLDI12, Batty-al:POPL12, Batty-al:POPL16} 
was dedicated to prove soundness of optimal compilation mappings 
with respect to formal models of hardware, 
and there the results were mostly positive.
Besides that, Flur~\etal~\cite{Flur-al:POPL17} have extended the model to support mixed-size accesses.
Finally, Nienhuis~\etal~\cite{Nienhuis-al:OOPSLA16} presented 
a formal executable semantics in terms of an abstract machine, 
equivalent to the original \CMM model. 

\paragraph{\JS Memory Model}

\JSMM is based on \CMM model. 
Like the latter, it also has the problem of thin-air values
and thus can only provide internal DRF-SC gurantee. 
Contrary to the \CMM, \JS model does not treat 
racy non-atomic accesses as undefined behavior. 

The main language primitive provided by the \JSMM
is \texttt{SharedArrayBuffer}, that is a linear mutable byte buffer.
Thus the model naturally supports mixed-size accesses.

\paragraph{A calculus for relaxed memory}

Crary and Sullivan~\cite{Crary-Sullivan:POPL15} proposed 
an alternative approach to relaxed shared memory concurrency.
Instead of deriving the ordering constraints from the annotations 
on memory accesses, they propose to directly specify 
the ordering between memory access in the source code. 
Their approach is highly generic and subsumes 
the traditional memory order annotation in the style of \CMM.
Their model is very weak and permits thin-air values. 
Yet the authors proved the internal DRF theorem.

\paragraph{Relaxed Atomic + Ordering}

Saraswat~\etal~\cite{Saraswat-al:PPoPP07} presented the \RAO memory model
where relaxed behaviors are explained through the transformations 
over sequentially consistent execution.
Although the authors claim their model provides external DRF,
it also permits thin-air values. 
These two properties known to be incompatible~\cite{Batty-al:ESOP15}.
We suppose that the external DRF can be achieved in their model 
only because of the fundamental restrictions on the input programming language 
(\eg the general conditional statemets are not supported~\cite{PichonPharabod-Sewell:POPL16}). 

\paragraph{A theory of speculative computation}

Boudol and Petri~\cite{Boudol-Petri:ESOP10} proposed a general 
framework to study the effects of speculative execution in
shared memory setting. 
They have also noticed that the external DRF doest not 
hold in the presence of unrestricted speculations, 
yet the internal DRF theorem still can be proven. 

\subsection{Program Order Preserving}
\label{sec:catalog:porf}

In this section we describe the memory models 
that preserve the program order and forbid any 
kind of speculative executions to tackle problem 
of thin-air values. 
In particular, we consider \RCMM and 
several derivatives of this model, 
as well as memory model of \OCaml, 
and a proposed revised model of \Java.  

\paragraph{RC11}

Lahav~\etal~\cite{Lahav-al:PLDI17} formalized 
a version of \CMM that preserves the order between load/store pairs, 
and also repairs the semantics of sequentially-consistent accesses.

The authors proved soundness of several program transformation 
(see \cref{table:cmp-mms} for details). 
Among the unsound transformation, 
the load/store reordering is forbedden for an obvious reasons, 
speculative load instroduction is not supported 
because of catch-fire semantics for racy programs, 
\CSE is not supported because relaxed accesses 
enforce coherence (while non-atomic accesses 
support this transformation, they entail 
undefined behavior in the presence of races).

The compilation mappings to x86 are not affected and remain optimal.
One of the possible compilation mappings 
to architectures like \ARM and \POWER 
requires to compile relaxed load as  
plain load followed by a spurious conditional branch.
Ou and Demsky~\cite{Ou-Demsky:OOPSLA18} have studied 
the performance penalty of this mapping on ARMv8 hardware.
They modified the LLVM compiler framework 
to enforce \RCMM memory model
by (1) adjusting the compiler optimization passes and 
(2) changing the compilation mappings.
Several compilation schemes were considered,
among them the one that uses spurious conditional branch
as descibed above has demonstrated the most promising results.  
The authors measured the running time on a set of benchmarks 
implementing various concurrent data-structures
(\eg locks, stacks, queues, deques, maps
from various open source libraries~\cite{CDSLib, FollyLib, JunctionLib})
and reported an overhead of 0\% on average and 6.3\% in maximum,
compared to the unmodified version of the compiler. 

\paragraph{RAR}

Doherty~\etal~\cite{Doherty-al:PPoPP19} developed an 
operational version of \RCMM supporting 
release-acquire and relaxed accesses. 
On top of it they built proof calculus for 
invariant-based reasoning and verified 
correctness of mutal exclusion algorithms. 

\paragraph{ORC11}

Dang~\etal~\cite{Dang-al:POPL19} developed yet another 
operational version of \RCMM which they called \ORCMM. 
Their motivation was to then develop a 
new program logic and show it's soundness
with respect to \ORCMM memory model. 
The program logic itself was then utilized to 
prove correctness of synchronization primitives 
from the standard library of \Rust~\cite{RustBook:19}.

\paragraph{Compositional Relaxed Concurrency}

Dodds~\etal~\cite{Dodds-al:ESOP18} proposed a denotational 
compositional semantics for the fragment of \CMM memory model, 
including non-atomic accesses with cath-fire semantics, 
release-acquire accesses, and sequantially-consistent fences. 
Based on this semantics the authors developed 
a tool for automatic verification of program transformations
in the considered fragment of the \CMM model. 
Since the relaxed fragment was not included, 
the authors avoided problems with thin-air values. 

\paragraph{OCaml Memory Model}

Dolan~\etal~\cite{Dolan-al:PLDI18} developed a new 
memory model for the \MOCaml project. 
They were the first to propose the local DRF property. 
The authors also hinted that the local DRF property 
is not compatible with load/store reordering.
This fact forced them to forbid this transformation
and adapt similar compilation scheme as for \RCMM. 
An important divergence of \OCaml memory model 
from \CMM-like models is that the former 
has a weaker notion of coherence.
The choice of the weaker coherence was deliberate 
with the purpose to enable common subexpression elimination
(see \cref{sec:analysis:coh} for details).

\paragraph{Java Access Modes}

Bender and Palsberg~\cite{Bender-Palsberg:OOPSLA19} formalized a new revision 
of the Java Memory Model~\cite{JDK9-VarHandle, JEP:193, JDK9-Modes}, 
which was developed to overcome 
the difficulties of the previous one~\cite{Manson-al:POPL05}
(see \ref{sec:catalog:jmm} for details).
The new version of the model was inspired by \RCMM. 
It introduced a system of annotations on memory accesses, 
called ``Java Access Modes'' (hence the name of the model --- \JAM),
similar to those present in \CMM like models.
The new model adopted the \RCMM solution to OOTA problem. 
It forbids load/store reorderings on the level of 
opaque (an analog of \CPP relaxed) or stronger accesses.
The model does not tackle the problem of 
thin-air values on the level of plain (\ie non-atomic) accesses.

\subsection{Syntactic Dependencies Preserving}
\label{sec:catalog:deprf}

Next we discuss the programming languge memory models 
that track syntactic dependencies.

\paragraph{Linux Kernel Memory Model}

\LKMM~\cite{Alglave-al:ASPLOS18} has adopted 
the idea to track syntactic dependencies in order to 
forbid thin-air values. Despite this choice 
ultimately limits the list of supported 
trace preserving transformations,
in the context of OS kernel development 
it can be justified by the following arguments. 
First, the Linux kernel targets 
a wide range of hardware architectures with a diverse
set of memory models. To simplify the reasoning about the code, 
it is reasonable to pick a model which is conceptually close
to those of hardware. 
Second, the kernel developers already utilze 
various techiniques to prevent certain compiler optimizations%
\cite{Alglave-al:ASPLOS18, LK-MemBarriers, LK-RCU-Deref}.

The authors of the models have empirically tested 
soundness of compilation mappings to 
\xTSO, \ARMv{7}, \ARMv{8}, and \POWER hardware. 
They also formalized the read-copy-update 
synchronization mechanism (RCU)~\cite{McKenney-RCU2007}, 
extensively used in Linux kernel development, 
and proved soundness of its implementation with respect to their model.

\paragraph{Operational Happens-Before Model}

In attempt to repair Java Memory Model (see \cref{sec:catalog:jmm})
Zhang and Feng have proposed the 
operational happens-before model \OHMM~\cite{Zhang-Feng:FCS16}.
Their abstract machine consists of global event buffer,
where the events might be reordered before they propogate into  
a global history based memory, and a replay mechanism 
used to simulate speculative executions. 
To avoid thin-air outcomes the model tracks syntactic dependencies 
between the events and forbids the reordering of dependent events. 
The authors proved external DRF and soundness of several 
program trasformations. 

\paragraph{Dependency Preserving Compiler}

Ou and Demsky~\cite{Ou-Demsky:OOPSLA18} studied 
the performance penalty induced by dependency preserving compiler. 
Again, they modified the LLVM compiler infrastructure 
and run benchmarks from \SPECCPU suite on ARMv8 hardware. 
They have observed 3.1\% overhead on average and 17.6\% in maximum. 

\subsection{Semantic Dependencies Preserving}
\label{sec:catalog:sdeprf}

Finally we discuss memory models 
which try to tackle thin-air values problem 
by development of the notion of semantic dependencies. 
In particular, this class includes the original Java Memory Model, 
Promising semantics, and several models based 
on \emph{event structures}~\cite{Winskel:86}.

\paragraph{Java Memory Model}
\label{sec:catalog:jmm}

The original formalized version of Java memory model \JMM~\cite{Manson-al:POPL05}
was a pioneering work in the area of programming language memory models 
While most of the memory model was formalized in axiomatic style, 
it also used an operational notion of \emph{commit sequence}, 
\ie a sequence of partial execution graphs, to forbid thin-air outcomes. 
The model was shown to adhere external DRF~\cite{Huisman-Petri:CONCUR07}.
However, the model failed to justify some program transformations 
which were expected to be sound~\cite{Sevcik-Aspinall:ECOOP08} 
(\eg redundunt load after load elimination, roach motel reodering, and others,
see \cref{table:cmp-mms} for details). 

\paragraph{Generative operational semantics}

Jagadeesan~\etal~\cite{Jagadeesan-al:ESOP10} attempted to fix \JMM 
and proposed an operational semantics with speculative execution.
To avoid thin-air values they have put stratification constraints 
on speculations. The authors prove the external DRF theorem. 
Also they verified a few program transformations 
(store/store reordering, load/load elimination, and roach motel reordering), 
but overall their study of transformations was not systematic.  

\paragraph{Promising Semantics}

Promising semantics~\cite{Kang-al:POPL17, Lee-al:PLDI20} 
presents the most complete to this day model of this class. 
Its key ingredient is the promising and certification machinery.
During the execution, the abstract machine can 
non-deterministically \emph{promise} to perform some store,
it then has to \emph{certify} the promise is feasible. 
The certification mechanism is defined in the way to forbid thin-air values to appear.
The authors of the model have proven formally 
that \Promising semantics admits many local and global program transformations,
with a notable exception of the thread inlining
(see \cref{table:cmp-mms} for details).

The model has a fully defined semantics even for plain accesses.  
Plain and relaxed accesses, however, have different semantics.
In particular, coherence is enforced for relaxed accesses 
but not for the plain ones.  
This design choice, in particular, allows to support 
\CSE on the level of plain accesses. 

Podkopaev~\etal~\cite{Podkopaev-al:ECOOP17, Podkopaev-al:POPL19} have proven formally
soundness of standard optimal compilation mappings to \xTSO, \ARMv{7}, \ARMv{8}, and \POWER.

One of a few limitations of the \Promising semantics is that 
it does not support sequentially consistent accesses. 

\paragraph{Weakestmo}

Chakraborty and Vafeiadis~\cite{Chakraborty-Vafeiadis:CGO17, Chakraborty-Vafeiadis:POPL19}
developed a memory model based on event structures. 
They utilize the event structures' capability to simultaneously encode 
multiple conflicting executions in order to model speculative executions.
Their model is close in spirit to conventional axiomatic models, 
however, instead of idividual execution graphs, they use 
the event structures upon which they put additional constraints. 
The model was shown to admit optimal compilation mappings~\cite{Moiseenko-al:ECOOP20},
several program transformation, and external DRF.
Unlike \Promising semantics, it also supports 
sequantially consistent accesses.

\paragraph{A Concurrency Semantics for Relaxed Atomics}

Pichon-Pharabod and Sewell~\cite{PichonPharabod-Sewell:POPL16} 
presented an operational memory model consisting of 
memory subsystem inspired by POWER and a thread subsystem, 
where each thread is represented as an event structure. 
At each step the abstract machine is allowed to either 
commit an event to the storage, or perform a transformation 
on one of the event structures. 
The authors have shown soundness of 
optimal compilation mappings to \xTSO and \POWER, 
as well as soundness of several program transformations.
It was later revealed though that the compilation scheme
to ARMv7 and ARMv8 is not optimal~\cite{PichonPharabod:PhD18}.

\paragraph{Well-Justified Event Structures}

Jeffrey and Riely~\cite{Jeffrey-Riely:LICS16} proposed 
a memory model based on event structures and a notion of 
\emph{well-justification} of events inspired by the game semantics. 
Well-justification is used to prevent thin-air values 
and prove external DRF. The authors do not study 
soundness of program transformations in their model. 
They show, however, a counterexample demonstrating that 
load/load reordering is unsound. It implies that 
the compilation mappings to \ARMv{7}, \ARMv{8}, and \POWER are not optimal.   

\paragraph{Modular Relaxed Dependencies}

Paviotti~\etal~\cite{Paviotti-al:ESOP20} constructed a 
denotational semantics based on event structures. 
They employ the event structures to capture 
semantic dependencies between the memory access events, 
which are in ture used to rule out thin-air outcomes.
The authors prove the external DRF-SC and 
soundness of optimal compilation mappings,
also they present a refinement relation which 
can be used to reason about validity of program transformations. 
However, they have not studied soundness of particular transformations. 
