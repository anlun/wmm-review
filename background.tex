\section{Criteria for Memory Models}
\label{sec:background}

In this section we will have a closer look on criteria for 
programming language memory models, 
namely an optimality of the compilation scheme, 
soundness of common program transformations, 
and provided reasoning guarantees.  
% But first we need to specify 

\subsection{Abstractions and Interfaces}
\label{sec:background-primitives}


The memory model defines the semantics of the shared memory 
in the presence of concurrently executing threads. 
More concretly, the memory model 
defines which values reads of shared variables 
can observe at each point of execution. 
Therefore the main abstraction of the memory model 
is the shared memory itself. 
The shared memory consists of a individual shared variables, 
each having unique address~\footnote{
Throughout the rest of paper, we use terms 
memory address and memory location interchangeably}.
Threads can access these variables by performing 
loads or stores~\footnote{We also use the terms 
load/stores and reads/writes interchangeably}.

\begin{figure}[t]
\[\def\arraystretch{1.2}
  \begin{array}{ll} 
    \readInst{o}{r}{x}                  & \text{Load}                   \\ 
    \writeInst{o}{x}{v}                 & \text{Store}                  \\ 
    \readArrayInst{o}{r}{x}{i}{j}       & \text{Mixed Size Load}        \\ 
    \writeArrayInst{o}{x}{v}{i}{j}      & \text{Mixed Size Store}       \\ 
    \casInst{o}{r}{x}{v_e}{v_d}         & \text{Compare-and-Swap}        \\ 
    \faddInst{o}{r}{x}{v}               & \text{Fetch-and-Add}          \\ 
    \lockInst{l}                        & \text{Lock}                   \\ 
    \unlockInst{l}                      & \text{Unlock}                 \\ 
    \fenceInst{o}                       & \text{Fence}                  \\ 
    x \in \mathsf{Var}                  & \text{Shared Variables}       \\ 
    r \in \mathsf{Reg}                  & \text{Thread-Local Registers} \\ 
    v \in \mathsf{Val}                  & \text{Values}                 \\ 
    l \in \mathsf{Lock}                 & \text{Locks}                  \\ 
    i,j \in \mathsf{Index}              & \text{Array Indices}          \\ 
    o \in \set{\na,\rlx,\acqrel,\sco}   & \text{Access Modes}           \\ 
  \end{array}
\]
\caption{Primitives of relaxed memory models}
\label{fig:wmm-abs}
\end{figure}

Most programming language memory models distinguish 
\emph{non-atomic} (sometimes also called \emph{plain})
and \emph{atomic} variables. 
The former generally should not be accessed 
concurrently from parallel threads. 
Depending on the particular programming language, 
concurrent accesses to non-atomic variables 
can be either prevented by the type-system 
(\eg \Haskell~\cite{Marlow-al:Haskell10, Vollmer-al:PPoPP17}, \Rust~\cite{RustBook:19}), 
have undefined behavior (\eg \CPP~\cite{Boehm-Adve:PLDI08, Batty-al:POPL11}), 
or have defined but very weak semantics with almost 
no guarantees on the order of accesses to non-atomics
that concurrent threads can observe (\eg \Java~\cite{Manson-al:POPL05}).

In addition to this some memory models distinguish 
several kinds of accesses to atomic variables.
In these models the accesses to shared memory are annotated by the 
so-called \emph{access modes}.
For example, the \CPP model (and a later revision of 
\Java~\cite{Bender-Palsberg:OOPSLA19}), distinguish 
three modes: \emph{relaxed} (\emph{opaque} in \Java terminology), 
\emph{acquire/release}, and \emph{sequentially consistent}.
They are denoted as $\rlx$, $\opq$, $\acqrel$, and $\sco$ correspondingly.
Non-atomic accesses are often considered to be the fourth access mode $\na$, 
but note that mixing non-atomic and atomic accesses to the same variable 
entails undefined behavior in \CPP.

The access modes are ordered by the guarantees they provide
in exchange of optimization opportunities, as the following 
diagram shows.

$$ \na \sqsubseteq \rlx \sqsubseteq \acqrel \sqsubseteq \sco $$

On the one end of the spectrum are sequentially consistent accesses. 
They guarantee to restore the \SC semantics, if used properly 
(see \cref{sec:background-drf} for details).
Non-atomic accesses, as we have already discussed, give 
from little to no guarantee. 
Relaxed accesses also have very weak guarantees, 
usually they provide the so-called \emph{coherence} property
(see \cref{sec:background-coh} for details).
Finally, in the middle there are the acquire/release accesses. 
They are designed to support the message passing idiom~\cite{Lahav-al:POPL16}.
The thread sends the message by performing a release write, 
the thread expecting a message can perform an acquire read. 
If the acquire read observes the release write, the two 
threads ``synchronize''. 

The memory model can also provide atomic \emph{read-modify-write} operations.
These include \emph{compare-and-swap}, and variations of atomic increment,
\eg \emph{fetch-and-add}, \emph{fetch-and-sub}, \etc 
Compare-and-swap (\CAS) operation takes the shared variable, the expected 
and the desired values. It reads the content of the variable
and compares it with the expected value. If the two are equal,
it substitutes the content of the variable to desired value. 
In either case, the value read from the variable is returned as a result. 
Note that the above operations is guaranteed to be performed atomically, 
no other write to the shared variable may happen ``in-between'' 
read and write parts of \CAS.
Fetch-and-add and similar primitives perfrom 
the operation (addition, substraction,~\emph{etc.}~)
atomically and unconditionally, returning 
the content of the shared variable prior to modification.  

Locks sometimes considered to be a part 
of the memory model~\cite{Manson-al:POPL05}, 
as well as memory fence operations~\cite{Batty-al:POPL11},
corresponding to hardware fence instructions
(see \cref{sec:background-compile} for details). 

Finally, the memory model can treat the shared memory 
not as set of disjoint typed variables, but rather as 
a raw byte sequence, and permit so-called \emph{mixed-size} 
concurrent accesses~\cite{Flur-al:POPL17}, as in the example below 
(assume $\declareArray{x}{0}{7}$ is byte array of size~\texttt{8}). 

\begin{align*}
\inarrIII{
  \writeArrayInst{}{x}{1}{0}{3}
}{
  \writeArrayInst{}{x}{1}{4}{7}
}{
  \readArrayInst{}{r}{x}{0}{7}
}
\tag{Mixed-Size}\label{ex:mixsz}
\end{align*}

\subsection{Optimal Compilation Scheme}
\label{sec:background-compile}

We next consider the first criterion for 
programming language memory models --- 
an optimality of compilation scheme. 
A compilation scheme is a mapping of 
programming language primitives into 
instructions of particular hardware architecture. 
In our setting we consider the primitives 
mentioned in \cref{sec:background-primitives}.
The hardware architectures provide a similar set 
of instructions which usually contain 
plain load and stores\footnote{Some architectures 
also provide various types of load and stores
matching the access modes annotations, 
\eg \texttt{lda} --- load acquire, 
and \texttt{stl} --- store release on \ARMv{8}}, 
read-modify-write operations, 
and also various memory fences.    

The compilation mapping should be \emph{sound}.
In the context of this paper it means that 
the set of outcomes permited by the 
memory model of the hardware should be 
a subset of outcomes permited by the 
programming language model. 

Let us consider an example. 
The program \ref{ex:sb} below is a 
fragment of the \ref{ex:Dekker} program 
from the \cref{sec:intro}.
Assume the memory model of the programming language
is sequential consistency, and we want 
to compile for \xTSO hardware. 
Then if one compile all loads and stores 
to plain load and store instructions of \xTSO,
the outcome $[r_1=0, r_2=0]$ would be 
allowed for the compiled program 
(and it can actually be observed in practice), 
since the memory model of \xTSO permits this outcome. 
It can be obtained as a result of \emph{store buffering}
optimization (hence the name of the program \ref{ex:sb}). 
The store $\writeInst{}{x}{1}$ can be buffered and 
executed after all other instructions of the program.  
Yet clearly the outcome $[r_1=0, r_2=0]$ is not sequentially consisestent. 
Therefore the proposed compilation scheme, 
which maps all load and stores to the plain one is unsound. 
As we have demonstrated in \cref{sec:intro} 
the unsoundness of compilation scheme has 
dramatic consequences, as it may break 
the correctness of the program. 

\begin{minipage}{0.45\linewidth}
\begin{equation*}
\inarrII{
   \writeInst{}{x}{1}   \\
   \emptyInst           \\
   \readInst{}{r_1}{y}  \\
}{
  \writeInst{}{y}{1}   \\
  \emptyInst           \\
  \readInst{}{r_2}{x}  \\
}
\tag{SB}\label{ex:sb}
\end{equation*}
\end{minipage}\hfill%
\begin{minipage}{0.45\linewidth}
\begin{equation*}
\inarrII{
   \writeInst{}{x}{1}   \\
   \mfenceInst          \\
   \readInst{}{r_1}{y}  \\
}{
  \writeInst{}{y}{1}   \\
  \mfenceInst          \\
  \readInst{}{r_2}{x}  \\
}
\tag{SB+MFENCE}\label{ex:sb-mfence}
\end{equation*}
\end{minipage}

A sound compilation scheme for sequential consistency 
can compile the stores as plain stores followed 
by the \texttt{mfence} instruction~\cite{Sewell-al:CACM10, Batty-al:POPL11}, 
as demonstrated in \ref{ex:sb-mfence}. 
The \texttt{mfence} is a special memory fence instruction
of \xTSO architecture which flushes the store buffer of the thread. 
For the program \ref{ex:sb-mfence} the outcome $[r_1=0, r_2=0]$
is forbidden by the \xTSO memory model. 

Although the modified compilation scheme is sound for \SC, 
it is not \emph{optimal}, in a sense that 
it requires to use memory fence instructions, 
which usually induces a performance penalty~\cite{Marino-al:PLDI11, Liu-al:OOPSLA17}.
Unfortunately, it is not possible to have compilation mapping 
for \SC model which is both \emph{sound and optimal}, 
assuming the modern hardware architectures.     
This fact makes \SC memory model unsuitable  
for the high-performance programming languages
and serves as one of the stimuls for the weakening 
of the memory model. 
 
In this paper, when speaking about the compilation schemes, 
we will consider the following hardware memory models:
\xTSO, \POWER, \ARMv{7}, and \ARMv{8}. 
This is for two main reasons. 
First, these hardware architectures are the 
most widespread today~\cite{}. 
Second, they have received a lot of attention 
from the research community recently. 
As the result of this effort, 
there were developed a rigorous formal 
specifications of these models~%
\cite{Sewell-al:CACM10, Sarkar-al:PLDI11, 
Flur-al:POPL16, Pulte-al:POPL18}. 

\subsection{Soundness of Program Transformations}
\label{sec:bgrnd-opt-sound}

\subsubsection{Local Transformations}

\paragraph{Trace preserving transformations (TR)}

\paragraph{Reordering of independent instructions (RI)}

\paragraph{Redundunt Load/Store Elimination (RE)}

\paragraph{Irrelevant Load Elimination (ILE)}

\paragraph{Speculative Load Introduction (SLI)}

\paragraph{Strengthening (S)}

\paragraph{Roach Motel Reordering (RM)}

\subsubsection{Global Transformations}

\paragraph{Register Promotion (RP)}

\paragraph{Value Range based Transformations (VR)}

\paragraph{Thread Inlining (TI)}

\subsection{Reasoning Guarantees}

\subsubsection{DRF Theorems}
\label{sec:background-drf}

\begin{itemize}
  \item Motivation: no weak behaviors for well-sync programs.
  \item Example
  \begin{itemize}
    \item Lock + increment?
    \item MP with atomic flag?
  \end{itemize}
  \item Definition of the race
  \item DRF theorem
  \begin{itemize}
    \item explain difference between internal/external DRF
  \end{itemize}
  \item Local DRF (?)
  \begin{itemize}
    \item Example
  \end{itemize}
\end{itemize}

\subsubsection{Out-of-Thin-Air Values}
\label{sec:bgrnd-oota}

\begin{itemize}
  \item Introduce three examples: LB, LB+dep, LB+fakedep.
  \item Explain the problem: some memory models cannot 
    correctly distinguish theese three programs.
  \item It leads to out-of-thin-air values.
\end{itemize}

% \begin{equation*}
% \inarrII{
%   \readInst{}{r_1}{x}      \\
%   \writeInst{}{y}{r_1}     \\
% }{
%   \readInst{}{r_2}{x}      \\
%   \writeInst{}{x}{r_2}     \\
% }
% \tag{LB+data}\label{ex:lb+data}
% \end{equation*}


\subsubsection{$\lPO\cup\lRF$ Acyclicity}

\begin{itemize}
  \item Motivation: forbid OOTA.
  \item Proposed solution: forbid causality cycles. 
  \item Explain execution graphs and $\lPO\cup\lRF$ cycles.
  \item LB examples againg, with their execution graphs.
  \item Drawback of this approach: some optimizations are unsound 
    (refer to analysis section).
\end{itemize}

\subsubsection{Syntactic Dependency Tracking}

\begin{itemize}
  \item Motivation: forbid OOTA, but enable more opts.
  \item Proposed solution: forbid causality cycles with syntactic dependencies. 
  \item Explain $\lPPO\cup\lRF$ cycles.
  \item LB examples againg, with their execution graphs and syntax.deps.
  \item Drawback of this approach: some optimizations are still unsound 
    (refer to analysis section).
\end{itemize}

\subsubsection{Semantic Dependency Tracking}

\begin{itemize}
  \item Motivation: forbid OOTA (precisely).
  \item Proposed solution: forbid causality cycles with semantic dependencies. 
  \item The exact notion of semantic dependency is model-specific.
        There is no conventional common definition.
  \item Drawback of this approach: complexity, no common ground.
\end{itemize}

\subsubsection{Undefined Behavior}
\label{sec:bgrnd-ub}

\begin{itemize}
  \item Motivation: enable more opts for non-atomics.
        (At the cost of model's predictability and programmability).
  \item Catch-fire semantics (race on non-atomics implies UB).
  \item Explain how it affects opts.
\end{itemize}

\subsubsection{Coherence (?)}
\label{sec:background-coh}

\subsection{Supported Features and Interfaces}

