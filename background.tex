\section{Criteria for Memory Models}
\label{sec:background}

In this section we will have a closer look on criteria for 
programming language memory models, 
namely on optimality of the compilation scheme, 
soundness of common program transformations, 
and provided reasoning guarantees.  
But first we need to introduce  
programming primitives provided 
by the shared memory abstraction. 

\subsection{Programming Primitives}
\label{sec:background:primitives}


A memory model defines the semantics of the shared memory 
in the presence of concurrently executing threads. 
% More concretly, the memory model 
% defines which values reads of shared variables 
% can observe at each point of execution. 
Therefore the main abstraction of the memory model 
is the shared memory itself. 
The shared memory consists of individual shared variables, 
each having a unique address.\footnote{
Throughout the rest of paper, we use terms 
memory address and memory location interchangeably}
Threads can access these variables by performing 
loads or stores.\footnote{We also use the terms 
load/stores and reads/writes interchangeably}

% \begin{figure}[t]
% \[\def\arraystretch{1.2}
%   \begin{array}{ll} 
%     \readInst{o}{r}{x}                  & \text{Load}                   \\ 
%     \writeInst{o}{x}{v}                 & \text{Store}                  \\ 
%     \readArrayInst{o}{r}{x}{i}{j}       & \text{Mixed Size Load}        \\ 
%     \writeArrayInst{o}{x}{v}{i}{j}      & \text{Mixed Size Store}       \\ 
%     \casInst{o}{r}{x}{v_e}{v_d}         & \text{Compare-and-Swap}        \\ 
%     \faddInst{o}{r}{x}{v}               & \text{Fetch-and-Add}          \\ 
%     \lockInst{l}                        & \text{Lock}                   \\ 
%     \unlockInst{l}                      & \text{Unlock}                 \\ 
%     \fenceInst{o}                       & \text{Fence}                  \\ 
%     x \in \mathsf{Var}                  & \text{Shared Variables}       \\ 
%     r \in \mathsf{Reg}                  & \text{Thread-Local Registers} \\ 
%     v \in \mathsf{Val}                  & \text{Values}                 \\ 
%     l \in \mathsf{Lock}                 & \text{Locks}                  \\ 
%     i,j \in \mathsf{Index}              & \text{Array Indices}          \\ 
%     o \in \set{\na,\rlx,\acqrel,\sco}   & \text{Access Modes}           \\ 
%   \end{array}
% \]
% \caption{Primitives of relaxed memory models}
% \label{fig:wmm-abs}
% \end{figure}

Most programming language memory models distinguish 
\emph{non-atomic} (sometimes also called \emph{plain})
and \emph{atomic} variables. 
The former generally should not be accessed 
concurrently from parallel threads. 
Depending on the particular programming language, 
concurrent accesses to non-atomic variables 
can be either prevented by the type-system 
(\eg \Haskell~\cite{Marlow-al:Haskell10, Vollmer-al:PPoPP17}, \Rust~\cite{RustBook:19}), 
have undefined behavior (\eg \CPP~\cite{Boehm-Adve:PLDI08, Batty-al:POPL11}), 
or being defined but have very weak semantics with almost 
no guarantees on the order in which concurrent
threads can observe these accesses (\eg \Java~\cite{Manson-al:POPL05}).

In turn, atomics are designed for concurrent accesses. 
Some memory models further distinguish 
several kinds of accesses to atomic variables.
In these models the accesses to shared memory are annotated by the 
so-called \emph{access modes}.
For example, the \CPP model (and a later revision of 
\Java~\cite{Bender-Palsberg:OOPSLA19}), distinguish 
three modes: \emph{relaxed} (\emph{opaque} in \Java terminology), 
\emph{acquire/release}, and \emph{sequentially consistent}
(\emph{volatile} in \Java).
They are denoted as $\rlx$, $\acq$, $rel$, and $\sco$ correspondingly.
Note that $\acq$ mode is only applicable to load operations,
while $\rel$ is applicable to store operations.
Non-atomic accesses are often considered to be the fourth access mode $\na$, 
but note that mixing non-atomic and atomic accesses to the same variable 
entails undefined behavior in \CPP.

The access modes are ordered by the guarantees they provide
in exchange of optimization opportunities, as the following 
diagram shows.

% $$ \na \sqsubseteq \rlx \sqsubseteq \acqrel \sqsubseteq \sco $$

 \[\inarr{
 \begin{tikzpicture}[yscale=0.6,xscale=1.0]
   \node (rlx) at (-1.3, 0) {$\rlx$};
   \node (rel) at (0, 1) {$\rel$};
   \node (acq) at (0,-1) {$\acq$};
   % \node (acqrel) at (1.5,0) {$\acqrel$};
   \node (sc) at (1.3,0) {$\sco$};

   \path[->] (rlx) edge[line width=0.742mm] node[fill=white, anchor=center, pos=0.5] {\rotatebox[origin=c]{45}{$\sqsubset$}} (rel);
   \path[->] (rlx) edge[line width=0.742mm] node[fill=white, anchor=center, pos=0.5] {\rotatebox[origin=c]{-45}{$\sqsubset$}} (acq);
   \path[->] (rel) edge[line width=0.742mm] node[fill=white, anchor=center, pos=0.5] {\rotatebox[origin=c]{-45}{$\sqsubset$}} (sc);
   \path[->] (acq) edge[line width=0.742mm] node[fill=white, anchor=center, pos=0.5] {\rotatebox[origin=c]{45}{$\sqsubset$}} (sc);
   % \path[->] (acqrel) edge[line width=0.742mm] node[fill=white, anchor=center, pos=0.5] {$\sqsubset$} (sc);

 \end{tikzpicture}
 }\]


On one end of the spectrum are sequentially consistent accesses. 
They guarantee to restore the sequentially consistent semantics, 
if used properly (see \cref{sec:background:drf} for details).
Non-atomic accesses, as we have already discussed, give a little guarantee. 
Relaxed accesses also have very weak guarantees, 
usually they provide only the \emph{coherence} property
(see \cref{sec:background:coh} for details).
Finally, in the middle there are the acquire/release accesses. 
They are designed to support the message passing idiom~\cite{Lahav-al:POPL16}.
The thread sends the message by performing a release write, 
the thread expecting a message can perform an acquire read. 
If the acquire read observes the release write, the two 
threads synchronize their views on shared memory. 

The memory model can also provide atomic \emph{read-modify-write} operations.
These include \emph{compare-and-swap}, \emph{exchange}, and variations of atomic increment,
\eg \emph{fetch-and-add}, \emph{fetch-and-sub}, \etc 
Compare-and-swap (\CAS) operation takes the shared variable, the expected 
and the desired values. It reads from the variable
and compares the result with the expected value. If the two are equal,
it substitutes the value of the variable to desired value. 
In either case, the value read from the variable is returned as a result. 
Note that the above operations are guaranteed to be performed atomically, 
no other write to the shared variable may happen in-between 
the read and write parts of \CAS.
Exchange operation atomically replaces the value 
of the variable and returns the previously held value.
Fetch-and-add and similar primitives perform 
the operation (addition, substraction,~\emph{etc.}~)
atomically and unconditionally, returning 
the content of the shared variable prior to modification.  

Locks sometimes considered to be a part 
of a memory model~\cite{Manson-al:POPL05}, 
as well as memory fence operations~\cite{Batty-al:POPL11},
which are related to hardware fence instructions
(see \cref{sec:background:compile} for details). 

Finally, a memory model can treat the shared memory 
not as set of disjoint typed variables, but rather as 
a raw byte sequence, and permit so-called \emph{mixed-size} 
concurrent accesses~\cite{Flur-al:POPL17}.
For example, in a mixed-size model it is 
allowed for an 8 byte load instruction 
to read from two concurrent adjacent 4 byte stores. 

\subsection{Compilation Scheme}
\label{sec:background:compile}

We next consider the first criterion for 
programming language memory models---an optimality 
of compilation scheme. 
A \emph{compilation scheme} is a mapping of 
programming language primitives into 
instructions of particular hardware architecture. 
In our setting we consider the primitives 
mentioned in \cref{sec:background:primitives}.
The hardware architectures provide a similar set 
of instructions which usually contain 
plain load and stores,\footnote{Some architectures 
also provide various types of load and stores
matching the access modes annotations, 
\eg \texttt{lda} --- load acquire, 
and \texttt{stl} --- store release on \ARMv{8}.} 
read-modify-write operations, 
and also various memory fences.    

The compilation mapping should be \emph{sound}.
In the context of this paper it means that 
the set of outcomes permitted by the 
memory model of the hardware 
for the compiled program should be 
a subset of outcomes permitted by the 
programming language model for the original program. 

Let us consider an example. 
The program \ref{ex:sb} below is a 
fragment of the \ref{ex:Dekker} from \cref{sec:intro}.

\begin{equation*}
\inarrII{
   \writeInst{}{x}{1}   \\
   \readInst{}{r_1}{y}  \\
}{
  \writeInst{}{y}{1}   \\
  \readInst{}{r_2}{x}  \\
}
\tag{SB}\label{ex:sb}
\end{equation*}

Assume that the memory model of the programming language
is sequential consistency, and we want 
to compile it to \Intel hardware. 
Then if one compile all loads and stores 
to plain load and store instructions of \Intel,
the outcome $[r_1=0, r_2=0]$ would be 
allowed for the compiled program 
(and it can actually be observed in practice as was stated before), 
since the memory model of \Intel permits this outcome. 
It can be obtained as a result of \emph{store buffering}
optimization (hence the name of the program \ref{ex:sb}). 
The store $\writeInst{}{x}{1}$ can be buffered and 
executed after all other instructions of the program.  
Yet clearly the outcome $[r_1=0, r_2=0]$ is not sequentially consisestent. 
Therefore the proposed compilation scheme, 
which maps all load and stores to the plain one is unsound. 
As we have demonstrated in \cref{sec:intro} 
unsoundness of compilation scheme has 
dramatic consequences, as it may break 
correctness of the program. 

Sound compilation scheme for sequential consistency 
can compiles the stores as plain stores followed 
by the \texttt{mfence} instruction~\cite{Sewell-al:CACM10, Batty-al:POPL11} 
as demonstrated below. 

\begin{equation*}
\inarrII{
   \writeInst{}{x}{1}   \\
   \mfenceInst          \\
   \readInst{}{r_1}{y}  \\
}{
  \writeInst{}{y}{1}   \\
  \mfenceInst          \\
  \readInst{}{r_2}{x}  \\
}
\tag{SB+MFENCE}\label{ex:sb-mfence}
\end{equation*}


The \texttt{mfence} is a special memory fence instruction
of \Intel architecture which flushes the store buffer of the thread. 
For the program \ref{ex:sb-mfence} the outcome $[r_1=0, r_2=0]$
is forbidden by the \Intel memory model. 

Although the modified compilation scheme is sound for \SC, 
it is not \emph{optimal}~\cite{OptimalCompilationCPP}, 
in a sense that it requires to use memory fence instructions, 
which usually induces a performance penalty
of about 10-30\%~\cite{Marino-al:PLDI11, Liu-al:OOPSLA17}
(see \cref{sec:catalog:sc} for details).
Unfortunately, it is not possible to have compilation mapping 
for the \SC model which is both \emph{sound and optimal}, 
to modern hardware architectures.     
This fact makes the \SC memory model unsuitable  
for high-performance programming languages
and serves as one of the stimuls for 
weakening of memory models. 
 
In this paper, when speaking about the compilation schemes, 
we will consider only the following hardware memory models: 
\Intel, \ARMv{7}, \ARMv{8}, and \POWER, 
for the two main reasons. 
First, these hardware architectures are the 
most widespread today.
Second, they have received a lot of attention 
from the research community recently. 
As the result of this effort, 
there were developed rigorous formal 
specifications of these models~%
\cite{Sewell-al:CACM10, Sarkar-al:PLDI11, 
Flur-al:POPL16, Pulte-al:POPL18}. 

\subsection{Program Transformations}
\label{sec:background:trans}

The next criterion for memory model is 
soundness of program transformations, 
that is source-to-source transfromations of 
the code which are applied during 
the optimization passes of the compiler. 

\emph{Sound} transformations should preserve the semantics 
of the program. In our context, similarly to the 
soundness of compilation scheme, it means that 
the set of outcomes of the transformed program 
should be a subset of outcomes of the original one. 

Going back to the \ref{ex:sb} example, 
assume the sequential consistency model again and
consider a transformation that reorders 
a store instruction past the following load 
instructions in the left thread, 
assuming the load and store operate on disjoint memory locations:

\begin{minipage}{0.45\linewidth}
\begin{equation*}
\inarrII{
   \writeInst{}{x}{1}   \\
   \readInst{}{r_1}{y}  \\
}{
  \writeInst{}{y}{1}   \\
  \readInst{}{r_2}{x}  \\
}
% \tag{SB}\label{ex:sb-src}
\end{equation*}
\end{minipage}\hfill%
\begin{minipage}{0.05\linewidth}
\Large~\\ $\leadsto$
\end{minipage}\hfill%
\begin{minipage}{0.45\linewidth}
\begin{equation*}
\inarrII{
   \readInst{}{r_1}{y}  \\
   \writeInst{}{x}{1}   \\
}{
  \writeInst{}{y}{1}   \\
  \readInst{}{r_2}{x}  \\
}
% \tag{SBtr}\label{ex:sb-tgt}
\end{equation*}
\end{minipage}

For the transformed version of the program (on the right),
the outcome $[r_1=0, r_2=0]$ is sequentially consistent.
Yet for the original one (on the left) it is not. 
It means that the aforementioned program transformation
is unsound for \SC. 

We next present a comprehensive list of 
various program transformations considered in
the literature on weak memory models 
with a short description of each one.
Note that the list is far from being complete regarding to  
transformations used in optimizing compilers~\cite{Muchnick:ACDI97}.
For example it lacks common loop optimizations, 
because the theory of relaxed memory models still
strugles with problems of liveness properties, 
needed for studing these transformations formally. 

The transformations we consider can be split into 
two subcategories: \emph{local} and \emph{global}.
Local transformations rewrite a small 
piece of code within a single thread.
Global transformations may need to consider 
the whole program (or a large part of it) 
spanning multiple threads in order 
to perform the rewriting.       
 
\subsubsection{Local Transformations}

\paragraph{Reordering of Independent Instructions} 

This transformation reorders two 
adjacent independent memory accessing instructions
operating on different memory locations.
Depending on the particular pair of instructions
it can be further split into store/load, store/store, 
load/load, and load/store reorderings.  
%
\[\def\arraystretch{1.4}\footnotesize
  \begin{array}{cccl} 

      \writeInst{}{x}{v} \seq \readInst{}{r}{y} 
    & \leadsto 
    & \readInst{}{r}{y} \seq \writeInst{}{x}{v}
    & \text{store/load}  \\ 

      \writeInst{}{x}{v} \seq \writeInst{}{y}{u} 
    & \leadsto 
    & \writeInst{}{y}{u} \seq \writeInst{}{x}{v}
    & \text{store/store}  \\ 

      \readInst{}{r}{x} \seq \readInst{}{s}{y} 
    & \leadsto 
    & \readInst{}{s}{y} \seq \readInst{}{r}{x}
    & \text{load/load}  \\ 

      \readInst{}{r}{x} \seq \writeInst{}{y}{v} 
    & \leadsto 
    & \writeInst{}{y}{v} \seq \readInst{}{r}{x}
    & \text{load/store}  \\ 

  \end{array}
\]

\paragraph{Elimination of Redundunt Access} 

In a pair of two adjacent memory access instructions 
one of them can be eliminated if its effect 
is subsumed by the another. 
For example, the two stores writing to the same variable 
can be replaced by a single store.  
Similarly to reorderings, there exist four kinds 
of eliminations depicted below. 
%
\[\def\arraystretch{1.4}\footnotesize
  \begin{array}{cccl} 

      \writeInst{}{x}{v} \seq \readInst{}{r}{x} 
    & \leadsto 
    & \writeInst{}{x}{v} \seq \assignInst{r}{v}
    & \text{store/load}  \\ 

      \readInst{}{r}{x} \seq \readInst{}{s}{x} 
    & \leadsto 
    & \readInst{}{r}{x} \seq \assignInst{s}{r}
    & \text{load/load}  \\ 

      \readInst{}{r}{x} \seq \writeInst{}{x}{r} 
    & \leadsto 
    & \readInst{}{r}{x} 
    & \text{load/store}  \\ 

      \writeInst{}{x}{v} \seq \writeInst{}{x}{u} 
    & \leadsto 
    & \writeInst{}{x}{u}
    & \text{store/store}  \\ 

  \end{array}
\]

\paragraph{Irrelevant Load Elimination}

Yet another elimination transformation 
which removes the load instruction if its 
result is never used. 
%
\[\def\arraystretch{1.4}\footnotesize
  \begin{array}{cccl} 

      \readInst{}{r}{x} 
    & \leadsto 
    & \epsInst
    & ~|~ \text{$r$ is never used}  \\ 

  \end{array}
\]

\paragraph{Speculative Load Introduction}

The inverse to the previous transformation, 
the load introduction inserts a load instruction 
in an arbitary place of the program.
%
\[\def\arraystretch{1.4}\footnotesize
  \begin{array}{cccl} 

      \epsInst
    & \leadsto 
    & \readInst{}{r}{x} 
    & ~|~ \text{$r$ is never used}  \\ 

  \end{array}
\]

It can be used in combination with 
load/load elimination to move a load 
instruction out from one branch of 
a conditional:
%
\[\def\arraystretch{1.4}\footnotesize
  \begin{array}{ccc} 

      \kw{if} (e)~ \kw{then} \{ \readInst{}{r}{x} \}
    & \leadsto 
    & \readInst{}{s}{x} \seq \kw{if} (e)~ \kw{then} \{ \assignInst{r}{s} \} \\
    & & ~|~ \text{$s$ is never used}  \\ 

  \end{array}
\]

\paragraph{Roach Motel Reordering}

This class of reorderings moves memory access instructions
into synchronization blocks. For example, a store 
can be moved past a lock acquisition. 
Intuitively, such reorderings can only increase 
the synchronization of the program, 
which means that the transformed program should 
exhibit less non-determinism and have fewer outcomes. 

Non-atomic accesses can be moved freely inside 
a critical section, \ie past a lock acquisition
or prior a lock release. 
Besides that, a store can be moved after \texttt{lock}, 
and load can be moved prior \texttt{unlock}.   
Similar rules apply to reordeings around 
acquire and release accesses and fences, 
where an acquire operation behaves similarly to \texttt{lock}, 
and release operation similarly to \texttt{unlock}.
% %
\[\def\arraystretch{1.4}\footnotesize
  \begin{array}{cccl} 

      \readInst{\na}{r}{x} \seq \lockInst{l} 
    & \leadsto 
    & \lockInst{l} \seq \readInst{\na}{r}{x}
    & ~ \\ 

      \writeInst{o}{x}{v} \seq \lockInst{l} 
    & \leadsto 
    & \lockInst{l} \seq \writeInst{o}{x}{v}
    & ~  \\ 

      \unlockInst{l} \seq \writeInst{\na}{x}{v} 
    & \leadsto 
    & \writeInst{\na}{x}{v} \seq \unlockInst{l}
    & ~ \\ 


      \unlockInst{l} \seq \readInst{o}{r}{x} 
    & \leadsto 
    & \readInst{o}{r}{x} \seq \unlockInst{l}
    & ~  \\ 

  \end{array}
\]


\paragraph{Strengthening}

Similarly to the roach motel reordeing, the strengthening
transformation increases synchronization by 
replacing an access mode of an operation by a stronger one. 
For example, a non-atomic access can be replaced by 
a sequentially consistent access: 
%
\[\def\arraystretch{1.4}\footnotesize
  \begin{array}{cccl} 

      \readInst{o}{r}{x} 
    & \leadsto 
    & \readInst{o'}{r}{x}
    & ~|~ o \sqsubset o' \\ 

      \writeInst{o}{x}{v}
    & \leadsto 
    & \writeInst{o'}{x}{v}
    & ~|~ o \sqsubset o'  \\ 

  \end{array}
\]

\paragraph{Trace Preserving Transformations}

This wide class contains all local transformations 
which do not change the set of traces of a thread~\cite{Sevcik-Aspinall:ECOOP08},
Trace is a sequence of visible side-effects performed by a thread. 
An example is classic \emph{constant folding}%
~\cite{Muchnick:ACDI97, Wegman-Zadeck:TOPLAS91} transformation.
Here is particular example of constant folding application:
%
\[\def\arraystretch{1.4}\footnotesize
  \begin{array}{cccl} 

      \writeInst{}{x}{0 + v} 
    & \leadsto 
    & \writeInst{}{x}{v}
    & \\ 

  \end{array}
\]
  
\paragraph{Common Subexpression Elimination}

\CSE is yet another classic transformation~\cite{Muchnick:ACDI97} 
which searches for instances of identical expressions 
and removes redundunt computations. 
Here is an example: 
%
\[\def\arraystretch{1.4}\footnotesize
  \begin{array}{cccl} 

      \readInst{}{r_1}{x + y} \seq \readInst{}{r_2}{x + y} 
    & \leadsto 
    & \readInst{}{r_1}{x + y} \seq \readInst{}{r_2}{r_1}
    & \\ 

  \end{array}
\]

\subsubsection{Global Transformations}

\paragraph{Register Promotion}

If the compiler can determine that a shared variable 
is accesses only from a single thread, it can replace 
the variable by a thread-local register. 
%
\[\def\arraystretch{1.4}\footnotesize
  \begin{array}{ccl} 

      \writeInst{}{x}{v} \seq \readInst{}{r}{x} 
    & \leadsto 
    & \assignInst{s}{v} \seq \assignInst{r}{s}
    \\ 
    
    & & |~ \text{\texttt{x} is not accessed from other threads} \\
    & & |~ \text{\texttt{s} is a fresh register} \\ 

  \end{array}
\]

\paragraph{Thread Inlining}

Sequentialization or thread inlining 
is a transformation that merges two threads into one.
Quite surprisingly, this seamingly harmless transformation
is challenging for most of memory models. 
%
\[\def\arraystretch{1.4}\footnotesize
  \begin{array}{cccl} 

      P \pll Q 
    & \leadsto 
    & P ~\seq Q
    & ~ \\ 
    
  \end{array}
\]


\paragraph{Value Range Based Transformations}

Transformations of this class can be applied 
if a program satisfies some invariant deduced 
by a global value-range analysis. 
For example, in a program below   
the conditional statement can be eliminated 
since a static analysis can deduce an invariant 
$\mathsf{x} \geq \mathsf{0}$.

{\footnotesize
\begin{minipage}{0.45\linewidth}
\begin{equation*}
\inarrII{
   \readInst{}{r_1}{x}             \\
   \kw{if} (r_1 \geq 0) ~\kw{then} \\
   \quad\writeInst{}{y}{1}         \\
}{
  \readInst{}{r_2}{x}               \\
  \writeInst{}{y}{r_2}              \\
}
\end{equation*}
\end{minipage}\hfill%
\begin{minipage}{0.05\linewidth}
\Large~\\ $\leadsto$
\end{minipage}\hfill%
\begin{minipage}{0.4\linewidth}
\begin{equation*}
\inarrII{
   \readInst{}{r_1}{x}             \\
   \writeInst{}{y}{1}              \\
}{
  \readInst{}{r_2}{x}               \\
  \writeInst{}{y}{r_2}              \\
}
\end{equation*}
\end{minipage}
}

\subsection{Reasoning Guarantees}

Finally, we discuss the guarantees provided by memory models 
which allow the programmers to reason about concurrent 
programs and their correctness. 

\subsubsection{DRF Theorems}
\label{sec:background:drf}

When reasoning about concurrent code, most programmers 
assume sequential consistent memory model.
Of course, it would be improper to require from 
the programmers to always keep in mind 
all the intricacy of weak memory models,
as it only complicates an already difficult task
of establishing correctness of concurrent programs. 
The \emph{data-race freedom}~\cite{Manson-al:POPL05} property, 
DRF for short, is designed to solve this problem. 
It guarantees that well-synchronized programs 
have only sequentially consistent outcomes. 
In other words, it allows to programmers assume 
simpler sequentially consistent model 
if they properly use synchonization primitives.

Let us consider an example. 
Remember the \ref{ex:sb} program from \cref{sec:background:compile}.
As we have demonstrated, under the weak memory model,
like the model of \Intel hardware, 
this program can have weak outcome $[r_1=0, r_2=0]$.
Nevertheless, one can restore the \SC semantics.
One way to do this is to use locks, as the following listing demonstrates:

\begin{equation*}
\inarrII{
   \lockInst{l}         \\
   \writeInst{}{x}{1}   \\
   \readInst{}{r_1}{y}  \\
   \unlockInst{l}       \\
}{
   \lockInst{l}         \\
   \writeInst{}{y}{1}   \\
   \readInst{}{r_2}{x}  \\
   \unlockInst{l}       \\
}
\tag{SB+LOCK}\label{ex:sb-lock}
\end{equation*}

A DRF compliant weak memory model should guarantee 
that this program has only sequentially consistent outcomes:
$\{[r_1=0, r_2=1]\}$, $\{[r_1=1,r_2=0]\}$, or $\{[r_1=1,r_2=1]\}$.

Alternatively, if the model provides $\sco$ access mode, 
a programmer can annotate all memory accesses by this mode
to restore sequential consistency:  
 
\begin{equation*}
\inarrII{
   \writeInst{\sco}{x}{1}   \\
   \readInst{\sco}{r_1}{y}  \\
}{
   \writeInst{\sco}{y}{1}   \\
   \readInst{\sco}{r_2}{x}  \\
}
\tag{SB+SC}\label{ex:sb-sc}
\end{equation*}

More formally, DRF theorem for a weak model $M$ states that 
a program has only sequentially consistent outcomes under $M$
if it has no data-races under sequentially consistent memory model
(or all accees participating in such race are annotated by~$\sco$).

The DRF theorem allows to reduce reasoning under weak memory model
to reasoning under sequential consistency.
It is sufficient to prove that a program has no data-races under \SC
in order to derive that this program has only \SC outcomes. 

The DRF theorem in the formulation given above is 
sometimes called \emph{external data-race freedom},
in order to distinguish it from \emph{internal data-race freeedom}. 
The latter guarantees the \SC semantics for a program 
under weak model $M$ only if the program 
has no races under \textbf{model $\mathbf{M}$ itself}.
Note that the internal DRF gives a weaker guarantee 
compared to external DRF. It does not allow to completly 
avoid the reasoning in term of the weak memory model, 
because one has to first show that the program 
is race-free under relaxed model. 
As we will demonstrate later (see \cref{sec:analysis:oota})
internal DRF is a reasonable compromise
for a certain class of memory models which 
cannot establish external DRF.

\subsubsection{Coherence}
\label{sec:background:coh}

\subsubsection{Undefined Behavior}
\label{sec:background:ub}

\subsubsection{Out-of-Thin-Air Values}
\label{sec:background:oota}

