\section{Analysis}

In this section we present the detailed comparison 
of the considered programming language memory models. 

We summarize our findings in \cref{table:summary}. 
Each row of the table corresponds to a memory model, denoted by its abbreviation. 
The correspondence between the memory model abbrevation, 
its full name and the research papers is given in \cref{table:mmodels}.
Columns of the table correspond to properties of the models. 
We split the properties into several subgroups. 

The first group is devoted to optimality of compilation mappings
to target hardware instructions. 
\todo{explain how we classify mappings as optimal/non-optimal}.

The second group is dedicated to soundness of various program transformations. 
The classification is binary: a transformation is either sound or unsound 
in the given memory model (in a sense stated in \cref{sec:bgrnd-opt-sound}).
We further split the transformations to global and local as in \cref{sec:bgrnd-opt-sound}.

The third group corresponds to reasoning principles guaranteed by the model. 
It includes the following properties. What kind of DRF guarantee the model provides.
(we distinguish the internal, external and local DRF theorems, see \cref{sec:bgrnd-drf}).
Whether the model has undefined behaviors (see \cref{sec:bgrnd-ub}).
Whether the model permits out-of-thin-air values (see \cref{sec:bgrnd-oota}).

Finally, the last group enumerates the list of memory access modes 
and fences supported by the model, as well as whether the model 
supports read-modify-write operations, locks, and mixed-size accesses.

Driven by our analysis of the models' properties, we partition all models into five classes. 
The models from the same class have similar compilation mappings, 
set of sound program transformations, and provided reasoning guarantees.
Our classes are ordered by the weakness of the memory models they consist of.  
The strongest models are located at the top rows of the table, 
while the weakest are at the bottom. 

We next discuss each class in more details
(note that the order is different from the one in the \cref{table:summary}). 
We also give some insight on the relationship
between compilation scheme optimality, 
soundness of transformations and reasoning guarantees.
In particular, we explain why the support of some reasoning guarantees 
disables some program transformations and requires more heavyweight 
compilation mappings to hardware.

\subsection{Strong SC-like Models}

\subsection{Strong TSO-like Models}

\subsection{OOTA Models}

\subsection{$\lPO\cup\lRF$ Acyclic Models}

\subsection{$\lPPO\cup\lRF$ Acyclic Models}

\subsection{no-OOTA Models}
 
\onecolumn

\begin{landscape}

\begin{table*}
\begin{tabular}{|c|c|c|c|c|c|c|c|c|c|c|c|c|c|c|c|c|c|c|c|c|c|c|c|c|c|c|c|c|c|}
 \hline

 \multirow{3}{*}{Model}                               & 
 \multicolumn{ 4}{c|}{\multirow{2}{*}{Compilation}}   &
 \multicolumn{10}{c|}{Optimizations}                  &
 \multicolumn{ 3}{c|}{\multirow{2}{*}{DRF}}           &
 \multicolumn{ 3}{c|}{\multirow{2}{*}{Properties}}    &
 \multicolumn{ 9}{c|}{\multirow{2}{*}{Features}}      \\ 

 \cline{6-15}

                             &
 \multicolumn{4}{c|}{}       &
 \multicolumn{7}{c|}{Local}  &
 \multicolumn{3}{c|}{Global} &

 \multicolumn{3}{c|}{}       &
 \multicolumn{3}{c|}{}       &
 \multicolumn{9}{c|}{}       \\ 
 
 \hline
                                      &
 \rotatebox[origin=c]{270}{x86}      & 
 \rotatebox[origin=c]{270}{Power}    & 
 \rotatebox[origin=c]{270}{ARMv7}    & 
 \rotatebox[origin=c]{270}{ARMv8}    & 
 
 \rotatebox[origin=c]{270}{T.P.}     &
 \rotatebox[origin=c]{270}{R.I.}     &
 \rotatebox[origin=c]{270}{R.E.}     &
 \rotatebox[origin=c]{270}{I.L.E.}   &
 \rotatebox[origin=c]{270}{S.L.I.}   &
 \rotatebox[origin=c]{270}{S.}       &
 \rotatebox[origin=c]{270}{R.M.}     &
 \rotatebox[origin=c]{270}{R.P.}     &
 \rotatebox[origin=c]{270}{V.R.}     &
 \rotatebox[origin=c]{270}{T.I.}     &

 % \makecell{Trace \\ Preserving}                &
 % \makecell{Reordering of \\Indep. \\Instr.}    & 
 % \makecell{Redundunt \\Load/Store \\Elim.}     &
 % \makecell{Irrelevant \\Load \\Elim.}          &
 % \makecell{Speculative \\Load \\Intro.}        &
 % \makecell{Strength.}                          &
 % \makecell{Roach Motel}                        &
 % \makecell{Register \\ Promotion}              &
 % \makecell{Value Range}                        &
 % \makecell{Thread \\ Inlining}                 &
 
 \rotatebox[origin=c]{270}{Int.} &
 \rotatebox[origin=c]{270}{Ext.} &
 \rotatebox[origin=c]{270}{Loc.} &

 \rotatebox[origin=c]{270}{UB}                                 &
 \rotatebox[origin=c]{270}{\makecell{$\lPO\cup\lRF$ \\ acyc.}} & 
 \rotatebox[origin=c]{270}{OOTA}                               &                              


 \rotatebox[origin=c]{270}{NA}                      &
 \rotatebox[origin=c]{270}{RLX}                     &
 \rotatebox[origin=c]{270}{RA}                      &
 \rotatebox[origin=c]{270}{SC}                      &
 \rotatebox[origin=c]{270}{F-RA}                    &
 \rotatebox[origin=c]{270}{F-SC}                    &
 \rotatebox[origin=c]{270}{RMW}                     &
 \rotatebox[origin=c]{270}{Lock}                    &
 \rotatebox[origin=c]{270}{\makecell{Mix.Sz.}}  \\ 
 
 \hline
 
 SC             & & & & & & & & & & & & & & & & & & & & & & & & & & & & & \\ \hline

 C11            & & & & & & & & & & & & & & & & & & & & & & & & & & & & & \\ \hline

 JMM            & & & & & & & & & & & & & & & & & & & & & & & & & & & & & \\ \hline

 RC11           & & & & & & & & & & & & & & & & & & & & & & & & & & & & & \\ \hline

 Promising      & & & & & & & & & & & & & & & & & & & & & & & & & & & & & \\ \hline

 Weakest        & & & & & & & & & & & & & & & & & & & & & & & & & & & & & \\ \hline

 RMD            & & & & & & & & & & & & & & & & & & & & & & & & & & & & & \\ \hline

 OCaml          & & & & & & & & & & & & & & & & & & & & & & & & & & & & & \\ \hline

 JavaScript     & & & & & & & & & & & & & & & & & & & & & & & & & & & & & \\ \hline

\end{tabular}
\caption{
  % Comparison of memory models. 
  % \textit{Eff. Comp.} --- does the model have efficient compilation scheme.
  % \textit{Opt.} --- are the common optimizations sound in the model.
  % \textit{DRF.} --- does the model provides Data-Race-Freedom guarantees.
  % \textit{UB.} --- does the model have undefined behaviors.
  % \textit{No OOTA.} --- does the model forbid out-of-thin-air behaviors.
  % \textit{MC.} --- is the model suitable for model checking.
  % \textit{Complex.} --- subjective complexity of the model.
  \textit{T.P.} --- trace preserving.
  \textit{R.I.} --- reordering of independent instructions.
  \textit{R.E.} --- redundunt load/store elimination.
  \textit{I.L.E.} --- irrelevant load elimination.
  \textit{S.L.I.} --- speculative load introduction.
  \textit{S.} --- strengthening.
  \textit{R.M.} --- roach motel reordering.
  \textit{R.P.} --- register promotion.
  \textit{V.R.} --- value range analysis based optimizations.
  \textit{T.I.} --- thread inlining (sequentialization).
  \textit{Int.} --- internal.
  \textit{Ext.} --- external.
  \textit{Loc.} --- local.
  \textit{UB} --- undefined behavior.
  \textit{OOTA} --- out-of-thin air values.
  \textit{Mix.Sz.} --- mixed-size accesses.
}
\label{table:summary}
\end{table*}

\end{landscape}

\twocolumn