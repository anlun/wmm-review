\section{Analysis}
\label{sec:analysis}

% We partitioned all models into six primary classes: 
% sequentially consistent models, models with total or partial order on stores, 
% program order preserving models, syntactic dependency preserving models, 
% semantic dependency preserving models, and model with out of thin-air values.

In this section we discuss each identified class 
of memory models in more detail.
Based on the data from \cref{table:cmp-cls},
we derive a relationship between a compilation scheme optimality, 
a soundness of transformations and reasoning guarantees.
In particular, we show how the support of some reasoning guarantees 
disables some program transformations and requires a more heavyweight 
compilation mapping to hardware.

We stat with a discussion of the class of 
sequentially consistent models~\cref{sec:analysis:seqcst}.
Then we proceed to the class of totally and partially 
store ordered models~\cref{sec:analysis:tso}.
After that we switch to the class of very weak 
models permitting thin-air values~\cref{sec:analysis:oota}.
Then we describe various solutions tackling 
the problem of thin-air values, namely
program order preserving models~\cref{sec:analysis:porf},
syntactic dependency preserving models~\cref{sec:analysis:deprf},
and semantic dependency preserving models~\cref{sec:analysis:sdeprf}.

In \cref{sec:analysis:other} we also discuss the particular properties 
of models, namely the coherence and the catch-fire semantics, 
which are orthogonal to the main partitioning into classes, 
but nonetheless they affect the soundness of 
certain program transformations.

\subsection{Sequential Consistency}
\label{sec:analysis:seqcst}

Sequential Consistency (\SC) is one of the most intuitive models of concurrency.
Under this model, one can represent a state of the memory as 
a simple mapping from memory locations to their values. 
Then each outcome can be obtained by 
sequential execution of some interleaving of threads' instructions.
% Each execution induces a total order 
% on individual memory access operations consistent with per-thread program order
% and, moreover, each read is guaranteed to read-from the most recent write. 

\SC renders many common transformations unsound, 
including all kind of instruction reorderings and the
common subexpression elimination~\cite{Marino-al:PLDI11, Sevcik-Aspinall:ECOOP08}.
The fact that instruction reorderings are forbidden 
makes the model expensive to implement on modern hardware
since even the relatively strong hardware model of \Intel
performs store/load reordering.
Therefore, in order to preserve the sequential consistency during compilation,
a compiler need to emit heavyweight memory fences 
between store and load instructions,
which makes the compilation mappings far from being optimal.  

In terms of reasoning guarantees, however, \SC is quite a pleasant model. 
It gives the external \DRF and the coherence properties for free, 
because it assigns to programs only sequentially consistent
outcomes by definition.
% thus it satisfies \DRF-SC without any preconditions.
% The coherence property is equivalent to sequential consistency per location.
% The fact that in SC model any execution is sequentially consistent implies trivially
% that accesses to each location are also sequentially consistent.

The conceptual simplicity of SC have inspired many researchers 
to adopt it and to try to mitigate the induced performance penalty.
The recurring idea was to somehow separate 
thread-local and shared mutable memory.
Accesses to thread-local memory can be compiled 
without memory fences and are subject to a wider range 
of local transformations.
To safely distinguish between two types of memory
researches proposed to utilize a type-system~\cite{Vollmer-al:PPoPP17},
static~\cite{Singh-al:ISCA12} or dynamic~\cite{Liu-al:PLDI19} analysis,
a hardware support~\cite{Singh-al:ISCA12, Marino-al:PLDI10}
or some combination of the above. 

Despite these efforts \SC still induces considerable slowdown,
especially on weak hardware.
For instance, on \ARMv{8} machines 
the slowdown can be up to 70\%~\cite{Liu-al:PLDI19} 
(see \cref{sec:catalog:sc} for details).
Moreover, while these optimization typically reduce 
penalties on thread-local accesses (a common case), 
they are likely to have a lesser impact on specific 
applications which heavily utilize concurrency,
for example, lock-free data structures.
Finally, modern compilers usually require 
a significant amount of engineering work and rewrite
in order to preserve \SC~\cite{Marino-al:PLDI11, Liu-al:PLDI19}.

\subsection{Total and Partial Store Order}
\label{sec:analysis:tso}

The next class of PL memory models we consider 
was inspired by the \emph{total store order}~\cite{Sewell-al:CACM10} (\TSO) 
and the \emph{partial store order}~\cite{Sparc:94} (\PSO) hardware models. 
In these models threads are equipped with \emph{store buffers}.
All store operations go to these buffers before they 
propagate into the main memory.
% The name of these models stems from their axiomatic formulation. 
% They require an existence of total (in case of \TSO) 
% or partial (in case of \PSO) order on write events, 
% consistent with program order and such that 
% each read either reads locally or from 
% the most recent write~\cite{Sewell-al:CACM10, Lahav-al:POPL16}. 

Models of this class can be compiled down to the \Intel hardware without any 
performance penalty, since \Intel implements \TSO model itself.
However, on weaker hardware like~\POWER 
a compiler need to emit essentially as many fences 
as to enforce~\SC~\cite{Lustig-al:AISCA15}. 

This class permits more program transformations than \SC.
Store buffers enable store/load reorderings in case of~\TSO,
and additionally store/store reorderings in case of~\PSO.
The \TSO and \PSO models are weaker than the \SC, 
but they are still relatively strong~---
the external \DRF and the coherence properties hold.

Therefore these models do not have any significant 
benefits in terms of the reasoning guarantees compared to \SC,
but induce a similar performance penalty 
on architectures weaker than \Intel. 
Hence a choice of \TSO and \PSO as a programming language level
memory model is reasonable only if the language targets
the \Intel hardware solely. 

\subsection{Out of Thin-Air Values}
\label{sec:analysis:oota}

We next move on to the other end of a memory models' spectrum. 
We consider the class uniting the weakest models from our list.
These models enable efficient compilation mappings and 
almost all reasonable program transformations, but at a cost of 
introducing thin-air values (\cref{sec:background:oota}).

Let us revisit the load buffering program:  

\begin{minipage}{0.43\linewidth}
\begin{equation*}
\small
\inarrII{
  \readInst{}{r_1}{x}     \\
  \writeInst{}{y}{1}      \\
}{
  \readInst{}{r_2}{y}     \\
  \writeInst{}{x}{r_2}    \\
}
\tag{LB}\label{ex:lbA}
\end{equation*}
\end{minipage}\hfill%
\begin{minipage}{0.09\linewidth}
\Large~\\ $\leadsto$
\end{minipage}\hfill%
\begin{minipage}{0.43\linewidth}
\begin{equation*}
\inarrII{
  \writeInst{}{y}{1}      \\
  \readInst{}{r_1}{x}     \\
}{
  \readInst{}{r_2}{y}     \\
  \writeInst{}{x}{r_2}    \\
}
\tag{LBtr}\label{ex:lbB}
\end{equation*}
\end{minipage}

Program on the right \ref{ex:lbB} can be obtained 
from the program on the left \ref{ex:lbA}
via the load/store reordering.
The outcome ${[r_1=1, r_2=1]}$ is valid for \ref{ex:lbB}.
Therefore under a memory model where the load/store reordering 
is a sound transformation, 
this outcome should also be valid for \ref{ex:lbA}.
As we demonstrated in \cref{sec:background:oota}
in order to allow this outcome, 
memory models needs to utilize speculative execution.

We also demonstrated that unrestricted speculations 
can lead to so called thin-air values which 
break fundamental reasoning principles~%
\cite{Boehm-Demsky:MSPC14, Batty-al:ESOP15}.
In the presence of thin-air values
type safety and security guarantees can be violated, 
and compositional reasoning is impossible.
Moreover, they are also incompatible with 
the external \DRF property.
To see this consider yet another variation of 
the load buffering program:

\begin{equation*}
\inarrII{
  \readInst{}{r_1}{x}      \\
  \kw{if} {(r_1)} ~\{      \\
  \quad\writeInst{}{y}{1}  \\
  \}
}{
  \readInst{}{r_2}{x}      \\
  \kw{if} {(r_2)} ~\{      \\
  \quad\writeInst{}{x}{1}  \\
  \}
}
\tag{LB+ctrl}\label{ex:lb+ctrl}
\end{equation*}

For a memory model admitting thin-air values 
(as \eg \CMM~\cite{Batty-al:POPL11}), 
the outcome ${[r_1=1, r_2=1]}$ is valid
(a justification is the same as for 
the \ref{ex:lb+data} example from \cref{sec:background:oota}).
Not only this outcome is completely unintuitive,
but it also contradicts to the external \DRF guarantee.
Indeed, under \SC model the program above has 
single valid execution with the outcome $[r_1=0, r_2=0]$ 
containing no data-races, thus under a \DRF compliant model 
it should also has this sole outcome.  

A counter-intuitive behavior of OOTA models, 
together with the fact that they break 
important reasoning principles,
has lead over the time to the consensus 
in the research community that these models 
are not suited well for the role of 
programming languages memory models~\cite{Boehm-Demsky:MSPC14, Batty-al:ESOP15}.
A lot of effort has been put to forbid problematic 
thin-air outcomes, while still keep a 
compilation scheme as efficient as possible
and enable as many transformations as possible.
In the rest of this section we describe various proposals 
tackling the thin-air problem. 

\subsection{Program Order Preserving}
\label{sec:analysis:porf}

The most straightforward way to forbid thin-air values 
was proposed by Boehm and Demsky~\cite{Boehm-Demsky:MSPC14}
The idea is to simply prohibit any kind of speculative execution, 
which can be achieved by forbidding load/store reorderings altogether. 
This fix not only restores the external \DRF~\cite{Lahav-al:PLDI17}
and other reasoning guarantees, but also leads to 
a much simpler model. The abstract machine implementing 
the memory model does not need to resort to speculative execution 
and can perform threads' instructions in-order. 
A memory storage can be implemented as a 
monotonically growing history of messages, 
with each thread having its own view on 
a frontier of this history~\cite{Dolan-al:PLDI18, Doherty-al:PPoPP19}.
% In terms of axiomatic semantics it is enough to just
% forbid cycles consisting of program order and reads-from relations. 

Lahav~\etal~\cite{Lahav-al:PLDI17} formalized this approach
to thin-air problem and studied it extensively. 
The authors shown that many program transformations 
are still sound in this setting, 
with the obvious exception of the load/store reordering itself
(see~\cref{table:cmp-cls} for details).

The compilation mapping to \Intel remain efficient, 
since this architecture already guarantee to preserve order 
between loads and subsequent stores. 
However, weaker architectures (\ARM, \POWER) do not guarantee that, 
and thus additional measures are required.
Boehm and Demsky~\etal~\cite{Boehm-Demsky:MSPC14} proposed to 
compile every relaxed load as 
a plain load followed by a spurious conditional branch,
which introduces a dependency between 
the load and subsequent stores. 
\ARM and \POWER hardware preserves this dependency, 
and thus it also retains the load/store ordering. 
Ou and Demsky~\cite{Ou-Demsky:OOPSLA18} studied 
a performance penalty required to preserve
load/store ordering between \textbf{relaxed atomics only}
on the \ARMv{8} hardware and reported a negligible overhead
of 0\% on average and 6.3\% in maximum on a set of benchmarks
implementing various concurrent data-structures, 
\eg locks, stacks, queues, deques, maps, \etc 
(see \ref{sec:catalog:porf} for details).
Note that the overhead is expected to be greater 
if the compilation scheme is required to preserve the ordering 
between non-atomic accesses as well. 

% In conclusion, the memory models forbidding load/store reorderings
% provide a simple solution to thin-air problem, 
% restore external \DRF-SC guarantee, preserve soundness of 
% most program transformations, induce no overhead when 
% compiled down to \Intel, and a moderate overhead 
% when compiled to \ARM or \POWER.

\subsection{Syntactic Dependencies Preserving}
\label{sec:analysis:deprf}

The alternative conceptually simple solution 
to thin-air values problem is to preserve 
\emph{syntactic dependencies}~\cite{Boehm-Demsky:MSPC14, Alglave-al:ASPLOS18}.
Under this approach reordering of independent load/store pairs is allowed.
However, reordering is forbidden if a store depends on the value 
read by a load either because this value 
was used to compute the value written by the store (\emph{data dependency}), 
or it was used to compute 
the memory address of the store (\emph{address dependency}),
or else a control-flow path lead to the store was dependent
on this value (\emph{control dependency}).
For example, giving the program \ref{ex:lb+data} 
the store $\writeInst{}{y}{r_1}$ depends 
on the load $\readInst{}{x}{r_1}$ since 
it writes the value read by the load.

Note that these kind of dependencies are computed following a 
syntax of a program (hence the name) as opposed 
to \emph{semantic dependencies}.
For example, giving the modified version of 
the \ref{ex:lb+data} program below, 
the store to \texttt{y} is still considered 
to be dependent on the previous load. 

\begin{equation*}
\inarrII{
  \readInst{}{r_1}{x}           \\
  \writeInst{}{y}{1 + 0 * r_1}  \\
}{
  \readInst{}{r_2}{x}      \\
  \writeInst{}{x}{r_2}     \\
}
\tag{LB+fakedata}\label{ex:lb+fakedata}
\end{equation*}

Here the syntactic dependency can be eliminated 
by the \emph{constant folding} transformation --- 
the expression $1 + 0 * r_1$ can be reduced to value~$1$.
Under a syntactic dependency preserving memory model 
a compiler, however, is prohibited to perform this optimization. 
Indeed, once a dependency is removed, nothing prevents 
to reorder a store before a preceding load. 
Even if a compiler itself does not perform this reordering,
after the compilation hardware can 
do this during the execution.   

This subtlety reveals the main drawback of 
syntactic dependency tracking models --- 
trace preserving transformations
(\eg constant folding) are unsound in these models. 
Constant folding is one of the classic optimizations 
that any compiler might want to apply, 
and the fact that it is unsound  
makes an adoption of this class of models problematic.
Note that hardware memory models apply a similar approach 
and usually have a notion of syntactic dependencies between 
memory operations~\cite{Sarkar-al:PLDI11, Alglave-al:TOPLAS14, Pulte-al:POPL18}.
Yet in this setting the unsoundness of 
trace preserving transformations is not a problem,
since hardware does not perform such complex optimizations.

Ou and Demsky~\cite{Ou-Demsky:OOPSLA18} examined
the performance penalty of a syntactic 
dependency tracking compiler.
They adjusted compiler optimization passes to preserve
dependencies between \textbf{non-atomic and relaxed} accesses.
They evaluated the dependency preserving 
version of the \LLVM compiler infrastructure 
on \ARMv{8} hardware using \SPECCPU benchmarks
and reported a moderate slowdown of 
3.1\% on average and 17.6\% in maximum. 
(see \ref{sec:catalog:deprf} for details).

\subsection{Semantic Dependencies Preserving}
\label{sec:analysis:sdeprf}

The last approach to tackle thin-air problem is to   
construct a notion of \emph{semantic dependencies}, 
which would precisely characterize what load/store 
pairs are independent and rule out fake dependencies 
like the one in \ref{ex:lb+fakedata}.
A practical payoff of this approach is that it 
does not require significant modifications to existing compilers or hardware, 
and thus should not impose performance penalties.  
The ultimate goal is to enable the optimal compilation mappings, 
preserve most of the existing compiler optimizations, 
and at the same time maintain the important 
reasoning guarantees like external \DRF. 

It turns out that this task is quite challenging 
and to this date there is no strong consensus on how to achieve it.
In order to give a satisfactory definition of semantic dependencies 
researchers had to resort to conceptually complex memory models%
~\cite{Jagadeesan-al:ESOP10, Kang-al:POPL17, Jeffrey-Riely:LICS16, 
PichonPharabod-Sewell:POPL16, Chakraborty-Vafeiadis:POPL19, Paviotti-al:ESOP20}.
The main challenge in this line of work was to formally prove 
that these complex models indeed satisfy all the desired properties. 

Currently the most complete approach of this class 
is the \Promising semantics~\cite{Kang-al:POPL17, Lee-al:PLDI20}. 
This model was proven to enable the optimal compilation schemes~\cite{Podkopaev-al:POPL19}, 
and permit most local and global program transformations
(with a notable exception of the thread inlining), 
while still preserving the external \DRF guarantee.

\subsection{Secondary Classes}
\label{sec:analysis:other}

We also identified an alternative division of memory models into groups, 
which correspond to particular properties of a memory model, 
namely the coherence and the catch-fire semantics which treats racy programs 
as erroneous. We demonstrate how presence of these properties 
affects the compilation mappings and the soundness of some program transformations.

\subsubsection{Coherent Models}
\label{sec:analysis:coh}

The coherence property (\ie \SC-per-location, \cref{sec:background:coh})
has a subtle effect on the common subexpression elimination optimization (\CSE),
which was was first observed in the context of an early version of the \Java 
memory model~\cite{Pugh:JAVA99}.
To see the problem, consider the program below
(on the left) and the transformed version 
of this program after application of \CSE (on the right).
Note that the optimization has replaced 
the second access to variable \texttt{x}
by the read from register. 

\begin{minipage}{0.45\linewidth}
\begin{equation*}
\small
\inarrII{
  \readInst{}{r_1}{x}      \\
  \readInst{}{r_2}{y}      \\
  \readInst{}{r_3}{x}      \\
}{
  \writeInst{}{y}{1}       \\
}
\label{ex:coh-rr}
\end{equation*}
\end{minipage}\hfill%
\begin{minipage}{0.05\linewidth}
\Large~\\ $\leadsto$
\end{minipage}\hfill%
\begin{minipage}{0.45\linewidth}
\begin{equation*}
\small
\inarrII{
  \readInst{}{r_1}{x}      \\
  \readInst{}{r_2}{y}      \\
  \assignInst{r_3}{r_1}    \\
}{
  \writeInst{}{y}{1}       \\
}
\label{ex:coh-rr}
\end{equation*}
\end{minipage}

Now assume that variables \texttt{x} and \texttt{y} 
point to the same memory location.
Under this assumption the outcome $[r_1=0, r_2=1, r_3=0]$
is forbidden for a memory model respecting coherence.
Indeed, the coherence guarantees sequential consistency per location, 
which means that for programs consisting of accesses 
to a single memory location 
(as the one above in the presence of aliasing) 
only the sequentially consistent outcomes are allowed.
The outcome $[r_1=0, r_2=1, r_3=0]$ cannot be obtained 
as the interleaving of instructions, and thus 
it should be forbidden.  
However, this outcome is allowed for 
the optimized version of the program. 

Note that the compiler still can apply \CSE to the program above, 
but only if it is able to prove that variables \texttt{x} and \texttt{y} 
point to disjoint memory locations, which can be achieved 
by the means of an alias analysis~\cite{Diwan-al:PLDI1998}.
In fact, in this case \CSE can be seen as a combination 
of instructiong reordering and elimination transformations.  

Therefore, the coherence property in general is not compatible 
with the common subexpression elimination.
As for compilation schemes, coherence does not require 
any changes here and thus does not impose any performance penalty.
It is because hardware models already guarantee coherence%
~\cite{Alglave-al:TOPLAS14, Sarkar-al:PLDI11, Sewell-al:CACM10, Lahav-al:PLDI17}. 

\subsubsection{Catch-Fire Models}
\label{sec:analysis:ub}

A catch-fire semantics which treats racy non-atomic 
accesses as undefined behavior also affects  
soundness of program transformations. 
As we already briefly discussed, it enables 
the optimal compilation mapping for non-atomic accesses and 
makes sequentially valid transformations applicable 
to them, but its impact is not limited to this observation. 
A catch-fire semantics also has an interesting interplay
with the speculative load introduction.

Consider the following example:

\begin{minipage}{0.43\linewidth}
\begin{equation*}
\small
\inarrII{
  \readInst{}{r}{x}      \\
  \kw{if} {(r)} ~\{      \\
  \quad\readInst{}{s}{y} \\
  \}

}{
  \writeInst{}{y}{1}       \\
}
\label{ex:sliA}
\end{equation*}
\end{minipage}\hfill%
\begin{minipage}{0.09\linewidth}
\Large~\\ $\leadsto$
\end{minipage}\hfill%
\begin{minipage}{0.43\linewidth}
\begin{equation*}
\inarrII{
  \readInst{}{r_1}{x}      \\
  \readInst{}{t}{y}        \\
  \kw{if} {(r)} ~\{        \\
  \quad\assignInst{s}{t}   \\
  \}

}{
  \writeInst{}{y}{1}       \\
}
\label{ex:sliB}
\end{equation*}
\end{minipage}
 
As we mentioned in \cref{sec:background:trans}, 
the speculative load introduction can be used 
in combination with the load/load elimination 
to move a load instruction out of one branch of a conditional.
In more detail, giving the example above, 
the speculative load introduction can be applied
to add the load $\readInst{}{t}{y}$ before the $\kw{if}$ statement, 
and then the load/load elimination can be used 
to replace the second load with an assignment. 

The subtle point here is that while the 
left program is race free even under \SC, 
the right program is racy under \SC semantics,
because of the race between load and store to \texttt{y}.
This fact implies that if all accesses in the programs above 
are non-atomic, then a catch-fire semantics should 
treat the right program as having undefined behavior.
In other words, the right program allows any outcome,
while the left program allows only the outcome ${[r=0]}$.
Soundness of program transformation requires 
a set of outcomes of a transformed program 
to be a subset of outcomes of the original program. 
This condition is clearly violated in our example. 

Put simply, speculative load introduction in general
is unsound in catch-fire memory models, 
because it can bring data-races into otherwise 
race-free programs. Since catch-fire semantics
is sensitive to the presence of data-races 
it is incompatible with this transformation. 

Note that this problem cannot be mitigated 
by forbidding only non-atomic load introduction
and allowing atomic load introduction. 
Indeed, an introduced atomic load access still 
can race with some non-atomic load or store
located elsewhere in a program.  
