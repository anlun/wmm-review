\section{Analysis}

In this section we present the detailed comparison 
of the considered programming language memory models. 
We summarize our findings in~\cref{table:summary}.
Each row of the table corresponds to a memory model, denoted by its abbreviation. 
Columns of the table correspond to the properties of the models discussed in~\cref{sec:background}.
We split the properties into several subgroups. 

The first group is devoted to optimality of compilation mappings
to target hardware architectures. In order to be concise, 
we chose the binary classification of optimality, 
that is, we classify the compilation scheme as either optimal or not,
in the following sense.
We chose the weakest possible access mode supported by the model
and consider the compilation scheme for the memory accesses annotated by this mode. 
For the memory models that treat racy non-atomic accesses
as undefined behavior, we consider the compilation mapping
for most relaxed access types the model provides.
This is because the catch-fire semantics for racy non-atomics 
trivially permits the most optimal compilation mappings (see~\cref{sec:bgrnd-ub}).
We say that compilation scheme is \emph{optimal} if the 
accesses annotated by the most relaxed mode 
can be compiled just as plain load and store instructions 
of the given hardware architecture (\ie without memory fences or extra dependencies). 

The second group is dedicated to soundness of various program transformations. 
The classification is also binary: a transformation is either sound or unsound 
in the given memory model (in a sense stated in~\cref{sec:bgrnd-opt-sound}).
Again, to be concise, we do not consider all the combinations 
of program tranformations and memory access modes. 
Instead, we consider the weakest possible accesses which have fully defined semantics. 
We further split the transformations into global and local as in~\cref{sec:bgrnd-opt-sound}.

The third group corresponds to reasoning principles guaranteed by the model. 
It includes the following properties. What kind of DRF guarantee the model provides.
(we distinguish the internal, external and local DRF theorems, see~\cref{sec:bgrnd-drf}).
Whether the model has undefined behaviors (see~\cref{sec:bgrnd-ub}).
Whether the model permits out-of-thin-air values (see~\cref{sec:bgrnd-oota}).

Finally, the last group enumerates the list of memory access modes 
and fences supported by the model, as well as whether the model 
supports read-modify-write operations, locks, and mixed-size accesses.

Driven by our analysis of the models' properties, we partition all models into five classes. 
\app{I think this is the place to name the classes and briefly introduce them.}
The models from the same class have similar compilation mappings, 
set of sound program transformations, and provided reasoning guarantees.
Our classes are ordered by the weakness of the memory models they consist of.  
The strongest models are located at the top rows of the table, 
while the weakest are at the bottom. 

We next discuss each class in more details
(note that the order is different from the one in the~\cref{table:summary}). 
\app{I think we should briefly explain why the order is different.}
We also give some insight on the relationship
between compilation scheme optimality, 
soundness of transformations and reasoning guarantees.
In particular, we explain why the support of some reasoning guarantees 
disables some program transformations and requires more heavyweight 
compilation mappings to hardware.

\subsection{Strong SC-like Models}

Sequential Consistency (\SC) is one of the most intuitive models of concurrency.
Under this model, one can represent the state of the system as 
a simple mapping from memory locations to their values. 
Then each outcome can be obtained by a sequential interleaved execution 
of thread's instructions. Each execution induces a total order 
on individual memory access operations consistent with per-thread program order. 

\SC renders many common transformations unsound, 
including all kind of instruction reorderings and 
common subexpression elimination~\cite{Marino-al:PLDI11, Sevcik-Aspinall:ECOOP08}.
The fact that instruction reorderings are forbidden 
makes the model expensive to implement on modern hardware
since even relatively strong hardware model of x86,
permits store/load reorderings.
Therefore, in order to preserve sequential consistency during compilation,
the compiler need to emit heavyweight memory fences between store and load instructions,
which makes compilation mappings far from being optimal.  

In terms of reasoning guarantees, however, SC is quite a pleasant model. 
It gives the DRF-SC and coherence (\eupp{s/coherence/SC-per-location?}) 
properties for free%
\footnote{The SC semantics assigns to programs only sequantially consistent
outcomes by definition, thus satisfying DRF-SC without any preconditions.
The coherence property is equivalent to sequential consistency per location.
The fact that in SC model any execution is sequeantially consistent implies trivially
that the accesses to each location are also sequeantially consistent.}
and it is naturally program order preserving.
\app{I don't get the last sub-sentence.}

The conceptual simplicity of SC have inspired many researchers 
to adopt it and to try to mitigate the induced performance 
penalty by some additional measures.



\app{As a reader, I'd expect to see the slowdown numbers in the table in some
unified manner.}

Given the results of performance evaluation above,
one can argue that the cost is worth the prize
of having simple SC-like model.
However, we point out that the two 
of the solutions above~\cite{Singh-al:ISCA12, Marino-al:PLDI10} 
require non-trivial modifications to the 
existing hardware and compilers.
In case of SC-Haskell~\cite{Vollmer-al:PPoPP17}, 
the programmers are obligated to follow 
a strict programming discipline enforced 
by the type system of the language.
In all research papers above \app{Very unclear reference.} the expirements 
of proposed solutions were mainly evaluated on x86 hardware, 
and the impact of enforcing SC on weaker hardware
(ARM, POWER) is less clear.
\app{I think that the current text dictates that we have to say smth stronger: not studied.
However, it is not true. Why don't we discuss Liu-al-PLDI19 here?}
Moreover, while the reported performance overhead 
is relatively small on average, 
it is more significant for particular 
kind of programs that heavily utilize shared 
mutable memory (like lock-free data structures).
\eupp{check the evaluation in the papers once again
to support this claim}

\subsection{Strong TSO-like Models}

The next class of PL memory models we consider 
was inspired by TSO~\cite{Sewell-al:CACM10} and PSO~\cite{Sparc:94} 
hardware models. In these models, threads usually 
are equipped with \emph{store buffers}.
All store operations go to these buffers before they 
propogate into the main memory.  
In essence, store buffers enable 
store/load reordering (in case of~TSO),
and store/load \& store/store reordegins (in case of~PSO).

The models based on store buffers idea 
\app{Something is wrong w/ phrasing}
can be compiled down to x86 hardware without any 
performance penalty, since x86 implements TSO model itself.
That is, the compilation mappings to x86 are optimal.
However, when compiled down to weaker hardware (e.g. ARM, POWER)
the compiler indeed needs to take additional measures 
to enforce TSO/PSO like memory model.
\eupp{Perhaps, we can cite some paper here?} 

The TSO/PSO models are weaker than SC, while 
they are still relatively strong.
The external DRF-SC and coherence still hold
and program order is preserved.

\subsection{OOTA Models}

We next move on to the other end of the memory models' spectrum. 
We consider the class uniting the weakest models of our analysis.
These models enable efficient compilation mappings and 
many program transformations, but at the cost of 
introducing thin-air values (see \cref{sec:bgrnd-oota}).
Hence we name this class of models as Out-of-Thin-Air (OOTA) models. 
 
The main common disadvantage of OOTA models is that 
they lack many fundamental reasoning 
principles~\cite{Boehm-Demsky:MSPC14, Batty-al:ESOP15}:
type safety and security gurantess can be violated, 
compositional reasoning is impossible, and
the external DRF-SC property cannot be established. 
The odd consequences of thin-air values manifest 
itself best on the following classical example~\cite{Boehm-Demsky:MSPC14}: 

\begin{equation*}
\inarrII{
  \readInst{}{r_1}{x}      \\
  \kw{if} {(r_1)} ~\{      \\
  \quad\writeInst{}{y}{1}  \\
  \}
}{
  \readInst{}{r_2}{x}      \\
  \kw{if} {(r_2)} ~\{      \\
  \quad\writeInst{}{x}{1}  \\
  \}
}
\tag{LB+ctrl}\label{ex:lb+ctrl}
\end{equation*}

For the memory model admitting thin-air values 
(as \eg \CMM~\cite{Batty-al:POPL11}), 
the outcome $[r_1=1, r_2=1]$ is perfectly valid
(one can see that the program above is analogous 
to the \ref{ex:lb+data}, except it has 
control dependencies between instructions 
in place of data dependencies).
Not only this outcome is completely unintuitive,
but it also contradicts to the external DRF-SC guarantee.
Indeed, in SC model the program above has 
a single valid execution with the outcome $[r_1=0, r_2=0]$ 
and no data-races, thus under DRF-SC compliant model 
it should also has this sole outcome.  

\subsection{$\lPO\cup\lRF$ Acyclic Models}
\label{sec:porf-acyc}

Counter-intuitive behavior of OOTA models, together with the fact that they break 
many important reasoning principles (external DRF-SC among them), 
has lead over the time to the consensus in the research community that these models 
are not suited well for the role of 
programming languages memory models~\cite{Boehm-Demsky:MSPC14, Batty-al:ESOP15}.
A lot of effort has been put to forbid problematic 
thin-air outcomes, while still keep compilation scheme as efficient as possible
and enable as many transformations as possible.

The most straightforward way to forbid thin-air values 
was proposed by Boehm and Demsky~\cite{Boehm-Demsky:MSPC14}
The idea is to simply prohibit any kind of speculative execution, 
which is equivalent to forbidding load/store reorderings altogether. 
This fix not only restores external DRF-SC~\cite{Lahav-al:PLDI17}
and other reasoning guarantees, but also leads to 
a much simpler mental model.  

Lahav~\etal~\cite{Lahav-al:PLDI17} formalized this approach and 
studied it extensively. They proposed a modified version 
of \CPP model called \RCMM (repaired \CMM).  
Besides the strengthening of the model to preserve 
the order between load/store pairs, 
the repaired version also corrects the semantics 
of sequentially-consistent accesses.

The authors have shown that many 
program transformations are still sound in \RCMM, 
with the obvious exception of load/store reordering itself
(see~\cref{table:summary} for details).
Also, the compilation mappings to x86 remain efficient, 
since the architecture already guarantee to preserve the order 
between loads and subsequent stores. 
However, weaker architectures (ARM, POWER) do not guarantee that, 
and thus additional measures are required.
Lahav~\etal~\cite{Lahav-al:PLDI17} proposed to compile relaxed load 
as plain load followed by a spurious conditional branch,
which introduces fake control dependency between 
the load and subsequent stores. 
ARM and POWER hardware preserves dependencies, 
and thus it has to also retain the load/store ordering. 

Ou and Demsky~\cite{Ou-Demsky:OOPSLA18} have studied 
the performance penalty needed to guarantee 
\RCMM memory model on ARMv8 hardware.
They modified the LLVM compiler framework 
to enforce \RCMM memory model
by (1) adjusting the compiler optimization passes and 
(2) changing the compilation mappings.
Several compilation schemes were considered,
among them the one that uses spurious conditional branch
as descibed above has demonstrated the most promising results.  
The authors measured the running time on a set of benchmarks 
implementing various concurrent data-structures
and reported an overhead of 0\% on average and 6.3\% in maximum,
compared to the unmodified version of the compiler. 

\subsection{$\lPPO\cup\lRF$ Acyclic Models}
\label{sec:pporf-acyc}

An alternative conceptually simple solution 
to thin-air values problem is to preserve 
\emph{syntactic dependencies}~\cite{Boehm-Demsky:MSPC14, Alglave-al:ASPLOS18}.
Under this approach reordering of independent load/store pairs is not forbidden.
However, the reordering is forbidden if the store depends on the value 
read by the load either because this value 
was used to compute the value written by the store (\emph{data dependency}), 
or the memory address used in the store (\emph{address dependency}),
or else the control-flow path lead to the store was dependent
on this value (\emph{control dependency}).
For example, giving the program \ref{ex:lb+data} 
the store $\writeInst{}{y}{r_1}$ depends 
on the load $\writeInst{}{x}{r_1}$ since 
it writes the value read by the load.

Note that these kind of dependencies are computed following the 
syntax of the program (hence the name) as opposed 
to \emph{semantic dependencies}.
For example, giving the modified version of the \ref{ex:lb+data} program below, 
the store to \texttt{y} is still considered to be dependent on the previous load. 

\begin{equation*}
\inarrII{
  \readInst{}{r_1}{x}           \\
  \writeInst{}{y}{1 + 0 * r_1}  \\
}{
  \readInst{}{r_2}{x}      \\
  \writeInst{}{x}{r_2}     \\
}
\tag{LB+fakedata}\label{ex:lb+fakedata}
\end{equation*}

Here the syntactic dependency can be eliminated 
by the \emph{constant folding} transformation --- 
the expression $1 + 0 * r_1$ can be reduced to just~$1$.
Under the syntactic dependency preserving memory model 
the compiler, however, is prohibited to perform this optimization. 
Indeed, once the dependency is removed, nothing prevents 
to reorder the store before the preceding load. 
Even if the compiler itself does not perform this reordering,
after the compilation the hardware can do this during the execution.   

This subtlety reveals the main drawback of 
syntactic dependency tracking models --- 
various trace preserving transformations
(\eg constant folding) are unsound in these models. 
Constant folding is one of the basic kind of optimizations that 
any compiler might want to apply, 
and the fact that it is unsound  
makes the adoption of this class of models problematic.

Note that harware models apply a similar approach 
and usually have a notion of dependencies between 
the memory operations~\cite{Sarkar-al:PLDI11, Alglave-al:TOPLAS14, Pulte-al:POPL18}.
Yet in this setting the unsoundness of 
trace preserving transformation is not a problem,
since the hardware does not perform such complex optimizations.

\subsection{no-OOTA Models}
\label{sec:prm-cert}

The last approach to tackle thin-air problem is to   
construct a notion of \emph{semantic dependencies}, 
which would precisely characterize what load/store 
pairs are ``independent'' and rule out 
``fake'' dependencies like the one in \ref{ex:lb+fakedata}.
The practical payoff of this approach is that it 
does not require significant modifications to existing compilers or hardware, 
\app{Here should be some reference to optimality of compilation schemes used in the compilers}
and thus should not impose performance penalties.  
The ultimate goal is to enable optimal compilation mappings, 
preserve most of the existing compiler optimizations, 
and at the same time maintain the important 
reasoning guarantees like external DRF. 

It turns out that this task is quite challenging 
and to this date there is no strong consensus on how to achieve it.
In order to give a satisfactory definition of semantic dependencies 
the researchers had to resort to conceptually complex memory models.
The main challenge in this line of work was to formally prove 
that these complex models indeed satisfy all the desired properties. 

\subsection{Secondary Classes}
\label{sec:other-classes}

\subsubsection{Coherent Models}

The choice of the weaker coherence was deliberate 
with the purpose to enable common subexpression elimination (CSE).
\eupp{Consider to move the details on Coherence vs. CSE
into separate subsection and then leave the reference here}.
The subtle effect of the strong coherence property 
on this classic compiler optimization was first 
observed in the context of \Java 
memory model~\cite{Pugh:JAVA99}.
To see the problem, consider the program below
(on the left) and the transformed version 
of this program after application of CSE (on the right).
Note that the optimization has replaced 
the second access to variable \texttt{x}
by a read from register. 

\begin{minipage}{0.45\linewidth}
\begin{equation*}
\small
\inarrII{
  \readInst{}{r_1}{x}      \\
  \readInst{}{r_2}{y}      \\
  \readInst{}{r_3}{x}      \\
}{
  \writeInst{}{y}{1}       \\
}
\label{ex:coh-rr}
\end{equation*}
\end{minipage}\hfill%
\begin{minipage}{0.05\linewidth}
\Large~\\ $\leadsto$
\end{minipage}\hfill%
\begin{minipage}{0.45\linewidth}
\begin{equation*}
\small
\inarrII{
  \readInst{}{r_1}{x}      \\
  \readInst{}{r_2}{y}      \\
  \assignInst{r_3}{r_1}    \\
}{
  \writeInst{}{y}{1}       \\
}
\label{ex:coh-rr}
\end{equation*}
\end{minipage}

Now assume that \texttt{x} and \texttt{y} point to the same memory location.
Under this assumption the outcome $[r_1=0, r_2=1, r_3=0]$
is forbidden for the memory model respecting coherence.
Indeed, the coherence guarantees sequential consistency per location, 
which means that for the programs consisting of accesses 
to the single memory location 
(as the one above in the presence of aliasing) 
only the sequentially consistent outcomes are allowed.
The outcome $[r_1=0, r_2=1, r_3=0]$ cannot be obtained 
by the interleaving of instructions, and thus 
it should be forbidden.  
However, this outcome is allowed for 
the optimized version of the program. 

\newpage
\onecolumn

\begin{landscape}

\begin{table*}
\begin{tabular}{|c|l|c|c|c|c|c|c|c|c|c|c|c|c|c|c|c|c|c|c|c|c|c|c|c|c|c|c|c|c|c|}
 \hline

                                                      &
 \multirow{3}{*}{Model}                               & 
 \multicolumn{ 4}{c|}{\multirow{2}{*}{Compilation}}   &
 \multicolumn{10}{c|}{Transformations}                &
 \multicolumn{ 6}{c|}{Reasoning}                      &
 \multicolumn{ 9}{c|}{\multirow{2}{*}{Features}}      \\ 

 \cline{7-22}

                             &
                             &
 \multicolumn{4}{c|}{}       &
 \multicolumn{7}{c|}{Local}  &
 \multicolumn{3}{c|}{Global} &

 \multicolumn{3}{c|}{DRF}    &
 \multicolumn{3}{c|}{}       &
 \multicolumn{9}{c|}{}       \\ 
 
 \hline
                                     &
                                     &
 \rotatebox[origin=c]{270}{x86}      & 
 \rotatebox[origin=c]{270}{Power}    & 
 \rotatebox[origin=c]{270}{ARMv7}    & 
 \rotatebox[origin=c]{270}{ARMv8}    & 
 
 \rotatebox[origin=c]{270}{TP}     &
 \rotatebox[origin=c]{270}{RI}     &
 \rotatebox[origin=c]{270}{RE}     &
 \rotatebox[origin=c]{270}{ILE}    &
 \rotatebox[origin=c]{270}{SLI}    &
 \rotatebox[origin=c]{270}{S}      &
 \rotatebox[origin=c]{270}{RM}     &
 \rotatebox[origin=c]{270}{RP}     &
 \rotatebox[origin=c]{270}{VR}     &
 \rotatebox[origin=c]{270}{TI}     &
 
 \rotatebox[origin=c]{270}{Int}    &
 \rotatebox[origin=c]{270}{Ext}    &
 \rotatebox[origin=c]{270}{Loc}    &

 \rotatebox[origin=c]{270}{UB}                                 &
 \rotatebox[origin=c]{270}{\makecell{$\lPO\cup\lRF$ \\ acyc.}} & 
 \rotatebox[origin=c]{270}{OOTA}                               &                              


 \rotatebox[origin=c]{270}{NA}                      &
 \rotatebox[origin=c]{270}{RLX}                     &
 \rotatebox[origin=c]{270}{RA}                      &
 \rotatebox[origin=c]{270}{SC}                      &
 \rotatebox[origin=c]{270}{F-RA}                    &
 \rotatebox[origin=c]{270}{F-SC}                    &
 \rotatebox[origin=c]{270}{RMW}                     &
 \rotatebox[origin=c]{270}{Lock}                    &
 \rotatebox[origin=c]{270}{\makecell{Mix.Sz.}}      \\ 
 
 \Xhline{2\arrayrulewidth}
 
 \multirow{3}{*}{\rotatebox[origin=c]{270}{\makecell{SeqCst}}}   

 & SC             & & & & & & & & & & & & & & & & & & & & & & & & & & & & & \\ \cline{2-31}
 & SC-Haskell     & & & & & & & & & & & & & & & & & & & & & & & & & & & & & \\ \cline{2-31}
 & DRFx           & & & & & & & & & & & & & & & & & & & & & & & & & & & & & \\ \Xhline{2\arrayrulewidth}

 \multirow{2}{*}{\rotatebox[origin=c]{270}{\makecell{TSO\\PSO}}}   

 & BMM            & & & & & & & & & & & & & & & & & & & & & & & & & & & & & \\ \cline{2-31}

 & Rlx Op.Sem.    & & & & & & & & & & & & & & & & & & & & & & & & & & & & & \\ \Xhline{2\arrayrulewidth}

 \multirow{5}{*}{\rotatebox[origin=c]{270}{\makecell{$\lPO\lRF$\\acyc}}}   

 & RC11           & & & & & & & & & & & & & & & & & & & & & & & & & & & & & \\ \cline{2-31}

 & ORC11          & & & & & & & & & & & & & & & & & & & & & & & & & & & & & \\ \cline{2-31}

 & OCaml MM       & & & & & & & & & & & & & & & & & & & & & & & & & & & & & \\ \cline{2-31}

 & JAM            & & & & & & & & & & & & & & & & & & & & & & & & & & & & & \\ \cline{2-31}

 & Rlx Compos.    & & & & & & & & & & & & & & & & & & & & & & & & & & & & & \\ \Xhline{2\arrayrulewidth}

 \multirow{2}{*}{\rotatebox[origin=c]{270}{\makecell{$\lPPO\lRF$\\acyc}}}   

 & LKMM           & & & & & & & & & & & & & & & & & & & & & & & & & & & & & \\ \cline{2-31}

 & OHMM           & & & & & & & & & & & & & & & & & & & & & & & & & & & & & \\ \Xhline{2\arrayrulewidth}

 \multirow{7}{*}{\rotatebox[origin=c]{270}{\makecell{no-OOTA}}}   

 & JMM            & & & & & & & & & & & & & & & & & & & & & & & & & & & & & \\ \cline{2-31}

 & Promising      & & & & & & & & & & & & & & & & & & & & & & & & & & & & & \\ \cline{2-31}

 & Weakestmo      & & & & & & & & & & & & & & & & & & & & & & & & & & & & & \\ \cline{2-31}

 & MRD            & & & & & & & & & & & & & & & & & & & & & & & & & & & & & \\ \cline{2-31}

 & P-P/S          & & & & & & & & & & & & & & & & & & & & & & & & & & & & & \\ \cline{2-31}

 & J/R            & & & & & & & & & & & & & & & & & & & & & & & & & & & & & \\ \cline{2-31}

 & Generative     & & & & & & & & & & & & & & & & & & & & & & & & & & & & & \\ \Xhline{2\arrayrulewidth}

 \multirow{5}{*}{\rotatebox[origin=c]{270}{\makecell{OOTA}}}   

 & C11            & & & & & & & & & & & & & & & & & & & & & & & & & & & & & \\ \cline{2-31}

 & JS MM          & & & & & & & & & & & & & & & & & & & & & & & & & & & & & \\ \cline{2-31}

 & RMC            & & & & & & & & & & & & & & & & & & & & & & & & & & & & & \\ \cline{2-31}

 & RAO            & & & & & & & & & & & & & & & & & & & & & & & & & & & & & \\ \cline{2-31}

 & Spec.Comp.     & & & & & & & & & & & & & & & & & & & & & & & & & & & & & \\ \Xhline{2\arrayrulewidth}


\end{tabular}
\caption{
  % \textit{T.P.} --- trace preserving.
  % \textit{R.I.} --- reordering of independent instructions.
  % \textit{R.E.} --- redundunt load/store elimination.
  % \textit{I.L.E.} --- irrelevant load elimination.
  % \textit{S.L.I.} --- speculative load introduction.
  % \textit{S.} --- strengthening.
  % \textit{R.M.} --- roach motel reordering.
  % \textit{R.P.} --- register promotion.
  % \textit{V.R.} --- value range analysis based optimizations.
  % \textit{T.I.} --- thread inlining (sequentialization).
  % \textit{Int.} --- internal.
  % \textit{Ext.} --- external.
  % \textit{Loc.} --- local.
  % \textit{UB} --- undefined behavior.
  % \textit{OOTA} --- out-of-thin air values.
  % \textit{Mix.Sz.} --- mixed-size accesses.
}
\label{table:summary}
\end{table*}

\end{landscape}

\twocolumn
