\section{Discussion}
\label{sec:discussion}

In this section we provide a summary of our findings  
described in the previous sections. 
We briefly compare the design choices and properties 
of various memory models and discuss their prons and cons.
Finally, We highlight the underdeveloped parts of the theoretical
and practical aspects of memory models, suggesting 
a possible directions for future research in the field. 

\subsection{Optimizations versus Complexity} 

Split the models into classes, for each class 
outline what reasoning guarantess are given and  
how the optimizations are affected.

\begin{itemize}
  \item strong SC-like models (mention SC-haskell)
  \item $\lPO \cup \lRF$ acyclic cyclic models
  \item $\lPPO \cup \lRF$ acyclic cyclic (?) (mention LKMM)
  \item $\lPO \cup \lRF$ cyclic models
  \item OOTA models
\end{itemize}

Mention the consequences of allowing undefined behavior.

On relationship between speculative optimizations and DRF (?)
(i.e. load introduction is unsound in models sensitive to DRF).

On relationship between trace preserving opts. and dependency tracking (?)

Support from PL (i.e. through type system) to mitigate performance problems 
(i.e. by separating local/shared objects statically). 
Examples: SC-haskell, OCaml (?), Rust can do this too. 

\subsection{Memory Model Style (?)}

Operational, axiomatic, denotational. 
Pros and cons of different styls w.r.t. memory models. 

\begin{itemize}
  \item Axiomatic. \\
    Easier to add features (?). 
    The most evolved models (in terms of \#features) are axiomatic.
    (exception: Promising, but even it has problems with SC).
  \item Denotational.
    The conventional way to prove soundness of optimizations in PL theory (?). 
    Could enable local reasoning (?).
    Yet today underdeveloped for weak memory models (only a couple of papers).
\end{itemize}

\subsection{Hints for PL Designers}
 
\begin{itemize}
  \item Separation between local/shared objects can mitigate
        performance slowdown due to strong memory model.
  \item Undefined behavior gives an escape hatch for optimizations.
  \item $\lPO \cup \lRF$ acyclicity gives (relatively) simple 
        cognitive model, enables almost all desired optimizations, efficient on x86.
  \item SC accesses (when combined with other types of accesses) behave in unituitive way (?).
  \item Theory of mixed-size accesses is underdeveloped (?)
        (i.e. it's good idea to forbid them in the language).
\end{itemize}

\subsection{Future Work}

\begin{itemize}
  \item Mixed-size accesses.
  \item Global optimization. 
        Sequentialization (noOOTA model + seq-on is an open problem).
  \item Local DRF. Local reasoning in general (denotational models?).
  \item Common ground between no-OOTA models (?). 
  \item Performance comparison. Quantitative study in general.
\end{itemize}