\section{Критерии сравнения моделей памяти}
\label{sec:background}

В этом разделе мы более подробно рассмотрим 
критерии сравнения моделей памяти языков программирования:
 оптимальность схем компиляции \ref{item:criteria:opt-comp},
корректность трансформаций \ref{item:criteria:sound-trans}
и предоставляемые гарантии \ref{item:criteria:reasoning}.
Эти критерии непосредственно связаны с примитивами, 
предоставляемыми абстракцией разделяемой памяти. 
Таким образом сперва нам необходимо ввести эти примитивы. 

\paragraph{Программные примитивы.}
\label{sec:background:primitives}

Модель памяти определяет семантику разделяемой памяти программы
при наличии  параллельно исполняемых потоков. 
Разделяемая память состоит из переменных, 
каждая из которых имеет уникальный адрес.% 
\footnote{В этой статье мы будем использовать 
термины "адрес" и "локация" как взаимозаменяемые.}
Потоки могут обращаться к этим переменным, 
выполняя операции чтения и записи. 

% \begin{figure}[t]
% \[\def\arraystretch{1.2}
%   \begin{array}{ll} 
%     \readInst{o}{r}{x}                  & \text{Load}                   \\ 
%     \writeInst{o}{x}{v}                 & \text{Store}                  \\ 
%     \readArrayInst{o}{r}{x}{i}{j}       & \text{Mixed Size Load}        \\ 
%     \writeArrayInst{o}{x}{v}{i}{j}      & \text{Mixed Size Store}       \\ 
%     \casInst{o}{r}{x}{v_e}{v_d}         & \text{Compare-and-Swap}        \\ 
%     \faddInst{o}{r}{x}{v}               & \text{Fetch-and-Add}          \\ 
%     \lockInst{l}                        & \text{Lock}                   \\ 
%     \unlockInst{l}                      & \text{Unlock}                 \\ 
%     \fenceInst{o}                       & \text{Fence}                  \\ 
%     x \in \mathsf{Var}                  & \text{Shared Variables}       \\ 
%     r \in \mathsf{Reg}                  & \text{Thread-Local Registers} \\ 
%     v \in \mathsf{Val}                  & \text{Values}                 \\ 
%     l \in \mathsf{Lock}                 & \text{Locks}                  \\ 
%     i,j \in \mathsf{Index}              & \text{Array Indices}          \\ 
%     o \in \set{\na,\rlx,\acqrel,\sco}   & \text{Access Modes}           \\ 
%   \end{array}
% \]
% \caption{Primitives of relaxed memory models}
% \label{fig:wmm-abs}
% \end{figure}

В большинстве языков программирования различаются 
\emph{неатомарные} (\emph{обычные})
и \emph{атомарные} переменные.
Первые не должны использоваться для  
обращений из различных параллельно исполняемых потоков программы. 
В зависимости от конкретного языка программирования
параллельные обращения к неатомарным переменным 
либо полностью запрещены системой типов
(например, в  \Haskell~\cite{Marlow-al:Haskell10, Vollmer-al:PPoPP17} и \Rust~\cite{RustBook:19}), 
либо имеют неопределенное поведение (например, в  \CPP~\cite{Boehm-Adve:PLDI08, Batty-al:POPL11}), 
либо обладают очень слабой семантикой,
не предоставляющей гарантий о порядке,
в котором  потоки могут наблюдать эти обращения
(например, в \Java~\cite{Manson-al:POPL05}). 

В свою очередь, атомарные переменные непосредственно 
предназначены для параллельных обращений. 
Некоторые модели памяти также вводят 
несколько типов обращений к атомарным переменным,  аннотируя их 
 \emph{режимом доступа} (\emph{access mode}).
Например, язык \CPP (и более новая версия \Java~\cite{Bender-Palsberg:OOPSLA19})
имеет четыре следующих режима: ослабленный (\emph{relaxed}
или \emph{opaque} в терминологии \Java),
режимы захвата и освобождения (\emph{acquire/release}), 
последовательно согласованный режим (\emph{sequentially consistent}
или  \emph{volatile} в \Java). 
Эти режимы обозначаются как $\rlx$, $\acq$, $\rel$ и $\sco$ соответственно.
Заметим, что режим $\acq$ может быть применен только к операциям чтения,
а режим $\rel$ --- только к операциям записи.
Неатомарные обращения иногда рассматриваются как дополнительный режим $\na$, 
однако  одновременное использование атомарных 
и неатомарных обращений к одной и той же переменной 
влечет неопределенное поведение в программы на \CPP.

Режимы обращения упорядочены по гарантиям, 
которые они предоставляют, как показано на следующей диаграмме. 

% $$ \na \sqsubseteq \rlx \sqsubseteq \acqrel \sqsubseteq \sco $$

 \[\inarr{
 \begin{tikzpicture}[yscale=0.6,xscale=1.0]
   \node (rlx) at (-1.3, 0) {$\rlx$};
   \node (rel) at (0, 1) {$\rel$};
   \node (acq) at (0,-1) {$\acq$};
   % \node (acqrel) at (1.5,0) {$\acqrel$};
   \node (sc) at (1.3,0) {$\sco$};

   \path[->] (rlx) edge[line width=0.742mm] node[fill=white, anchor=center, pos=0.5] {\rotatebox[origin=c]{45}{$\sqsubset$}} (rel);
   \path[->] (rlx) edge[line width=0.742mm] node[fill=white, anchor=center, pos=0.5] {\rotatebox[origin=c]{-45}{$\sqsubset$}} (acq);
   \path[->] (rel) edge[line width=0.742mm] node[fill=white, anchor=center, pos=0.5] {\rotatebox[origin=c]{-45}{$\sqsubset$}} (sc);
   \path[->] (acq) edge[line width=0.742mm] node[fill=white, anchor=center, pos=0.5] {\rotatebox[origin=c]{45}{$\sqsubset$}} (sc);
   % \path[->] (acqrel) edge[line width=0.742mm] node[fill=white, anchor=center, pos=0.5] {$\sqsubset$} (sc);

 \end{tikzpicture}
 }\]


На одном из концов спектра находятся последовательно согласованные обращения.
При правильном использовании они гарантируют семантику 
последовательной согласованности
(детали этого обсуждаются в разделе \cref{sec:background:drf}).
Неатомарные обращения либо не дают никаких гарантий, 
либо предоставляют минимальные гарантии. 
Ослабленные обращения также имеют слабую семантику, 
тем не менее обычно они предоставляют свойство \emph{когерентности}
(\emph{см.} раздел \cref{sec:background:coh}).
Наконец, в середине находится обращения с режимом захвата/освобождения. 
Они необходимы для поддержи идиомы передачи сообщений~\cite{Lahav-al:POPL16}.
Поток, желающий отправить сообщение, может выполнить операцию освобождающей записи, 
другой поток, ожидающий это сообщение, должен выполнить операцию захватывающего чтения. 
Если операция чтение наблюдает выполненную операцию освобождающей записи, 
тогда два потока синхронизируются.

Модель памяти также может предоставлять атомарные операции 
\emph{чтения-модификации-записи} (\emph{read-modify-write}).
Они включают в себя операции сравнения и замены (\emph{compare-and-swap}), 
атомарного обмена (\emph{exchange}) и разные вариации атомарного инкремента, 
например, \emph{fetch-and-add}, \emph{fetch-and-sub} и т.д. 
Операция сравнения и замены (\CAS) принимает на вход 
адрес разделяемой переменной, ожидаемое и желаемое значение.
Она выполняет чтение из переменной и сравнивает полученное значение 
с ожидаемым. Если они равны, то выполняется замена значения переменной 
на желаемое. Также, прочитанное значение возвращается как результат, 
вне зависимости от успеха проверки. 
Заметим, что описанные выше действия происходят атомарно, 
ни одна другая запись не может произойти между 
чтением и записью операции \CAS.
Операция обмена (\emph{exchange}) атомарно 
заменит значение переменной и возвратит её 
значение до замены. 
Атомарный инкремент (\emph{fetch-and-add} и др.) выполняет
атомарную операцию и возвращает значение переменной до модификации.  

Блокировки (\emph{locks}) иногда рассматриваются 
как часть модели памяти~\cite{Manson-al:POPL05}, 
также как и барьеры (\emph{fence})~\cite{Batty-al:POPL11},
соответствующие инструкциям барьеров памяти, 
которые выполняют процессоры  
(см. раздел \cref{sec:background:compile}). 

Наконец, модель памяти может рассматривать разделяемую память 
не как множество независимых типизированных переменных, 
а как последовательность байт, и допускать 
так называемые \emph{смешанные} (\emph{mixed-size}) 
параллельные обращения~\cite{Flur-al:POPL17}. 
Например, в подобной модели операция чтения восьми байт по определенному адресу
может прочитать значение, записанное двумя параллельными 
операциями записи по четыре байта.

\subsection{Схема компиляции}
\label{sec:background:compile}

Теперь рассмотрим первый критерий~\ref{item:criteria:opt-comp}
сравнения моделей памяти языков программирования --- 
оптимальность схемы компиляции. 
\emph{Схема компиляции} --- это отображение
примитивов языка программирования в инструкции 
конкретного семейства процессоров.  
Мы будем рассматривать примитивы, представленные в 
разделе \cref{sec:background:primitives}.
Архитектура процессора обычно предоставляет 
схожий набор примитивов, который включает 
инструкции для выполнения обычных операций чтения и записи,%  
\footnote{Некоторые архитектуры
предоставляют дополнительные инструкции чтения и 
записи с режимом доступа, \eg 
\eg \texttt{lda} --- захватывающее чтение (load acquire), 
и \texttt{stl} --- освобождающая запись (store release) в \ARMv{8}.} 
операции чтения-модификации-записи, и 
также различные типы барьеров памяти.

Схема компиляции должна быть корректной. 
В контексте данной статьи мы будем подразумевать под 
этим следующее --- множество сценариев поведения допустимых
моделью памяти процессора для скомпилированной программы 
должно являться подмножеством сценариев поведения допустимых 
моделью языка программирования для исходной программы. 

Рассмотрим пример. 
Программа \ref{ex:sb} ниже является 
фрагментом \ref{ex:Dekker} из \cref{sec:intro}.

\begin{equation*}
\inarrII{
   \writeInst{}{x}{1}   \\
   \readInst{}{r_1}{y}  \\
}{
  \writeInst{}{y}{1}   \\
  \readInst{}{r_2}{x}  \\
}
\tag{SB}\label{ex:sb}
\end{equation*}

Предположим, что язык программирования предоставляет 
модель последовательной согласованности, а программа
должна быть скомпилирована для процессоров \Intel. 
Если компилировать чтения и записи как обычные 
инструкции чтения и записи \Intel,\footnote{
В архитектуре \Intel инструкция \texttt{MOV} 
используется как для чтения, так и для записи в память.}
тогда результат работы скомпилированной программы
${[r_1=0, r_2=0]}$ будет допустим спецификацией модели памяти~\Intel
(и может наблюдаться на практике). 
Данный результат может появиться как результат 
\emph{буфферизации записи}
(отсюда и название программы \ref{ex:sb} --- \emph{store buffering}) ---  
запись ${\writeInst{}{x}{1}}$ может быть буфферизирована 
и исполнена после всех остальных инструкций программы. 

Отметим, что результат ${[r_1=0, r_2=0]}$ не является последовательно согласованным
и следовательно рассмотренная схема компиляции не является корректной. 
Как было продемонстрировано в разделе \cref{sec:intro}, 
некорректность схемы компиляции может иметь 
печальные последствия и нарушать корректность программы. 

Корректная схема компиляции для модели последовательной согласованности 
под архитектуру \Intel может компилировать 
запись как инструкцию записи, за которой следует 
инструкция \texttt{mfence}~\cite{Sewell-al:CACM10, Batty-al:POPL11}, 
как продемонстрировано ниже:

\begin{equation*}
\inarrII{
   \writeInst{}{x}{1}   \\
   \mfenceInst          \\
   \readInst{}{r_1}{y}  \\
}{
  \writeInst{}{y}{1}   \\
  \mfenceInst          \\
  \readInst{}{r_2}{x}  \\
}
\tag{SB+MFENCE}\label{ex:sb-mfence}
\end{equation*}

Инструкция \texttt{mfence} это специальный барьер памяти 
архитектуры \Intel, которая выполняет сброс буффера записей в основную память. 
Для программы \ref{ex:sb-mfence} результат ${[r_1=0, r_2=0]}$
запрещен моделью памяти \Intel. 

Несмотря на то, что модифицированная схема компиляции является корректной, 
она не \emph{оптимальна}~\cite{OptimalCompilationCPP}, 
в том смысле что она использует барьеры памяти,
которые обычно влекут замедление программы 
на 10-30\%~\cite{Marino-al:PLDI11, Liu-al:OOPSLA17}
(\emph{см.} \cref{sec:catalog:sc}).
К сожалению, невозможно иметь \emph{корректную и оптимальную} 
схему компиляции модели последовательной согласованности 
под современные процессоры. 
Этот факт делает модель \SC неподходящей 
для высокопроизводительных языков программирования 
и служит одним из стимулов ослабления моделей памяти. 

В рамках этой статьи при обсуждении 
схем компиляции мы будем рассматривать процессоры семейств
\Intel, \ARMv{7}, \ARMv{8} и \POWER 
по двум основным причинам. 
Во-первых, эти архитектуры наиболее 
распространены на сегодняшний день. 
Во-вторых, модели этих процессоров 
всесторонне изучались исследовательским сообществом, 
что привело к созданию строгих 
формальных спецификаций~% 
\cite{Sewell-al:CACM10, Sarkar-al:PLDI11, 
Flur-al:POPL16, Pulte-al:POPL18}. 

\subsection{Трансформации кода}
\label{sec:background:trans}

Следующий критерий~\ref{item:criteria:sound-trans} ---
корректность трансформаций, то есть правил переписывания 
исходного кода, применяемых в ходе оптимизирующих проходов компилятора. 

\emph{Корректная} трансформация должна сохранять семантику программы. 
В нашем контексте, как и в случае корректности схемы компиляции,
это означает, что множество допустимых сценариев поведения 
программы после применения трансформации должно 
быть подмножеством допустимых сценариев поведения оригинальной программы.

Возвращаясь к примеру \ref{ex:sb}
предположим вновь модель последовательной согласованности, 
и рассмотрим трансформацию, которая переставляет
инструкцию записи после инструкции чтения в левом потоке, 
в предположении что эти две инструкции 
оперируют над различными локациями в памяти:

\begin{minipage}{0.45\linewidth}
\begin{equation*}
\inarrII{
   \writeInst{}{x}{1}   \\
   \readInst{}{r_1}{y}  \\
}{
  \writeInst{}{y}{1}   \\
  \readInst{}{r_2}{x}  \\
}
% \tag{SB}\label{ex:sb-src}
\end{equation*}
\end{minipage}\hfill%
\begin{minipage}{0.05\linewidth}
\Large~\\ $\leadsto$
\end{minipage}\hfill%
\begin{minipage}{0.45\linewidth}
\begin{equation*}
\inarrII{
   \readInst{}{r_1}{y}  \\
   \writeInst{}{x}{1}   \\
}{
  \writeInst{}{y}{1}   \\
  \readInst{}{r_2}{x}  \\
}
% \tag{SBtr}\label{ex:sb-tgt}
\end{equation*}
\end{minipage}

Для преобразованной версии программы (справа),
результат $[r_1=0, r_2=0]$ является последовательно согласованным. 
Тем не менее, для оригинальой версии программы (слева) это неверно. 
Следовательно, вышеупомянутая трансформация 
является некорректной для модели \SC. 

Далее мы представим список различных трансформаций,
рассматриваемых в исследованиях по моделям памяти, 
с краткими пояснениям каждой трансформации. 
Заметим, что этот список далеко не полон 
и не включает многие оптимизации 
выполняемые компиляторами~\cite{Muchnick:ACDI97}.
Например, он не включает оптимизации над циклами,
так как в теории моделей памяти ещё недостаточно 
проработаны темы гарантий живучести 
(\emph{liveness properties}), 
которые необходимы для формального 
изучения этих трансформаций. 

Трансформации, которые мы рассматриваем, могут быть 
разделены на два подкласса: \emph{локальные} и \emph{глобальные}.
Локальные трансформации выполняют переписывание 
маленького участка кода в пределах одного потока. 
Для выполнения глобальных транформаций 
необходимо рассматривать всю программу 
(или большую часть программы), 
захватывающую несколько потоков.      
 
\subsubsection{Локальные трансформации}

\paragraph{
Переупорядочивание независимых инструкций
(Reordering of Independent Instructions)
} 

Эта трансформация выполняет перестановку 
двух смежных инструкций, выполняющих обращение к 
различным адресам памяти. 
В зависимости от конкретной пары обращений разделяют 
четыре класса переупорядочиваний:
записи/чтения, записи/записи,
чтения/чтения и чтения/записи. 
%
\[\def\arraystretch{1.4}\footnotesize
  \begin{array}{cccl} 

      \writeInst{}{x}{v} \seq \readInst{}{r}{y} 
    & \leadsto 
    & \readInst{}{r}{y} \seq \writeInst{}{x}{v}
    & \text{store/load}  \\ 

      \writeInst{}{x}{v} \seq \writeInst{}{y}{u} 
    & \leadsto 
    & \writeInst{}{y}{u} \seq \writeInst{}{x}{v}
    & \text{store/store}  \\ 

      \readInst{}{r}{x} \seq \readInst{}{s}{y} 
    & \leadsto 
    & \readInst{}{s}{y} \seq \readInst{}{r}{x}
    & \text{load/load}  \\ 

      \readInst{}{r}{x} \seq \writeInst{}{y}{v} 
    & \leadsto 
    & \writeInst{}{y}{v} \seq \readInst{}{r}{x}
    & \text{load/store}  \\ 

  \end{array}
\]

\paragraph{
Элиминация избыточного обращения
(Elimination of Redundant Access)
} 

В паре двух смежных обращений к памяти
одно из них может быть удалено 
если его эффект покрывается другим. 
Например, две записи в одну и ту же переменную 
одного и того же значения могут быть заменены 
на одну запись. 
Аналогично переупорядочиваниям 
выделяют четыре класса элиминаций
представленных ниже. 
%
\[\def\arraystretch{1.4}\footnotesize
  \begin{array}{cccl} 

      \writeInst{}{x}{v} \seq \readInst{}{r}{x} 
    & \leadsto 
    & \writeInst{}{x}{v} \seq \assignInst{r}{v}
    & \text{store/load}  \\ 

      \readInst{}{r}{x} \seq \readInst{}{s}{x} 
    & \leadsto 
    & \readInst{}{r}{x} \seq \assignInst{s}{r}
    & \text{load/load}  \\ 

      \readInst{}{r}{x} \seq \writeInst{}{x}{r} 
    & \leadsto 
    & \readInst{}{r}{x} 
    & \text{load/store}  \\ 

      \writeInst{}{x}{v} \seq \writeInst{}{x}{u} 
    & \leadsto 
    & \writeInst{}{x}{u}
    & \text{store/store}  \\ 

  \end{array}
\]

\paragraph{
Элиминация нерелевантного чтения
(Irrelevant Load Elimination)
}

Эта трансформация удаляет инструкцию чтения, 
если её результат никогда не используется. 
%
\[\def\arraystretch{1.4}\footnotesize
  \begin{array}{cccl} 

      \readInst{}{r}{x} 
    & \leadsto 
    & \epsInst
    & ~|~ \text{$r$ is never used}  \\ 

  \end{array}
\]

\paragraph{
Введение спекулятивного чтения
(Speculative Load Introduction)
}

Обратная к предыдущей, эта трансформация 
вставляет инструкцию чтения в произвольное место программы.
%
\[\def\arraystretch{1.4}\footnotesize
  \begin{array}{cccl} 

      \epsInst
    & \leadsto 
    & \readInst{}{r}{x} 
    & ~|~ \text{$r$ is never used}  \\ 

  \end{array}
\]

В комбинации с элиминацей чтения/чтения 
эта трансформация может быть использована 
для того, чтобы вынести чтение из 
одной из веток условного оператора:
%
\[\def\arraystretch{1.4}\footnotesize
  \begin{array}{ccc} 

      \kw{if} (e)~ \kw{then} \{ \readInst{}{r}{x} \}
    & \leadsto 
    & \readInst{}{s}{x} \seq \kw{if} (e)~ \kw{then} \{ \assignInst{r}{s} \} \\
    & & ~|~ \text{$s$ is never used}  \\ 

  \end{array}
\]

\paragraph{
(Roach Motel Reordering)
}

Этот класс переупорядочиваний позволяет 
вносить инструкции в блоки синхронизации. 
Например, запись может быть передвинута 
после операции захвата блокировки. 
Интуитивно, такие перестановки могут 
только увеличить синхронизацию в программе, 
то есть преобразованная программа 
должна обладать меньшим недетерминизмом 
и иметь меньшее количество допустимых сценариев поведения. 

Неатомарные обращения могут быть передвинуты 
в критическую секцию без дополнительных предусловий. 
Кроме того, запись может быть перемещена после 
операции захвата блокировки, а чтение 
может быть перемещено до операции освобождения блокировки. 
Схожие правила применяются к переупорядочиваниям вокруг 
захватывающих (\emph{acquire}) и освобождающих (\emph{release}) операций. 
% %
\[\def\arraystretch{1.4}\footnotesize
  \begin{array}{cccl} 

      \readInst{\na}{r}{x} \seq \lockInst{l} 
    & \leadsto 
    & \lockInst{l} \seq \readInst{\na}{r}{x}
    & ~ \\ 

      \writeInst{o}{x}{v} \seq \lockInst{l} 
    & \leadsto 
    & \lockInst{l} \seq \writeInst{o}{x}{v}
    & ~  \\ 

      \unlockInst{l} \seq \writeInst{\na}{x}{v} 
    & \leadsto 
    & \writeInst{\na}{x}{v} \seq \unlockInst{l}
    & ~ \\ 


      \unlockInst{l} \seq \readInst{o}{r}{x} 
    & \leadsto 
    & \readInst{o}{r}{x} \seq \unlockInst{l}
    & ~  \\ 

  \end{array}
\]


\paragraph{
Усиление обращений
(Strengthening)
}

Подобно предыдущей трансформации, 
усиление обращений к памяти 
увеличивает синхронизацию в программе 
путем замены аннотации режима обращения на более строгую.
Например, неатомарное обращение может быть заменено на 
последовательно согласованное: 
%
\[\def\arraystretch{1.4}\footnotesize
  \begin{array}{cccl} 

      \readInst{o}{r}{x} 
    & \leadsto 
    & \readInst{o'}{r}{x}
    & ~|~ o \sqsubset o' \\ 

      \writeInst{o}{x}{v}
    & \leadsto 
    & \writeInst{o'}{x}{v}
    & ~|~ o \sqsubset o'  \\ 

  \end{array}
\]

\paragraph{
Трансформации сохраняющие трассы
(Trace Preserving Transformations)
}

Этот широкий класс включает все трансформации, 
которые не изменяют множество трасс потока~\cite{Sevcik-Aspinall:ECOOP08}.
Трассой называется последовательной видимых побочных эффектов,
производимых во время исполнения кода потока 
(чтения и записи в разделяему память тоже считаются эффектами).
Классический пример подобной трансформации --- 
\emph{распространение констант}%
~\cite{Muchnick:ACDI97, Wegman-Zadeck:TOPLAS91}.
Ниже приведен пример применения данной трансформации: 
%
\[\def\arraystretch{1.4}\footnotesize
  \begin{array}{cccl} 

      \writeInst{}{x}{0 + v} 
    & \leadsto 
    & \writeInst{}{x}{v}
    & \\ 

  \end{array}
\]
  
\paragraph{
Удаление общих подвыражений
(Common Subexpression Elimination)
}

\CSE является ещё одной классической трансформацией~\cite{Muchnick:ACDI97}, 
которая выполняет поиск и удаление идентичных подвыражений.
Пример работы трансформации:
%
\[\def\arraystretch{1.4}\footnotesize
  \begin{array}{cccl} 

      \readInst{}{r_1}{x + y} \seq \readInst{}{r_2}{x + y} 
    & \leadsto 
    & \readInst{}{r_1}{x + y} \seq \readInst{}{r_2}{r_1}
    & \\ 

  \end{array}
\]

\subsubsection{Глобальные трансформации}

\paragraph{
Продвижение регистров
(Register Promotion)
}

Если компилятор может определить, что 
обращения к разделяемой переменной 
происходят только из одного потока,
тогда он может заменить эту переменную на регистр.
%
\[\def\arraystretch{1.4}\footnotesize
  \begin{array}{ccl} 

      \writeInst{}{x}{v} \seq \readInst{}{r}{x} 
    & \leadsto 
    & \assignInst{s}{v} \seq \assignInst{r}{s}
    \\ 
    
    & & |~ \text{\texttt{x} is not accessed from other threads} \\
    & & |~ \text{\texttt{s} is a fresh register} \\ 

  \end{array}
\]

\paragraph{Слияние потоков (Thread Inlining)}

Трансформация, объединяющая два потока в один.
Оказывается что эта на первый взгляд безобидная
трансформация не является корректной во многих моделях памяти. 
%
\[\def\arraystretch{1.4}\footnotesize
  \begin{array}{cccl} 

      P \pll Q 
    & \leadsto 
    & P ~\seq Q
    & ~ \\ 
    
  \end{array}
\]


\paragraph{
Трансформации основанные на анализе диапозона значений
(Value Range Based Transformations)
}

Трансформации этого класса могут быть применены
если программа удовлетворяет некоторому инварианту,
выведенному с помощью глобального анализа 
диапазона возможных значений переменной.
Например, в программе ниже условный оператор
может быть удален, так как статический 
анализ может вывести инвариант 
$\mathsf{x} \geq \mathsf{0}$.

{\footnotesize
\begin{minipage}{0.45\linewidth}
\begin{equation*}
\inarrII{
   \readInst{}{r_1}{x}             \\
   \kw{if} (r_1 \geq 0) ~\kw{then} \\
   \quad\writeInst{}{y}{1}         \\
}{
  \readInst{}{r_2}{x}               \\
  \writeInst{}{y}{r_2}              \\
}
\end{equation*}
\end{minipage}\hfill%
\begin{minipage}{0.05\linewidth}
\Large~\\ $\leadsto$
\end{minipage}\hfill%
\begin{minipage}{0.4\linewidth}
\begin{equation*}
\inarrII{
   \readInst{}{r_1}{x}             \\
   \writeInst{}{y}{1}              \\
}{
  \readInst{}{r_2}{x}               \\
  \writeInst{}{y}{r_2}              \\
}
\end{equation*}
\end{minipage}
}

\subsection{Гарантии}

Далее мы обсуждаем третий критерий~\ref{item:criteria:reasoning} ---
гарантии о поведении программ, предоставляемые моделью памяти.

\subsubsection{\DRF теоремы}
\label{sec:background:drf}

При рассуждении о многопоточном коде 
большинство программистов подразумевают 
модель последовательной согласованности. 
Действительно, было бы неправильно ожидать
от программистов знания всех деталей слабых моделей, 
так как это только усложнило бы и без того 
тяжелую задачу доказательства корректности 
многопоточных программ. 
Свойство \emph{свободы от гонок}
(\emph{data-race freedom})~\cite{Manson-al:POPL05}, 
или кратко \DRF, призвано решить эту проблему. 
Это свойство гарантирует, что правильно 
синхронизированные программы будут иметь 
только последовательно согласованные 
сценарии поведения в слабой модели памяти. 
Другими словами, эта гарантия позволяет программистам 
подразумевать модель последовательной согласованности
если они правильно используют примитивы синхронизации. 

Рассмотрим пример. 
Вернемся к программе \ref{ex:sb} из \cref{sec:background:compile}.
Как было продемонстрировано ранее, в слабой модели 
эта программа может допускать результат ${[r_1=0, r_2=0]}$.
Тем не менее, семантика последовательной согласованности
может быть восстановлена, например, при помощи блокировок, 
как показано в примере ниже:

\begin{equation*}
\inarrII{
   \lockInst{l}         \\
   \writeInst{}{x}{1}   \\
   \readInst{}{r_1}{y}  \\
   \unlockInst{l}       \\
}{
   \lockInst{l}         \\
   \writeInst{}{y}{1}   \\
   \readInst{}{r_2}{x}  \\
   \unlockInst{l}       \\
}
\tag{SB+LOCK}\label{ex:sb-lock}
\end{equation*}

Совместимая с \DRF слабая модель памяти должна гарантировать, 
что для программы выше допустимы только 
последовательно согласованные сценарии поведения с результатами
${[r_1=0, r_2=1]}$, ${[r_1=1,r_2=0]}$, или ${[r_1=1,r_2=1]}$.

Если модель предоставляет последовательно согласованный 
режим доступа, тогда программист также может 
аннотировать все обращения как последовательно согласованные
и таким образом восстановить семантику \SC:
 
\begin{equation*}
\inarrII{
   \writeInst{\sco}{x}{1}   \\
   \readInst{\sco}{r_1}{y}  \\
}{
   \writeInst{\sco}{y}{1}   \\
   \readInst{\sco}{r_2}{x}  \\
}
\tag{SB+SC}\label{ex:sb-sc}
\end{equation*}

Более формально, \DRF свойство для слабой модели $M$
утверждает, что если программа не содержит гонок в модели 
последовательной согласованности, тогда модель $M$
допускает только последовательно согласованные 
сценарии поведения для этой программы.

Свойство \DRF позволяет свести рассуждения о поведении программы 
в слабой модели к рассуждениям в модели последовательной согласованности.
Достаточно лишь показать, что программа не имеет гонок 
в модели \SC, чтобы вывести что она будет иметь только \SC 
сценарии поведения в слабой модели. 

Свойство \DRF в формулировке выше иногда также называется
\emph{внешней свободой от гонок} (\eDRF), 
чтобы отличать её от \emph{внутренней свободы от гонок} (\iDRF).
Последняя гарантирует для программы семантику \SC
в слабой модели $M$ только если программа 
не имеет гонок в \textbf{самой модели $M$}.
Это свойство предоставляет более слабую гарантию
по сравнению с внешней свободой от гонок. 
Оно не позволяет полностью избежать рассуждений 
в терминах слабой модели, так как 
сначала необходимо показать что программа не имеет 
гонок именно в слабой модели. 
Как будет продемонстрировано далее (\emph{см.} \cref{sec:analysis:oota}), 
внутреняя свобода от гонок является компромиссной гарантией 
для определенного класса моделей, которые не могут 
предоставить внешнюю свободу от гонок. 

\subsubsection{Когерентность}
\label{sec:background:coh}

Как было показано раньше, современные процессоры
не предоставляют модель последовательной согласованности. 
Тем не менее, обычно они предоставляют более слабую 
гарантию \emph{последовательной согласованности 
по каждой локации в памяти}, именуемую также 
\emph{когерентностью}~\cite{Alglave-al:TOPLAS14}. 
Следуя за моделями процеесоров, модели 
для языков программирования также зачастую предоставляют эту гарантию.

Когерентность гарантирует, что все записи 
по определенному адресу памяти будут полностью упорядочены,
и что получающийся в результате \emph{порядок когерентности} 
(\emph{coherence order}) отражает порядок 
в котором записи попадают из потоков в основную память.
В частности, из свойства когерентности следует, 
что программа, состоящая из обращений только 
к одной локации в памяти, должна 
обладать семантикой последовательной согласованности.
Например, рассмотрим следующую программу:

\begin{equation*}
\inarrII{
   \writeInst{}{x}{1}   \\
   \readInst{}{r_1}{x}  \\
}{
   \writeInst{}{x}{2}   \\
   \readInst{}{r_2}{x}  \\
}
\tag{COH}\label{ex:coh}
\end{equation*}

Когерентность предписывает модели памяти 
присвоить этой программе только последовательно 
согласованные сценарии поведения с результатами
${[r_1=1, r_2=2]}$, ${[r_1=1, r_2=1]}$, или ${[r_1=2, r_2=2]}$.
Для модели, не удовлетворяющей свойству когерентности, 
допустимым также является результат ${[r_1=2, r_2=1]}$.
Например, модель памяти \Java допускает подобный сценарий поведения~\cite{Manson-al:POPL05}.

\subsubsection{Неопределенное поведени}
\label{sec:background:ub}

Как мы уже кратко упоминали, некоторые модели памяти,
\eg \CPP, рассматривают программы с гонками 
на неатомарных обращениях как имеющие 
\emph{неопределенное поведение}~\cite{Boehm-Adve:PLDI08}.
Другими словами, для таких программ любое поведение считается допустимым. 
Это свойство также иногда называется 
\emph{возгорающейся семантикой} (\emph{catch-fire semantics})
 
Практическая польза такого подхода заключается в том,
что он допускает оптимальную схему компиляции для 
неатомарных обращений и позволяет применять к ним 
любые оптимизации, корректные для последовательных программ.
Дело в том, что эффекты от оптимизаций, производимых 
процессором или компилятором, могут наблюдаться 
только вследствии конкурентных обращений 
из параллельных потоков. Если таким обращениям 
предписывается неопределенное поведение и на них 
не распространяются никакие гарантии, 
то тогда эффекты этих оптимизаций 
становятся неразличимы с точки зрения семантики. 
 
\subsubsection{Спекулятивное исполнение и значения из воздуха}
\label{sec:background:oota}

Чтобы представить последние два свойства, мы рассмотрим ещё один пример:

\begin{equation*}
\inarrII{
  \readInst{}{r_1}{x}     \\
  \writeInst{}{y}{1}      \\
}{
  \readInst{}{r_2}{y}     \\
  \writeInst{}{x}{r_2}    \\
}
\tag{LB}\label{ex:lb}
\end{equation*}

Предположим, что модель памяти допускает 
результат ${[r_1=1, r_2=1]}$ для этой программы. 
Например, модели семейств процессоров 
\ARMv{7}, \ARMv{8}, и \POWER
допускают такой сценарий поведения, 
и он может наблюдаться на некоторых процессорах семейства 
\ARMv{7}~\cite{Maranget-al:Tutorial2012}.

Результат ${[r_1=1, r_2=1]}$ не может быть получен 
путем исполнения инструкции согласно их порядку внутри потоков.
Чтобы получить подобное поведение, 
модели памяти необходимо использовать 
некоторую форму \emph{спекулятивного исполнения}~\cite{Boudol-Petri:ESOP10, Boehm-Demsky:MSPC14}.
То есть во время исполнения чтение $\readInst{}{r_1}{x}$
должно быть буфферизировано, а запись $\writeInst{}{y}{1}$ 
должна выполнится вне очереди 
(отсюда и название программы выше --- 
буфферизация чтения \emph{load buffering}).

Однако неограниченые спекуляции могут привести 
к нежелательным последствиям. 
Запись, исполненная вне очереди, может обернуться
самоисполняющемся пророчеством~\cite{Boehm-Demsky:MSPC14}. 
Рассмотрим следующий вариант программы с буфферизацией чтения:

\begin{equation*}
\small
\inarrII{
  \readInst{}{r_1}{x}   \\
  \writeInst{}{y}{r_1}  \\
}{
  \readInst{}{r_2}{y}   \\
  \writeInst{}{x}{r_2}  \\
}
\tag{LB+data}\label{ex:lb+data}
\end{equation*}

Здесь гипотетическая абстрактная машина 
может спекулятивно исполнить запись в переменную \texttt{y}
значения \texttt{1} в левом потоке, 
затем прочитать это значение в правом потоке, 
записать его в переменную \texttt{x} и прочитать обратно из 
первого потока, таким образом сформировав парадоксальный цикл 
причинно-следственных связей.  
Значение \texttt{1} в примере выше появляется \emph{из воздуха}
(\emph{out of thin-air}) и приводит 
к неожиданному результату ${[r_1=1, r_2=1]}$.

Как будет показано в \cref{sec:analysis},
спекулятивные исполнение необходимо для того, чтобы 
поддержать в модели памяти некоторый класс трансформаций программ. 
Тем не менее, спекулятивное исполнения необходимо 
ограничить должным образом, чтобы избежать 
появление значений из воздуха. 
В \cref{sec:analysis:porf,sec:analysis:deprf,sec:analysis:sdeprf}
мы рассмотрим как различные модели памяти подходят 
к решению этой проблемы. 
