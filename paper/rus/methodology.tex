\section{Методология}
\label{sec:methodology}

Основной задачей нашей работы было изучение
компромиссов в дизайне моделей памяти 
языков программирования. 
Более строгие модели предоставляют больше гарантий программисту,
в то время как более слабые модели предоставляют 
больше возможностей для проведения оптимизаций. 
Мы хотим ответить на следующий вопрос. 

\begin{itemize}
  \item Как гарантии корректного поведения программ, 
    предоставляемые моделью памяти, ограничивают возможности 
    по оптимизации этих программ для потенциальных реализаций
    данного языка программирования?
\end{itemize}

Для сравнения моделей в свете данного вопроса мы использовали стандартные критерии,
встречающиеся в работах по исследованию моделей памяти. 

\begin{enumerate}[label=\textbf{C.\arabic*}]
  
  \item \label{item:criteria:opt-comp}
    \emph{Оптимальность схемы компиляции.}
    Язык с моделью памяти, поддерживающей оптимальные 
    схемы компиляции, может быть эффективно реализован
    на современных процессорах. 
    Напротив, использование неоптимальных схем компиляции
    приводит к замедлению при исполнении программы, 
    но в то же время может предотвратить появление 
    слабых сценариев поведения, допустимых спецификацией данной архитектуры. 

  \item \label{item:criteria:sound-trans}
    \emph{Корректность трансформаций над программным кодом.} 
    При оптимизации компилятор
    применят различные трансформации к исходному коду. 
    Чем больше транформаций допускается моделью памяти языка программирования, 
    тем больше оптимизаций потенциально может применить компилятор 
    к программам на данном языке. 

  \item \label{item:criteria:reasoning}
    Наличие  различных \emph{гарантии для рассуждения о поведении программ}
    позволяет упростить  выполнение доказательств  корректности 
    многопоточных программ  на данном языке. 
  
\end{enumerate}

Чтобы отобрать модели памяти для нашего исследования, 
мы провели следующую процедуру поиска. 
На \emph{первом этапе} мы вручную отобрали 10 рецензированных статей 
предлагающих новые модели памяти~%
\cite{
Manson-al:POPL05,
Batty-al:POPL11,
Lahav-al:PLDI17,
Dolan-al:PLDI18,
Watt-al:PLDI2020,
Jeffrey-Riely:LICS16,
PichonPharabod-Sewell:POPL16,
Kang-al:POPL17,
Chakraborty-Vafeiadis:POPL19,
Paviotti-al:ESOP20
},
которые были представлены на высоко рейтинговых конференциях
в области языков программирования, таких как
``Symposium on Principles of Programming Languages'' (POPL),
``Conference on Programming Language Design and Implementation'' (PLDI), 
и другие. Затем мы взяли список ключевых слов из этих статей. 
Мы исключили ключевые слова, которые были слишком общими
или, наоборот, слишком специфичными. 
В результате мы получили три ключевые фразы: Relaxed Memory Models, Weak Memory Models, Weak Memory Consistency.

На \emph{втором этапе} мы использовали эти фразы как 
поисковые запросы в Google Scholar\footnote{https://scholar.google.com/}.
По каждому запросу мы взяли первые 1000 результатов.%
\footnote{Все поисквые запросы были выполнены 24 сентября 2020 года.}
В итоге мы получили список из 2493 статей. 
Мы удостоверились, что каждая из 10 изначально выбранных статей 
попала в эту выборку. 

На \emph{третьем этапе} мы удалили из выборки дубликаты и нерецензированные статьи. 
Также мы удалили технические отчеты, диссертации, 
публикации не на английском языке и короткие статьи (меньше 4 страниц). 
В результате осталось 1077 статей. 

На \emph{четвертом этапе} мы продолжили отбор статей, изучив 
заголовки и аннотации статей. 
Мы оставили только те статьи, которые напрямую относятся 
к теме моделей памяти языков программирования, 
и, напротив, исключили статьи, которые только используют 
существующие результаты о моделях или статьи, относящиеся 
к смежным темам, таким как
 модели памяти архитектур процессоров, гетерогенных и распределенных систем; семантика транзакций и персистентности; методы верификации программ в контексте слабых моделей памяти.
В результате осталось 105 статей.

На заключительном \emph{пятом этапе} мы изучили содержание оставшихся статей. 
Мы оставили только те статьи, в которых основным результатом была либо новая модель памяти~ЯП, либо изучение/уточнение существующей модели памяти~ЯП.
В итоге осталось 40 статей.