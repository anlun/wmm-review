\documentclass[a4paper,twoside,11pt]{article}
\usepackage[utf8]{inputenc}
\usepackage[english,russian]{babel}
\usepackage{fancyhdr}
\usepackage{newprog1e}

\usepackage{amsfonts,amsmath,amssymb,amsthm}
\usepackage{mathtools}
\usepackage{graphicx}
\usepackage{caption}
\usepackage[table]{xcolor}
\usepackage{tikz}
\usepackage{hyperref}
\usepackage[russian]{cleveref}
\usepackage{sansmath}
\usepackage{xspace}
\usepackage{cite}
\usepackage{multirow}
\usepackage{makecell}
\usepackage{pifont}
\usepackage{hhline}
\usepackage{enumitem}

\usepackage{lscape}
% \usepackage{pdflscape}

\usepackage{longtable}

\usepackage{etoolbox}

\tolerance=1000

\newcommand{\defi}{\stackrel{\mathrm{def}}{=}}

\numberwithin{equation}{section}
\newtheorem{theorem}{Theorem}

\newtheorem{definition}{Definition}

\journalnumber{2}
\curyear{2021}
\authorlist{}
\titlehead{}
\headerdef

\udk{004.421.6}
%92+004.94}
\rubrika{}
\dateinput{17.05.2021}

\rusabstr{}

\author{
{\bfseries Евгений~Моисеенко$^{*, \ddagger}$, Антон~Подкопаев$^{\dagger \ddagger}$, Дмитрий~Кознов$^{*}$}
\\ {\itshape $~^{*}$ Санкт-Петербургский Государственный Университет}
\\ {\slshape 198504} {\itshape Санкт-Петербург, Петергоф, Университетский пр.} {\slshape 28}
\\ {\itshape $~^{\dagger}$ Национальный исследовательский университет «Высшая школа экономики»}
\\ {\slshape 194100} {\itshape Санкт-Петербург, Кантемировская ул.} {\slshape 3} {\itshape b.} {\slshape 1}
\\ {\itshape $~^{\ddagger}$ JetBrains Research}
\\ {\slshape 197342} {\itshape Санкт-Петербург, Кантемировская ул.} {\slshape 2}
\\ {\itshape E-mail: e.moiseenko@2012.spbu.ru, apodkopaev@hse.ru, d.koznov@spbu.ru}}

\title{Модели памяти языков программирования: обзор и тенденции}
%\thanks{~}

\date{}

%% \newcommand{\excl}[1]{{\color{purple} #1}}
\newcommand{\ifext}[2]{\ifdefined\extflag{#1}\else{#2}\fi}

\newcommand{\Reads}{{\sf Reads}}
%% \newcommand{\Reads}{\{ \angled{\tId, \cpath, \loc} \mid \tId, \cpath, \loc \}}

\newcommand{\excl}[1]{}
\newcommand\tT{\mathbf{t}}

\newcommand{\ListOf}[1]{\mathit{List}\;{#1}}
\newcommand{\Label}{\mathit{Label}}

\newcommand\viewf{\textbf{\sf viewf}}

\newcommand{\deltaToView}{\delta\textup{\sf -to-view}}
\newcommand{\deltaHmap}{\textup{\sf comb-time}}
\newcommand{\invDeltaDefOne}{\inv_{\delta\textup{\sf -con-1}}}
\newcommand{\invDeltaDefTwo}{\inv_{\delta\textup{\sf -con-2}}}
\newcommand{\invDeltaDefThree}{\inv_{\delta\textup{\sf -con-3}}}
\newcommand{\invDeltaDefFour}{\inv_{\delta\textup{\sf -con-4}}}

\newcommand{\tIdState}{\textup{\sf thread-state}}
\newcommand{\lastCommittedWrite}{\textup{\sf last-write-com}}

\newcommand\certifiableTid{\mathsf{certifiable}_{\tId}}
\newcommand\certifiable{\mathsf{certifiable}}

\newcommand\lastInstr[1]{#1.\textup{\sf last}}

\newcommand\length{\textup{\sf length}}
\newcommand\prefix{\textup{\sf prefix}}

\newcommand{\ExclReadType}{\mathit{er}}
\newcommand{\ExclWriteType}{\mathit{ew}}
\newcommand{\ComWriteState}{\mathit{im}}

\newcommand\InMemory{im}
\newcommand{\NotInMemory}{\textup{\sf no-mem}}
\newcommand{\IssuedToMemory}{\textup{\sf mem}}
\newcommand{\ExclIssuedToMemory}{\textup{\sf excl-mem}}

\newcommand{\readInst }[3]{#2\;:=_{#1}\;[#3]}
\newcommand{\writeInst}[3]{[#2]\;:=_{#1}\;#3}
\newcommand{\fenceInst}[1]{\fence{#1}}
%% \newcommand{\dmbSY}{\fenceInst{\SY}}
%% \newcommand{\dmbLD}{\fenceInst{\LD}}

\newcommand\kw[1]{\textsf{#1}~}

\newcommand{\writeExclInst}[3]{\textup{\sf atomic-write}(#1, [#2], #3)}
\newcommand{\assignInst}[2]{#1\;:=\;#2}

\newcommand{\restrict}[2]{#1{\restriction_{#2}}}

%% \newcommand{\writeRI}[3][, \ExclType]{#2:#3\excl{, #1}}
\newcommand{\writeRI}[3][]{#2:#3\excl{#1}}

%% \newcommand\ARM{\mathrm{ARM}}
\newcommand\ARMt{\mathrm{ARM}{+}\tau}
%% \newcommand\Promise{\mathrm{Promise}}

\newcommand{\tstamp}[1]{\mbox{\small\color{brown!60!black}\bf{#1}}}

\newcommand{\OrdPrevRequest}{\textup{\sf prev-Ord-req}}

\newcommand{\uniqueTimeLoc}{\textup{\sf uniq-time-loc}}
\newcommand{\ordPrevRequest}{\textup{\sf ord-prev-req}}
\newcommand{\coherentThread}{\textup{\sf coherent-thread}}
\newcommand{\timeRangeCondition}{\textup{\sf time-range}}
\newcommand{\sameMemory}{\textup{\sf same-memory}}
\newcommand{\instToLbl}{\textup{\sf inst-to-lbl}}
\newcommand{\cmdsToLbls}{\textup{\sf cmds-to-lbls}}
\newcommand{\cmdsToLblsAux}{\textup{\sf cmds-to-lbls-aux}}

\newcommand{\prevInstrCommitted}{\textup{\sf prev-instr-committed}}
\newcommand{\prevReadsCommitted}{\textup{\sf prev-reads-committed}}
\newcommand{\prevFencesCommitted}{\textup{\sf prev-fences-committed}}
\newcommand{\prevBrCommitted}{\textup{\sf prev-branches-committed}}
\newcommand{\prevBrFencesCommitted}{\textup{\sf prev-branches-and-fences-committed}}
\newcommand{\prevCmdDetermined}{\textup{\sf prev-fully-determined}}
\newcommand{\prevNoRestart}{\textup{\sf no-prev-restartable-reads-from-loc}}
\newcommand{\prevExclCommitted}{\textup{\sf prev-excl-to-loc-committed}}
\newcommand{\noFollowingWcom}{\textup{\sf no-following-com-writes-to-loc}}

%% \newcommand\reorderableRel{\leftrightarrow} 
%% \newcommand\notReorderableRel{\not \leftrightarrow} 
\newcommand\reorderableRel[2]{#1 \hookrightarrow #2} 
\newcommand\notReorderableRel[2]{#1 \not \hookrightarrow #2} 
%% \newcommand\ReorderingFunction{is\_reorderable}
%% \newcommand{\checkReorderings}[1]{\textup{\sf check-reorderings}(#1)}
\newcommand{\checkReorderings}[1]{#1 \setminus {\reorderableRel{}{}}}

\newcommand{\deleteUpdReads}{\textup{\sf delete-upd-reads}}
\newcommand{\acceptRequest}{\textup{\sf accept-request}}
\newcommand{\acceptExclWrite}{\textup{\sf accept-excl-write}}

\newcommand{\readsBetweenCommitted}{\textup{\sf reads-in-between-committed}}
\newcommand{\noWritesBetween}{\textup{\sf no-writes-to-loc-in-between}}
\newcommand{\noDiffReadsBetween}{\textup{\sf no-different-write-reads-in-between}}
\newcommand{\samePropagated}{\textup{\sf propagated-to-same-threads}}
\newcommand{\fullyPropagated}{\textup{\sf fully-propagated}}
\newcommand{\getNewTapeCell}{\textup{\sf get-new-tapecell}}
\newcommand{\prevReadFromOther}{\textup{\sf prev-read-from-other-write}}
%\newcommand{\nextPath}{\textup{\sf next-path}}
\newcommand{\tapeUpdRestart}{\textup{\sf tape-upd-restart}}
\newcommand{\doesntPreventExcl}{\textup{\sf doesnt-prevent-excl}}
\newcommand{\tapeUpdWcom}{\textup{\sf tape-upd-Wcom}}
\newcommand{\tapeUpdRsat}{\textup{\sf tape-upd-Rsat}}
\newcommand{\tapeUpdIf}{\textup{\sf tape-upd-IfGoto}}
\newcommand{\noExclInBetween}{\textup{\sf no-excl-in-between}}
\newcommand{\getLoc}{\textup{\sf get-loc}}

%% PROBLEMS WITH: \C, \next, \a
  \newcommand\armStepWriteCommit{\armStepgen{\transenv{Write commit} \; \tId \; \cpath \; x \; \stval \; \tau}}
  \newcommand\armStepWriteCommitLoc{\armStepgen{\transenv{Write commit} \; \tId \; \cpath \; \loc \; \stval \; \tau}}
  \newcommand\armStepWriteCommitPrime{\armStepgen{\transenv{Write commit} \; \tId' \; \cpath' \; y \; \stval' \; \tau}}
  \newcommand\armStepWriteCommitP{\armStepPgen{\transenv{Write commit} \; \tId \; \cpath \; x \; \stval}}
  \newcommand\armStepWriteCommitPLoc{\armStepPgen{\transenv{Write commit} \; \tId \; \cpath \; \loc \; \stval}}
  \newcommand\armStepWriteCommitPrimeP{\armStepPgen{\transenv{Write commit} \; \tId' \; \cpath' \; y \; \stval'}}


  \newcommand\promStepBranch{\promStepgen{\transenv{Branch commit} \; \tId}}
  \newcommand\promTStepBranch{\promTStepgen{\transenv{Branch commit}}}

  \newcommand\promStepAcquire{\promStepgen{\transenv{Acquire fence commit} \; \tId}}
  \newcommand\promTStepAcquire{\promTStepgen{\transenv{Acquire fence commit}}}
  \newcommand\promStepRelease{\promStepgen{\transenv{Release fence commit} \; \tId}}
  \newcommand\promTStepRelease{\promTStepgen{\transenv{Release fence commit}}}

  \newcommand\promStepRead{\promStepgen{\transenv{Read from memory} \; \tId \; \writeEvt{x}{\stval}{\tau}{\R}}}
  \newcommand\promTStepRead{\promTStepgen{\transenv{Read from memory} \; \writeEvt{x}{\stval}{\tau}{\R}}}
  \newcommand\promTStepReadLoc{\promTStepgen{\transenv{Read from memory} \; \writeEvt{\loc}{\stval}{\tau}{\R}}}

  \newcommand\promStepPromise{\promStepgen{\transenv{Promise write} \; \tId \; \writeEvt{x}{\stval}{\tau}{\R}}}
  \newcommand\promStepPromiseRPrime{\promStepgen{\transenv{Promise write} \; \tId \; \writeEvt{x}{\stval}{\tau}{\R'}}}
  \newcommand\promTStepPromise{\promTStepgen{\transenv{Promise write} \; \writeEvt{x}{\stval}{\tau}{\R}}}
  \newcommand\promTStepPromiseLoc{\promTStepgen{\transenv{Promise write} \; \writeEvt{\loc}{\stval}{\tau}{\R}}}

  \newcommand\promStepFulfill{\promStepgen{\transenv{Fulfill promise} \; \tId \; \writeEvt{x}{\stval}{\tau}{\R}}}
  \newcommand\promTStepFulfill{\promTStepgen{\transenv{Fulfill promise} \; \writeEvt{x}{\stval}{\tau}{\R}}}
  \newcommand\promTStepFulfillLoc{\promTStepgen{\transenv{Fulfill promise} \; \writeEvt{\loc}{\stval}{\tau}{\R}}}

  \newcommand\promStepAssign{\promStepgen{\transenv{Local variable assignment} \; \tId}}
  \newcommand\promTStepAssign{\promTStepgen{\transenv{Local variable assignment}}}

  \newcommand\promStepNop{\promStepgen{\transenv{Execution of $\nop$} \; \tId}}
  \newcommand\promTStepNop{\promTStepgen{\transenv{Execution of $\nop$}}}

  \newcommand{\tapeToCertificate}{\mathsf{tape}\textsf{-}\mathsf{to}\textsf{-}\mathsf{certificate}}
  %% \newcommand{\Timestamp}{\mathit{Time}}

  \newcommand{\graybox}[1]{$\colorbox{gray!20}{$#1\!$}$}

  %\newcommand\Rarm{\R_{\ARM}}
  \newcommand\Rarm{\mathsf{view}_{\ARM}}
  \newcommand\transenv[1]{\textcolor{darkgray}{\textup{\textsf{\textbf{\mathversion{bold}#1}}}}}

  \newcommand\promMessage{\mathit{msg}}
  \newcommand\promMessageSet{\mathit{Msg}}

  \newcommand\FtypeARM{\mathit{fmod}_{\ARM}}
  %% \newcommand\FtypeProm{\mathit{ftype}_{\Promise}}
  \newcommand\FtypeProm{\mathit{fmod}}

  \newcommand\RtypeProm{\mathit{rtype}_{\Promise}}
  \newcommand\WtypeProm{\mathit{wtype}_{\Promise}}

  %% \newcommand\V{\mathit{V}}
  \newcommand\R{\mathit{view}}
  %% \newcommand\View{\mathit{View}}
  \newcommand\Rsc{\R_{\mathrm{sc}}}
  \newcommand\Rcur{\R_{\mathrm{cur}}}
  \newcommand\Racq{\R_{\mathrm{acq}}}
  \newcommand\Rrel{\R_{\mathrm{rel}}}
  
  \newcommand\lessUpToDelta[3]{#2 <_{#1} #3}

  \newcommand\StmtARM{S}
  \newcommand\StmtProm{S_{\Promise}}


  %% \newcommand\C{C}
  \newcommand\Cf{\mathit{Prog}}%\textsf{Prog}
  \newcommand\Carm{{cmds}}
  \newcommand\CARM{{Cmds}}
  \newcommand\Cfarm{\Cf}
  \newcommand\Cprom{{cmds}}
  \newcommand\Cfprom{\Cf}
  %% \newcommand\Carm{\textsf{C}_{\ARM}}
  %% \newcommand\Cfarm{\Cf_{\ARM}}
  %% \newcommand\Cprom{\textsf{C}_{\Promise}}
  %% \newcommand\Cfprom{\Cf_{\Promise}}
  
  %% \newcommand\PromSet{\mathit{promises}}
  %% \newcommand\PromState{\mathit{st}}

  \newcommand\scAcqRead[2]{#1 := [#2]_{\sf LDAR}}
  \newcommand\scRelWrite[2]{[#1]_{\sf STLR} := #2}

  \newcommand\acqFence{\fence{\sf acquire}}
  \newcommand\relFence{\fence{\sf release}}
  \newcommand\scFence{\fence{\sf sc}}
  \newcommand\fence[1]{\mathsf{fence}({#1})}

  \newcommand\syFence{\fence{\sf sy}}
  \newcommand\ldFence{\fence{\sf ld}}
  
  \newcommand\reqInfoRead[1]{{\sf rd} \; #1}
  \newcommand\reqInfoWrite[2]{{\sf wr} \; #1:#2}
  \newcommand\reqInfoFence{{\sf dmb}}

  \newcommand\stRequest[3]{\angled{#1, #2, #3}}
  \newcommand\stRequestWrite[4]{\angled{#1, #2, \reqInfoWrite{#3}{#4}}}
  \newcommand\stRequestRead[3]{\angled{#1, #2, \reqInfoRead{#3}}}
  \newcommand\stRequestFence[2]{\angled{#1, #2, \reqInfoFence}}

  \newcommand\moTau{\textsf{tedges}}
  \newcommand\invTT{\textsf{inv}}

  % \newcommand\opstau{op\_\loc\_\tau}
  \newcommand\opstau{\textsf{com-writes-time}}
  \newcommand\readsSatisfiedR{\textsf{sat-reads-view}}
  \newcommand\readsCommittedR{\textsf{com-reads-view}}

  \newcommand\hmap{\mathit{H}}
  \newcommand\tmap{\mathit{H}_{\tau}}
  \newcommand\rmap{\mathit{H}_{\mathsf{view}}}
  \newcommand\omap{\mathit{H}_{\le}}

  \newcommand\StateARM{\mathsf{State}_{\ARM}}
  \newcommand\StateARMtau{\mathsf{State}_{\ARMt}}
  \newcommand\StateProm{\mathsf{State}_{\Promise}}
  \newcommand\TStateProm{\mathsf{TState}_{\Promise}}

  \newcommand\armStep{\xrightarrow[\ARMt]{}}
  \newcommand\armStepl{\xrightarrow[\ARMt]{}}
  \newcommand\armStepgen[1]{\xrightarrow[\ARMt]{#1}}
  \newcommand\armStepP{\xrightarrow[\ARM]{}}
  \newcommand\armStepPgen[1]{\xrightarrow[\ARM]{#1}}

  \newcommand\armStepPrSat{\armStepPgen{\transenv{Read satisfy} \; \tId \; \cpath \; \tId' \; \cpath' \; x \; \stval}}
  \newcommand\armStepPrSatLoc{\armStepPgen{\transenv{Read satisfy} \; \tId \; \cpath \; \tId' \; \cpath' \; \loc \; \stval}}
  \newcommand\armStepRSat{\armStepgen{\transenv{Read satisfy} \; \tId \; \cpath \; \tId' \; \cpath' \; x \; \stval}}
  \newcommand\armStepPrSatFail{\armStepPgen{\transenv{Read satisfy (fail)} \; \tId \; \cpath \; \tId' \; \cpath' \; x \; \stval}}
  \newcommand\armStepPrSatFailLoc{\armStepPgen{\transenv{Read satisfy (fail)} \; \tId \; \cpath \; \tId' \; \cpath' \; \loc \; \stval}}
  \newcommand\armStepRSatFail{\armStepgen{\transenv{Read satisfy (fail)} \; \tId \; \cpath \; \tId' \; \cpath' \; x \; \stval}}
  \newcommand\armStepPrInFlightSat{\armStepPgen{\transenv{Read satisfy from in-flight write} \; \tId \; \cpath \; \cpath' \; x \; \stval}}
  \newcommand\armStepPrInFlightSatLoc{\armStepPgen{\transenv{Read satisfy from in-flight write} \; \tId \; \cpath \; \cpath' \; \loc \; \stval}}
  \newcommand\armStepRInFlightSat{\armStepgen{\transenv{Read satisfy from in-flight write} \; \tId \; \cpath \; \cpath' \; x \; \stval}}

  \newcommand\promStep{\xrightarrow[\Promise]{}}
  \newcommand\promStepl{\xrightarrow[\Promise]{}}
  \newcommand\promStepgen[1]{\xrightarrow[\Promise]{#1}}
  \newcommand\promTStepgen[1]{\xrightarrow[\Promise \; \tId]{#1}}
  \newcommand\promTStep{\xrightarrow[\Promise \; \tId]{}}

  \newcommand\semState[2]{|[#1|] ^{#2}}
  \newcommand\semf[2]{|[#1|] ^{#2}}
  \newcommand\semfcom[2]{|[#1|] ^{#2}_\mathsf{com}}

  \newcommand\textIf{\text{\underline{if}} \;}
  \newcommand\textElif{\text{\underline{elif}} \;}
  \newcommand\textElse{\text{\underline{else}} \;}
  \newcommand\textThen{\text{\underline{then}} \;}
  \newcommand\textLet{\text{\underline{let}} \;}
  \newcommand\textIn{\text{\underline{in}}}

  %% \newcommand\nextPathCom[3]{nextPath_{com}(#1, #2, #3)}
  \newcommand\nextPathCom[3]{\mathsf{next}\textsf{-}\mathsf{path}(#1, #2, #3)}
  \newcommand\nextPathProm{\mathsf{next}\textsf{-}\mathsf{path}_{\Promise}}

  \newcommand\comShift{\phantom{{}_\mathsf{com}}}

  \newcommand\extendExpr{\expr\_extend}

  \newcommand\regst{\mathsf{regf}}
  \newcommand\regstcom{\mathsf{regf}_\mathsf{com}}
  \newcommand\regf{\mathit{regf}}
  
  \newcommand\Mpop{\mathit{M}_{\mathrm{POP}}}
  %% \newcommand\Mprom{\mathit{M}_{\mathrm{Promise}}}
  \newcommand\Mprom{\mathit{M}}
  \newcommand{\Mcomp}[3]{\angled{#1, #2, #3}}


  \newcommand\Request{\textit{req}}
  \newcommand\RequestSet{\mathit{ReqSet}}
  \newcommand\RequestInfo{\mathit{reqinfo}}
  \newcommand\RequestInfoSet{\mathit{ReqInfoSet}}
  \newcommand\Evt{\mathit{Evt}}
  \newcommand\Ord{\mathit{Ord}}
  \newcommand\Prop{\mathit{Prop}}
  \newcommand\ExclMap{\mathit{Excl}}

  \newcommand\IssuingOrder{\mathit{iord}}
  \newcommand\IssuingOrderf{\mathit{iordf}}

  \newcommand\Issued[1]{\mathsf{requested} \; #1}
  \newcommand\Satisfied{\mathsf{sat}}
  \newcommand\SatisfiedInFlight{\mathsf{inflight}}
  \newcommand\Committed{\mathsf{com}}
  \newcommand\Plain{\mathsf{pln}}
  \newcommand\Exclusive{\mathsf{excl}}
  \newcommand\None{\mathsf{none}}
  \newcommand\Any{\mathsf{any}}

  %% \newcommand\LD{\mathsf{LD}}
  %% \newcommand\SY{\mathsf{SY}}
  %% \newcommand\ST{\mathsf{ST}}

  \newcommand\tId{\mathit{tid}}
  %% \newcommand\Tid{\mathit{Tid}}

  \newcommand\Tape{\mathit{Tape}}
  \newcommand\tape{\mathit{tape}}
  \newcommand\tapef{\mathit{tapef}}
  \newcommand\TapeCell{\mathit{TapeCell}}
  \newcommand\tapeCell{\mathit{tapecell}}
  \newcommand\Taken{\mathsf{taken}}
  \newcommand\Ignored{\mathsf{ignored}}

  \newcommand\cpath{\mathit{path}}
  \newcommand\cpathSY{\cpath^{\SY}}
  \newcommand\cpathLD{\cpath^{\LD}}
  \newcommand\cpathLDSY{\cpath^{\LD\SY}}
  \newcommand\Path{Path}
  
  \newcommand\tapeRead[1]{\textsf{R} \; #1}
  \newcommand\tapeFence[2]{\textsf{F} \; #1 \; #2}
  \newcommand\tapeWrite[1]{\textsf{W} \; #1}
  \newcommand\tapeIfGoto[2]{\textsf{If} \; #1 \; #2}
  \newcommand\tapeNop{\textsf{Nop}}
  \newcommand\tapeAssign{\textsf{Assign}}

  \newcommand\tapeSatisfied[2]{\Satisfied \; #1 \; #2}
  \newcommand\satisfiedState{\mathit{sat\text{-}state}}

  \newcommand\tapePending[2]{\textsf{pending} \; #1 \; #2}
  \newcommand\tapeWriteCommitted[3]{\Committed \; #1 \; #2 \; #3}

  \newcommand\Fstate{\mathit{st}_{\textup{\rm fence}}}
  \newcommand\Rstate{\mathit{st}_{\textup{\rm read}}}
  \newcommand\Wstate{\mathit{st}_{\textup{\rm write}}}
  \newcommand\IfState{\mathit{st}_{\textup{\rm ifgoto}}}

  \newcommand\locvar{\iota}
  %% \newcommand\loc{\ell}
  %% \newcommand\Loc{Loc}
  \newcommand\Reg{\mathit{Reg}}
  \newcommand\reg{\mathit{reg}}
  %% \newcommand\val{\mathit{val}}
  \newcommand\stval{\mathit{val}}
  \newcommand\Stval{\mathit{Val}}
  \newcommand\expr{\mathit{expr}}
  \newcommand\Expr{\mathit{Expr}}
  \newcommand\z{\mathit{k}}

  \newcommand\dmb{\textsf{dmb}}
  \newcommand\nop{\textsf{nop}}

  \newcommand\ImmediateEdge{ImmediateEdge}
   
  \newcommand\armState[1]{\ARM_{state}(#1)}
  \newcommand\angled[1]{\langle #1 \rangle}
  %% \newcommand\TSfprom{TSf}
  %% \newcommand\TSfprom{TSf_{\Promise}}
  %% \newcommand\TS{TS_{\Promise}}
  %% \newcommand\TSfprom{\textsf{tsf}}
  %% \newcommand\TS{\textsf{ts}}
  %% \newcommand\TSfprom{\mathit{tsf}}
  %% \newcommand\TS{\mathit{ts}}

  %% \newcommand\M{M}
  \newcommand\e{e}
  \newcommand\w{w}
  %% \newcommand\dom[1]{\mathsf{dom}(#1)}
  \newcommand\taumapping{\mathsf{map}_{\tau}}

  \newcommand\Nop{Nop}
  \newcommand\Write[1]{Write \; #1}
  \newcommand\WritePending[1]{\Write(\Pending \; #1)}
  \newcommand\Read[1]{Read \; #1}
  \newcommand\ReadIssued[1]{Read \; (\Issued \; #1)}
  \newcommand\ReadSatisfied[4]{Read \; (\Satisfied \; #1 \; #2 \; #3 \; #4)}
  %% \newcommand\next[2]{next(#1, #2)}
  \newcommand\nextPath[2]{\mathsf{next}\textsf{-}\mathsf{path}(#1, #2)}
  \newcommand\lastIndex{\mathsf{last}\textsf{-}\mathsf{index}}
  \newcommand\last[1]{last(#1)}
  \newcommand\lastSY{\mathsf{last}\SY}
  \newcommand\lastCF{\mathsf{lastCF}}
  \newcommand\lastLD{\mathsf{last}\LD}
  \newcommand\lastLDSY{\mathsf{last}\LD\SY}
  \newcommand\Fence[2]{Fence \; #1 \; #2}
  \newcommand\FenceSY[1]{\Fence{#1}{SY}}
  \newcommand\FenceLD[1]{\Fence{#1}{LD}}
  \newcommand\IfGoto[2]{IfGoto \; #1 \; #2}
  \newcommand\IfGotoK[1]{\IfGoto{#1}{k}}
  \newcommand\tick{✓}
  \newcommand\Certificate{certificate}
  \newcommand\CommandState{instrPlan}
  %% \newcommand\State{State}
  \newcommand\writeEvt[4]{\angled{#1:#2@#3,#4}}
  %% \newcommand\writeEvt[4]{#1:#2@#3,#4}
  \newcommand\simrel{\mathcal{I}}
  \newcommand\simrelPre{\mathcal{I}_{\textup{\rm pre}}}
  \newcommand\simrelBase{\mathcal{I}_{\textup{\rm base}}}
  \newcommand\inv{\mathcal{I}}
  \newcommand\s{\mathbf{s}}
  \newcommand\sfst{\mathbf{s}_0}
  \newcommand\ssnd{\mathbf{s}_1}
  \newcommand\aT{\mathbf{a}}
  \newcommand\afst{\mathbf{a}_0}
  \newcommand\asnd{\mathbf{a}_1}
  \newcommand\p{\mathbf{p}}
  \newcommand\ptid{\mathbf{p}\textsf{-}\tId}
%  \newcommand\ordPlusMO{S_{\Ord \cup mo}}
  \newcommand\ordPlusMO{\textsf{S-Ord-mo}}

  \newcommand\finalStateP{\mathsf{Final}^{\ARM}}
  \newcommand\finalStateA{\mathsf{Final}^{\ARMt}}
  \newcommand\finalStateProm{\mathsf{Final}^{\Promise}}

  \newcommand\sinit{\mathbf{s}^{\rm init}}
  \newcommand\ainit{\mathbf{a}^{\rm init}}
  \newcommand\pinit{\mathbf{p}^{\rm init}}

  \newcommand\invI[1]{\inv_{\textup{\rm #1}}}
  \newcommand\invARM[1]{\inv^\ARM_{\textup{\rm #1}}}

  \newcommand\invTidWriteComCERT{\inv^{\tId}_{\mathsf{w-cert}}}
  \newcommand\invStateCERT{\invI{state-cert}}
  \newcommand\invViewDeltaCERT{\invI{view-cert}}
  \newcommand\invViewWriteCERT{\invI{view-write-cert}}
  \newcommand\invViewRelCERT{\invI{write-rel-cert}}
  \newcommand\invViewReadCERT{\invI{view-read-cert}}
  \newcommand\invWriteTimestampCERT{\invI{write-time-cert}}
  %% \newcommand\invViewCERT{\invI{view-cert}}
  \newcommand\invMemZeroCERT{\invI{mem-1-tid-cert}}
  \newcommand\invMemOneCERT{\invI{mem-1-com-cert}}
  \newcommand\invMemTwoCERT{\invI{mem-2-cert}}
  \newcommand\simrelBaseCERT{\mathcal{I}_{\textup{\rm base-cert}}}

  \newcommand\invCf{\invI{prg}}
  \newcommand\invTId{\invI{tid}}
  \newcommand\invPrefix{\invI{prefix}}
  \newcommand\invSPrefix{\invI{strong prefix}}
  %% \newcommand\invMemOne{\invI{mem1}}
  %% \newcommand\invMemTwo{\invI{mem2}}
  %% \newcommand\invMemThree{\invI{mem3}}
  %% \newcommand\invView{\invI{view}}
  %% \newcommand\invState{\invI{state}}
  \newcommand\invReach{\invI{reach}}
  \newcommand\invComWrite{\invI{com-SY}}
  \newcommand\invCert{\invI{cert}} %\Certificate}}
  \newcommand\invCertTid{\inv_{cert \; \tId}} %\Certificate}}
  \newcommand\simrelTid{\inv_{exec \; \tId}} %\Certificate}}
  \newcommand\invApartialOrderOrd{\inv^{ARM}_{\Ord \; \text{is a partial order}}}
  \newcommand\invAuniqWrite{\invARM{unique write}}
  \newcommand\invAtransClosedOrd{\invARM{\Ord = transitive\_closure(\Ord)}}
  \newcommand\invAimmediateEdge{\invARM{Immediate edge}}
  \newcommand\invAimmediatePath{\invARM{Immediate path}}
  \newcommand\invAmaxPath{\invARM{max \; \cpath}}
  \newcommand\invPromUptoARM{\invI{Promise is up to ARM}}
  \newcommand\invPromUptoARMtId{\inv^{\tId}_{\textup{Promise is up to ARM}}}
  \newcommand\invPromUptoARMnot{\invI{Promise isn't up to ARM}}


  \newcommand\correctStateA{\inv^{\ARMt}_{\textup{\rm correct}}}
  \newcommand\correctStateP{\invARM{correct}}
  \newcommand\invATapeCf{\invARM{tape-Prg}}
  \newcommand\invATapeCfState{\invARM{{tape-Prg-State}}}
  \newcommand\invAReadWrite{\invARM{{Read-Write}}}
  \newcommand\invAReadRead{\invARM{{Read-Read}}}
  \newcommand\invAWriteWriteRead{\invARM{{Write-Write-Read}}}
  \newcommand\invAReadCommittedWrite{\invARM{{Read-Write-Committed}}}
  \newcommand\invAview{\invARM{{View}}}
  \newcommand\invAviewWrite{\invARM{{View-Write}}}
  \newcommand\invAviewRead{\invARM{{View-Read}}}
  \newcommand\invAWriteView{\invARM{{Write-View}}}
  \newcommand\invAnextCommitted{\invARM{{Next-Committed}}}
  \newcommand\invACf{\invARM{Prg}}
  \newcommand\invAtId{\invARM{tid}}
  \newcommand\invAldRead{\invARM{LD-Read}}
  \newcommand\invAtypePreservation{\invARM{{tape-Type}}}
  \newcommand\invAcommittedPreservation{\invARM{{Committed-Preserve}}}
  \newcommand\invAstatePreservation{\invARM{{State-Preserve}}}
  \newcommand\invAstateCom{\invARM{S-Scom}}
  \newcommand\invAcomFences{\invARM{{Committed-Fences}}}
  \newcommand\invAordMOacyclic{\invARM{{Ord-mo-acyclic}}}
  \newcommand\invAordProp{\invARM{{Ord-Prop}}}
  \newcommand\invApropOrd{\invARM{{Prop-Ord}}}
  \newcommand\invAord{\invARM{{Ord-acyclic}}}
  \newcommand\invAevtTape{\invARM{{Evt-tape}}}
  \newcommand\invAtapeEvt{\invARM{{tape-Evt}}}
  \newcommand\invAtapeOrd{\invARM{{tape-Ord}}}
  \newcommand\invAReadWriteOne{\invARM{{Read-Write-1}}}
  \newcommand\invAReadWriteTwo{\invARM{{Read-Write-2}}}

  \newcommand\invPMP{\inv^{\Promise}_{\textsf{M-P}}}
  \newcommand\invPmessageView{\inv^{\Promise}_{\textup{\rm Message-View}}}

  \newcommand\armStepFetch{\overset{fetch \; \tId \; \cpath}{\armStepl}}
  \newcommand\armStepFetchPrime{\overset{fetch \; \tId' \; \cpath'}{\armStepl}}
  \newcommand\armStepProp{\overset{e \rightsquigarrow \tId}{\armStepl}}
  \newcommand\armStepPropPrime{\overset{e' \rightsquigarrow \tId'}{\armStepl}}
  \newcommand\armStepWritePending{\overset{\dashrightarrow \stRequestWrite{\tId}{ \cpath}{ x}{\stval}}{\armStepl}}
  \newcommand\armStepWritePendingPrime{\overset{\dashrightarrow \stRequestWrite{\tId'}{ \cpath'}{ y}{\stval'}}{\armStepl}}

  \newcommand\armStepReadRequest{\overset{\dashrightarrow \stRequestRead{\tId}{ \cpath}{ x}}{\armStepl}}
  \newcommand\armStepReadRequestPrime{\overset{\dashrightarrow \stRequestRead{\tId'}{ \cpath'}{ y}}{\armStepl}}
  \newcommand\armStepCondBranch{\overset{\dashrightarrow \tId, \cpath, \text{choose branch}}{\armStepl}}
  \newcommand\armStepCondBranchPrime{\overset{\dashrightarrow \tId', \cpath', \text{choose branch}}{\armStepl}}
  \newcommand\armStepFenceCommit{\overset{\dashrightarrow \stRequestFence{\tId}{ \cpath}}{\armStepl}}
  \newcommand\armStepFenceCommitPrime{\overset{\dashrightarrow \stRequestFence{\tId'}{ \cpath'}}{\armStepl}}
  \newcommand\armStepReadSatisfy{\overset{\dashrightarrow \stRequestRead{\tId}{ \cpath}{ x}, \stRequestWrite{\tId'}{ \cpath'}{ x}{\stval}}{\armStepl}}
  \newcommand\armStepReadSatisfyPrime{\overset{\dashrightarrow \stRequestRead{\tId'}{ \cpath'}{ y}, \stRequestWrite{\tId''}{ \cpath''}{ y}{\stval'}}{\armStepl}}
  \newcommand\armStepReadSatisfyFail{\overset{\not \dashrightarrow \stRequestRead{\tId}{ \cpath}{ x}, \stRequestWrite{\tId'}{ \cpath'}{ x}{\stval}}{\armStepl}}
  \newcommand\armStepReadSatisfyFailPrime{\overset{\not \dashrightarrow \stRequestRead{\tId'}{ \cpath'}{ y}, \stRequestWrite{\tId''}{ \cpath''}{ y}{\stval'}}{\armStepl}}
  \newcommand\armStepReadSatisfyInF{\overset{\dashrightarrow \stRequestRead{\tId}{ \cpath}{ x}, \stRequestWrite{\tId}{ \cpath'}{ x}{\stval}}{\armStepl}}
  \newcommand\armStepReadSatisfyInFPrime{\overset{\dashrightarrow \stRequestRead{\tId'}{ \cpath'}{ y}, \stRequestWrite{\tId'}{ \cpath''}{ y}{\stval'}}{\armStepl}}
  \newcommand\armStepReadCommit{\overset{\dashrightarrow \tId, \cpath, \text{read commit}}{\armStepl}}
  \newcommand\armStepReadCommitPrime{\overset{\dashrightarrow \tId', \cpath', \text{read commit}}{\armStepl}}

  \newcommand\armStepFetchP{\overset{\text{fetch} \; \tId \; \cpath}{\armStepP}}
  \newcommand\armStepFetchPrimeP{\overset{\text{fetch} \; \tId' \; \cpath'}{\armStepP}}
  \newcommand\armStepPropP{\overset{e \rightsquigarrow \tId}{\armStepP}}
  \newcommand\armStepPropPrimeP{\overset{e' \rightsquigarrow \tId'}{\armStepP}}
  \newcommand\armStepWritePendingP{\overset{\dashrightarrow \stRequestWrite{\tId}{ \cpath}{ x}{\stval}}{\armStepP}}
  \newcommand\armStepWritePendingPrimeP{\overset{\dashrightarrow \stRequestWrite{\tId'}{ \cpath'}{ y}{\stval'}}{\armStepP}}
  \newcommand\armStepReadRequestP{\overset{\dashrightarrow \stRequestRead{\tId}{ \cpath}{ x}}{\armStepP}}
  \newcommand\armStepReadRequestPrimeP{\overset{\dashrightarrow \stRequestRead{\tId'}{ \cpath'}{ y}}{\armStepP}}
  \newcommand\armStepCondBranchP{\overset{\dashrightarrow \tId, \cpath, \text{choose branch}}{\armStepP}}
  \newcommand\armStepCondBranchPrimeP{\overset{\dashrightarrow \tId', \cpath', \text{choose branch}}{\armStepP}}
  \newcommand\armStepFenceCommitP{\overset{\dashrightarrow \stRequestFence{\tId}{ \cpath}}{\armStepP}}
  \newcommand\armStepFenceCommitPrimeP{\overset{\dashrightarrow \stRequestFence{\tId'}{ \cpath'}}{\armStepP}}
  \newcommand\armStepReadSatisfyP{\overset{\dashrightarrow \stRequestRead{\tId}{ \cpath}{ x}, \stRequestWrite{\tId'}{ \cpath'}{ x}{\stval}}{\armStepP}}
  \newcommand\armStepReadSatisfyPrimeP{\overset{\dashrightarrow \stRequestRead{\tId'}{ \cpath'}{ y}, \stRequestWrite{\tId''}{ \cpath''}{ y}{\stval'}}{\armStepP}}
  \newcommand\armStepReadSatisfyFailP{\overset{\not \dashrightarrow \stRequestRead{\tId}{ \cpath}{ x}, \stRequestWrite{\tId'}{ \cpath'}{ x}{\stval}}{\armStepP}}
  \newcommand\armStepReadSatisfyFailPrimeP{\overset{\not \dashrightarrow \stRequestRead{\tId'}{ \cpath'}{ y}, \stRequestWrite{\tId''}{ \cpath''}{ y}{\stval'}}{\armStepP}}
  \newcommand\armStepReadSatisfyInFP{\overset{\dashrightarrow \stRequestRead{\tId}{ \cpath}{ x}, \stRequestWrite{\tId}{ \cpath'}{ x}{\stval}}{\armStepP}}
  \newcommand\armStepReadSatisfyInFPrimeP{\overset{\dashrightarrow \stRequestRead{\tId'}{ \cpath'}{ y}, \stRequestWrite{\tId'}{ \cpath''}{ y}{\stval'}}{\armStepP}}
  \newcommand\armStepReadCommitP{\overset{\dashrightarrow \tId, \cpath, \text{read commit}}{\armStepP}}
  \newcommand\armStepReadCommitPrimeP{\overset{\dashrightarrow \tId', \cpath', \text{read commit}}{\armStepP}}


\newcommand{\event}[3]{#1#2#3}
\tikzset{
   every path/.style={>=stealth},
   po/.style={->,color=brown,,shorten >=-0.5mm,shorten <=-0.5mm},
   rf/.style={->,color=green!60!black,dashed,,shorten >=-0.5mm,shorten <=-0.5mm},
   fr/.style={->,color=red,thick,shorten >=-0.5mm,shorten <=-0.5mm},
   mo/.style={->,color=orange!60!red,dotted,thick,shorten >=-0.5mm,shorten <=-0.5mm},
   no/.style={->,dotted,thick,shorten >=-0.5mm,shorten <=-0.5mm},
   deps/.style={->,color=violet,dotted,thick,shorten >=-0.5mm,shorten <=-0.5mm},
}

\newcommand{\IssuedSet}{I}
\newcommand{\issuable}{{\sf Issuable}}

\newcommand{\comment}[1]{\color{teal}{~~\texttt{/\!\!/}\textit{#1}}}

\newcommand{\lEID}{\lE_{tid}}
\newcommand{\TransSet}{{\rm Transitions}}
\newcommand{\LabelSet}{{\rm Labels}}

\newcommand{\pseudoCompileF}{{\rm pseudo\text{-}compile}}
\newcommand{\pseudoCompileReq}{{\rm pseudo\text{-}compile\text{-}req}}
\newcommand{\compileReq}{{\rm compile\text{-}req}}

\newcommand{\scRel}{{\rm sc\text{-}rel}}
\newcommand{\scRelF}{{\rm sc\text{-}rel\text{-}fun}}
\newcommand{\acqRel}{{\rm acq\text{-}rel}}
\newcommand{\acqRelF}{{\rm acq\text{-}rel\text{-}fun}}
\newcommand{\curRel}{{\rm cur\text{-}rel}}
\newcommand{\curRelF}{{\rm cur\text{-}rel\text{-}fun}}
\newcommand{\relRelFF}{{\rm rel\text{-}rel\text{-}fun}}
\newcommand{\relRel}{{\rm rel\text{-}rel}}
\newcommand{\msgRelF}{{\rm msg\text{-}rel\text{-}fun}}
\newcommand{\msgRel}{{\rm msg\text{-}rel}}

%% listings

\newcommand{\inarrC}[1]{\begin{array}{@{}c@{}}#1\end{array}}
\newcommand{\inpar}[1]{\left(\begin{array}{@{}l@{}}#1\end{array}\right)}
\newcommand{\inset}[1]{\left\{\begin{array}{@{}l@{}}#1\end{array}\right\}}
\newcommand{\inarr}[1]{\begin{array}{@{}l@{}}#1\end{array}}
\newcommand{\inarrII}[2]{\begin{array}{@{}l@{~~}||@{~~}l@{}}\inarr{#1}&\inarr{#2}\end{array}}
\newcommand{\inarrIII}[3]{\begin{array}{@{}l@{~~}||@{~~}l@{~~}||@{~~}l@{}}\inarr{#1}&\inarr{#2}&\inarr{#3}\end{array}}
\newcommand{\inarrIV}[4]{\begin{array}{@{}l@{~~}||@{~~}l@{~~}||@{~~}l@{~~}||@{~~}l@{}}\inarr{#1}&\inarr{#2}&\inarr{#3}&\inarr{#4}\end{array}}
\newcommand{\inarrV}[5]{\begin{array}{@{}l@{~~}||@{~~}l@{~~}||@{~~}l@{~~}||@{~~}l@{~~}||@{~~}l@{}}\inarr{#1}&\inarr{#2}&\inarr{#3}&\inarr{#4}&\inarr{#5}\end{array}}

\renewcommand{\comment}[1]{\color{teal}{~~\texttt{/\!\!/}\textit{#1}}}
\newcommand{\nocomment}[1]{\color{red!60!black}{~~\texttt{/\!\!/}\textit{#1}}}

%% abbrevations

\newcommand{\ie}{\emph{i.e.,} }
\newcommand{\eg}{\emph{e.g.,} }
\newcommand{\etc}{\emph{etc.} }
\newcommand{\sth}{\emph{s.t.} }
\newcommand{\etal}{\emph{et~al.}}
\newcommand{\wrt}{w.r.t.~}
\newcommand{\aka}{a.k.a.~}

\newcommand{\app}[1]{{\color{blue}\textbf{ANTON: #1}}}
\newcommand{\eupp}[1]{{\color{orange!70!black}\textbf{Evgenii: #1}}}
\newcommand{\todo}[1]{{\color{red!70!black}\textbf{TODO: #1}}}

%% memory models

\newcommand{\Java}{Java\xspace}
\newcommand{\CPP}{C/C++\xspace}
\newcommand{\LLVM}{LLVM\xspace}
\newcommand{\OpenCL}{OpenCL\xspace}
\newcommand{\JS}{JavaScript\xspace}
\newcommand{\WASM}{WebAssembly\xspace}
\newcommand{\OCaml}{OCaml\xspace}

\newcommand{\MM}[1]{\ensuremath{\mathsf{#1}}\xspace}
\newcommand{\SC}{\MM{SC}}
\newcommand{\JMM}{\MM{JMM}}
\newcommand{\CMM}{\MM{C11}}
\newcommand{\RCMM}{\MM{RC11}}
\newcommand{\Promising}{\MM{Promising}}
\newcommand{\Weakest}{\MM{Weakest}}
\newcommand{\MRD}{\MM{MRD}}
\newcommand{\OCMM}{\MM{OCamlMM}}
\newcommand{\JSMM}{\MM{JSMM}}

% \captionsetup[table]{name=Table}

\crefformat{section}{\S#2#1#3} 
\crefformat{subsection}{\S#2#1#3}
\crefformat{subsubsection}{\S#2#1#3}

\crefrangeformat{section}{\S#3#1#4-\S#5#2#6}
\crefrangeformat{subsection}{\S#3#1#4-\S#5#2#6}
\crefrangeformat{subsubsection}{\S#3#1#4-\S#5#2#6}


%% abbrevations

\newcommand{\ie}{\emph{т.е.,} }
\newcommand{\eg}{\emph{напр.,} }
\newcommand{\etc}{\emph{итд.} }
\newcommand{\sth}{\emph{т.ч.} }
\newcommand{\etal}{\emph{et~al.} }
% \newcommand{\wrt}{\emph{w.r.t.} }
% \newcommand{\aka}{\emph{a.k.a.} }
\newcommand{\see}{\emph{см.} }

%% rus language
\togglefalse{langeng}

\begin{document}

\maketitle
%\tableofcontents

\renewcommand{\abstractname}{Abstract}

\begin{abstract}

Memory model defines the semantics of 
concurrent programs operating on shared memory.
The main trade-off in the design of a memory model
is between programmability and optimization opportunities.
Stronger models, like sequential consistency, 
provide simpler and more predictable semantics,
but have to abandon certain hardware and compiler optimizations,
while weaker models enable many optimizations 
at the cost of less intuitive semantics.  

In recent years a plenty of memory models for 
various programming languages were proposed, 
which made various compromises with respect 
to programmability versus optimization opportunities.
In this paper we present a survey of these models,
classify them, and discuss their properties and limitations. 

\end{abstract}

\section{Introduction}

A main challenge in concurrent programming is 
to establish a proper synchronization between threads executed in parallel.     
Usually it is done with the help of synchronization primitives
provided by the programming language or libraries,
for example locks, barriers, channels \etc
Sometimes, however, the usage of these primitives
is not possible or undesirable. 
Examples of such cases are the implementation 
of synchronization primitive themselves
or lock-free data structures.
In these cases one has to resort to 
lower-level programming discipline and 
use the mutable shared variables. 
At this point things get complicated.

%The memory model defines which values reads of shared variables can observe at each point of execution. 
%In other words, it defines the semantics of concurrent program.

Let us consider a concrete example.
Here is a simplified version of Dekker's algorithm for mutual exclusion.

\begin{equation*}
\inarrII{
  \writeInst{}{x}{1} \\
  \readInst{}{r_1}{y}  \\
  \kw{if} {r_1 = 0} ~\{ \\
  ~~ \comment{critical section} \\
  \}
}{
  \writeInst{}{y}{1} \\
  \readInst{}{r_2}{x}  \\
  \kw{if} {r_2 = 0} ~\{ \\
  ~~ \comment{critical section} \\
  \}
}
\tag{Dekker}\label{ex:Dekker}
\end{equation*}

In this program, there are two threads that compete to enter critical section.
In order to indicate their intention threads set 
variables $x$ and $y$ correspondingly.%
\footnote{We enclose names of shared variables into square brackets
(\ie $x$, $y$), in order to distinguish them 
from local registers (\ie $r_1$ and $r_2$).}
The one who manages to set the variable first wins.
The algorithm relies on the fact that both threads cannot read value~\texttt{0}%
\footnote{From here and through the rest of the paper we assume that initially all 
variables are initialized with zeros, unless the other is stated explicitly}.
Otherwise, the two threads would have been able 
to enter critical section simultaneously, 
thus breaking the correctness of the algorithm.

Indeed, after running this program on a multi-core system, one would expect to see 
one of the following outcomes: $[r_1=0, r_2=1]$, $[r_1=1,r_2=0]$, and $[r_1=1,r_2=1]$.
These outcomes are \emph{sequential consistent}~\cite{Lamport:TC79} meaning
that they may be obtained by an interleaving of threads' instructions.

%A memory model that admits only these behaviours is known under the name \emph{sequential consistency} (SC) [Lamport:TC79].

%% ANTON: If we have enough space, I'd put a figure w/ the interleavings.)
%% ANTON: IMO, we should more carefully distinguish terms "behavior" and "outcome".
%% ANTON: Also, we should state somewhere that, in the context of concurrent programs, 
%%        we use terms "semantics" and "memory model" interchangeably.

However, not all behaviors which are observable on real concurrent systems are sequentially consistent. 
For example, if one ports the \ref{ex:Dekker} program
from the pseudocode to the C language, compile it with the GCC compiler, 
and run on a processor from the x86/x64 family,
she may observe yet non sequentially consistent outcome $[r_1=0, r_2=0]$,
which is called \emph{weak}.

Weak outcomes appear because of compiler and CPU optimizations.
For example, in the \ref{ex:Dekker} program,
the optimizer may observe that the store to $x$ and the load from $y$ in the left thread
are independent instructions and thus they can be reordered
(this optimization is perfectly valid for single-threaded programs).
For the optimized program, the outcome $[r_1=0, r_2=0]$
is sequentially consistent.

The exact set of allowed outcomes for a given program 
is defined by a semantics of a concurrent system, or a \emph{memory model}.
The memory model permitting only sequentially consistent outcomes 
is called \emph{sequential consistency} (SC).
Memory models admitting weak behaviors are called \emph{weak memory models}.

Neither modern hardware nor programming languages 
guarantee sequential consistency since this model forbids many important optimizations.
The main question then is how \emph{weak} their memory models should be,
\ie how big is a set of allowed weak behaviors for a given program.
A stronger model allows less behaviors, thus giving more guarantees to a programmer
and simplifying reasoning about programs, but a weaker model permits more optimizations,
thus allowing a compiler to produce more efficient code.

It turns out that this question is challenging
especially in the context of programming language (PL) memory models.
Thus over the last two decades a plenty of memory models have been proposed~%
\cite{Manson-al:POPL05, Batty-al:POPL11, Batty-el:POPL16, 
Dolan-al:PLDI18, Watt-el:OOPSLA19, Watt-el:PLDI2020, 
Jeffrey-Riely:LICS16, PichonPharabod-Sewell:POPL16, 
Podkopaev-al:CoRR16, Kang-al:POPL17, Chakraborty-Vafeiadis:POPL19, 
Paviotti-el:ESOP20, Lee-el:PLDI20}. 
These memory models have different design goals, trade-offs, and limitations.
For the people unfamiliar with all the subtleties 
of weak memory models, it can be hard to navigate in this large zoo.
Despite the long history of the field and recent progress made, 
there is no single source that would summare the prior knowledge
and give comprehensive comparison of different memory models
of programming languages.

The aim of this paper is to close this gap.
We provide an overview of the existing approaches to 
formal programming languages memory models,
discuss their design choices, trade-offs, and limitations.
Besides that, we compare the existing memory models 
in terms of what optimizations opportunities 
and what guarantees for formal reasoning they provide.

We hope that our work will be useful to the programming language researches 
who want to dive into the theory of weakly consistent memory models,
and also to the system-level developers, 
who are working on new programming languages, compilers or virtual machines, 
and thus have to choose the memory model for their system.

The rest of the paper is organized as follows.
\todo{}.
% In section [1] we will discuss in more detail the requirements to the programming language memory models.
% On the way we will also look at specification of memory models for the C/C++ and Java languages
% and see why these models do not meet the desired requirements.
% In section [2] we will consider several proposed solutions to fix C/C++ MM. 
% Section [3] contains an overview of memory models for JavaScript/WebAssembly and OCaml languages. 
% Both of these models features some interesting properties that are currently lack in other models.
% In section [4] we compare all of the memory models presented in the paper.
% Finally, section [5] concludes with the discussion and open problems. 

\section{Related Work}

One of the first formally specified memory models 
for shared memory multiprocessors was sequential consistency~\cite{Lamport:TC79}
It was early on realized that sequential consistency 
is too costly to be implemented in the hardware. 
The pioneering work on weak memory models started with the attempt to 
invent a more optimization-friendly memory model~\cite{Adve:PhD93, Adve:Comp96}.
Since then a large amount of effots were put to formally specify weak memory models of various 
hardware architectures~\cite{Chong-ASPLOS08, Alglave-DAMP09, Sewell-al:CACM10, Sarkar-al:PLDI11, Flur-al:POPL16}.
The work~\cite{Alglave-al:TOPLAS14} summarized different studies in the area 
of hardware weak memory models and provides a comprehensive overview of existing models.  
Besides that, it also proposed a general framework for specification, 
testing and verification of hardware weak memory models.

In the context of programming languages, 
it was realized that that multithreading cannot
be implemented as a library~\cite{Boehm:ACM05} (e.g. \texttt{pthreads} library)
without the concise specification of the concurrency semantics on the language level.
This is due to the fact that some compiler optimization 
valid for single threaded programs are not correct in the multithreading enviroment.

After that a lot of work has been done to formally specify 
weak memory models of programming languages.
Memory models have been proposed for the languages
Java~\cite{Manson-al:POPL05}, 
C/C++~\cite{Boehm-Adve:PLDI08, Batty-al:POPL11}, 
OpenCL~\cite{Batty-el:POPL16}, LLVM~\cite{Chakraborty-Vafeiadis:CGO17}, 
OCaml~\cite{Dolan-al:PLDI18}, JavaScript/WebAssembly~\cite{Watt-el:OOPSLA19, Watt-el:PLDI2020}.
%% TODO: recent works on Java opaque accesses (?)

For some of these models it was then shown that they are flawed.
Some compiler optimizations are unsound for Java model~\cite{Sevcik-Aspinall:ECOOP08}, 
that is the model is too strong.
As for C/C++ model it was observed that the model both is too strong 
to permit some compiler optimizations~\cite{Vafeiadis-al:POPL15} and too weak to 
be suitable for any kind of formal or informal reasoning~\cite{Boehm-Demsky:MSPC14}. 
OpenCL model follow closely the C/C++ model and thus inherent some of its problems.
OCaml and WebAssembly/JavaScript have chosen consciously to strengthening their memory models
to overcome problems specific to C/C++ and thus sacrifice performance. 

The problem of C/C++ model being overly weak \app{Some details} has 
been formally stated and studied in~\cite{Batty-al:ESOP15}.
Many of the subsequent works were dedicated 
to strengthen the model ``just enough'' to outlaw undesired behaviours
and yet support efficient compilation and 
common optimizations~\cite{Jeffrey-Riely:LICS16, PichonPharabod-Sewell:POPL16, 
Podkopaev-al:CoRR16, Kang-al:POPL17, Chakraborty-Vafeiadis:POPL19, 
Paviotti-el:ESOP20, Lee-el:PLDI20}. 

We are unaware of any work that would try to give 
a survey of programming language memory models.
Yet with the large number of new models proposed recently,
we argue such work would be of current interest. 

\section{Методология}
\label{sec:methodology}

Основной задачей нашей работы было изучение
компромиссов в  моделях памяти 
языков программирования. Мы хотим ответить на следующий вопрос. 

\begin{itemize}
  \item Как гарантии корректного поведения программ, 
    предоставляемые моделью памяти, ограничивают возможности 
    по оптимизации этих программ?
\end{itemize}

Для сравнения моделей в свете данного вопроса мы использовали стандартные критерии,
встречающиеся в литературе. 

\begin{enumerate}[label=\textbf{C.\arabic*}]
  
  \item \label{item:criteria:opt-comp}
    \emph{Оптимальность схемы компиляции.}
    Язык с моделью памяти, поддерживающей оптимальные 
    схемы компиляции, может быть эффективно реализован
    на современных процессорах. 
    Напротив, использование неоптимальных схем компиляции
    приводит к замедлению при исполнении программы, 
    но в то же время может предотвратить появление 
    слабых сценариев поведения, допустимых спецификацией данной архитектуры. 

  \item \label{item:criteria:sound-trans}
    \emph{Корректность трансформаций над программным кодом.} 
    При оптимизации компилятор
    применят различные трансформации к исходному коду. 
    Чем больше транформаций допускается моделью памяти языка программирования, 
    тем больше оптимизаций потенциально может применить компилятор 
    к программам на данном языке. 

  \item \label{item:criteria:reasoning}
    Наличие  различных \emph{гарантии для рассуждения о поведении программ}
    позволяет упростить  выполнение доказательств  корректности 
    многопоточных программ  на данном языке. 
  
\end{enumerate}

Чтобы отобрать модели памяти для нашего исследования, 
мы провели следующую процедуру поиска. 
На \emph{первом этапе} мы вручную отобрали 10 рецензированных статей 
предлагающих новые модели памяти~%
\cite{
Manson-al:POPL05,
Batty-al:POPL11,
Lahav-al:PLDI17,
Dolan-al:PLDI18,
Watt-al:PLDI2020,
Jeffrey-Riely:LICS16,
PichonPharabod-Sewell:POPL16,
Kang-al:POPL17,
Chakraborty-Vafeiadis:POPL19,
Paviotti-al:ESOP20
},
которые были представлены на высоко рейтинговых конференциях
в области языков программирования, таких как
``Symposium on Principles of Programming Languages'' (POPL),
``Conference on Programming Language Design and Implementation'' (PLDI) 
и др. Затем мы взяли список ключевых слов из этих статей. 
Мы исключили ключевые слова, которые были слишком общими
или, наоборот, слишком специфичными. 
В результате мы получили три ключевые фразы: Relaxed Memory Models, Weak Memory Models, Weak Memory Consistency.

На \emph{втором этапе} мы использовали эти фразы в
качестве поисковых запросов в Google Scholar\footnote{https://scholar.google.com/}.
По каждому запросу мы взяли первые 1000 результатов.%
\footnote{Все поисквые запросы были выполнены 24 сентября 2020 года.}
В итоге мы получили список из 2493 статей. 
Мы удостоверились, что каждая из 10 изначально выбранных статей 
попала в эту выборку. 

На \emph{третьем этапе} мы удалили из выборки дубликаты и нерецензированные статьи. 
Также мы удалили технические отчеты, диссертации, 
публикации не на английском языке и короткие статьи (меньше 4 страниц). 
В результате осталось 1077 статей. 

На \emph{четвертом этапе} мы продолжили отбор статей, изучив 
заголовки и аннотации. 
Мы оставили только те статьи, которые напрямую относятся 
к теме моделей памяти языков программирования, 
и, напротив, исключили статьи, которые только используют 
существующие результаты о моделях или статьи, относящиеся 
к смежным темам, таким как
 модели памяти архитектур процессоров, гетерогенных и распределенных систем; семантика транзакций и персистентности; методы верификации программ в контексте слабых моделей памяти.
В результате осталось 105 статей.

На заключительном \emph{пятом этапе} мы изучили содержание оставшихся статей. 
В итовом списке мы оставили только те статьи, в которых основным результатом была либо новая модель памяти~ЯП, либо изучение/уточнение существующей модели памяти~ЯП.
В итоге осталось 40 статей.
\section{Criteria for Memory Models}
\label{sec:background}

In this section we will have a closer look on criteria for 
programming language memory models, 
namely an optimality of the compilation scheme \ref{item:criteria:opt-comp}, 
a soundness of common program transformations \ref{item:criteria:sound-trans}, 
and provided reasoning guarantees \ref{item:criteria:reasoning}.  
But first we need to introduce  
programming primitives provided 
by the shared memory abstraction. 

\paragraph{Programming Primitives}
\label{sec:background:primitives}


A memory model defines the semantics of the shared memory 
in the presence of concurrently executing threads. 
% More concretly, the memory model 
% defines which values reads of shared variables 
% can observe at each point of execution. 
% Therefore the main abstraction of the memory model 
% is the shared memory itself. 
The shared memory consists of individual variables, 
each having a unique address.\footnote{
Throughout the rest of paper, we use terms 
memory address and memory location interchangeably}
Threads can access these variables by performing 
loads or stores.\footnote{We also use terms 
load/stores and reads/writes interchangeably}

% \begin{figure}[t]
% \[\def\arraystretch{1.2}
%   \begin{array}{ll} 
%     \readInst{o}{r}{x}                  & \text{Load}                   \\ 
%     \writeInst{o}{x}{v}                 & \text{Store}                  \\ 
%     \readArrayInst{o}{r}{x}{i}{j}       & \text{Mixed Size Load}        \\ 
%     \writeArrayInst{o}{x}{v}{i}{j}      & \text{Mixed Size Store}       \\ 
%     \casInst{o}{r}{x}{v_e}{v_d}         & \text{Compare-and-Swap}        \\ 
%     \faddInst{o}{r}{x}{v}               & \text{Fetch-and-Add}          \\ 
%     \lockInst{l}                        & \text{Lock}                   \\ 
%     \unlockInst{l}                      & \text{Unlock}                 \\ 
%     \fenceInst{o}                       & \text{Fence}                  \\ 
%     x \in \mathsf{Var}                  & \text{Shared Variables}       \\ 
%     r \in \mathsf{Reg}                  & \text{Thread-Local Registers} \\ 
%     v \in \mathsf{Val}                  & \text{Values}                 \\ 
%     l \in \mathsf{Lock}                 & \text{Locks}                  \\ 
%     i,j \in \mathsf{Index}              & \text{Array Indices}          \\ 
%     o \in \set{\na,\rlx,\acqrel,\sco}   & \text{Access Modes}           \\ 
%   \end{array}
% \]
% \caption{Primitives of relaxed memory models}
% \label{fig:wmm-abs}
% \end{figure}

Most programming language memory models distinguish 
\emph{non-atomic} (sometimes also called \emph{plain})
and \emph{atomic} variables. 
The former generally should not be accessed 
concurrently from parallel threads. 
Depending on the particular programming language, 
concurrent accesses to non-atomic variables 
can be either prohibited by a type-system 
(\eg \Haskell~\cite{Marlow-al:Haskell10, Vollmer-al:PPoPP17}, \Rust~\cite{RustBook:19}), 
have undefined behavior (\eg \CPP~\cite{Boehm-Adve:PLDI08, Batty-al:POPL11}), 
or being defined but have very weak semantics with almost 
no guarantees on the order in which concurrent
threads can observe these accesses (\eg \Java~\cite{Manson-al:POPL05}).

In turn, atomics are designed for concurrent accesses. 
Some memory models further distinguish 
several kinds of accesses to atomic variables.
In these models the accesses to shared memory are annotated by the 
so-called \emph{access modes}.
For example, the \CPP model (and a later revision of 
the \Java~\cite{Bender-Palsberg:OOPSLA19} model), distinguish 
three modes: \emph{relaxed} (\emph{opaque} in \Java terminology), 
\emph{acquire/release}, and \emph{sequentially consistent}
(\emph{volatile} in \Java).
They are denoted as $\rlx$, $\acq$, $\rel$, and $\sco$ correspondingly.
Note that $\acq$ mode is only applicable to load operations,
while $\rel$ is applicable to store operations.
Non-atomic accesses are often considered to be the fourth access mode $\na$, 
but note that mixing non-atomic and atomic accesses to the same variable 
entails undefined behavior in \CPP.

The access modes are ordered by the guarantees they provide
in exchange of optimization opportunities, as the following 
diagram shows.

% $$ \na \sqsubseteq \rlx \sqsubseteq \acqrel \sqsubseteq \sco $$

 \[\inarr{
 \begin{tikzpicture}[yscale=0.6,xscale=1.0]
   \node (rlx) at (-1.3, 0) {$\rlx$};
   \node (rel) at (0, 1) {$\rel$};
   \node (acq) at (0,-1) {$\acq$};
   % \node (acqrel) at (1.5,0) {$\acqrel$};
   \node (sc) at (1.3,0) {$\sco$};

   \path[->] (rlx) edge[line width=0.742mm] node[fill=white, anchor=center, pos=0.5] {\rotatebox[origin=c]{45}{$\sqsubset$}} (rel);
   \path[->] (rlx) edge[line width=0.742mm] node[fill=white, anchor=center, pos=0.5] {\rotatebox[origin=c]{-45}{$\sqsubset$}} (acq);
   \path[->] (rel) edge[line width=0.742mm] node[fill=white, anchor=center, pos=0.5] {\rotatebox[origin=c]{-45}{$\sqsubset$}} (sc);
   \path[->] (acq) edge[line width=0.742mm] node[fill=white, anchor=center, pos=0.5] {\rotatebox[origin=c]{45}{$\sqsubset$}} (sc);
   % \path[->] (acqrel) edge[line width=0.742mm] node[fill=white, anchor=center, pos=0.5] {$\sqsubset$} (sc);

 \end{tikzpicture}
 }\]


On the one end of the spectrum are sequentially consistent accesses. 
They guarantee to restore the sequentially consistent semantics, 
if used properly (see \cref{sec:background:drf} for details).
Non-atomic accesses, as we have already discussed, give little guarantee. 
Relaxed accesses also have very weak semantics, 
usually they provide only the \emph{coherence} property
(see \cref{sec:background:coh} for details).
Finally, in the middle there are the acquire/release accesses. 
They are designed to support the message passing idiom~\cite{Lahav-al:POPL16}.
A thread sends the message by performing a release write, 
another thread expecting a message can perform an acquire read. 
If the acquire read observes the release write, the two 
threads synchronize their views on shared memory. 

The memory model can also provide atomic \emph{read-modify-write} operations.
These include \emph{compare-and-swap}, \emph{exchange}, and variations of atomic increment,
\eg \emph{fetch-and-add}, \emph{fetch-and-sub}, \etc 
Compare-and-swap (\CAS) operation takes a shared variable, expected 
and desired values. It reads from the variable
and compares the result with the expected value. If the two are equal,
it substitutes the value of the variable to the desired value. 
In either case, the value read from the variable is returned as a result. 
Note that the above operations are guaranteed to be performed atomically, 
no other write to the shared variable can happen in-between 
the read and write parts of \CAS.
Exchange operation atomically replaces the value 
of the variable and returns the previously held value.
Fetch-and-add and similar primitives perform 
the operation (addition, subtraction,~\emph{etc.}~)
atomically and unconditionally, returning 
the content of the shared variable prior to modification.

Locks sometimes considered to be a part 
of a memory model~\cite{Manson-al:POPL05}, 
as well as memory fence operations~\cite{Batty-al:POPL11},
which are related to hardware fence instructions
(see \cref{sec:background:compile} for details). 

Finally, a memory model can treat the shared memory 
not as a set of disjoint typed variables, but rather as 
a raw byte sequence, and permit so-called \emph{mixed-size} 
concurrent accesses~\cite{Flur-al:POPL17}.
For example, in a mixed-size model it is 
allowed for an 8 byte load instruction 
to read from two concurrent adjacent 4 byte stores. 

\subsection{Compilation Scheme}
\label{sec:background:compile}

We next consider the first criterion~\ref{item:criteria:opt-comp}
for programming language memory models---an optimality of a compilation scheme. 
A \emph{compilation scheme} is a mapping of 
programming language primitives into 
instructions of particular hardware architecture. 
In our setting we consider the primitives 
mentioned in \cref{sec:background:primitives}.
The hardware architectures provide a similar set 
of instructions which usually contain 
plain load and stores,\footnote{Some architectures 
also provide various types of load and stores
matching the access modes annotations, 
\eg \texttt{lda} --- load acquire, 
and \texttt{stl} --- store release on \ARMv{8}.} 
read-modify-write operations, 
and also various memory fences. 

A compilation mapping should be \emph{sound}.
In the context of this paper it means that 
a set of outcomes permitted by a 
memory model of hardware for a compiled program 
should be a subset of outcomes permitted by a 
programming language model for the original program. 

Let us consider an example. 
The program \ref{ex:sb} below is a 
fragment of \ref{ex:Dekker} from \cref{sec:intro}.

\begin{equation*}
\inarrII{
   \writeInst{}{x}{1}   \\
   \readInst{}{r_1}{y}  \\
}{
  \writeInst{}{y}{1}   \\
  \readInst{}{r_2}{x}  \\
}
\tag{SB}\label{ex:sb}
\end{equation*}

Assume that the memory model of the programming language
is sequential consistency, and 
it should be compiled to \Intel hardware. 
If one would compile all loads and stores 
to plain load and store instructions of \Intel,\footnote{
On \Intel \texttt{MOV} instruction 
is used as both plain load from memory and 
plain store to memory instructions.}
the outcome $[r_1=0, r_2=0]$ would be 
allowed for the compiled program 
(and it can actually be observed in practice as was stated before), 
since the memory model of \Intel permits this outcome. 
It can be obtained as a result of \emph{store buffering}
optimization (hence the name of the program \ref{ex:sb}). 
The store ${\writeInst{}{x}{1}}$ can be buffered and 
executed after all other instructions of the program.  
Yet clearly the outcome ${[r_1=0, r_2=0]}$ is not sequentially consistent. 
Therefore the proposed compilation scheme, 
which maps all loads and stores to the 
plain load and stores is unsound. 
As we demonstrated in \cref{sec:intro}, 
unsoundness of a compilation scheme has 
dramatic consequences as it may break 
the correctness of a program. 

A sound compilation scheme for the sequential consistency 
can compile a store as a plain store followed 
by the \texttt{mfence} instruction~\cite{Sewell-al:CACM10, Batty-al:POPL11} 
as demonstrated below: 

\begin{equation*}
\inarrII{
   \writeInst{}{x}{1}   \\
   \mfenceInst          \\
   \readInst{}{r_1}{y}  \\
}{
  \writeInst{}{y}{1}   \\
  \mfenceInst          \\
  \readInst{}{r_2}{x}  \\
}
\tag{SB+MFENCE}\label{ex:sb-mfence}
\end{equation*}


The \texttt{mfence} is a special memory fence instruction
of \Intel architecture which flushes a store buffer of a thread. 
For the program \ref{ex:sb-mfence} the outcome $[r_1=0, r_2=0]$
is forbidden by the \Intel memory model. 

Although the modified compilation scheme is sound for \SC, 
it is not \emph{optimal}~\cite{OptimalCompilationCPP}, 
in a sense that it requires to use memory fence instructions, 
which usually induce a performance penalty
of about 10-30\%~\cite{Marino-al:PLDI11, Liu-al:OOPSLA17}
(see \cref{sec:catalog:sc} for details).
Unfortunately, it is not possible to have compilation mapping 
to modern hardware architectures     
for the \SC model which is both \emph{sound and optimal}. 
This fact makes the \SC memory model unsuitable  
for high-performance programming languages
and serves as one of the stimulus for 
weakening of memory models. 
 
In this paper, when speaking about compilation schemes, 
we will consider only the following hardware memory models: 
\Intel, \ARMv{7}, \ARMv{8}, and \POWER, 
for the two main reasons. 
First, these hardware architectures are the 
most widespread today.
Second, they have received a lot of attention 
from the research community recently. 
As a result of this effort, 
there were developed rigorous formal 
specifications of these models~%
\cite{Sewell-al:CACM10, Sarkar-al:PLDI11, 
Flur-al:POPL16, Pulte-al:POPL18}. 

\subsection{Program Transformations}
\label{sec:background:trans}

The next criterion~\ref{item:criteria:sound-trans} 
for memory models is a soundness of program transformations, 
that is source-to-source transformations of code which are applied during 
optimization passes of a compiler. 

\emph{Sound} transformations should preserve the semantics 
of a program. In our context, similarly to the 
soundness of compilation scheme, it means that 
a set of outcomes of a transformed program 
should be a subset of outcomes of the original one. 

Going back to the \ref{ex:sb} example, 
assume the sequential consistency model again and
consider a transformation that reorders 
the store instruction past the following load 
instructions in the left thread, 
assuming the load and store operate on 
the disjoint memory locations:

\begin{minipage}{0.45\linewidth}
\begin{equation*}
\inarrII{
   \writeInst{}{x}{1}   \\
   \readInst{}{r_1}{y}  \\
}{
  \writeInst{}{y}{1}   \\
  \readInst{}{r_2}{x}  \\
}
% \tag{SB}\label{ex:sb-src}
\end{equation*}
\end{minipage}\hfill%
\begin{minipage}{0.05\linewidth}
\Large~\\ $\leadsto$
\end{minipage}\hfill%
\begin{minipage}{0.45\linewidth}
\begin{equation*}
\inarrII{
   \readInst{}{r_1}{y}  \\
   \writeInst{}{x}{1}   \\
}{
  \writeInst{}{y}{1}   \\
  \readInst{}{r_2}{x}  \\
}
% \tag{SBtr}\label{ex:sb-tgt}
\end{equation*}
\end{minipage}

For the transformed version of the program (on the right),
the outcome $[r_1=0, r_2=0]$ is sequentially consistent.
Yet for the original one (on the left) it is not. 
It means that the aforementioned program transformation
is unsound for \SC. 

We next present a comprehensive list of 
various program transformations considered in
the literature on weak memory models 
with a short description of each one.
Note that the list is far from being complete regarding to  
transformations used in optimizing compilers~\cite{Muchnick:ACDI97}.
For example, it lacks common loop optimizations, 
because the theory of relaxed memory models still
struggles with problems of liveness properties, 
needed for studying these transformations formally. 

The transformations we consider can be split into 
two subcategories: \emph{local} and \emph{global}.
Local transformations rewrite a small 
piece of code within a single thread.
Global transformations may need to consider 
a whole program (or a large part of it) 
spanning multiple threads in order 
to perform a rewriting.       
 
\subsubsection{Local Transformations}

\paragraph{Reordering of Independent Instructions} 

This transformation reorders two 
adjacent independent memory accessing instructions
operating on different memory locations.
Depending on a particular pair of instructions
it can be further split into store/load, store/store, 
load/load, and load/store reorderings.  
%
\[\def\arraystretch{1.4}\footnotesize
  \begin{array}{cccl} 

      \writeInst{}{x}{v} \seq \readInst{}{r}{y} 
    & \leadsto 
    & \readInst{}{r}{y} \seq \writeInst{}{x}{v}
    & \text{store/load}  \\ 

      \writeInst{}{x}{v} \seq \writeInst{}{y}{u} 
    & \leadsto 
    & \writeInst{}{y}{u} \seq \writeInst{}{x}{v}
    & \text{store/store}  \\ 

      \readInst{}{r}{x} \seq \readInst{}{s}{y} 
    & \leadsto 
    & \readInst{}{s}{y} \seq \readInst{}{r}{x}
    & \text{load/load}  \\ 

      \readInst{}{r}{x} \seq \writeInst{}{y}{v} 
    & \leadsto 
    & \writeInst{}{y}{v} \seq \readInst{}{r}{x}
    & \text{load/store}  \\ 

  \end{array}
\]

\paragraph{Elimination of Redundant Access} 

In a pair of two adjacent memory accessing instructions 
one of them can be eliminated if its effect 
is subsumed by another. 
For example, two stores writing to the same variable 
can be replaced by a single store.  
Similarly to the reorderings, there exist four kinds 
of eliminations depicted below. 
%
\[\def\arraystretch{1.4}\footnotesize
  \begin{array}{cccl} 

      \writeInst{}{x}{v} \seq \readInst{}{r}{x} 
    & \leadsto 
    & \writeInst{}{x}{v} \seq \assignInst{r}{v}
    & \text{store/load}  \\ 

      \readInst{}{r}{x} \seq \readInst{}{s}{x} 
    & \leadsto 
    & \readInst{}{r}{x} \seq \assignInst{s}{r}
    & \text{load/load}  \\ 

      \readInst{}{r}{x} \seq \writeInst{}{x}{r} 
    & \leadsto 
    & \readInst{}{r}{x} 
    & \text{load/store}  \\ 

      \writeInst{}{x}{v} \seq \writeInst{}{x}{u} 
    & \leadsto 
    & \writeInst{}{x}{u}
    & \text{store/store}  \\ 

  \end{array}
\]

\paragraph{Irrelevant Load Elimination}

Yet another elimination transformation 
which removes a load instruction if its 
result is never used. 
%
\[\def\arraystretch{1.4}\footnotesize
  \begin{array}{cccl} 

      \readInst{}{r}{x} 
    & \leadsto 
    & \epsInst
    & ~|~ \text{$r$ is never used}  \\ 

  \end{array}
\]

\paragraph{Speculative Load Introduction}

An inverse to the previous transformation, 
the load introduction inserts a load instruction 
in an arbitrary place of a program.
%
\[\def\arraystretch{1.4}\footnotesize
  \begin{array}{cccl} 

      \epsInst
    & \leadsto 
    & \readInst{}{r}{x} 
    & ~|~ \text{$r$ is never used}  \\ 

  \end{array}
\]

It can be used in combination with the
load/load elimination to move a load 
instruction out from one branch of 
a conditional:
%
\[\def\arraystretch{1.4}\footnotesize
  \begin{array}{ccc} 

      \kw{if} (e)~ \kw{then} \{ \readInst{}{r}{x} \}
    & \leadsto 
    & \readInst{}{s}{x} \seq \kw{if} (e)~ \kw{then} \{ \assignInst{r}{s} \} \\
    & & ~|~ \text{$s$ is never used}  \\ 

  \end{array}
\]

\paragraph{Roach Motel Reordering}

This class of reorderings moves memory access instructions
into synchronization blocks. For example, a store 
can be moved past a lock acquisition. 
Intuitively, such reorderings can only increase 
synchronization of a program, 
which means that the transformed program should 
exhibit less non-determinism and have fewer outcomes. 

Non-atomic accesses can be moved freely inside 
a critical section, \ie past a lock acquisition
or prior a lock release. 
Besides that, a store can be moved after a \texttt{lock}, 
and load can be moved prior an \texttt{unlock}.   
Similar rules apply to reorderings around 
acquire and release accesses and fences, 
where an acquire operation behaves similarly to \texttt{lock}, 
and release operation similarly to \texttt{unlock}.
% %
\[\def\arraystretch{1.4}\footnotesize
  \begin{array}{cccl} 

      \readInst{\na}{r}{x} \seq \lockInst{l} 
    & \leadsto 
    & \lockInst{l} \seq \readInst{\na}{r}{x}
    & ~ \\ 

      \writeInst{o}{x}{v} \seq \lockInst{l} 
    & \leadsto 
    & \lockInst{l} \seq \writeInst{o}{x}{v}
    & ~  \\ 

      \unlockInst{l} \seq \writeInst{\na}{x}{v} 
    & \leadsto 
    & \writeInst{\na}{x}{v} \seq \unlockInst{l}
    & ~ \\ 


      \unlockInst{l} \seq \readInst{o}{r}{x} 
    & \leadsto 
    & \readInst{o}{r}{x} \seq \unlockInst{l}
    & ~  \\ 

  \end{array}
\]


\paragraph{Strengthening}

Similarly to the roach motel reordering, the strengthening
transformation increases synchronization by 
replacing an access mode of an operation by a stronger one. 
For example, a non-atomic access can be replaced by 
a sequentially consistent access: 
%
\[\def\arraystretch{1.4}\footnotesize
  \begin{array}{cccl} 

      \readInst{o}{r}{x} 
    & \leadsto 
    & \readInst{o'}{r}{x}
    & ~|~ o \sqsubset o' \\ 

      \writeInst{o}{x}{v}
    & \leadsto 
    & \writeInst{o'}{x}{v}
    & ~|~ o \sqsubset o'  \\ 

  \end{array}
\]

\paragraph{Trace Preserving Transformations}

This wide class contains all local transformations 
which do not change a set of traces of a thread~\cite{Sevcik-Aspinall:ECOOP08},
Trace is a sequence of visible side-effects performed by a thread
(loads and stores to shared memory also viewed as side-effects). 
An example is the classic \emph{constant folding}%
~\cite{Muchnick:ACDI97, Wegman-Zadeck:TOPLAS91} transformation.
Here is a particular example of the constant folding application:
%
\[\def\arraystretch{1.4}\footnotesize
  \begin{array}{cccl} 

      \writeInst{}{x}{0 + v} 
    & \leadsto 
    & \writeInst{}{x}{v}
    & \\ 

  \end{array}
\]
  
\paragraph{Common Subexpression Elimination}

\CSE is yet another classic transformation~\cite{Muchnick:ACDI97} 
which searches for instances of identical expressions 
and removes redundant computations. 
Here is an example: 
%
\[\def\arraystretch{1.4}\footnotesize
  \begin{array}{cccl} 

      \readInst{}{r_1}{x + y} \seq \readInst{}{r_2}{x + y} 
    & \leadsto 
    & \readInst{}{r_1}{x + y} \seq \readInst{}{r_2}{r_1}
    & \\ 

  \end{array}
\]

\subsubsection{Global Transformations}

\paragraph{Register Promotion}

If a compiler can determine that a shared variable 
is accessed only from a single thread, it can replace 
the variable by a thread-local register. 
%
\[\def\arraystretch{1.4}\footnotesize
  \begin{array}{ccl} 

      \writeInst{}{x}{v} \seq \readInst{}{r}{x} 
    & \leadsto 
    & \assignInst{s}{v} \seq \assignInst{r}{s}
    \\ 
    
    & & |~ \text{\texttt{x} is not accessed from other threads} \\
    & & |~ \text{\texttt{s} is a fresh register} \\ 

  \end{array}
\]

\paragraph{Thread Inlining}

Sequentialization or thread inlining 
is a transformation that merges two threads into one.
Quite surprisingly, this seemingly harmless transformation
is challenging for many memory models. 
%
\[\def\arraystretch{1.4}\footnotesize
  \begin{array}{cccl} 

      P \pll Q 
    & \leadsto 
    & P ~\seq Q
    & ~ \\ 
    
  \end{array}
\]


\paragraph{Value Range Based Transformations}

Transformations of this class can be applied 
if a program satisfies some invariant deduced 
by a global value-range analysis. 
For example, in a program below   
the conditional statement can be eliminated 
since a static analysis can deduce an invariant 
$\mathsf{x} \geq \mathsf{0}$.

{\footnotesize
\begin{minipage}{0.45\linewidth}
\begin{equation*}
\inarrII{
   \readInst{}{r_1}{x}             \\
   \kw{if} (r_1 \geq 0) ~\kw{then} \\
   \quad\writeInst{}{y}{1}         \\
}{
  \readInst{}{r_2}{x}               \\
  \writeInst{}{y}{r_2}              \\
}
\end{equation*}
\end{minipage}\hfill%
\begin{minipage}{0.05\linewidth}
\Large~\\ $\leadsto$
\end{minipage}\hfill%
\begin{minipage}{0.4\linewidth}
\begin{equation*}
\inarrII{
   \readInst{}{r_1}{x}             \\
   \writeInst{}{y}{1}              \\
}{
  \readInst{}{r_2}{x}               \\
  \writeInst{}{y}{r_2}              \\
}
\end{equation*}
\end{minipage}
}

\subsection{Reasoning Guarantees}

% Finally, we discuss the guarantees provided by memory models 
% which allow the programmers to reason about concurrent 
% programs and their correctness. 

Finally, we discuss the third criterion~\ref{item:criteria:reasoning}---%
reasoning guarantees provided by memory models. 

\subsubsection{\DRF Theorems}
\label{sec:background:drf}

When reasoning about concurrent code, most programmers 
assume sequential consistent memory model.
Of course, it would be improper to require from 
programmers to always keep in mind 
all the intricacy of weak memory models,
as it only complicates an already difficult task
of establishing the correctness of concurrent programs. 
The \emph{data-race freedom}~\cite{Manson-al:POPL05} property, 
\DRF for short, is designed to solve this problem. 
It guarantees that well-synchronized programs 
have only sequentially consistent outcomes. 
In other words, it allows to programmers assume 
simpler sequentially consistent model 
if they properly use synchronization primitives.

Let us consider an example. 
Remember the \ref{ex:sb} program from \cref{sec:background:compile}.
As we demonstrated, under a weak memory model 
%, like the model of \Intel hardware, 
this program can have the weak outcome ${[r_1=0, r_2=0]}$.
Nevertheless, one can restore the \SC semantics.
One way to do this is to use locks, as the following listing demonstrates:

\begin{equation*}
\inarrII{
   \lockInst{l}         \\
   \writeInst{}{x}{1}   \\
   \readInst{}{r_1}{y}  \\
   \unlockInst{l}       \\
}{
   \lockInst{l}         \\
   \writeInst{}{y}{1}   \\
   \readInst{}{r_2}{x}  \\
   \unlockInst{l}       \\
}
\tag{SB+LOCK}\label{ex:sb-lock}
\end{equation*}

A \DRF compliant weak memory model should guarantee 
that this program has only sequentially consistent outcomes:
${[r_1=0, r_2=1]}$, ${[r_1=1,r_2=0]}$, or ${[r_1=1,r_2=1]}$.

Alternatively, if model provides $\sco$ access mode, 
a programmer can annotate all memory accesses by this mode
to restore sequential consistency:  
 
\begin{equation*}
\inarrII{
   \writeInst{\sco}{x}{1}   \\
   \readInst{\sco}{r_1}{y}  \\
}{
   \writeInst{\sco}{y}{1}   \\
   \readInst{\sco}{r_2}{x}  \\
}
\tag{SB+SC}\label{ex:sb-sc}
\end{equation*}

More formally, \DRF theorem for a weak model $M$ states that 
a program has only sequentially consistent outcomes under $M$
if it has no data-races under sequentially consistent memory model
(or all accesses participating in such race are annotated by~$\sco$).

The \DRF theorem allows to reduce reasoning under a weak memory model
to reasoning under the sequential consistency.
It is sufficient to prove that a program has no data-races under the \SC
in order to derive that this program has only \SC outcomes. 

The \DRF theorem in the formulation given above is 
sometimes called \emph{external data-race freedom} (\eDRF),
in order to distinguish it from the \emph{internal data-race freedom} (\iDRF). 
The latter guarantees the \SC semantics for a program 
under weak model $M$ only if the program 
has no races under \textbf{model $\mathbf{M}$ itself}.
Note that the internal \DRF gives a weaker guarantee 
compared to the external \DRF. It does not allow to completely 
avoid the reasoning in term of the weak memory model, 
because one has to first show that the program 
is race-free under relaxed model. 
As we will demonstrate later (see \cref{sec:analysis:oota})
the internal \DRF is a compromise for a certain class 
of memory models which cannot establish the external \DRF.

\subsubsection{Coherence}
\label{sec:background:coh}

As we demonstrated, memory models of 
modern hardware architectures do not 
provide the sequentially consistent semantics.
Yet they usually provide a weaker property 
called \emph{sequential consistency per location},
also known as \emph{coherence}~\cite{Alglave-al:TOPLAS14}.
Following hardware models many programming language level
memory models also provide this property. 

The coherence property ensures that 
all stores to each particular location 
can be totally ordered and that the 
resulting order, the \emph{coherence order}, reflects 
the order in which stores propagate from threads
into the main memory. 
In particular, coherence implies that  
programs consisting only of accesses to 
a single memory location have 
the sequentially consistent semantics.
For example, consider the following program:

\begin{equation*}
\inarrII{
   \writeInst{}{x}{1}   \\
   \readInst{}{r_1}{x}  \\
}{
   \writeInst{}{x}{2}   \\
   \readInst{}{r_2}{x}  \\
}
\tag{COH}\label{ex:coh}
\end{equation*}

The coherence prescribes to a memory model 
assign to this program only the 
sequentially consistent outcomes: 
${[r_1=1, r_2=2]}$, ${[r_1=1, r_2=1]}$, or ${[r_1=2, r_2=2]}$.
A non-coherent model additionally may permit 
the following outcome ${[r_1=2, r_2=1]}$.
For example, the \Java memory model actually 
allows this outcome~\cite{Manson-al:POPL05}.

\subsubsection{Undefined Behavior}
\label{sec:background:ub}

As we already briefly mentioned, some memory models, 
\eg \CPP, treat racy programs as having 
\emph{undefined behavior}~\cite{Boehm-Adve:PLDI08}
if with at least one of the accesses participating 
in a race is a non-atomic access. 
In other words, for these programs any outcome is possible. 
This property is also sometimes called the \emph{catch-fire semantics}.
  
The practical payoff of this approach  
is that it enables the optimal compilation scheme 
for non-atomic accesses and makes any sequentially valid 
transformation applicable to them.  
Indeed, effects of hardware and compiler 
optimizationz can only be observed due to racy accesses
from concurrent threads. If such accesses are said 
to imply undefined behavior and give no guarantee, 
effects of these optimizations become indistinguishable.

\subsubsection{Speculative Execution and Out~of~Thin-Air~Values}
\label{sec:background:oota}

In order to introduce the last two properties, 
we again turn to an example: 

\begin{equation*}
\inarrII{
  \readInst{}{r_1}{x}     \\
  \writeInst{}{y}{1}      \\
}{
  \readInst{}{r_2}{y}     \\
  \writeInst{}{x}{r_2}    \\
}
\tag{LB}\label{ex:lb}
\end{equation*}

Assume a weak memory model admitting 
the outcome ${[r_1=1, r_2=1]}$ for this program.
For example, hardware memory models of 
\ARMv{7}, \ARMv{8}, and \POWER
allow this outcome, and it can even be 
actually observed on some \ARMv{7} 
machines~\cite{Maranget-al:Tutorial2012}.

The outcome ${[r_1=1, r_2=1]}$ cannot be obtained by some 
\emph{in-order} execution of the program. 
To enable this kind of behaviors for programs, 
a memory model has to utilize some form of 
\emph{speculative execution}~\cite{Boudol-Petri:ESOP10, Boehm-Demsky:MSPC14}.
That is, during the execution, the load $\readInst{}{r_1}{x}$
needs to be buffered and the store $\writeInst{}{y}{1}$ 
needs to be executed out of order
(hence the name of the program LB --- \emph{load buffering}).

However, unrestricted speculations can lead to disruptive results. 
A store executed out of order can turn into 
a self-satisfying prophecy~\cite{Boehm-Demsky:MSPC14}.
Consider the following variation of the load buffering program. 

\begin{equation*}
\small
\inarrII{
  \readInst{}{r_1}{x}   \\
  \writeInst{}{y}{r_1}  \\
}{
  \readInst{}{r_2}{y}   \\
  \writeInst{}{x}{r_2}  \\
}
\tag{LB+data}\label{ex:lb+data}
\end{equation*}

Here, a hypothetical abstract machine can speculate 
to perform a store of value \texttt{1} into the variable \texttt{y}
from the left thread, then read this value in the right thread, 
write it to the variable \texttt{x} and then read it back in the
left thread closing the paradoxical causality cycle.
The value \texttt{1} in the example above appears \emph{out of thin-air}
and then justifies itself leading to the confusing outcome ${[r_1=1, r_2=1]}$.

As we will see in \cref{sec:analysis}, speculative execution 
is required to enable certain program transformations. 
However, speculations should be properly constrained
in order to prevent an appearance of thin-air values. 
In \cref{sec:analysis:porf,sec:analysis:deprf,sec:analysis:sdeprf}
we will see how various memory models deal with this problem. 

\section{Comparison}
\label{sec:comparison}

We performed a comparison of the memory models 
found via the search procedure described in \cref{sec:methodology} 
by the criteria given in \cref{sec:background}. 
A particular challenge of this comparison was the fact that 
consulted research papers often use different terminology, 
have incomplete information about models, and 
sometimes they even contradict each other. 
We tried to approach these challenges by the following means.
First, we used consistent terminology 
to denote the properties of the memory models, 
as presented in \cref{sec:background}.
Second, we complemented the information about 
each particular memory model from different sources.
If after this procedure some particular property 
was still unclear, we left it as unknown. 

Based on our comparison of the memory models, 
we identified six classes of them:
sequentially consistent models,
models with total or partial order on stores,
program order preserving models, 
syntactic dependency preserving models, 
semantic dependency preserving models, 
and models with out of thin-air values. 
The models from the same class have the similar compilation mappings, 
the set of sound program transformations, and the provided reasoning guarantees.
We first present the result of our comparison on a per-class basis 
(see~\cref{table:cmp-cls} and~\cref{sec:analysis}), and 
then give a more detailed comparison with respect to individual models
(see~\cref{table:cmp-mms} and~\cref{sec:catalog}).
Thus we have an opportunity to first discuss common principles
behind programming language relaxed memory models in general, 
and then dive deeper into the details of each particular model. 

In both \cref{table:cmp-cls} and \cref{table:cmp-mms} we order
the memory models by their weakness.  
The strongest models are located at the top rows of the tables, 
while the weakest are at the bottom.  

Columns of both tables correspond 
to the properties of the memory models.
In order to be concise, we chose a binary classification for all properties,
\ie the model is either said to satisfy a given property or not.  
We also split properties into several subgroups.

The first group is devoted to an optimality 
of compilation mappings to target hardware architectures. 
We classify a compilation scheme as either optimal or not,
in the following sense.
We chose a weakest possible access mode supported by a model
and consider the compilation scheme for memory accesses annotated by this mode. 
For memory models that treat racy non-atomic accesses
as undefined behavior, we consider the compilation mapping
for a weakest atomic access mode that model provides.
This is because the catch-fire semantics for racy non-atomics 
trivially permits the optimal compilation mapping (see~\cref{sec:background:ub}).
We say that the compilation scheme is \emph{optimal} if  
accesses annotated by the most relaxed mode 
can be compiled just as plain load and store instructions 
of a given hardware architecture 
(\ie without use of memory fences or other auxiliary code). 

The second group is dedicated to a soundness of various program transformations. 
The classification is also binary: a transformation is either sound or unsound 
in a given memory model (in the sense stated in~\cref{sec:background:trans}).
Again, to be concise, we do not consider all combinations 
of program transformations and memory access modes. 
Instead, we consider the weakest possible accesses 
which have fully defined semantics. 
We further split the transformations into 
global and local as in~\cref{sec:background:trans}.

The third group corresponds to reasoning 
principles guaranteed by the model. In particular, we check 
whether a model provides the external DRF guarantee (see~\cref{sec:background:drf}), 
whether it provides the coherence property (see~\cref{sec:background:coh}),
whether it has fully defined semantics for all types of accesses, 
\ie the model does not treat racy non-atomic accesses 
as undefined behavior (see~\cref{sec:background:ub}),
whether the model utilizes in-order execution 
(as opposed to speculative out-of-order executon),
and whether it forbids out of thin-air values (see~\cref{sec:background:oota}).

In \cref{table:cmp-mms} each row corresponds to 
a specific memory model, denoted by its abbreviation, 
and thus each cell describes a particular property 
of that particular model. 
We marked a cell by \cmark~ if the corresponding model satisfies the given property,
and we marked it by \xmark~ otherwise.
If the property was not studied in the research papers, 
we color the cell in gray~% 
{\protect\tikz \protect\draw[fill=colorQmark] (0,0) rectangle ++(0.35,0.35);}.

Each row of the \cref{table:cmp-cls} corresponds to a class of memory models. 
We marked a cell by \cmark~ if the majority of models 
in the given class satisfy the property. 
If less than the majority of models satisfy the property we mark 
the corresponding cell by \wmark.
Finally, if none of the models satisfy the property, we mark the cell by \xmark. 
Note that when counting the majority, we omit the unknowns.
Also, if a given property was not studied in the context of some class of models 
(\ie, in~\cref{table:cmp-mms} it is marked by \qmark for all models in this class)
in \cref{table:cmp-cls} we mark the corresponding cell by \xmark. 
That is, in \cref{table:cmp-cls} symbols \cmark~ and \wmark~ 
denote positive knowledge,
while~\xmark~ denotes negative knowledge or
an absence of information.

Besides \cref{table:cmp-cls,table:cmp-mms} which describe 
properties of the memory models, 
we also present \cref{table:features}
that provides a list of features (see \cref{sec:background:primitives}) 
supported by the models.
In this table each row corresponds to a particular memory model. 
Columns correspond to supported features. 
In particular, we check what types of access modes are supported:
non-atomic (NA), relaxed (RLX), release/acquire (RA), sequentially-consistent (SC); 
what types of fences are supported: release/acquire (F-RA) 
and sequentially-consistent (F-SC);
whether the atomic read-modify-write operations are supported (RMW),
whether the model handles locks explicitly (LK),
and whether it supports mixed-size accesses (MIX). 

% Finally, the last group enumerates the list of memory access modes 
% and fences supported by the model, as well as whether the model 
% supports read-modify-write operations, locks, and mixed-size accesses.
\section{Analysis}

In this section we present the detailed comparison 
of the considered programming language memory models. 
We summarize our findings in~\cref{table:summary}.
Each row of the table corresponds to a memory model, denoted by its abbreviation. 
Columns of the table correspond to the properties of the models discussed in~\cref{sec:background}.
We split the properties into several subgroups. 

The first group is devoted to optimality of compilation mappings
to target hardware architectures. In order to be concise, 
we chose the binary classification of optimality, 
that is, we classify the compilation scheme as either optimal or not,
in the following sense.
We chose the weakest possible access mode supported by the model
and consider the compilation scheme for the memory accesses annotated by this mode. 
For the memory models that treat racy non-atomic accesses
as undefined behavior, we consider the compilation mapping
for most relaxed access types the model provides.
This is because the catch-fire semantics for racy non-atomics 
trivially permits the most optimal compilation mappings (see~\cref{sec:bgrnd-ub}).
We say that compilation scheme is \emph{optimal} if the 
accesses annotated by the most relaxed mode 
can be compiled as plain load and store instructions 
of the given hardware architecture. 

The second group is dedicated to soundness of various program transformations. 
The classification is also binary: a transformation is either sound or unsound 
in the given memory model (in a sense stated in~\cref{sec:bgrnd-opt-sound}).
Again, to be concise, we do not consider all the combinations 
of program tranformations and memory access modes. 
Instead, we consider the weakest possible accesses which have fully defined semantics. 
We further split the transformations into global and local as in~\cref{sec:bgrnd-opt-sound}.

The third group corresponds to reasoning principles guaranteed by the model. 
It includes the following properties. What kind of DRF guarantee the model provides.
(we distinguish the internal, external and local DRF theorems, see~\cref{sec:bgrnd-drf}).
Whether the model has undefined behaviors (see~\cref{sec:bgrnd-ub}).
Whether the model permits out-of-thin-air values (see~\cref{sec:bgrnd-oota}).

Finally, the last group enumerates the list of memory access modes 
and fences supported by the model, as well as whether the model 
supports read-modify-write operations, locks, and mixed-size accesses.

Driven by our analysis of the models' properties, we partition all models into five classes. 
\app{I think this is the place to name the classes and briefly introduce them.}
The models from the same class have similar compilation mappings, 
set of sound program transformations, and provided reasoning guarantees.
Our classes are ordered by the weakness of the memory models they consist of.  
The strongest models are located at the top rows of the table, 
while the weakest are at the bottom. 

We next discuss each class in more details
(note that the order is different from the one in the~\cref{table:summary}). 
We also give some insight on the relationship
between compilation scheme optimality, 
soundness of transformations and reasoning guarantees.
In particular, we explain why the support of some reasoning guarantees 
disables some program transformations and requires more heavyweight 
compilation mappings to hardware.

\subsection{Strong SC-like Models}

The sequential consistency is the strongest model one can imagine. 
\app{The statement is too strong. I.e., sequential execution of the threads
is even stronger.}
It renders many common transformations unsound, 
including all kind of instruction reorderings and 
common subexpression elimination~\cite{Marino-el:PLDI11, Sevcik-Aspinall:ECOOP08}.
The fact that instruction reorderings are forbidden 
makes the model expensive to implement on modern hardware
since even the strongest hardware model, namely the x86-TSO,
\app{The same problem. I'd not use ``strongest'' that often.}
permits store/load reordering as an optimization.
Therefore, to preserve sequential consistency during compilation,
the compiler need to emit heavyweight full memory fences,
which makes compilation mappings far from optimal.  

In terms of reasoning guarantees, however, SC is quite a pleasant model. 
It gives the DRF-SC and coherence (\eupp{s/coherence/SC-per-location?}) 
properties for free%
\footnote{The SC semantics assigns to programs only sequantially consistent
outcomes by definition, thus satisfying DRF-SC without any preconditions.
The coherence property is equivalent to sequential consistency per location.
The fact that in SC model any execution is sequeantially consistent implies trivially
that the accesses to each location are also sequeantially consistent.}
and it is naturally program order preserving.
\app{I don't get the last sub-sentence.}

The conceptual simplicity of SC have inspired many researchers 
to adopt it and to try to mitigate the induced performance 
penalty by some additional measures.
Marino et al~\cite{Marino-el:PLDI11, Singh-el:ISCA12} 
examined the performance penalties to ensure end-to-end SC
enforced by (1) modified SC-preserving version 
of LLVM compiler infrastructure and 
(2) a modified version of x86-TSO hardware. 
To mitigate the induced overhead the authors 
utilized the observation that hardware need to 
enforce SC only for memory accesses to shared mutable variables. 
To classify the memory regions as either thread-local,
shared immutable, or shared mutable they have used 
a combination of static compiler analysis and 
dynamic analysis powered by modified hardware. 
They evaluated their approach on a number of benchmarks
and reported performance overhead of 6.2\% on average 
and ~17\% in maximum, compared to stock LLVM compiler 
and regular x86 hardware. 

The SC-Haskell memory model~\cite{Vollmer-el:PPoPP17}
were inspired by the same idea of separation
between the thread-local and shared mutable memory. 
To safely distinguish between the two 
the authors utilized the powerful strong type system of Haskell. 
The consequence of this approach is that the 
programmers need to follow a stricter discipline 
in order to please the type checker. 
The authors modified the GHC to preserve SC 
and then run 1,279 benchmarks on x86-64 hardware
to measure the performance penalties.
They reported 0.4\% geometric mean slowdown,
and noticed that only 12 benchmarks experienced 
slowdown greater than 10\%.

The DRFx~\cite{Marino-el:PLDI10} is another 
SC preserving memory model. In this memory model
the runtime system is guaranteed to raise 
an exception if the program has data-races, 
and yeild only sequantially consistent outcomes otherwise.
In order to make the runtime data-race detection feasible 
in practice, the authors propose several modifications 
to existing hardware.
The authors claim that any sequentially valid optimization 
(\eg instruction reorderings or common subexpression elimination),
is \textbf{sound} in DRFx model.
The only limitation is that these transformations can only be performed
withing the bounds of compiler-designated program regions.
Importantly, any transformation that introduces 
speculative reads or writes is \textbf{unsound},
since speculative optimizations can bring
data-races into otherwise race-free programs.
The expected performance overhead of the model 
is reported to be 3.25\% on average
assuming the efficient implementation 
of data-race detection in hardware. 
(compared to stock compiler and x86 hardware). 

\app{As a reader, I'd expect to see the slowdown numbers in the table in some
unified manner.}

Given the results of performance evaluation above,
one can argue that the cost is worth the prize
of having simple SC-like model.
However, we point out that the two 
of the solutions above~\cite{Singh-el:ISCA12, Marino-el:PLDI10} 
require non-trivial modifications to the 
existing hardware and compilers.
In case of SC-Haskell~\cite{Vollmer-el:PPoPP17}, 
the programmers are obligated to follow 
a strict programming discipline enforced 
by the type system of the language.
In all research papers above \app{Very unclear reference.} the expirements 
of proposed solutions were mainly evaluated on x86 hardware, 
and the impact of enforcing SC on weaker hardware
(ARM, POWER) is less clear.
\app{I think that the current text dictates that we have to say smth stronger: not studied.
However, it is not true. Why don't we discuss Liu-al-PLDI19 here?}
Moreover, while the reported performance overhead 
is relatively small on average, 
it is more significant for particular 
kind of programs that heavily utilize shared 
mutable memory (like lock-free data structures).
\eupp{check the evaluation in the papers once again
to support this claim}

\subsection{Strong TSO-like Models}

The next class of PL memory models we consider 
was inspired by TSO~\cite{Sewell-al:CACM10} and PSO~\cite{Sparc:94} 
hardware models. In these models, threads usually 
are equipped with \emph{store buffers}.
All store operations go to these buffers before they 
propogate into the main memory.  
In essence, store buffers enable 
store/load reordering (in case of~TSO),
and store/load \& store/store reordegins (in case of~PSO).

The models based on store buffers idea 
can be compiled down to x86 hardware without any 
performance penalty, since x86 implements TSO model itself.
That is, the compilation mappings to x86 are optimal.
However, when compiled down to weaker hardware (e.g. ARM, POWER)
the compiler indeed needs to take additional measures 
to enforce TSO/PSO like memory model.
\eupp{Perhaps, we can cite some paper here?} 

The TSO/PSO models are weaker than SC, while 
they are still relatively strong.
The external DRF-SC and coherence still hold
and program order is preserved.

We are aware of two papers that propose to use TSO/PSO 
like models as PL level memory models.

Demange et al.~\cite{Demange-el:POPL13} presented 
the \emph{Buffered Memory Model} (or BMM in short)
as a candidate model for Java language.
Their motivation, however, stemmed not from the desire 
to fully replace the Java Memory Model, but rather 
from the goal to build a verified version of 
Java Virtual Machine (akin to CompCertTSO project).
By taking a relatively simple and yet pragmatic memory model
as a first target they hoped to made this task feasible. 
The authors proved soundness of several program transformations
(including store/load reordering, speculative load introduction,
and several others, see 
the~\cref{table:summary} for details%
\footnote{\eupp{If we'll decide to mention unsoundness of 
thread inlining w.r.t. TSO
then cite~\cite{Lahav-Vafeiadis:FM16} 
(since the paper itself doesn't mention this transformation)}})
and the external DRF-SC theorem. 
They also modified existing open-source implementation of 
JVM~\cite{Pizlo-el:ECCS10} to preserve BMM and 
reported only~1\% average overhead 
compared to original version of the virtual machine. 
Again, the authors used only x86 hardware in their 
experiments, and the performance penalties 
are expected to be more significant on weaker hardware.   

In~\cite{Boudol-el:POPL09} the authors propose 
an approach to formal semantics of relaxed memory models 
based on the abstract machine with the main memory 
and the hierarchial structure of store buffers 
with stores to different locations possibly 
propagating to the main memory out-of-order
(similarly to PSO model).
The authors present a proof of DRF-SC theorem,
but do not provide an extensive study 
of program transformations' soundness.

\subsection{OOTA Models}

We next move on to the other end of the memory models' spectrum. 
We consider the class uniting the weakest models of our analysis.
These models enable efficient compilation mappings and 
many program transformations, but at the cost of 
introducing thin-air values (see \cref{sec:bgrnd-oota}).
Hence we name this class of models as Out-of-Thin-Air (OOTA) models. 
 
The main common disadvantage of OOTA models is that 
they lack many fundamental reasoning 
principles~\cite{Boehm-Demsky:MSPC14, Batty-al:ESOP15}:
type safety and security gurantess can be violated, 
compositional reasoning is impossible, and
the external DRF-SC property cannot be established. 
The odd consequences of thin-air values manifest 
itself best on the following classical example~\cite{Boehm-Demsky:MSPC14}: 

\begin{equation*}
\inarrII{
  \readInst{}{r_1}{x}      \\
  \kw{if} {(r_1)} ~\{      \\
  \quad\writeInst{}{y}{1}  \\
  \}
}{
  \readInst{}{r_2}{x}      \\
  \kw{if} {(r_2)} ~\{      \\
  \quad\writeInst{}{x}{1}  \\
  \}
}
\tag{LB+ctrl}\label{ex:lb+ctrl}
\end{equation*}

For the memory model admitting thin-air values 
(as \eg \CMM~\cite{Batty-al:POPL11}), 
the outcome $[r_1=1, r_2=1]$ is perfectly valid
(one can see that the program above is analogous 
to the \ref{ex:lb+data}, except it has 
control dependencies between instructions 
in place of data dependencies).
Not only this outcome is completely unintuitive,
but it also contradicts to the external DRF-SC guarantee.
Indeed, in SC model the program above has 
a single valid execution with the outcome $[r_1=0, r_2=0]$ 
and no data-races, thus under DRF-SC compilant model 
it should also has this sole outcome.  

The most notable member of the OOTA class is \CMM model~\cite{Batty-al:POPL11}.
The C and C++ languages are widely known as low-level languages 
for system programming which focus on the efficiency of compiled code. 
The main design principle of these languages is to provide
so-called zero-cost abstraction that should be compiled 
into efficient assembly code and leave the room 
for the aggressive optimizations. 
The main purpose of the \CMM model was to adhere 
to the same principles. The memory model 
was meant to provide efficient compilation mappings 
and as many transformation as possible.
It was later shown that the formal model actually 
fails in achieving this goal. 
Vafeiadis~\etal~\cite{Vafeiadis-al:POPL15} have shown
that many program transformation that deemed to be correct
are actually unsound according to the formal model. 
Batty~\etal~\cite{Batty-al:ESOP15} have shown that 
the model also fails to provide exteranl DRF guarantee, 
and that it is ultimately not possible to provide this guarantee
at all within the style of the \CMM formal semantics.
%Yet the authors show that the internal DRF 
A lot of work~\cite{Batty-al:POPL11, Sarkar-al:PLDI12, Batty-al:POPL12, Batty-el:POPL16} 
was dedicated to prove soundness of optimal compilation mappings 
with respect to formal models of hardware, 
and there the results were mostly positive.
\eupp{should we mention that SC proof was broken and 
also cite RC11 here?}. 
Flur~\etal~\cite{Flur-el:POPL17} have extended the model 
to support mixed-size accesses.
Finally, Nienhuis~\etal~\cite{Nienhuis-el:OOPSLA16} presented 
a formal executable semantics in terms of an abstract machine, 
equivalent to the original \CMM model. 

The JavaScript memory model \JSMM inhereted some properties from \CMM.
Like the latter, it also has the problem of thin-air values
and thus can only provide internal DRF-SC gurantee. 
Contrary to the \CMM, \JS model gives a fully defined 
semantics to non-atomics 
(\ie no undefined behavior for racy non-atomic accesses).  
Also, unlike \CMM, where the main language primitive is 
individual mutable atomic variables, \JSMM uses 
the primitive of \texttt{SharedArrayBuffer},
that is a linear mutable byte buffer.
Thus the model naturally supports mixed-size accesses.

Crary and Sullivan~\cite{Crary-Sullivan:POPL15} proposed 
an alternative approach to relaxed shared memory concurrency.
Instead of deriving the ordering constraints from the annotations 
on memory accesses, they propose to directly specify 
the ordering between memory access in the source code. 
Their approach is highly generic and subsumes 
the traditional memory order annotation in the style of \CMM.
Their model is very weak and permits thin-air values. 
Still the authors prove the internal DRF theorem.

Saraswat~\etal~\cite{Saraswat-el:PPoPP07} presented the \RAO memory model
where relaxed behaviors are explained through the transformations 
over sequentially consistent execution.
Depending on the exact choice of transformations 
their model is either (1) permits thin-air values or 
(2) preserves external DRF-SC. 
Since the transformations (\eg reorderings, value range based, \etc)
are baked into the model, their soundness follows immediatelly.  

Boudol and Petri~\cite{Boudol-Petri:ESOP10} proposed a general 
framework to study the effects of speculative execution in
shared memory setting. 
They have also noticed that the external DRF doest not 
hold in the presence of unrestricted speculations, 
yet the internal DRF theorem still can be proven. 

\subsection{$\lPO\cup\lRF$ Acyclic Models}
\label{sec:porf-acyc}

Contrintuitive behavior of OOTA models, together with the fact that they break 
many important reasoning principles (external DRF-SC among them), 
has lead over the time to the consensus in the research community that these models 
are not suited well for the role of 
programming languages memory models~\cite{Boehm-Demsky:MSPC14, Batty-al:ESOP15}
A lot of effort has been put to forbid problematic 
thin-air outcomes, while still keep compilation scheme as efficient as possible
and enable as many transformations as possible.

The most straightforward way to forbid thin-air values 
was proposed by Boehm and Demsky~\cite{Boehm-Demsky:MSPC14}
The idea is to simply prohibit any kind of speculative execution, 
which is equivalent to forbidding load/store reorderings altogether. 
This fix not only restores external DRF-SC~\cite{Lahav-al:PLDI17}
and other reasoning guarantees, but also leads to 
a much simpler mental model.  

Lahav~\etal~\cite{Lahav-al:PLDI17} formalized this approach and 
studied it extensively. They proposed a modified version 
of \CPP model called \RCMM (repaired \CMM).  
Besides the strengthening of the model to preserve 
the order between load/store pairs, 
the repaired version also corrects the semantics 
of sequentially-consistent accesses.

The authors have shown that many 
program transformations are still sound in \RCMM, 
with the obvious exception of load/store reordering itself
(see~\cref{table:summary} for details).
Also, the compilation mappings to x86 remain efficient, 
since the architecture already guarantee to preserve the order. 
between loads and subsequent stores. 
However, weaker architectures (ARM, POWER) do not guarantee that, 
and thus additional measures are required.
Lahav~\etal~\cite{Lahav-al:PLDI17} proposed to compile relaxed loads 
as plain load followed by a spurious conditional branch,
which introduces fake control dependency between 
the load and subsequent stores. 
ARM and POWER hardware preserves dependencies, 
and thus it has to also retain the load/store ordering. 

Ou and Demsky \cite{Ou-Demsky:OOPSLA18} have studied 
the performance penalty needed to guarantee 
\RCMM memory model on ARMv8 hardware.
They modified the LLVM compiler framework 
to enforce \RCMM memory model
by (1) adjusting the compiler optimization passes and 
(2) changing the compilation mappings.
Several compilation schemes were considered,
among them the one that uses spurious conditional branch
as descibed above has demonstrated the most promising results.  
The authors measured the running time on a set of benchmarks 
implementing various concurrent data-structures
and reported an overhead of 0\% on average and 6.3\% in maximum,
compared to the unmodified version of the compiler. 

A plenty of memory models have utilized the \RCMM
solution to forbid thin-air values by 
preserving the order between load/store pairs. 

Dang~\etal~\cite{Dang-el:POPL19} developed an operational 
version of \RCMM in terms of the abstract machine, 
which they called \ORCMM. Their motivation was to 
then develop a new program logic and show it's soundness
with respect to \ORCMM memory model. 
The program logic itself was then utilized to 
prove correctness of some parts of 
the \Rust~\cite{RustBook:19} standard library.
\eupp{Maybe we can omit this paper, since it's mostly about program logics?} 

Dolan~\etal~\cite{Dolan-al:PLDI18} developed a new 
memory model for the \MOCaml project. 
They were the first to propose the local DRF property. 
The authors also hinted that the local DRF property 
is not compatible with load/store reordering.
This fact forced them to forbid this transformation
and adapt similar compilation scheme as for \RCMM. 

An important divergence of \OCaml memory model 
from \CMM-like models is that the former 
has a weaker notion of coherence.
The choice of the weaker coherence was deliberate 
with the purpose to enable common subexpression elimination (CSE).
\eupp{Consider to move the details on Coherence vs. CSE
into separate subsection and then leave the reference here}.
The subtle effect of the strong coherence property 
on this classic compiler optimization was first 
observed in the context of \Java 
memory model~\cite{Pugh:JAVA99}.
To see the problem, consider the program below
(on the left) and the transformed version 
of this program after application of CSE (on the right).
Note that the optimization has replaced 
the second access to variable \texttt{x}
by a read from register. 

\begin{minipage}{0.5\linewidth}
\begin{equation*}
\inarrII{
  \readInst{}{r_1}{x}      \\
  \readInst{}{r_2}{y}      \\
  \readInst{}{r_3}{x}      \\
}{
  \writeInst{}{y}{1}       \\
}
\tag{}\label{ex:coh-rr}
\end{equation*}
\end{minipage}
%
\begin{minipage}{0.5\linewidth}
\begin{equation*}
\inarrII{
  \readInst{}{r_1}{x}      \\
  \readInst{}{r_2}{y}      \\
  \assignInst{r_3}{r_1}    \\
}{
  \writeInst{}{y}{1}       \\
}
\tag{}\label{ex:coh-rr}
\end{equation*}
\end{minipage}

Now assume that \texttt{x} and \texttt{y} are aliased,
\ie they point to the same memory location.
Under this assumption the outcome $[r_1=0, r_2=1, r_3=0]$
is forbidden for the memory model respecting coherence.
Indeed, the coherence guarantees sequential consistency per location, 
which means that for the programs consisting of accesses 
to the single memory location 
(as the one above in the presence of aliasing) 
only the sequentially consistent outcomes are allowed.
The outcome $[r_1=0, r_2=1, r_3=0]$ cannot be obtained 
by the interleaving of instructions, and thus 
it should be forbidden.  
However, this outcome is allowed for 
the optimized version of the program. 

Bender and Palsberg~\cite{Bender-Palsberg:OOPSLA19} formalized a new revision 
of the Java Memory Model~\cite{JDK9-VarHandle, JEP:193, JDK9-Modes}, 
which was developed to overcome 
the difficulties of the previous one~\cite{Manson-al:POPL05}
(see \cite{sec:prm-cert} for details).
The new version of the model was inspired by \RCMM. 
It introduced a system of annotations on memory accesses, 
called "Java access modes" (hence the name of the model --- \JAM),
similar to those present in \CMM like models.
The new model adopted the \RCMM solution to OOTA problem. 
It forbids load/store reorderings on the level of 
opaque (an analog of \CPP relaxed) or stronger accesses.
The model does not tackle the problem of 
thin-air values on the level of plain (\ie non-atomic) accesses.

Dodds~\etal~\cite{Dodds-el:ESOP18} proposed a denotational 
compositional semantics for the fragment of \CMM memory model, 
including non-atomic accesses with cath-fire semantics, 
release-acquire accesses, and sequantially-consistent fences. 
Based on this semantics the authors developed 
a tool for automatic verification of program transformations
in the considered fragment of the \CMM model. 
Since the relaxed fragment was not included, 
the authors avoided problems with thin-air values. 

\subsection{$\lPPO\cup\lRF$ Acyclic Models}
\label{sec:pporf-acyc}


\begin{itemize}
  \item Pros.
  \begin{itemize}
    \item (?)
  \end{itemize}
  \item Cons.
  \begin{itemize}
    \item Trace preserving transformations are unsound.
  \end{itemize}
\end{itemize}

\subsection{no-OOTA Models}
\label{sec:prm-cert}

\begin{itemize}
  \item Pros.
  \begin{itemize}
    \item Cheap compilation to hardware.
    \item Almost all transformations are sound (including all reorderings). 
  \end{itemize}
  \item Cons.
  \begin{itemize}
    \item Complexity.
    \item No common ground between various models.
    \item It's unknown how to make thread inlining sound (?).
  \end{itemize}
\end{itemize}

\newpage
\onecolumn

\begin{landscape}

\begin{table*}
\begin{tabular}{|c|l|c|c|c|c|c|c|c|c|c|c|c|c|c|c|c|c|c|c|c|c|c|c|c|c|c|c|c|c|c|}
 \hline

                                                      &
 \multirow{3}{*}{Model}                               & 
 \multicolumn{ 4}{c|}{\multirow{2}{*}{Compilation}}   &
 \multicolumn{10}{c|}{Transformations}                &
 \multicolumn{ 6}{c|}{Reasoning}                      &
 \multicolumn{ 9}{c|}{\multirow{2}{*}{Features}}      \\ 

 \cline{7-22}

                             &
                             &
 \multicolumn{4}{c|}{}       &
 \multicolumn{7}{c|}{Local}  &
 \multicolumn{3}{c|}{Global} &

 \multicolumn{3}{c|}{DRF}    &
 \multicolumn{3}{c|}{}       &
 \multicolumn{9}{c|}{}       \\ 
 
 \hline
                                     &
                                     &
 \rotatebox[origin=c]{270}{x86}      & 
 \rotatebox[origin=c]{270}{Power}    & 
 \rotatebox[origin=c]{270}{ARMv7}    & 
 \rotatebox[origin=c]{270}{ARMv8}    & 
 
 \rotatebox[origin=c]{270}{TP}     &
 \rotatebox[origin=c]{270}{RI}     &
 \rotatebox[origin=c]{270}{RE}     &
 \rotatebox[origin=c]{270}{ILE}    &
 \rotatebox[origin=c]{270}{SLI}    &
 \rotatebox[origin=c]{270}{S}      &
 \rotatebox[origin=c]{270}{RM}     &
 \rotatebox[origin=c]{270}{RP}     &
 \rotatebox[origin=c]{270}{VR}     &
 \rotatebox[origin=c]{270}{TI}     &
 
 \rotatebox[origin=c]{270}{Int}    &
 \rotatebox[origin=c]{270}{Ext}    &
 \rotatebox[origin=c]{270}{Loc}    &

 \rotatebox[origin=c]{270}{UB}                                 &
 \rotatebox[origin=c]{270}{\makecell{$\lPO\cup\lRF$ \\ acyc.}} & 
 \rotatebox[origin=c]{270}{OOTA}                               &                              


 \rotatebox[origin=c]{270}{NA}                      &
 \rotatebox[origin=c]{270}{RLX}                     &
 \rotatebox[origin=c]{270}{RA}                      &
 \rotatebox[origin=c]{270}{SC}                      &
 \rotatebox[origin=c]{270}{F-RA}                    &
 \rotatebox[origin=c]{270}{F-SC}                    &
 \rotatebox[origin=c]{270}{RMW}                     &
 \rotatebox[origin=c]{270}{Lock}                    &
 \rotatebox[origin=c]{270}{\makecell{Mix.Sz.}}      \\ 
 
 \Xhline{2\arrayrulewidth}
 
 \multirow{3}{*}{\rotatebox[origin=c]{270}{\makecell{SeqCst}}}   

 & SC             & & & & & & & & & & & & & & & & & & & & & & & & & & & & & \\ \cline{2-31}
 & SC-Haskell     & & & & & & & & & & & & & & & & & & & & & & & & & & & & & \\ \cline{2-31}
 & DRFx           & & & & & & & & & & & & & & & & & & & & & & & & & & & & & \\ \Xhline{2\arrayrulewidth}

 \multirow{2}{*}{\rotatebox[origin=c]{270}{\makecell{TSO\\PSO}}}   

 & BMM            & & & & & & & & & & & & & & & & & & & & & & & & & & & & & \\ \cline{2-31}

 & Rlx Op.Sem.    & & & & & & & & & & & & & & & & & & & & & & & & & & & & & \\ \Xhline{2\arrayrulewidth}

 \multirow{5}{*}{\rotatebox[origin=c]{270}{\makecell{$\lPO\lRF$\\acyc}}}   

 & RC11           & & & & & & & & & & & & & & & & & & & & & & & & & & & & & \\ \cline{2-31}

 & ORC11          & & & & & & & & & & & & & & & & & & & & & & & & & & & & & \\ \cline{2-31}

 & OCaml MM       & & & & & & & & & & & & & & & & & & & & & & & & & & & & & \\ \cline{2-31}

 & JAM            & & & & & & & & & & & & & & & & & & & & & & & & & & & & & \\ \cline{2-31}

 & Rlx Compos.    & & & & & & & & & & & & & & & & & & & & & & & & & & & & & \\ \Xhline{2\arrayrulewidth}

 \multirow{2}{*}{\rotatebox[origin=c]{270}{\makecell{$\lPPO\lRF$\\acyc}}}   

 & LKMM           & & & & & & & & & & & & & & & & & & & & & & & & & & & & & \\ \cline{2-31}

 & OHMM           & & & & & & & & & & & & & & & & & & & & & & & & & & & & & \\ \Xhline{2\arrayrulewidth}

 \multirow{7}{*}{\rotatebox[origin=c]{270}{\makecell{no-OOTA}}}   

 & JMM            & & & & & & & & & & & & & & & & & & & & & & & & & & & & & \\ \cline{2-31}

 & Promising      & & & & & & & & & & & & & & & & & & & & & & & & & & & & & \\ \cline{2-31}

 & Weakestmo      & & & & & & & & & & & & & & & & & & & & & & & & & & & & & \\ \cline{2-31}

 & MRD            & & & & & & & & & & & & & & & & & & & & & & & & & & & & & \\ \cline{2-31}

 & P-P/S          & & & & & & & & & & & & & & & & & & & & & & & & & & & & & \\ \cline{2-31}

 & J/R            & & & & & & & & & & & & & & & & & & & & & & & & & & & & & \\ \cline{2-31}

 & Generative     & & & & & & & & & & & & & & & & & & & & & & & & & & & & & \\ \Xhline{2\arrayrulewidth}

 \multirow{5}{*}{\rotatebox[origin=c]{270}{\makecell{OOTA}}}   

 & C11            & & & & & & & & & & & & & & & & & & & & & & & & & & & & & \\ \cline{2-31}

 & JS MM          & & & & & & & & & & & & & & & & & & & & & & & & & & & & & \\ \cline{2-31}

 & RMC            & & & & & & & & & & & & & & & & & & & & & & & & & & & & & \\ \cline{2-31}

 & RAO            & & & & & & & & & & & & & & & & & & & & & & & & & & & & & \\ \cline{2-31}

 & Spec.Comp.     & & & & & & & & & & & & & & & & & & & & & & & & & & & & & \\ \Xhline{2\arrayrulewidth}


\end{tabular}
\caption{
  % \textit{T.P.} --- trace preserving.
  % \textit{R.I.} --- reordering of independent instructions.
  % \textit{R.E.} --- redundunt load/store elimination.
  % \textit{I.L.E.} --- irrelevant load elimination.
  % \textit{S.L.I.} --- speculative load introduction.
  % \textit{S.} --- strengthening.
  % \textit{R.M.} --- roach motel reordering.
  % \textit{R.P.} --- register promotion.
  % \textit{V.R.} --- value range analysis based optimizations.
  % \textit{T.I.} --- thread inlining (sequentialization).
  % \textit{Int.} --- internal.
  % \textit{Ext.} --- external.
  % \textit{Loc.} --- local.
  % \textit{UB} --- undefined behavior.
  % \textit{OOTA} --- out-of-thin air values.
  % \textit{Mix.Sz.} --- mixed-size accesses.
}
\label{table:summary}
\end{table*}

\end{landscape}

\twocolumn

\section{Discussion}
\label{sec:discussion}

In this section we provide a summary of our findings described in the previous sections. 
We again briefly compare various classes of memory models 
and present a short guide for the researchers and system-level developers 
on how to choose a memory model based 
on the design principles of the programming language.   
Finally, we highlight underdeveloped parts of the theoretical
and practical aspects of memory models, suggesting 
possible directions for future work in the field. 


\subsection{On the Choice of  Memory Model}

Clearly, design of the memory model of a programming language
should reflect the design of the language itself. 

A language that seeks to provide clear semantics and 
high-level programming abstractions, \eg \Haskell, at the cost 
of some performance losses most definetly should 
adhere to simple memory models like sequential consistency. 

Programming languages focusing on efficiency 
of compiled code, like, for example, \CPP, 
have to resort to weak models admitting 
optimal compilation and wide range of 
various program transformations. 
For these languages it would be natural 
to pick some semantic dependency preserving model. 
However, there are a couple of subtle points here. 
First, models of these class are quite complex, 
which makes reasoning about programs challenging. 
Second, these models are still an active topic 
of research and thus are subject to further modifications.

In the middle between two extremes are program order preserving and 
syntactic dependency preserving models.
Those are reasonable choice for programming languages
which can affort a moderate performance overhead 
in exchange of simpler and more predictable model~\cite{Ou-Demsky:OOPSLA18}.
An example of such language is \OCaml --- 
a high-level programming language 
with an emphasis on functional paradigm which is 
at the same time is actively used in performance sensitive areas
like the development of compilers and verification tools. 

For languages adopting strong models which require non-optimal
compilation mappings and forbid certain program transformations
there are some general optimization techiniques and design decisions
which can partly mitigate induced performance penalties.

A type system can serve as a great help in this task. 
Languages like \Haskell, \OCaml, \Rust that 
statically distinguish and isolate memory regions 
which can be accessed and modified concurrently have a great advantage.
These languages can identify precisely 
immutable and thread local variables
and compile accesses to them without insertion of fences.
Moreover memory accesses of these classes are subject to 
a wide range of program transformations proven to be
sound for single threaded programs. 
 
Languages like, for example, \Java or \Kotlin, which cannot utilize the type system 
to fully prevent racy accesses to non-atomic variables 
because of backward compatibility, still can 
use conservative static escape analysis~\cite{Choi-al:OOPSLA1999}
or various dynamic techinques~\cite{Liu-al:PLDI19} 
in order to approximate the set of thread local variables.   

Functional programming languages encourage 
the programmers to use immutable data whenever possible.
This style of programming minimizes the use 
shared memory and mitigates the performance impact 
of strong memory model~\cite{Vollmer-al:PPoPP17}. 

Finally, if the language tolerates undefined behavior, as, for example, \CPP, 
an alternative to complex semantic dependency preserving model
could be a program order preserving model (or syntactic dependency preserving one) 
which treats data races on non-atomic accesses as 
undefined behavior~\cite{Boehm-Demsky:MSPC14, Ou-Demsky:OOPSLA18}.
In this case the compiler can use optimal compilation mappings 
and wide range of transformations for non-atomics 
and at the same time have a simpler semantics for atomics. 

\subsection{Future Work}

% \begin{itemize}
%   \item Mixed-size accesses.
%   \item Global optimization. 
%         Sequentialization (noOOTA model + seq-on is an open problem).
%   \item Local DRF. Local reasoning in general (denotational models?).
%   \item Common ground between no-OOTA models (?). 
%   \item Performance comparison. Quantitative study in general.
% \end{itemize}
\section{Заключение}
\label{sec:conclusion}

В данной работе мы представили обзор существующих моделей памяти языков программирования. 
Мы сравнили эти модели  по ряду критериев 
и идентифицировали шесть основных классов моделей памяти. 
Также, мы предложили рекомендации по выбору эффективной модели памяти 
для различных языков программирования. 
Мы надеемся, что наша работа будет полезна 
для исследователей и разработчиков в области языков программирования
и послужит введением в сложную тематику слабых моделей памяти. 


На основе нашего анализа мы также можем сделать следующие предположения 
о будущих направления работы в данной области. 

Проблема оптимальности схем компиляции и корректности локальных трансформаций кода 
на сегодняшний день относительно хорошо изучена.
Более новые модели, такие как 
RC11~\cite{Lahav-al:PLDI17}, \OCaml MM~\cite{Dolan-al:PLDI18},
\Promising~\cite{Kang-al:POPL17,Lee-al:PLDI20}
и \Weakestmo~\cite{Chakraborty-Vafeiadis:POPL19},
поддерживают широкий диапазон локальных трансформаций 
и имеют ясные компромиссы относительно оптимальности схем компиляции. 
Исключением являются  локальные трансформации, 
использующие  циклы или рекурсию, так как их корректность всё ещё недостаточно изучена. 
Глобальным трансформациям также уделялось мало внимания, за важным исключением работ~\cite{PichonPharabod-Sewell:POPL16, Lee-al:PLDI20}.
Влияние этих трансформаций на дизайн модели памяти 
ещё предстоит  изучить.

Глобальное свойство свободы от гонок на настоящий момент детально   изучено. 
Напротив, локальное свойство свободы от гонок~\cite{Dolan-al:PLDI18}  
является новой концепцией. 
Стоит ожидать, что ему, как и другим локальным гарантиям~%
\cite{Dodds-al:ESOP18, Jagadeesan-al:OOPSLA2020, Cho-al:PLDI21}, 
будет уделено  внимание исследователей в ближайшем будущем. 

Смешанные обращения~\cite{Flur-al:POPL17}, 
используемые в модели памяти \JS~\cite{Watt-al:PLDI2020}, 
а также в некоторых приложениях, например, в кодовой базе 
ядра \Linux~\cite{Flur-al:POPL17},
на сегодняшний день недостаточно изучены 
даже в контексте моделей памяти процессоров. 
Таким образом, более глубокое понимание семантики смешанных обращений 
является ещё одним важным направлением исследований. 

Модели памяти, сохраняющие семантические зависимости, по-прежнему являются  темой активных исследований~%
\cite{Kang-al:POPL17, Lee-al:PLDI20, Cho-al:PLDI21,
Chakraborty-Vafeiadis:POPL19, Paviotti-al:ESOP20, 
Jagadeesan-al:OOPSLA2020}.
Следует ожидать, что они будут более детально разрабатываться и уточняться в ближайшем будущем. 
Интересным направлением работ в этой области 
является разработка новых гарантий, 
помимо свойства свободы от гонок, 
которые помогут усовершенствовать мета-теорию 
этих моделей и упростить рассуждение о корректности программ. 

Наконец, детальное исследование 
накладных расходов на время исполнения программ с различными моделям памяти также являются очень важной задачей. Несмотря на наличие здесь некоторого количества работ %
\cite{Singh-al:ISCA12, Liu-al:OOPSLA17, Liu-al:PLDI19, 
Vollmer-al:PPoPP17, Dolan-al:PLDI18, Ou-Demsky:OOPSLA18}, 
полная картина все ещё остается недостаточно ясной. 

\section*{Благодарности}

Мы выражаем благодарность Ори Лахаву за ценные комментарии к данной статье. 

Исследование выполнено при финансовой поддержке РФФИ в рамках научного проекта~№~20-31-90088. \\

% WARNING: do not modify rus/main.bib. 
% The bibiligraphy is intended to be shared 
% between eng/rus versions of the paper
% and thus it is located in common/main.bib. 
% However, because of the techinical limitations 
% of Overleaf we have to keep the .bib file 
% in the same folder as main.tex
\bibliography{main} 
% \bibliography{../common/main} 


\bibliographystyle{ieeetr}

\clearpage
\appendix

\section{A Catalog of Memory Models}
\label{sec:catalog}

In this section we present a more detailed view 
on each memory model we consider in our study. 

\newpage
\onecolumn

\begin{landscape}

\begin{table*}
\scriptsize
\begin{tabular}{|c|l|c|c|c|c|c|c|c|c|c|c|c|c|c|c|c|c|c|c|c|c|c|c|c|c|c|c|c|c|c|c|c|c|c|c|} % 35
 \hline

                                                      &
 \multirow{3}{*}{Model}                               & 
 \multicolumn{ 4}{c|}{\multirow{2}{*}{Compilation}}   &
 \multicolumn{17}{c|}{Transformations}                &
 \multicolumn{ 4}{c|}{\multirow{2}{*}{Reasoning}}     &
 \multicolumn{ 9}{c|}{\multirow{2}{*}{Features}}      \\ 

 \cline{7-23}
                             &
                             &
 \multicolumn{4}{c|}{}       &
 \multicolumn{14}{c|}{Local} &
 \multicolumn{3}{c|}{Global} &
 \multicolumn{4}{c|}{}       &
 \multicolumn{9}{c|}{}       \\ 
 
 \hline
                                     &
                                     &
 \rotatebox[origin=c]{270}{x86}      & 
 \rotatebox[origin=c]{270}{Power}    & 
 \rotatebox[origin=c]{270}{ARMv7}    & 
 \rotatebox[origin=c]{270}{ARMv8}    & 
 
 \multicolumn{4}{c|}{\rotatebox[origin=c]{270}{RI}}   &
 \multicolumn{4}{c|}{\rotatebox[origin=c]{270}{RE}}   &

 \rotatebox[origin=c]{270}{ILE}    &
 \rotatebox[origin=c]{270}{SLI}    &
 \rotatebox[origin=c]{270}{RM}     &
 \rotatebox[origin=c]{270}{S}      &
 \rotatebox[origin=c]{270}{TP}     &
 \rotatebox[origin=c]{270}{CSE}    &
 \rotatebox[origin=c]{270}{RP}     &
 \rotatebox[origin=c]{270}{VR}     &
 \rotatebox[origin=c]{270}{TI}     &
 
 \rotatebox[origin=c]{270}{DRF}                                &
 \rotatebox[origin=c]{270}{COH}                                &
 \rotatebox[origin=c]{270}{no-UB}                              &
 % \rotatebox[origin=c]{270}{\makecell{$\lPO\lRF$}}              & 
 \rotatebox[origin=c]{270}{no-OOTA}                            &                              

 \rotatebox[origin=c]{270}{NA}                      &
 \rotatebox[origin=c]{270}{RLX}                     &
 \rotatebox[origin=c]{270}{RA}                      &
 \rotatebox[origin=c]{270}{SC}                      &
 \rotatebox[origin=c]{270}{F-RA}                    &
 \rotatebox[origin=c]{270}{F-SC}                    &
 \rotatebox[origin=c]{270}{RMW}                     &
 \rotatebox[origin=c]{270}{Lock}                    &
 \rotatebox[origin=c]{270}{\makecell{Mix.Sz.}}      \\ 

 \hline
  & & & & & &
 \rotatebox[origin=c]{270}{SL}       &
 \rotatebox[origin=c]{270}{SS}       &
 \rotatebox[origin=c]{270}{LL}       &
 \rotatebox[origin=c]{270}{LS}       &
 \rotatebox[origin=c]{270}{SL}       &
 \rotatebox[origin=c]{270}{SS}       &
 \rotatebox[origin=c]{270}{LL}       &
 \rotatebox[origin=c]{270}{LS}       &
 & & & & & & & & & & & & & & & & & & & & & \\
  
 \Xhline{2\arrayrulewidth}
 
 \multirow{3}{*}{\rotatebox[origin=c]{270}{\makecell{SeqCst}}}   

 & SC 
     &             
     % compilation 
     \badcell & \badcell & \badcell & \badcell & 
     % reorderings
     \badcell & \badcell & \badcell & \badcell &
     % eliminations
     \okcell & \okcell & \okcell & \okcell &
     % irrelevant load elim & speculative load intro 
     \okcell & \okcell &
     % roach motel 
     \okcell & 
     % strength.
     \unkwcell &
     % trace preserving 
     \okcell &
     % common subexpression elimination
     \badcell &
     % global: register promotion, value range, thread inlining 
     \okcell & \unkwcell & \okcell &
     % DRF, COH, no-UB, no-OOTA
     \ldrf & \okcell & \okcell & \okcell &
     % NA, RLX, RA, SC 
     \badcell & \badcell & \badcell & \okcell & 
     % F-RA, F-SC
     \badcell & \badcell & 
     % RMW, Lock
     \badcell & \badcell & 
     % Mix.sz.
     \badcell 
     \\ \cline{2-36}

 & SC-Hs 
     &
     % compilation 
     \badcell & \badcell & \badcell & \badcell & 
     % reorderings
     \badcell & \badcell & \badcell & \badcell & 
     % eliminations
     \okcell & \okcell & \okcell & \okcell &
     % irrelevant load elim & speculative load intro 
     \okcell & \okcell &
     % roach motel 
     \okcell & 
     % strength.
     \unkwcell &
     % trace preserving 
     \okcell &
     % common subexpression elimination
     \badcell &
     % global: register promotion, value range, thread inlining 
     \okcell & \unkwcell & \okcell & 
     % DRF, COH, no-UB, no-OOTA
     \ldrf & \okcell & \okcell & \okcell &
     % NA, RLX, RA, SC 
     \okcell & \badcell & \badcell & \okcell & 
     % F-RA, F-SC
     \badcell & \badcell & 
     % RMW, Lock
     \okcell & \badcell &
     % Mix.sz. 
     \badcell 
     \\ \cline{2-36}

 & DRFx
     &           
     % compilation 
     \badcell & \badcell & \badcell & \badcell & 
     % reorderings
     \badcell & \badcell & \badcell & \badcell & 
     % eliminations
     \okcell & \okcell & \okcell & \okcell &
     % irrelevant load elim & speculative load intro 
     \okcell & \badcell &
     % roach motel 
     \okcell & 
     % strength.
     \unkwcell &
     % trace preserving 
     \okcell &
     % common subexpression elimination
     \badcell &
     % global: register promotion, value range, thread inlining 
     \okcell & \unkwcell & \okcell &                                              
     % DRF, COH, no-UB, no-OOTA
     \ldrf & \okcell & \warncell & \okcell &
     % NA, RLX, RA, SC 
     \okcell & \badcell & \badcell & \okcell & 
     % F-RA, F-SC
     \badcell & \badcell & 
     % RMW, Lock
     \badcell & \badcell & 
     % Mix.sz.
     \badcell 
     \\ \Xhline{2\arrayrulewidth}

 \multirow{2}{*}{\rotatebox[origin=c]{270}{\makecell{TSO\\PSO}}}   

 & BMM
     &
     % compilation 
     \okcell & \badcell & \badcell & \badcell & 
     % reorderings
     \okcell & \badcell & \badcell & \badcell & 
     % eliminations
     \okcell & \okcell & \okcell & \badcell &  
     % irrelevant load elim & speculative load intro 
     \okcell & \okcell &
     % roach motel 
     \badcell & 
     % strength.
     \unkwcell &
     % trace preserving 
     \okcell &
     % common subexpression elimination
     \badcell &
     % global: register promotion, value range, thread inlining 
     \unkwcell & \unkwcell & \badcell &
     % DRF, COH, no-UB, no-OOTA
     \edrf & \okcell & \okcell & \okcell &
     % NA, RLX, RA, SC 
     \okcell & \badcell & \badcell & \okcell & 
     % F-RA, F-SC
     \badcell & \badcell & 
     % RMW, Lock
     \badcell & \okcell & 
     % Mix.sz.
     \badcell 
     \\ \cline{2-36}

 & RMMOA
     &
     % compilation 
     \okcell & \badcell & \badcell & \badcell & 
     % reorderings
     \okcell & \okcell & \badcell & \badcell & 
     % eliminations
     \unkwcell & \unkwcell & \unkwcell & \unkwcell &  
     % irrelevant load elim & speculative load intro 
     \unkwcell & \unkwcell &
     % roach motel 
     \unkwcell & 
     % strength.
     \unkwcell &
     % trace preserving 
     \unkwcell &
     % common subexpression elimination
     \unkwcell &
     % global: register promotion, value range, thread inlining 
     \unkwcell & \unkwcell & \unkwcell &
     % DRF, COH, no-UB, no-OOTA
     \edrf & \unkwcell & \okcell & \okcell &
     % NA, RLX, RA, SC 
     \okcell & \badcell & \badcell & \badcell & 
     % F-RA, F-SC
     \badcell & \badcell & 
     % RMW, Lock
     \badcell & \okcell & 
     % Mix.sz.
     \badcell 
     \\ \Xhline{2\arrayrulewidth}

 \multirow{5}{*}{\rotatebox[origin=c]{270}{\makecell{$\lPO\lRF$\\acyc}}}   

 & RC11
     &
     % compilation 
     \okcell & \badcell & \badcell & \badcell &  
     % reorderings
     \okcell & \okcell & \okcell & \badcell & 
     % eliminations
     \okcell & \okcell & \okcell & \badcell & 
     % irrelevant load elim & speculative load intro 
     \unkwcell & \badcell &
     % roach motel 
     \okcell & 
     % strength.
     \okcell &
     % trace preserving 
     \okcell &
     % common subexpression elimination
     \badcell &
     % global: register promotion, value range, thread inlining 
     \okcell & \unkwcell & \okcell &                                              
     % DRF, COH, no-UB, no-OOTA
     \edrf & \okcell & \warncell & \okcell &
     % NA, RLX, RA, SC 
     \okcell & \okcell & \okcell & \okcell & 
     % F-RA, F-SC
     \okcell & \okcell & 
     % RMW, Lock
     \okcell & \badcell & 
     % Mix.sz.
     \badcell 
     \\ \cline{2-36}

 & ORC11
     &
     % compilation 
     \okcell & \badcell & \badcell & \badcell &  
     % reorderings
     \okcell & \okcell & \okcell & \badcell & 
     % eliminations
     \okcell & \okcell & \okcell & \badcell & 
     % irrelevant load elim & speculative load intro 
     \unkwcell & \badcell &
     % roach motel 
     \okcell & 
     % strength.
     \okcell &
     % trace preserving 
     \okcell &
     % common subexpression elimination
     \badcell &
     % global: register promotion, value range, thread inlining 
     \okcell & \unkwcell & \okcell &
     % DRF, COH, no-UB, no-OOTA
     \edrf & \okcell & \warncell & \okcell &
     % NA, RLX, RA, SC 
     \okcell & \okcell & \okcell & \badcell & 
     % F-RA, F-SC
     \okcell & \badcell & 
     % RMW, Lock
     \okcell & \badcell & 
     % Mix.sz.
     \badcell 
     \\ \cline{2-36}

 & RAR
     & 
     % compilation 
     \okcell & \badcell & \badcell & \badcell &  
     % reorderings
     \okcell & \okcell & \okcell & \badcell & 
     % eliminations
     \okcell & \okcell & \okcell & \badcell & 
     % irrelevant load elim & speculative load intro 
     \unkwcell & \badcell &
     % roach motel 
     \okcell & 
     % strength.
     \okcell &
     % trace preserving 
     \okcell &
     % common subexpression elimination
     \badcell &
     % global: register promotion, value range, thread inlining 
     \okcell & \unkwcell & \okcell &
     % DRF, COH, no-UB, no-OOTA
     \edrf & \okcell & \okcell & \okcell &
     % NA, RLX, RA, SC 
     \badcell & \okcell & \okcell & \badcell & 
     % F-RA, F-SC
     \badcell & \badcell & 
     % RMW, Lock
     \okcell & \badcell & 
     % Mix.sz.
     \badcell 
     \\ \cline{2-36}

 & OCMM
     & 
     % compilation 
     \okcell & \badcell & \badcell & \badcell &  
     % reorderings
     \okcell & \okcell & \okcell & \badcell & 
     % eliminations
     \okcell & \okcell & \okcell & \badcell & 
     % irrelevant load elim & speculative load intro  
     \unkwcell & \unkwcell &
     % roach motel 
     \unkwcell & 
     % strength.
     \unkwcell &
     % trace preserving 
     \okcell &
     % common subexpression elimination
     \okcell &
     % global: register promotion, value range, thread inlining 
     \unkwcell & \unkwcell & \unkwcell &
     % DRF, COH, no-UB, no-OOTA
     \ldrf & \warncell & \okcell & \okcell &
     % NA, RLX, RA, SC 
     \okcell & \badcell & \badcell & \okcell & 
     % F-RA, F-SC
     \badcell & \badcell & 
     % RMW, Lock
     \okcell & \badcell & 
     % Mix.sz.
     \badcell 
     \\ \cline{2-36}

 & JAM
     & 
     % compilation 
     \okcell & \badcell & \badcell & \badcell &  
     % reorderings
     \unkwcell & \unkwcell & \unkwcell & \unkwcell &  
     % eliminations
     \unkwcell & \unkwcell & \unkwcell & \unkwcell & 
     % irrelevant load elim & speculative load intro  
     \unkwcell & \unkwcell &
     % roach motel 
     \unkwcell & 
     % strength.
     \okcell &
     % trace preserving 
     \unkwcell &
     % common subexpression elimination
     \badcell &
     % global: register promotion, value range, thread inlining 
     \unkwcell & \unkwcell & \unkwcell &
     % DRF, COH, no-UB, no-OOTA
     \edrf & \okcell & \okcell & \okcell &
     % NA, RLX, RA, SC 
     \okcell & \okcell & \okcell & \okcell & 
     % F-RA, F-SC
     \okcell & \okcell & 
     % RMW, Lock
     \okcell & \badcell & 
     % Mix.sz.
     \badcell 
     \\ \cline{2-36}

 & CRC
     &
     % compilation 
     \okcell & \badcell & \badcell & \badcell &  
     % reorderings
     \okcell & \badcell & \badcell & \badcell &  
     % eliminations
     \okcell & \okcell & \okcell & \badcell &  
     % irrelevant load elim & speculative load intro  
     \badcell & \badcell &
     % roach motel 
     \unkwcell & 
     % strength.
     \unkwcell &
     % trace preserving 
     \unkwcell &
     % common subexpression elimination
     \badcell &
     % global: register promotion, value range, thread inlining 
     \unkwcell & \unkwcell & \unkwcell &
     % DRF, COH, no-UB, no-OOTA
     \edrf & \okcell & \warncell & \okcell &
     % NA, RLX, RA, SC 
     \okcell & \badcell & \okcell & \badcell & 
     % F-RA, F-SC
     \badcell & \okcell & 
     % RMW, Lock
     \okcell & \badcell & 
     % Mix.sz.
     \badcell 
     \\ \Xhline{2\arrayrulewidth}

 \multirow{2}{*}{\rotatebox[origin=c]{270}{\makecell{$\lPPO\lRF$\\acyc}}}   

 & LKMM
     &           
     % compilation 
     \okcell & \okcell & \okcell & \okcell &  
     % reorderings
     \okcell & \okcell & \okcell & \okcell &
     % eliminations
     \unkwcell & \unkwcell & \unkwcell & \unkwcell &  
     % irrelevant load elim & speculative load intro 
     \unkwcell & \unkwcell &
     % roach motel 
     \unkwcell & 
     % strength.
     \unkwcell &
     % trace preserving 
     \badcell &
     % common subexpression elimination
     \badcell &
     % global: register promotion, value range, thread inlining 
     \unkwcell & \unkwcell & \unkwcell &
     % DRF, COH, no-UB, no-OOTA
     \edrf & \okcell & \okcell & \okcell &
     % NA, RLX, RA, SC 
     \badcell & \okcell & \okcell & \badcell & 
     % F-RA, F-SC
     \okcell & \okcell & 
     % RMW, Lock
     \okcell & \badcell & 
     % Mix.sz.
     \badcell 
     \\ \cline{2-36}

 & OHMM
     &
     % compilation 
     \unkwcell & \unkwcell & \unkwcell & \unkwcell &
     % reorderings
     \okcell & \okcell & \okcell & \okcell &
     % eliminations
     \okcell & \okcell & \okcell & \okcell &
     % irrelevant load elim & speculative load intro 
     \okcell & \okcell &
     % roach motel 
     \okcell & 
     % strength.
     \unkwcell &
     % trace preserving 
     \badcell &
     % common subexpression elimination
     \unkwcell &
     % global: register promotion, value range, thread inlining 
     \unkwcell & \unkwcell & \unkwcell &         
     % DRF, COH, no-UB, no-OOTA
     \edrf & \unkwcell & \okcell & \okcell &
     % NA, RLX, RA, SC 
     \okcell & \badcell & \badcell & \okcell & 
     % F-RA, F-SC
     \badcell & \badcell & 
     % RMW, Lock
     \badcell & \okcell & 
     % Mix.sz.
     \badcell 
     \\ \Xhline{2\arrayrulewidth}

 \multirow{7}{*}{\rotatebox[origin=c]{270}{\makecell{no-OOTA}}}   

 & JMM
     &            
     % compilation 
     & & & &
     % reorderings
     \okcell & \okcell & \okcell & \okcell &
     % eliminations
     \okcell & \okcell & \badcell & \badcell &
     % irrelevant load elim & speculative load intro  
     \okcell & \badcell &
     % roach motel 
     \badcell & 
     % strength.
     \unkwcell &
     % trace preserving 
     \okcell &
     % common subexpression elimination
     \badcell &
     % global: register promotion, value range, thread inlining 
     \unkwcell & \unkwcell & \badcell &
     % DRF, COH, no-UB, no-OOTA
     \edrf & \warncell & \okcell & \okcell &
     % NA, RLX, RA, SC 
     \okcell & \badcell & \badcell & \okcell & 
     % F-RA, F-SC
     \badcell & \badcell & 
     % RMW, Lock
     \okcell & \okcell & 
     % Mix.sz.
     \badcell 
     \\ \cline{2-36}

 & Prm
     &
     % compilation 
     \okcell & \okcell & \okcell & \okcell &  
     % reorderings
     \okcell & \okcell & \okcell & \okcell &
     % eliminations
     \okcell & \okcell & \okcell & \okcell &  
     % irrelevant load elim & speculative load intro  
     \okcell & \okcell &
     % roach motel 
     \okcell & 
     % strength.
     \okcell &
     % trace preserving 
     \okcell &
     % common subexpression elimination
     \okcell &
     % global: register promotion, value range, thread inlining 
     \okcell & \okcell & \badcell &
     % DRF, COH, no-UB, no-OOTA
     \edrf & \okcell & \okcell & \okcell &
     % NA, RLX, RA, SC 
     \okcell & \okcell & \okcell & \badcell & 
     % F-RA, F-SC
     \okcell & \okcell & 
     % RMW, Lock
     \okcell & \okcell & 
     % Mix.sz.
     \badcell 
     \\ \cline{2-36}

 & Wkmo
     &
     % compilation 
     \okcell & \okcell & \okcell & \okcell &
     % reorderings
     \okcell & \okcell & \okcell & \okcell &
     % eliminations
     \okcell & \okcell & \okcell & \okcell &  
     % irrelevant load elim & speculative load intro  
     \unkwcell & \okcell &
     % roach motel 
     \badcell & 
     % strength.
     \okcell &
     % trace preserving 
     \unkwcell &
     % common subexpression elimination
     \badcell &
     % global: register promotion, value range, thread inlining 
     \unkwcell & \unkwcell & \badcell &
     % DRF, COH, no-UB, no-OOTA
     \edrf & \okcell & \okcell & \okcell &
     % NA, RLX, RA, SC 
     \okcell & \okcell & \okcell & \okcell & 
     % F-RA, F-SC
     \okcell & \okcell & 
     % RMW, Lock
     \okcell & \badcell & 
     % Mix.sz.
     \badcell 
     \\ \cline{2-36}

 & MRD
     &
     % compilation 
     \okcell & \okcell & \okcell & \okcell &
     % reorderings
     \unkwcell & \unkwcell & \unkwcell & \unkwcell &
     % eliminations
     \unkwcell & \unkwcell & \unkwcell & \unkwcell &
     % irrelevant load elim & speculative load intro 
     \unkwcell & \unkwcell &
     % roach motel 
     \unkwcell & 
     % strength.
     \unkwcell &
     % trace preserving 
     \unkwcell &
     % common subexpression elimination
     \unkwcell &
     % global: register promotion, value range, thread inlining 
     \unkwcell & \unkwcell & \unkwcell &                                              
     % DRF, COH, no-UB, no-OOTA
     \edrf & \okcell & \okcell & \okcell &
     % NA, RLX, RA, SC 
     \badcell & \okcell & \badcell & \badcell & 
     % F-RA, F-SC
     \badcell & \badcell & 
     % RMW, Lock
     \badcell & \okcell & 
     % Mix.sz.
     \badcell 
     \\ \cline{2-36}

 & P-P/S
     &
     % compilation 
     \okcell & \okcell & \badcell & \badcell &
     % reorderings 
     \okcell & \okcell & \okcell & \okcell &
     % eliminations
     \okcell & \okcell & \okcell & \badcell &  
     % irrelevant load elim & speculative load intro  
     \okcell & \badcell &
     % roach motel 
     \okcell & 
     % strength.
     \badcell &
     % trace preserving 
     \unkwcell &
     % common subexpression elimination
     \badcell &
     % global: register promotion, value range, thread inlining 
     \unkwcell & \okcell & \badcell &
     % DRF, COH, no-UB, no-OOTA
     \unkwcell & \okcell & \warncell & \okcell &
     % NA, RLX, RA, SC 
     \okcell & \okcell & \okcell & \badcell & 
     % F-RA, F-SC
     \badcell & \badcell & 
     % RMW, Lock
     \okcell & \okcell & 
     % Mix.sz.
     \badcell 
     \\ \cline{2-36}

 & J/R
     &
     % compilation 
     \unkwcell & \badcell & \badcell & \badcell &
     % reorderings
     \unkwcell & \unkwcell & \badcell & \unkwcell &
     % eliminations
     \unkwcell & \unkwcell & \badcell & \badcell &  
     % irrelevant load elim & speculative load intro 
     \unkwcell & \unkwcell &
     % roach motel 
     \unkwcell & 
     % strength.
     \unkwcell &
     % trace preserving 
     \unkwcell &
     % common subexpression elimination
     \unkwcell &
     % global: register promotion, value range, thread inlining 
     \unkwcell & \unkwcell & \unkwcell &                                              
     % DRF, COH, no-UB, no-OOTA
     \edrf & \warncell & \okcell & \okcell &
     % NA, RLX, RA, SC 
     \badcell & \okcell & \badcell & \badcell & 
     % F-RA, F-SC
     \badcell & \badcell & 
     % RMW, Lock
     \okcell & \okcell & 
     % Mix.sz.
     \badcell 
     \\ \cline{2-36}

 & GOS
     &
     % compilation 
     \unkwcell & \unkwcell & \unkwcell & \unkwcell &
     % reorderings 
     \unkwcell & \unkwcell & \unkwcell & \unkwcell &
     % eliminations 
     \unkwcell & \unkwcell & \unkwcell & \unkwcell &
     % irrelevant load elim & speculative load intro 
     \unkwcell & \unkwcell &
     % roach motel 
     \unkwcell & 
     % strength.
     \unkwcell &
     % trace preserving 
     \unkwcell &
     % common subexpression elimination
     \unkwcell &
     % global: register promotion, value range, thread inlining 
     \unkwcell & \unkwcell & \unkwcell &                                              
     % DRF, COH, no-UB, no-OOTA
     \edrf & \unkwcell & \okcell & \okcell &
     % NA, RLX, RA, SC 
     \okcell & \badcell & \badcell & \badcell & 
     % F-RA, F-SC
     \badcell & \badcell & 
     % RMW, Lock
     \badcell & \okcell & 
     % Mix.sz.
     \badcell 
     \\ \Xhline{2\arrayrulewidth}

 \multirow{5}{*}{\rotatebox[origin=c]{270}{\makecell{OOTA}}}   

 & C11
     &            
     % compilation 
     \okcell & \okcell & \okcell & \okcell &
     % reorderings
     \okcell & \okcell & \okcell & \okcell &
     % eliminations
     \okcell & \okcell & \okcell & \badcell &  
     % irrelevant load elim & speculative load intro 
     \unkwcell & \badcell &
     % roach motel 
     \badcell & 
     % strength.
     \badcell &
     % trace preserving 
     \okcell &
     % common subexpression elimination
     \badcell &
     % global: register promotion, value range, thread inlining 
     \unkwcell & \unkwcell & \badcell &
     % DRF, COH, no-UB, no-OOTA
     \idrf & \okcell & \warncell & \badcell &
     % NA, RLX, RA, SC 
     \okcell & \okcell & \okcell & \okcell & 
     % F-RA, F-SC
     \okcell & \okcell & 
     % RMW, Lock
     \okcell & \okcell & 
     % Mix.sz.
     \okcell 
     \\ \cline{2-36}

 & JSMM
     &
     % compilation 
     \okcell & \okcell & \okcell & \okcell &
     % reorderings
     \unkwcell & \unkwcell & \unkwcell & \unkwcell &
     % eliminations
     \unkwcell & \unkwcell & \unkwcell & \unkwcell &
     % irrelevant load elim & speculative load intro 
     \unkwcell & \unkwcell &
     % roach motel 
     \unkwcell & 
     % strength.
     \unkwcell &
     % trace preserving 
     \unkwcell &
     % common subexpression elimination
     \unkwcell &
     % global: register promotion, value range, thread inlining 
     \unkwcell & \unkwcell & \unkwcell &
     % DRF, COH, no-UB, no-OOTA
     \idrf & & \okcell & \badcell &
     % NA, RLX, RA, SC 
     \okcell & \badcell & \badcell & \okcell & 
     % F-RA, F-SC
     \badcell & \badcell & 
     % RMW, Lock
     \okcell & \okcell & 
     % Mix.sz.
     \okcell 
     \\ \cline{2-36}

 & RMC
     &
     % compilation 
     \okcell & \okcell & \okcell & \okcell &
     % reorderings
     \unkwcell & \unkwcell & \unkwcell & \unkwcell &
     % eliminations
     \unkwcell & \unkwcell & \unkwcell & \unkwcell &
     % irrelevant load elim & speculative load intro 
     \unkwcell & \unkwcell &
     % roach motel 
     \unkwcell & 
     % strength.
     \unkwcell &
     % trace preserving 
     \unkwcell &
     % common subexpression elimination
     \unkwcell &
     % global: register promotion, value range, thread inlining 
     \unkwcell & \unkwcell & \unkwcell &
     % DRF, COH, no-UB, no-OOTA
     \idrf & \okcell & \okcell & \badcell &
     % NA, RLX, RA, SC 
     \badcell & \okcell & \okcell & \okcell & 
     % F-RA, F-SC
     \okcell & \okcell & 
     % RMW, Lock
     \okcell & \badcell & 
     % Mix.sz.
     \badcell 
     \\ \cline{2-36}


 & RAO
     &
     % compilation 
     \unkwcell & \unkwcell & \unkwcell & \unkwcell &
     % reorderings
     \unkwcell & \unkwcell & \unkwcell & \unkwcell &
     % eliminations
     \unkwcell & \unkwcell & \unkwcell & \unkwcell &
     % irrelevant load elim & speculative load intro 
     \unkwcell & \unkwcell &
     % roach motel 
     \unkwcell & 
     % strength.
     \unkwcell &
     % trace preserving 
     \unkwcell &
     % common subexpression elimination
     \unkwcell &
     % global: register promotion, value range, thread inlining 
     \unkwcell & \unkwcell & \unkwcell & 
     % DRF, COH, no-UB, no-OOTA
      & \warncell & \okcell & \badcell &
     % NA, RLX, RA, SC 
     \okcell & \badcell & \badcell & \okcell & 
     % F-RA, F-SC
     \badcell & \badcell & 
     % RMW, Lock
     \badcell & \badcell & 
     % Mix.sz.
     \badcell 
     \\ \cline{2-36}

 & TSC
     % compilation 
     & \unkwcell & \unkwcell & \unkwcell & \unkwcell &
     % reorderings
     \unkwcell & \unkwcell & \unkwcell & \unkwcell &
     % eliminations
     \unkwcell & \unkwcell & \unkwcell & \unkwcell &
     % irrelevant load elim & speculative load intro 
     \unkwcell & \unkwcell &
     % roach motel 
     \unkwcell & 
     % strength.
     \unkwcell &
     % trace preserving 
     \unkwcell &
     % common subexpression elimination
     \unkwcell &
     % global: register promotion, value range, thread inlining 
     \unkwcell & \unkwcell & \unkwcell & 
     % DRF, COH, no-UB, no-OOTA
     \idrf & \unkwcell & \okcell & \badcell &
     % NA, RLX, RA, SC 
     \okcell & \badcell & \badcell & \okcell & 
     % F-RA, F-SC
     \badcell & \badcell & 
     % RMW, Lock
     \badcell & \okcell & 
     % Mix.sz.
     \badcell 
     \\ \Xhline{2\arrayrulewidth}


\end{tabular}
\caption{
  % \textit{T.P.} --- trace preserving.
  % \textit{R.I.} --- reordering of independent instructions.
  % \textit{R.E.} --- redundunt load/store elimination.
  % \textit{I.L.E.} --- irrelevant load elimination.
  % \textit{S.L.I.} --- speculative load introduction.
  % \textit{S.} --- strengthening.
  % \textit{R.M.} --- roach motel reordering.
  % \textit{R.P.} --- register promotion.
  % \textit{V.R.} --- value range analysis based optimizations.
  % \textit{T.I.} --- thread inlining (sequentialization).
  % \textit{Int.} --- internal.
  % \textit{Ext.} --- external.
  % \textit{Loc.} --- local.
  % \textit{UB} --- undefined behavior.
  % \textit{OOTA} --- out-of-thin air values.
  % \textit{Mix.Sz.} --- mixed-size accesses.
}
\label{table:cmp-mms}
\end{table*}

\end{landscape}

\twocolumn



\begin{table}[t]

\newcommand{\rotateAngle}{270}
\newcommand{\lastcol}{11}

\def\arraystretch{1}
\setlength\tabcolsep{2pt}

\begin{tabular}{|c|l|c|c|c|c|c|c|c|c|c|} 

  \hline

  \multirow{2}{*}{Class}          &
  \multirow{2}{*}{Model}          &
  \multicolumn{ 9}{c|}{Features}   
  \\ 

  \cline{3-\lastcol}

                                                            &
                                                            &
  \rotatebox[origin=c]{\rotateAngle}{NA}                    &
  \rotatebox[origin=c]{\rotateAngle}{RLX}                   &
  \rotatebox[origin=c]{\rotateAngle}{RA}                    &
  \rotatebox[origin=c]{\rotateAngle}{SC}                    &
  \rotatebox[origin=c]{\rotateAngle}{F-RA}                  &
  \rotatebox[origin=c]{\rotateAngle}{F-SC}                  &
  \rotatebox[origin=c]{\rotateAngle}{RMW}                   &
  \rotatebox[origin=c]{\rotateAngle}{LK}                    &
  \rotatebox[origin=c]{\rotateAngle}{MIX}      
  \\[9pt] 

  \Xhline{2\arrayrulewidth}

  \multirow{4}{*}{\clsSC}

  & \EtESC
     & 
     % NA, RLX, RA, SC 
     \badcell & \badcell & \badcell & \okcell & 
     % F-RA, F-SC
     \badcell & \badcell & 
     % RMW, LK
     \badcell & \badcell & 
     % MIX
     \badcell 
     \\ \cline{2-\lastcol}

  & \VbD
     & 
     % NA, RLX, RA, SC 
     \okcell & \badcell & \badcell & \okcell & 
     % F-RA, F-SC
     \badcell & \badcell & 
     % RMW, LK
     \okcell & \okcell & 
     % MIX
     \badcell 
     \\ \cline{2-\lastcol}

  & \SCHs 
     &
     % NA, RLX, RA, SC 
     \okcell & \badcell & \badcell & \okcell & 
     % F-RA, F-SC
     \badcell & \badcell & 
     % RMW, LK
     \okcell & \badcell &
     % MIX 
     \badcell 
     \\ \cline{2-\lastcol}

  & \DRFx
     &           
     % NA, RLX, RA, SC 
     \okcell & \badcell & \badcell & \okcell & 
     % F-RA, F-SC
     \badcell & \badcell & 
     % RMW, LK
     \badcell & \badcell & 
     % MIX
     \badcell 
     \\ \Xhline{2\arrayrulewidth}

  \multirow{2}{*}{\clsTSO}

  & \BMM
     &
     % NA, RLX, RA, SC 
     \okcell & \badcell & \badcell & \okcell & 
     % F-RA, F-SC
     \badcell & \badcell & 
     % RMW, Lock
     \badcell & \okcell & 
     % Mix.sz.
     \badcell 
     \\ \cline{2-\lastcol}

  & \RMMOA
     &
     % NA, RLX, RA, SC 
     \okcell & \badcell & \badcell & \badcell & 
     % F-RA, F-SC
     \badcell & \badcell & 
     % RMW, Lock
     \badcell & \okcell & 
     % Mix.sz.
     \badcell 
     \\ \Xhline{2\arrayrulewidth}

  \multirow{6}{*}{\clsPO}   

  & \RCMM
     &
     % NA, RLX, RA, SC 
     \okcell & \okcell & \okcell & \okcell & 
     % F-RA, F-SC
     \okcell & \okcell & 
     % RMW, Lock
     \okcell & \badcell & 
     % Mix.sz.
     \badcell 
     \\ \cline{2-\lastcol}

  & \ORCMM
     &
     % NA, RLX, RA, SC 
     \okcell & \okcell & \okcell & \badcell & 
     % F-RA, F-SC
     \okcell & \badcell & 
     % RMW, Lock
     \okcell & \badcell & 
     % Mix.sz.
     \badcell 
     \\ \cline{2-\lastcol}

  & \RAR
     & 
     % NA, RLX, RA, SC 
     \badcell & \okcell & \okcell & \badcell & 
     % F-RA, F-SC
     \badcell & \badcell & 
     % RMW, Lock
     \okcell & \badcell & 
     % Mix.sz.
     \badcell 
     \\ \cline{2-\lastcol}

  & \CRC
     &
     % NA, RLX, RA, SC 
     \okcell & \badcell & \okcell & \badcell & 
     % F-RA, F-SC
     \badcell & \okcell & 
     % RMW, Lock
     \okcell & \badcell & 
     % Mix.sz.
     \badcell 
     \\ \cline{2-\lastcol}

  & \OCMM
     & 
     % NA, RLX, RA, SC 
     \okcell & \badcell & \badcell & \okcell & 
     % F-RA, F-SC
     \badcell & \badcell & 
     % RMW, Lock
     \okcell & \badcell & 
     % Mix.sz.
     \badcell 
     \\ \cline{2-\lastcol}

  & \JAM
     &
     % NA, RLX, RA, SC 
     \okcell & \okcell & \okcell & \okcell & 
     % F-RA, F-SC
     \okcell & \okcell & 
     % RMW, Lock
     \okcell & \badcell & 
     % Mix.sz.
     \badcell 
     \\ \Xhline{2\arrayrulewidth}

  \multirow{2}{*}{\clsSyDEP}   

  & \LKMM
     &           
     % NA, RLX, RA, SC 
     \badcell & \okcell & \okcell & \badcell & 
     % F-RA, F-SC
     \okcell & \okcell & 
     % RMW, Lock
     \okcell & \badcell & 
     % Mix.sz.
     \badcell 
     \\ \cline{2-\lastcol}

  & \OHMM
     &
     % NA, RLX, RA, SC 
     \okcell & \badcell & \badcell & \okcell & 
     % F-RA, F-SC
     \badcell & \badcell & 
     % RMW, Lock
     \badcell & \okcell & 
     % Mix.sz.
     \badcell 
     \\ \Xhline{2\arrayrulewidth}

  \multirow{7}{*}{\clsSemDEP}   

  & \JMM
     &            
     % NA, RLX, RA, SC 
     \okcell & \badcell & \badcell & \okcell & 
     % F-RA, F-SC
     \badcell & \badcell & 
     % RMW, Lock
     \okcell & \okcell & 
     % Mix.sz.
     \badcell 
     \\ \cline{2-\lastcol}

  & \PRM
     &
     % NA, RLX, RA, SC 
     \okcell & \okcell & \okcell & \badcell & 
     % F-RA, F-SC
     \okcell & \okcell & 
     % RMW, Lock
     \okcell & \okcell & 
     % Mix.sz.
     \badcell 
     \\ \cline{2-\lastcol}

  & \WMO
     &
     % NA, RLX, RA, SC 
     \okcell & \okcell & \okcell & \okcell & 
     % F-RA, F-SC
     \okcell & \okcell & 
     % RMW, Lock
     \okcell & \badcell & 
     % Mix.sz.
     \badcell 
     \\ \cline{2-\lastcol}

  & \CSRA
     &
     % NA, RLX, RA, SC 
     \okcell & \okcell & \okcell & \badcell & 
     % F-RA, F-SC
     \badcell & \badcell & 
     % RMW, Lock
     \okcell & \okcell & 
     % Mix.sz.
     \badcell 
     \\ \cline{2-\lastcol}

  & \WJES
     &
     % NA, RLX, RA, SC 
     \badcell & \okcell & \badcell & \badcell & 
     % F-RA, F-SC
     \badcell & \badcell & 
     % RMW, Lock
     \okcell & \okcell & 
     % Mix.sz.
     \badcell 
     \\ \cline{2-\lastcol}

  & \MRD
     &
     % NA, RLX, RA, SC 
     \badcell & \okcell & \badcell & \badcell & 
     % F-RA, F-SC
     \badcell & \badcell & 
     % RMW, Lock
     \badcell & \okcell & 
     % Mix.sz.
     \badcell 
     \\ \cline{2-\lastcol}

  & \GOS
     &
     % NA, RLX, RA, SC 
     \okcell & \badcell & \badcell & \badcell & 
     % F-RA, F-SC
     \badcell & \badcell & 
     % RMW, Lock
     \badcell & \okcell & 
     % Mix.sz.
     \badcell 
     \\ \Xhline{2\arrayrulewidth}

  \multirow{5}{*}{\clsOOTA}   

  & \CMM
     &            
     % NA, RLX, RA, SC 
     \okcell & \okcell & \okcell & \okcell & 
     % F-RA, F-SC
     \okcell & \okcell & 
     % RMW, Lock
     \okcell & \okcell & 
     % Mix.sz.
     \okcell 
     \\ \cline{2-\lastcol}

  & \JSMM
     &
     % NA, RLX, RA, SC 
     \okcell & \badcell & \badcell & \okcell & 
     % F-RA, F-SC
     \badcell & \badcell & 
     % RMW, Lock
     \okcell & \okcell & 
     % Mix.sz.
     \okcell 
     \\ \cline{2-\lastcol}

  & \RMC
     &
     % NA, RLX, RA, SC 
     \badcell & \okcell & \okcell & \okcell & 
     % F-RA, F-SC
     \okcell & \okcell & 
     % RMW, Lock
     \okcell & \badcell & 
     % Mix.sz.
     \badcell 
     \\ \cline{2-\lastcol}


  & \RAO
     &
     % NA, RLX, RA, SC 
     \okcell & \badcell & \badcell & \okcell & 
     % F-RA, F-SC
     \badcell & \badcell & 
     % RMW, Lock
     \badcell & \badcell & 
     % Mix.sz.
     \badcell 
     \\ \cline{2-\lastcol}

  & \TSC
     &
     % NA, RLX, RA, SC 
     \okcell & \badcell & \badcell & \okcell & 
     % F-RA, F-SC
     \badcell & \badcell & 
     % RMW, Lock
     \badcell & \okcell & 
     % Mix.sz.
     \badcell 
     \\ \Xhline{2\arrayrulewidth}

\end{tabular}

\iftoggle{langeng}{%
  \caption{Features supported by memory models}
}{%
  \caption{Поддержка примитивов моделями памяти}
}


\label{table:features}

\end{table}

\subsection{Sequential Consistency}
\label{sec:catalog:sc}

We start with a description of several attempts 
to adopt sequential consistency as memory model for 
existing languages and runtimes. 
Most of the proposed solutions share similar properties, 
and thus in \cref{table:cmp-mms} we unite them 
in a single row under the name \SC. 
The only exception is \DRFx model which implements
catch-fire semantics for racy programs 
and thus has a slightly different properties. 

\paragraph{End-to-end Sequential Consistency}

Marino et al~\cite{Marino-al:PLDI11, Singh-al:ISCA12} 
examined the performance penalties to ensure end-to-end SC
enforced by (1) modified SC-preserving version 
of \LLVM compiler infrastructure and 
(2) a modified version of x86-TSO hardware. 
To mitigate the induced overhead the authors 
utilized the observation that hardware need to 
enforce SC only for memory accesses to shared mutable variables. 
To classify the memory regions as either thread-local,
shared immutable, or shared mutable they have used 
a combination of static compiler analysis and 
dynamic analysis powered by modified hardware. 
They evaluated their approach on a number of benchmarks
and reported performance overhead of 6.2\% on average 
and ~17\% in maximum, compared to stock \LLVM compiler 
and regular x86 hardware. 

\paragraph{Volatile-by-default}

Liu~\etal~\cite{Liu-al:OOPSLA17, Liu-al:PLDI19} studied 
sequential consistency in the context of \Java.  
They proposed a \emph{volatile-by-default} semantics,
where each memory access is treated as volatile 
(\ie sequentially consistent) by default. 
The experiments showed a considerable performance penalty:
28\% slowdown on average with 81\% in maximum on x86 hardware,
and 57\% slowdown on average with 157\% in maximum on \ARMv{8} hardware.
The authors tried to mitigate performance overhead and presented  
a novel optimization technique for language-level SC
compatible with \emph{just-in-time} compilation. 
They propose to treat each object as thread-local speculatively 
and compile memory accesses without fences. 
If later during the execution concurrent accesses to the object  
are detected, the code is recompiled to place necessary fences.
A modified version of \JVM which implements the technique
described above was reported to incur 37\% slowdown on average 
with 73\% in maximum on \ARMv{8} hardware.

\paragraph{SC-Haskell}

SC-Haskell memory model~\cite{Vollmer-al:PPoPP17}
were inspired by the similar idea of separation
between the thread-local and shared mutable memory. 
To safely distinguish between the two 
the authors utilized the powerful strong type system of Haskell. 
The consequence of this approach is that the 
programmers need to follow a stricter discipline 
in order to please the type checker. 
The authors modified the GHC to preserve SC 
and then run 1,279 benchmarks on x86-64 hardware
to measure the performance penalties.
They reported 0.4\% geometric mean slowdown,
and noticed that only 12 benchmarks experienced 
slowdown greater than 10\%.

\paragraph{\DRFx}

The \DRFx~\cite{Marino-al:PLDI10} is another 
SC preserving memory model. In this model
the runtime system is guaranteed to raise 
an exception if the program has data-races, 
and yield only sequentially consistent outcomes otherwise.
In order to make the runtime data-race detection feasible 
in practice, the authors propose several modifications 
to existing hardware.

The authors claim that any sequentially valid optimization 
(\eg instruction reorderings or common subexpression elimination),
is sound in \DRFx model, the only limitation is that
these transformations can only be performed
withing the bounds of compiler-designated program regions.
Besides that any transformation that introduces 
speculative reads or writes is unsound,
since speculative optimizations can bring
data-races into otherwise race-free programs.

Note that in \cref{table:cmp-mms} we still do not consider 
many of the transformations that claimed to be sound by the authors
as actually being sound because of our convention described in \cref{sec:comparison}.
We consider transformations sound only if they are 
sound in a fully-defined semantics. 
The \DRFx model does not meet this criterion as 
it provides catch-fire semantics.

The expected performance overhead of the model 
is reported to be 3.25\% on average
assuming the efficient implementation 
of data-race detection in hardware. 
(compared to stock compiler and x86 hardware). 

\subsection{Total and Partial Store Order}

In this section we consider PL memory models 
inspired by \TSO and \PSO.  

\paragraph{Buffered Memory Model}

Demange et al.~\cite{Demange-al:POPL13} presented 
the \emph{Buffered Memory Model} (or \BMM in short)
as a candidate model for Java language.
Their motivation, however, stemmed not from the desire 
to fully replace the Java Memory Model, but rather 
from the goal to build a verified version of 
Java Virtual Machine (akin to CompCertTSO project~\cite{Sevcik-al:JACM13}).
A more simple yet pragmatic memory \TSO model 
was considered as a first step to achieve this goal. 

The authors proved soundness of several program transformations
(including store/load reordering, speculative load introduction,
and several others (see \cref{table:cmp-mms})
and the external \DRF theorem. 
They also modified existing open-source implementation of 
JVM~\cite{Pizlo-al:ECCS10} to preserve \BMM and 
reported only~1\% average overhead 
compared to original version of the virtual machine. 
Again, the authors used only \Intel hardware in their 
experiments, and the performance penalties 
are expected to be more significant on weaker hardware.   

\paragraph{Relaxed Memory Models: an Operational Approach}

Boudol and Petri~\cite{Boudol-Petri:POPL09} proposed 
an approach to formal semantics of relaxed memory models 
based on the abstract machine with the main memory 
and the hierarchical structure of store buffers 
with stores to different locations possibly 
propagating to the main memory out-of-order
(similarly to \PSO model).
The authors present a proof of external \DRF theorem,
but do not provide an extensive study 
of soundness of program transformations.

\subsection{Out of Thin-Air Values}

Next we discuss memory models admitting thin-air values. 

\paragraph{C11}

The most notable member of the OOTA class is the \CMM model~\cite{Batty-al:POPL11}.
The main purpose of the \CMM model was to adhere to the fundamental principle of \CPP, 
\ie to provide so-called zero-cost abstraction. 
In other words, the memory model was meant to provide 
efficient compilation mappings and support as many transformation as possible.
It was later shown that the formal model partially fails in achieving these goals.

Vafeiadis~\etal~\cite{Vafeiadis-al:POPL15} have shown that several program transformation 
(load/store elimination, strengthening, roach motel reorderings, sequentialization) 
that deemed to be correct are actually unsound according to the formal model.
They proposed several local fixes to the model which 
partly repair soundness of transformations and improve 
its meta-theoretical properties. 

Batty~\etal~\cite{Batty-al:ESOP15} have shown that 
the model also fails to provide external \DRF guarantee, 
and that it is ultimately not possible to provide this guarantee
at all within the style of the \CMM formal semantics,
only the internal \DRF can be achieved. 

A lot of work~\cite{Batty-al:POPL11, Sarkar-al:PLDI12, Batty-al:POPL12, Batty-al:POPL16} 
was dedicated to prove soundness of optimal compilation mappings 
with respect to formal models of hardware, 
and there the results were mostly positive.
Besides that, Flur~\etal~\cite{Flur-al:POPL17} have extended the model to support mixed-size accesses.
Finally, Nienhuis~\etal~\cite{Nienhuis-al:OOPSLA16} presented 
a formal executable semantics in terms of an abstract machine, 
equivalent to the original \CMM model. 

\paragraph{\JS Memory Model}

\JSMM is based on \CMM model. 
Like the latter, it also has the problem of thin-air values
and thus can only provide internal \DRF guarantee. 
Contrary to the \CMM, \JS model does not treat 
racy non-atomic accesses as undefined behavior. 

The main language primitive provided by the \JSMM
is \texttt{SharedArrayBuffer}, that is a linear mutable byte buffer.
Thus the model naturally supports mixed-size accesses.

\paragraph{A calculus for relaxed memory}

Crary and Sullivan~\cite{Crary-Sullivan:POPL15} proposed 
an alternative approach to relaxed shared memory concurrency.
Instead of deriving the ordering constraints from the annotations 
on memory accesses, they propose to directly specify 
the ordering between memory access in the source code. 
Their approach is highly generic and subsumes 
the traditional memory order annotation in the style of \CMM.
Their model is very weak and permits thin-air values. 
Yet the authors proved the internal \DRF theorem.

\paragraph{Relaxed Atomic + Ordering}

Saraswat~\etal~\cite{Saraswat-al:PPoPP07} presented the \RAO memory model
where relaxed behaviors are explained through the transformations 
over sequentially consistent execution.
Although the authors claim their model provides external \DRF,
it also permits thin-air values. 
These two properties known to be incompatible~\cite{Batty-al:ESOP15}.
We suppose that the external \DRF can be achieved in their model 
only because of the fundamental restrictions on the input programming language 
(\eg the general conditional statements are not supported~\cite{PichonPharabod-Sewell:POPL16}). 

\paragraph{A theory of speculative computation}

Boudol and Petri~\cite{Boudol-Petri:ESOP10} proposed a general 
framework to study the effects of speculative execution in
shared memory setting. 
They have also noticed that the external \DRF does not 
hold in the presence of unrestricted speculations, 
yet the internal \DRF theorem still can be proven. 

\subsection{Program Order Preserving}
\label{sec:catalog:porf}

In this section we describe the memory models 
that preserve the program order and forbid any 
kind of speculative executions to tackle problem 
of thin-air values. 
In particular, we consider \RCMM and 
several derivatives of this model, 
as well as memory model of \OCaml, 
and a proposed revised model of \Java.  

\paragraph{RC11}

Lahav~\etal~\cite{Lahav-al:PLDI17} formalized 
a version of \CMM that preserves the order between load/store pairs, 
and also repairs the semantics of sequentially-consistent accesses.

The authors proved soundness of several program transformation 
(see \cref{table:cmp-mms} for details). 
Among the unsound transformation, 
the load/store reordering is forbidden for an obvious reasons, 
speculative load introduction is not supported 
because of catch-fire semantics for racy programs, 
\CSE is not supported because relaxed accesses 
enforce coherence (while non-atomic accesses 
support this transformation, they entail 
undefined behavior in the presence of races).

The compilation mappings to x86 are not affected and remain optimal.
One of the possible compilation mappings 
to architectures like \ARM and \POWER 
requires to compile relaxed load as  
plain load followed by a spurious conditional branch.
Ou and Demsky~\cite{Ou-Demsky:OOPSLA18} have studied 
the performance penalty of this mapping on ARMv8 hardware.
They modified the \LLVM compiler framework 
to enforce \RCMM memory model
by (1) adjusting the compiler optimization passes and 
(2) changing the compilation mappings.
Several compilation schemes were considered,
among them the one that uses spurious conditional branch
as described above has demonstrated the most promising results.  
The authors measured the running time on a set of benchmarks 
implementing various concurrent data-structures
(\eg locks, stacks, queues, deques, maps
from various open source libraries~\cite{CDSLib, FollyLib, JunctionLib})
and reported an overhead of 0\% on average and 6.3\% in maximum,
compared to the unmodified version of the compiler. 

\paragraph{RAR}

Doherty~\etal~\cite{Doherty-al:PPoPP19} developed an 
operational version of \RCMM supporting 
release-acquire and relaxed accesses. 
On top of it they built proof calculus for 
invariant-based reasoning and verified 
correctness of mutual exclusion algorithms. 

\paragraph{ORC11}

Dang~\etal~\cite{Dang-al:POPL19} developed yet another 
operational version of \RCMM which they called \ORCMM. 
Their motivation was to then develop a 
new program logic and show it's soundness
with respect to \ORCMM memory model. 
The program logic itself was then utilized to 
prove correctness of synchronization primitives 
from the standard library of \Rust~\cite{RustBook:19}.

\paragraph{Compositional Relaxed Concurrency}

Dodds~\etal~\cite{Dodds-al:ESOP18} proposed a denotational 
compositional semantics for the fragment of \CMM memory model, 
including non-atomic accesses with catch-fire semantics, 
release-acquire accesses, and sequentially-consistent fences. 
Based on this semantics the authors developed 
a tool for automatic verification of program transformations
in the considered fragment of the \CMM model. 
Since the relaxed fragment was not included, 
the authors avoided problems with thin-air values. 

\paragraph{OCaml Memory Model}

Dolan~\etal~\cite{Dolan-al:PLDI18} developed a new 
memory model for the \MOCaml project. 
An important divergence of \OCaml memory model 
from \CMM-like models is that the former 
has a weaker notion of coherence.
The choice of the weaker coherence was deliberate 
with the purpose to enable common subexpression elimination
(see \cref{sec:analysis:coh} for details).

The authors also were the first to propose the local \DRF property (\lDRF),
a strengthening of external \DRF (\eDRF). 
While the latter requires an absence of data-races 
for the whole program as a prerequisite, 
the former bounds the effect of races 
to a portion of a program, thus 
enabling the compositional reasoning 
about behavior of the program. 
The authors discovered that the \lDRF property 
is not compatible with load/store reordering.
This fact forced them to forbid this transformation
and adapt similar compilation scheme as for \RCMM. 

\paragraph{Java Access Modes}

Bender and Palsberg~\cite{Bender-Palsberg:OOPSLA19} formalized a new revision 
of the Java Memory Model~\cite{JDK9-VarHandle, JEP:193, JDK9-Modes}, 
which was developed to overcome 
the difficulties of the previous one~\cite{Manson-al:POPL05}
(see \ref{sec:catalog:jmm} for details).
The new version of the model was inspired by \RCMM. 
It introduced a system of annotations on memory accesses, 
called ``Java Access Modes'' (hence the name of the model --- \JAM),
similar to those present in \CMM like models.
The new model adopted the \RCMM solution to OOTA problem. 
It forbids load/store reorderings on the level of 
opaque (an analog of \CPP relaxed) or stronger accesses.
The model does not tackle the problem of 
thin-air values on the level of plain (\ie non-atomic) accesses.

\subsection{Syntactic Dependencies Preserving}
\label{sec:catalog:deprf}

Next we discuss the programming language memory models 
that track syntactic dependencies.

\paragraph{Linux Kernel Memory Model}

\LKMM~\cite{Alglave-al:ASPLOS18} has adopted 
the idea to track syntactic dependencies in order to 
forbid thin-air values. Despite this choice 
ultimately limits the list of supported 
trace preserving transformations,
in the context of OS kernel development 
it can be justified by the following arguments. 
First, the Linux kernel targets 
a wide range of hardware architectures with a diverse
set of memory models. To simplify the reasoning about the code, 
it is reasonable to pick a model which is conceptually close
to those of hardware. 
Second, the kernel developers already utilize 
various techniques to prevent certain compiler optimizations%
\cite{Alglave-al:ASPLOS18, LK-MemBarriers, LK-RCU-Deref}.

The authors of the models have empirically tested 
soundness of compilation mappings to 
\Intel, \ARMv{7}, \ARMv{8}, and \POWER hardware. 
They also formalized the read-copy-update 
synchronization mechanism (RCU)~\cite{McKenney-RCU2007}, 
extensively used in Linux kernel development, 
and proved soundness of its implementation with respect to their model.

\paragraph{Operational Happens-Before Model}

In attempt to repair Java Memory Model (see \cref{sec:catalog:jmm})
Zhang and Feng have proposed the 
operational happens-before model \OHMM~\cite{Zhang-Feng:FCS16}.
Their abstract machine consists of global event buffer,
where the events might be reordered before they propagate into  
a global history based memory, and a replay mechanism 
used to simulate speculative executions. 
To avoid thin-air outcomes the model tracks syntactic dependencies 
between the events and forbids the reordering of dependent events. 
The authors proved external \DRF and soundness of several 
program transformations. 

\paragraph{Dependency Preserving Compiler}

Ou and Demsky~\cite{Ou-Demsky:OOPSLA18} studied 
the performance penalty induced by dependency preserving compiler. 
Again, they modified the \LLVM compiler infrastructure 
and run benchmarks from \SPECCPU suite on ARMv8 hardware. 
They have observed 3.1\% overhead on average and 17.6\% in maximum. 

\subsection{Semantic Dependencies Preserving}
\label{sec:catalog:sdeprf}

Finally we discuss memory models 
which try to tackle thin-air values problem 
by development of the notion of semantic dependencies. 
In particular, this class includes the original Java Memory Model, 
Promising semantics, and several models based 
on \emph{event structures}~\cite{Winskel:86}.

\paragraph{Java Memory Model}
\label{sec:catalog:jmm}

The original formalized version of Java memory model \JMM~\cite{Manson-al:POPL05}
was a pioneering work in the area of programming language memory models 
While most of the memory model was formalized in axiomatic style, 
it also used an operational notion of \emph{commit sequence}, 
\ie a sequence of partial execution graphs, to forbid thin-air outcomes. 
The model was shown to adhere external \DRF~\cite{Huisman-Petri:CONCUR07}.
However, the model failed to justify some program transformations 
which were expected to be sound~\cite{Sevcik-Aspinall:ECOOP08} 
(\eg redundant load after load elimination, roach motel reordering, and others,
see \cref{table:cmp-mms} for details). 

\paragraph{Generative operational semantics}

Jagadeesan~\etal~\cite{Jagadeesan-al:ESOP10} attempted to fix \JMM 
and proposed an operational semantics with speculative execution.
To avoid thin-air values they have put stratification constraints 
on speculations. The authors prove the external \DRF theorem. 
Also they verified a few program transformations 
(store/store reordering, load/load elimination, and roach motel reordering), 
but overall their study of transformations was not systematic.  

\paragraph{Promising Semantics}

Promising semantics~\cite{Kang-al:POPL17, Lee-al:PLDI20} 
presents the most complete to this day model of this class. 
Its key ingredient is the promising and certification machinery.
During the execution, the abstract machine can 
non-deterministically \emph{promise} to perform some store,
it then has to \emph{certify} the promise is feasible. 
The certification mechanism is defined in the way to forbid thin-air values to appear.
The authors of the model have proven formally 
that \Promising semantics admits many local and global program transformations,
with a notable exception of the thread inlining
(see \cref{table:cmp-mms} for details).

Podkopaev~\etal~\cite{Podkopaev-al:ECOOP17, Podkopaev-al:POPL19} have proven formally
soundness of standard optimal compilation mappings to \Intel, \ARMv{7}, \ARMv{8}, and \POWER.

The model has a fully defined semantics even for plain accesses.  
Plain and relaxed accesses, however, have different semantics.
In particular, coherence is enforced for relaxed accesses 
but not for the plain ones.  
This design choice, in particular, allows to support 
\CSE on the level of plain accesses. 

One of a few limitations of the \Promising semantics is that 
it does not support sequentially consistent accesses. 

\paragraph{Weakestmo}

Chakraborty and Vafeiadis~\cite{Chakraborty-Vafeiadis:CGO17, Chakraborty-Vafeiadis:POPL19}
developed a memory model based on event structures. 
They utilize the event structures' capability to simultaneously encode 
multiple conflicting executions in order to model speculative executions.
Their model is close in spirit to conventional axiomatic models, 
however, instead of individual execution graphs, they use 
the event structures upon which they put additional constraints. 
The model was shown to admit optimal compilation mappings~\cite{Moiseenko-al:ECOOP20},
several program transformation, and external \DRF.
Unlike \Promising semantics, it also supports 
sequentially consistent accesses.

\paragraph{A Concurrency Semantics for Relaxed Atomics}

Pichon-Pharabod and Sewell~\cite{PichonPharabod-Sewell:POPL16} 
presented an operational memory model consisting of 
memory subsystem inspired by POWER and a thread subsystem, 
where each thread is represented as an event structure. 
At each step the abstract machine is allowed to either 
commit an event to the storage, or perform a transformation 
on one of the event structures. 
The authors have shown soundness of 
optimal compilation mappings to \Intel and \POWER, 
as well as soundness of several program transformations.
It was later revealed though that the compilation scheme
to ARMv7 and ARMv8 is not optimal~\cite{PichonPharabod:PhD18}.

\paragraph{Well-Justified Event Structures}

Jeffrey and Riely~\cite{Jeffrey-Riely:LICS16} proposed 
a memory model based on event structures and a notion of 
\emph{well-justification} of events inspired by the game semantics. 
Well-justification is used to prevent thin-air values 
and prove external \DRF. The authors do not study 
soundness of program transformations in their model. 
They show, however, a counterexample demonstrating that 
load/load reordering is unsound. It implies that 
the compilation mappings to \ARMv{7}, \ARMv{8}, and \POWER are not optimal.   

\paragraph{Modular Relaxed Dependencies}

Paviotti~\etal~\cite{Paviotti-al:ESOP20} constructed a 
denotational semantics based on event structures. 
They employ the event structures to capture 
semantic dependencies between the memory access events, 
which are in turn used to rule out thin-air outcomes.
The authors prove the external \DRF and 
soundness of optimal compilation mappings,
also they present a refinement relation which 
can be used to reason about validity of program transformations. 
However, they have not studied soundness of particular transformations. 


\end{document}
