\renewcommand{\abstractname}{Аннотация}

\begin{abstract}
Модель памяти языка программирования определяет семантику
многопоточных программ, оперирующих с разделяемой памятью.
Наиболее известна  модель последовательной согласованности, которая 
является слишком строгой, так как запрещает многие 
сценарии поведения, наблюдаемые при исполнении программ на современных процессорах.  
Попытки формально описать эти сценарии  привели к возникновению 
так называемых слабых моделей памяти.
В последние годы было предложено значительное количество 
слабых моделей памяти для различных языков программирования. 
Эти модели предлагают различные компромиссы относительно  простоты/сложности рассуждения 
о поведении параллельных программ 
и возможностей их оптимизации. 
Цель данной статьи заключается в обзоре существующих 
моделей памяти для языков программирования 
и выработке общих рекомендации для 
разработки языков и их стандартов, а также  компиляторов 
по выбору/созданию соответствующей модели памяти. 

Для данного обзора мы рассмотрели более 2000 статей, найденных
по ключевым словам ``Relaxed Memory Models'', ``Weak Memory Models'',
и ``Weak Memory Consistency'' поисковой системой Google Scholar.
Используя разные критерии,  мы сузили этот множество до 40 статей, 
предлагающих и описывающих модели памяти 
языков программирования.
Мы разделили эти модели на шесть основных классов и 
проанализировали их свойства и ограничения. 
В заключении мы показали, как 
дизайн языка программирования влияет на 
выбор модели памяти и обсудили 
возможные направления дальнейших исследований в этой области. %слабых моделей памяти.
\end{abstract}
