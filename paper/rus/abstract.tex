\renewcommand{\abstractname}{Abstract}

\begin{abstract}
Модель памяти определяет семантику
конкурентных программ оперирующих с разделяемой памятью.
Наиболее известная и интуитивная модель последовательной согласованности 
является слишком строгой, так как запрещает многие 
поведения наблюдаемые на современных процессорах 
в результате выполнения компилятором и процессором различных оптимизаций.  
Попытки описать эти поведения привели к возникновению 
так называемых слабых моделей памяти.
В последние годы было предложено множество 
слабых моделей памяти для различных языков программирования. 
Эти модели предлагают различные компромиссы в соотношении
простоты рассуждения о поведении конкурентных программ 
и возможностей для их оптимизации. 
Цель данной статьи заключается в обзоре существующих 
моделей памяти для языков программирования 
и выработке общих рекомендации для 
разработчиков языков и компиляторов 
по выбору модели. 

Для данного обзора мы отобрали более 2000 статей
по ключевым словам ``Relaxed Memory Models'', ``Weak Memory Models'',
и ``Weak Memory Consistency'', используя поисковую систему Google Scholar.
Затем мы сузили этот список до 40 статей 
предлагающих и описывающих модели памяти 
языков программирования.
Мы разделили эти модели на 6 основных классов и 
проанализировали их свойства и ограничения. 
В заключении, мы рассматриваем как 
дизайн языка программирования вляет на 
выбор модели памяти и обсуждаем 
возможные направления дальнейших исследований в области. %слабых моделей памяти.
\end{abstract}
