\section{Введение}
\label{sec:intro}

Главная трудность многопоточного программирования 
заключается в необходимости обеспечить  
синхронизацию между различными потоками программы. 
Обычно это достигается при помощи специальных 
примитивов синхронизации, 
таких как блокировки, барьеры, каналы и т.д. 
Тем не менее, иногда использование этих примитивов 
нежелательно или невозможно. 
Примерами таких случаев является 
непосредственно реализация самих примитивов синхронизации,
а также различные неблокирующие (lock-free) структуры данных.
В подобных случаях необходимо обращаться к средствам
низкоуровневого программирования и 
взаимодействовать с разделяемой памятью напрямую.
Здесь начинаются сложности. 

Рассмотрим конкретный пример.
Ниже представлена упрощенная версия 
алгоритма блокировки Деккера~\cite{Dijkstra:68}:

\begin{equation*}
\inarrII{
  \writeInst{}{x}{1} \\
  \readInst{}{r_1}{y}  \\
  \kw{if} {r_1 = 0} ~\{ \\
  ~~ \comment{critical section} \\
  \}
}{
  \writeInst{}{y}{1} \\
  \readInst{}{r_2}{x}  \\
  \kw{if} {r_2 = 0} ~\{ \\
  ~~ \comment{critical section} \\
  \}
}
\tag{Dekker's Lock}\label{ex:Dekker}
\end{equation*}

В этой программе два потока соревнуются за доступ к критической секции.
Чтобы обозначить свое намерение войти в критическую секцию,
потоки устанавливают значение переменных $x$ и $y$ соответственно%
\footnote{В этой статье  переменные,  разделяемые разными потоками программы, мы обозначаем следующим образом ---
 $x$, $y$, $z$..., а локальные переменные  потока --- $r_1$, $r_2$, $r_3$...}.
Поток, который первым установит значение переменной и 
прочитает значение другой переменной до его установки,
получает право войти в критическую секцию.
Алгоритм полагается на тот факт, что оба 
потока не могут одновременно прочитать значение ~\texttt{0}%
\footnote{Здесь и далее  мы подразумеваем, 
что переменные инициализированы значением 0, если иное не указано явно.}.
В противном случае два потока способны одновременно 
войти в критическую секцию, таким образом нарушая корректность алгоритма. 

Логично ожидать, что эта программа 
завершится с одним из следующих результатов: 
${[r_1=0, r_2=1]}$, ${[r_1=1,r_2=0]}$ или ${[r_1=1,r_2=1]}$. 
Соответствующие сценарии поведения называются 
\emph{последовательно согласованными} 
(\emph{sequentially consistent})~\cite{Lamport:TC79}, и это означает, 
что они получены в результате поочередного 
исполнения инструкций потоков. 

Тем не менее не все сценарии поведения, наблюдаемые в
многопоточных  системах, являются последовательно согласованными.
Например, если перевести псевдокод алгоритма Деккера
на язык C, выполнить компиляцию полученной программы с помощью компилятора GCC
и  запустить на процессорах семейства x86/x64, то
можно  наблюдать, помимо перечисленных выше результатов, еще и результат  $[r_1=0, r_2=0]$. 
Сценарий поведения, который приводит к этому результату, 
является примером  слабого (\emph{weak}) поведения.

Слабые сценарии поведения появляются в результате оптимизаций программ
компиляторами и процессорами. Например, 
рассматривая программу \ref{ex:Dekker}, 
оптимизатор может заметить, что запись в переменную $x$
и чтение из $y$ в левом потоке являются независимыми инструкциями 
и, следовательно, могут быть переупорядочены
(заметим, что эта оптимизация является корректной 
для случая однопоточных программ).
Для оптимизированной программы поведение с результатом
$[r_1=0, r_2=0]$ является последовательно согласованным.

Множество допустимых сценариев поведения программы определяется
семантикой многопоточной системы или \emph{моделью памяти}.
Модель \emph{последовательной согласованности}
(\emph{sequential consistency}, SC) допускает 
только последовательно согласованные сценарии поведения.
Модели памяти, которые   допускают слабые сценарии поведения, 
называются, соответственно, \emph{слабыми моделями памяти}
(\emph{weak memory models}).

Современные процессоры и языки программирования 
не ограничиваются моделью последовательной согласованности, 
так как она запрещает многие важные оптимизации.
Основной вопрос таким образом заключается в том, насколько слабой 
должна быть модель. Более строгая модель допускает меньше сценариев поведения 
и предоставляет больше гарантии программисту.
С другой стороны, более слабая модель позволяет
выполнять большее количество оптимизаций, что повышает производительность программы. Последнее обстоятельство является очень важным для таких программ, как ядра операционных систем или системы управления СУБД. 

Оказывается, что этот вопрос весьма сложен, 
особенно в контексте моделей памяти языков программирования (ЯП).
Это привело к тому, что за последнее 20 лет было предложено 
множество моделей для различных языков программирования, например, для 
\Java~\cite{Manson-al:POPL05, Bender-Palsberg:OOPSLA19}, \CPP~\cite{Batty-al:POPL11}, 
\LLVM~\cite{Chakraborty-Vafeiadis:CGO17}, \JS~\cite{Watt-al:PLDI2020}, 
\OCaml~\cite{Manson-al:POPL05}, \Haskell~\cite{Vollmer-al:PPoPP17} и т.д.
Эти модели преследуют разные цели, делают различные компромиссы
и имеют разнообразные ограничения.
Более того, в данной области продолжают появляться новые исследования. 
По нашим подсчетам, за последние 10 лет в течении 
каждого года было опубликовано не менее 50 статей по этой тематике.%
\footnote{Эти подсчеты подкреплены данными, полученными 
с помощью поисковой системы Google Scholar, 
подробности представлены \cref{sec:methodology}.}
Тем не менее, несмотря на долгую историю этой области и существенный прогресс,
нет единого источника, который бы суммировал 
известную информацию и сравнивал существующие 
модели памяти различных языков программирования. 
Целью данной статьи является создание такого обзора. 

Мы рассматриваем существующие модели памяти языков программирования, 
обсуждаем их дизайн, компромиссы и ограничения. 
Также мы сравниваем эти модели на предмет того,
какие оптимизации они поддерживают 
и какие гарантии предоставляют программистам. 

Мы надеемся, что наша работа будет полезна для 
исследователей в области языков программирования, 
желающих ознакомиться с темой слабых моделей памяти, 
а также для разработчиков компиляторов и виртуальных машин, 
которым необходимо выбрать модель памяти для их системы.

Данная статья организована следующим образом. 
В \cref{sec:related} представлен обзор литературы. 
Далее мы описываем методологию нашего исследования \cref{sec:methodology}.
Затем мы вводим критерии сравнения моделей памяти \cref{sec:background}, 
в частности, оптимальность схем компиляции, 
корректность преобразований программ
и предоставляемые гарантии для рассуждения 
о поведении программ.  
Далее мы описываем то, каким образом мы  
сравниваем  модели \cref{sec:comparison}. 
В \cref{sec:analysis} мы классифицируем модели на основе их свойств
и обсуждаем каждый класс в отдельности. 
Также, в приложении \ref{sec:catalog} мы кратко описываем каждую рассмотренную модель. 
В \cref{sec:discussion} мы представляем набор рекомендации 
по выбору модели памяти на основе требований к языку программирования 
и рассматриваем их на примере языка Kotlin\footnote{https://kotlinlang.org/}.
Наконец, в \cref{sec:conclusion} мы подводим итоги 
и рассматриваем возможные направления дальнейших исследований.
