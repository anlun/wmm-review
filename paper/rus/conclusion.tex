\section{Заключение}
\label{sec:conclusion}

В данной работе мы представили обзор существующих моделей памяти языков программирования. 
Мы сравнили эти модели  по ряду критериев 
и идентифицировали шесть основных классов моделей памяти. 
Также, мы предложили рекомендации по выбору эффективной модели памяти 
для различных языков программирования. 
Мы надеемся, что наша работа будет полезна 
для исследователей и разработчиков в области языков программирования
и послужит введением в сложную тематику слабых моделей памяти. 


На основе нашего анализа мы также можем сделать следующие предположения 
о будущих направления работы в данной области. 

Проблема оптимальности схем компиляции и корректности локальных трансформаций кода 
на сегодняшний день относительно хорошо изучена.
Более новые модели, такие как 
RC11~\cite{Lahav-al:PLDI17}, \OCaml MM~\cite{Dolan-al:PLDI18},
\Promising~\cite{Kang-al:POPL17,Lee-al:PLDI20}
и \Weakestmo~\cite{Chakraborty-Vafeiadis:POPL19},
поддерживают широкий диапазон локальных трансформаций 
и имеют ясные компромиссы относительно оптимальности схем компиляции. 
Исключением являются  локальные трансформации, 
использующие  циклы или рекурсию, так как их корректность всё ещё недостаточно изучена. 
Глобальным трансформациям также уделялось мало внимания, за важным исключением работ~%
\cite{PichonPharabod-Sewell:POPL16, Lee-al:PLDI20}.
Влияние этих трансформаций на дизайн модели памяти 
ещё предстоит  изучить.

Глобальное свойство свободы от гонок на настоящий момент детально изучено~%
\cite{Manson-al:POPL05, Batty-al:ESOP15, Lahav-al:PLDI17, Kang-al:POPL17}. 
Напротив, локальное свойство свободы от гонок~\cite{Dolan-al:PLDI18}  
является новой концепцией. 
Стоит ожидать, что ему, как и другим локальным гарантиям~%
\cite{Dodds-al:ESOP18, Jagadeesan-al:OOPSLA2020, Cho-al:PLDI21}, 
будет уделено  внимание исследователей в ближайшем будущем. 

Смешанные обращения~\cite{Flur-al:POPL17}, 
используемые в модели памяти \JS~\cite{Watt-al:PLDI2020}, 
а также в некоторых приложениях, например, в кодовой базе 
ядра \Linux~\cite{Flur-al:POPL17},
на сегодняшний день недостаточно изучены 
даже в контексте моделей памяти процессоров. 
Таким образом, более глубокое понимание семантики смешанных обращений 
является ещё одним важным направлением исследований. 

Модели памяти, сохраняющие семантические зависимости, по-прежнему являются  темой активных исследований~%
\cite{Kang-al:POPL17, Lee-al:PLDI20, Cho-al:PLDI21,
Chakraborty-Vafeiadis:POPL19, Paviotti-al:ESOP20, 
Jagadeesan-al:OOPSLA2020}.
Следует ожидать, что они будут более детально разрабатываться и уточняться в ближайшем будущем. 
Интересным направлением работ в этой области 
является разработка новых гарантий, 
помимо свойства свободы от гонок, 
которые помогут усовершенствовать мета-теорию 
этих моделей и упростить рассуждение о корректности программ. 

Наконец, детальное исследование 
накладных расходов на время исполнения программ с различными моделям памяти также являются очень важной задачей. Несмотря на наличие здесь некоторого количества работ %
\cite{Singh-al:ISCA12, Liu-al:OOPSLA17, Liu-al:PLDI19, 
Vollmer-al:PPoPP17, Dolan-al:PLDI18, Ou-Demsky:OOPSLA18}, 
полная картина все ещё остается недостаточно ясной. 