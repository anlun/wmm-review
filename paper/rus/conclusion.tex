\section{Заключение}
\label{sec:conclusion}

В данной работе мы представили обзор моделей памяти 
предложенных для различных языков программирования. 
Мы сравнили их по множеству общих критериев, 
разработанных в исследованиях по моделям памяти, 
и идентифицировали шесть основных классов моделей памяти. 
Также, мы предложили рекомендации по выбору модели 
для языка программирования на основе его дизайна. 
Мы надеемся, что наша работа будет полезна 
для исследователей и разработчиков в области языков программирования, 
и послужит для них введением в сложную тему слабых моделей памяти. 
На основе нашего анализа мы также можем сделать предположения 
о будущих направления работы в области. 

Проблемы оптимальности схем компиляции и корректности локальных трансформаций кода 
на сегодняшний день относительно хорошо изучены.
Более новые модели, такие как 
RC11~\cite{Lahav-al:PLDI17}, \OCaml MM~\cite{Dolan-al:PLDI18},
\Promising~\cite{Kang-al:POPL17,Lee-al:PLDI20},
и \Weakestmo~\cite{Chakraborty-Vafeiadis:POPL19},
поддерживают широкий диапозон локальных трансформаций 
и имеют ясные компромиссы относительно оптимальности схем компиляции. 
Исключеним являются разве что локальные трансформации, 
задействующие циклы или рекурсию, так как их корректность не была изучена формально. 
Глобальным трансформациям также уделялось мало внимания, 
за важным исключением работ~\cite{PichonPharabod-Sewell:POPL16, Lee-al:PLDI20}.
Влияние этих трансформаций на дизайн модели памяти 
ещё предстоит внимательно изучить.

Глобальное свойство свободы от гонок также было всесторонее изучено. 
Напротив, локальное свойство свободы от гонок~\cite{Dolan-al:PLDI18}  
является новой концепцией. 
Стоит ожидать, что оно, как и другие локальные гарантии~%
\cite{Dodds-al:ESOP18, Jagadeesan-al:OOPSLA2020, Cho-al:PLDI21}, 
получат больше внимания в ближайшем будущем. 

Смешанные обращения~\cite{Flur-al:POPL17}, 
используемые в модели памяти \JS~\cite{Watt-al:PLDI2020} 
а также в некоторых приложениях, например, в кодовой базе 
ядра \Linux~\cite{Flur-al:POPL17},
недостаточно изучены на сегодняшний день, 
даже в контексте моделей памяти процессоров. 
Более глубокое понимание семантики смешанных обращений 
является важным направлением исследований. 

Модели памяти, сохраняющие семантические зависимости, 
по-прежнему являются активной темой исследований~%
\cite{Kang-al:POPL17, Lee-al:PLDI20, Cho-al:PLDI21,
Chakraborty-Vafeiadis:POPL19, Paviotti-al:ESOP20, 
Jagadeesan-al:OOPSLA2020}.
Мы ожидаем, что они будут уточняться в будущем. 
Интересным направлением работ в этой области 
будет разработка новых гарантий, 
помимо свойства свободы от гонок, 
которые помогут усовершенствовать мета-теорию 
этих моделей и упростить рассуждение о корректности программ. 

Наконец, всесторонние количественные исследования 
накладных расходов на время исполнения программ, 
вызванных выбором модели памяти, также являются очень ценными. 
Несмотря на наличие некоторого количества работ по этой проблеме%
\cite{Singh-al:ISCA12, Liu-al:OOPSLA17, Liu-al:PLDI19, 
Vollmer-al:PPoPP17, Dolan-al:PLDI18, Ou-Demsky:OOPSLA18}, 
полная картина все ещё остается недостаточно ясной. 