\section{Comparison}
\label{sec:comparison}

We performed a comparison of the memory models 
found via the search procedure described in \cref{sec:methodology} 
by the criteria given in \cref{sec:background}. 
A particular challenge of this comparison was the fact that 
consulted research papers often use different terminology, 
have incomplete information about models, and 
sometimes they even contradict each other. 
We tried to approach these challenges by the following means.
First, we used consistent terminology 
to denote the properties of the memory models, 
as presented in \cref{sec:background}.
Second, we complemented the information about 
each particular memory model from different sources.
If after this procedure some particular property 
was still unclear, we left it as unknown. 

Based on our comparison of the memory models, 
we identified six classes of them:
sequentially consistent models,
models with total or partial order on stores,
program order preserving models, 
syntactic dependency preserving models, 
semantic dependency preserving models, 
and models with out of thin-air values. 
The models from the same class have the similar compilation mappings, 
the set of sound program transformations, and the provided reasoning guarantees.
We first present the result of our comparison on a per-class basis 
(see~\cref{table:cmp-cls} and~\cref{sec:analysis}), and 
then give a more detailed comparison with respect to individual models
(see~\cref{table:cmp-mms} and~\cref{sec:catalog}).
Thus we have an opportunity to first discuss common principles
behind programming language relaxed memory models in general, 
and then dive deeper into the details of each particular model. 

% \begin{landscape}

\begin{table*}[t]
\scriptsize

\newcommand{\rotateAngle}{270}
\newcommand{\lastcol}{28}

\def\arraystretch{1}
\setlength\tabcolsep{3pt} %\setlength\tabcolsep{2pt}
\setlength\extrarowheight{6pt}

\begin{center}

\begin{tabular}{|c|c|c|c|c|c|c|c|c|c|c|c|c|c|c|c|c|c|c|c|c|c|c|c|c|c|c|c|} 
 
 \hline
  
 \multirow{3}{*}{Class}                                                &
 \multirow{3}{*}[-8pt]{\rotatebox[origin=c]{\rotateAngle}{\# Models}}  &
 \multicolumn{ 4}{ c|}{\multirow{2}{*}{Compilation}}                   &
 \multicolumn{17}{ c|}{Transformations}                                &
 \multicolumn{ 5}{ c|}{\multirow{2}{*}{Reasoning}}     % &
 % \multicolumn{ 9}{c|}{\multirow{2}{*}{Features}}      
 \\ 

 \cline{7-23}

 \multicolumn{1}{|c|}{}      &
 \multicolumn{1}{c|}{}       &
 \multicolumn{4}{c|}{}       &
 \multicolumn{14}{c|}{Local} &
 \multicolumn{3}{c|}{Global} &
 \multicolumn{5}{c|}{}       % &
 % \multicolumn{9}{c|}{}       
 \\ 
 
 \cline{3-\lastcol}

 \multicolumn{1}{|c|}{}                                              &
 \multicolumn{1}{c|}{}                                               &
 \multirow{2}{*}[-6pt]{\rotatebox[origin=c]{\rotateAngle}{\Intel}}   & 
 \multirow{2}{*}[-6pt]{\rotatebox[origin=c]{\rotateAngle}{\POWER}}   & 
 \multirow{2}{*}[-5pt]{\rotatebox[origin=c]{\rotateAngle}{\ARMv{7}}} & 
 \multirow{2}{*}[-5pt]{\rotatebox[origin=c]{\rotateAngle}{\ARMv{8}}} & 
 
 \multicolumn{4}{c|}{Reordering}    &
 \multicolumn{4}{c|}{Elimination}   &

 \multirow{2}{*}[-6pt]{\rotatebox[origin=c]{\rotateAngle}{ILE}}    &
 \multirow{2}{*}[-6pt]{\rotatebox[origin=c]{\rotateAngle}{SLI}}    &
 \multirow{2}{*}[-6pt]{\rotatebox[origin=c]{\rotateAngle}{RM}}     &
 \multirow{2}{*}[-6pt]{\rotatebox[origin=c]{\rotateAngle}{S}}      &
 \multirow{2}{*}[-6pt]{\rotatebox[origin=c]{\rotateAngle}{TP}}     &
 \multirow{2}{*}[-6pt]{\rotatebox[origin=c]{\rotateAngle}{CSE}}    &
 \multirow{2}{*}[-6pt]{\rotatebox[origin=c]{\rotateAngle}{RP}}     &
 \multirow{2}{*}[-6pt]{\rotatebox[origin=c]{\rotateAngle}{TI}}     &
 \multirow{2}{*}[-6pt]{\rotatebox[origin=c]{\rotateAngle}{VR}}     &
 
 \multirow{2}{*}[-6pt]{\rotatebox[origin=c]{\rotateAngle}{\eDRF}}             &
 \multirow{2}{*}[-6pt]{\rotatebox[origin=c]{\rotateAngle}{COH}}               &
 \multirow{2}{*}[-6pt]{\rotatebox[origin=c]{\rotateAngle}{no-UB}}             &
 \multirow{2}{*}[-4pt]{\rotatebox[origin=c]{\rotateAngle}{In-Order}}          &
 \multirow{2}{*}[-2pt]{\rotatebox[origin=c]{\rotateAngle}{no-OOTA}}         % &               

 % \multirow{2}{*}[-6pt]{\rotatebox[origin=c]{\rotateAngle}{NA}}                      &
 % \multirow{2}{*}[-6pt]{\rotatebox[origin=c]{\rotateAngle}{RLX}}                     &
 % \multirow{2}{*}[-6pt]{\rotatebox[origin=c]{\rotateAngle}{RA}}                      &
 % \multirow{2}{*}[-6pt]{\rotatebox[origin=c]{\rotateAngle}{SC}}                      &
 % \multirow{2}{*}[-6pt]{\rotatebox[origin=c]{\rotateAngle}{F-RA}}                    &
 % \multirow{2}{*}[-6pt]{\rotatebox[origin=c]{\rotateAngle}{F-SC}}                    &
 % \multirow{2}{*}[-6pt]{\rotatebox[origin=c]{\rotateAngle}{RMW}}                     &
 % \multirow{2}{*}[-6pt]{\rotatebox[origin=c]{\rotateAngle}{Lock}}                    &
 % \multirow{2}{*}[-6pt]{\rotatebox[origin=c]{\rotateAngle}{\makecell{Mix.Sz.}}}       
 \\[9pt] 

 \cline{7-14}
 \multicolumn{1}{|c|}{} & &
      & & & &
 SL         &
 SS         &
 LL         &
 LS         &
 SL         &
 SS         &
 LL         &
 LS         &
  & & & & & & & & & & & & & \\[3pt]
  
  \Xhline{2\arrayrulewidth}

  \multicolumn{1}{|c|}{\clsSC} & 2
     &             

     % compilation 
     \badcelln{0}{2} & 
     \badcelln{0}{2} & 
     \badcelln{0}{2} & 
     \badcelln{0}{2} & 

     % reorderings
     \badcelln{0}{2} & 
     \badcelln{0}{2} & 
     \badcelln{0}{2} & 
     \badcelln{0}{2} &

     % eliminations
     \okcelln{2}{2} & 
     \okcelln{2}{2} & 
     \okcelln{2}{2} & 
     \okcelln{2}{2} &

     % irrelevant load elim & speculative load intro 
     \okcelln{2}{2} & 
     \okcelln{1}{2} &

     % roach motel 
     \okcelln{2}{2} & 

     % strength.
     \okcelln{2}{2} &

     % trace preserving 
     \okcelln{2}{2} &

     % common subexpression elimination
     \badcelln{0}{2} &

     % global: register promotion, thread inlining, value range 
     \okcelln{2}{2}  & 
     \okcelln{2}{2}  & 
     \badcelln{0}{0} &

     % eDRF, COH, no-UB, In-order, no-OOTA
     \okcelln{2}{2}   & 
     \okcelln{2}{2}   & 
     \okcelln{1}{2}   &
     \okcelln{2}{2}   &
     \okcelln{2}{2}   

     \\ \hline

  \multicolumn{1}{|c|}{\makecell{Total/Partial\\Store Order}} & 2
     &

     % compilation 
     \okcelln{2}{2}  & 
     \badcelln{0}{2} & 
     \badcelln{0}{2} & 
     \badcelln{0}{2} & 

     % reorderings
     \okcelln{2}{2}   & 
     \okcelln{1}{2}   & 
     \badcelln{0}{2}  & 
     \badcelln{0}{2}  & 

     % eliminations
     \okcelln{1}{1}  & 
     \okcelln{1}{1}  & 
     \okcelln{1}{1}  & 
     \badcelln{0}{1} &  

     % irrelevant load elim & speculative load intro 
     \okcelln{1}{1} & 
     \okcelln{1}{1} &

     % roach motel 
     \badcelln{0}{1} & 

     % strength.
     \badcelln{0}{0} &

     % trace preserving 
     \okcelln{1}{1} &

     % common subexpression elimination
     \badcelln{0}{1} &

     % global: register promotion, thread inlining, value range 
     \badcelln{0}{0} &
     \badcelln{0}{1} & 
     \badcelln{0}{0} &

     % eDRF, COH, no-UB, In-order, no-OOTA
     \okcelln{2}{2} & 
     \okcelln{1}{1} & 
     \okcelln{2}{2} & 
     \okcelln{2}{2} &
     \okcelln{2}{2} 

     \\ \hline

  \multicolumn{1}{|c|}{\clsPO} & 3
     &

     % compilation 
     \okcelln{3}{3}  & 
     \badcelln{0}{3} & 
     \badcelln{0}{3} & 
     \badcelln{0}{3} &  

     % reorderings
     \okcelln{2}{2}  & 
     \okcelln{2}{2}  & 
     \okcelln{2}{2}  & 
     \badcelln{0}{2} & 

     % eliminations
     \okcelln{2}{2}  & 
     \okcelln{2}{2}  & 
     \okcelln{2}{2}  & 
     \badcelln{0}{2} & 

     % irrelevant load elim & speculative load intro 
     \badcelln{0}{0} & 
     \badcelln{0}{1} &

     % roach motel 
     \okcelln{1}{1} & 

     % strength.
     \okcelln{2}{2} &

     % trace preserving 
     \okcelln{2}{2} &

     % common subexpression elimination
     \warncelln{1}{3} &

     % global: register promotion, thread inlining, value range 
     \okcelln{1}{1}  & 
     \okcelln{1}{1}  &
     \badcelln{0}{0} & 

     % eDRF, COH, no-UB, In-order, no-OOTA
     \okcelln{3}{3} & 
     \okcelln{2}{3} & 
     \okcelln{2}{3} &
     \okcelln{3}{3} & 
     \okcelln{3}{3} 

     \\ \hline

  \multicolumn{1}{|c|}{\clsSyDEP} & 2
     &

     % compilation 
     \okcelln{1}{1} & 
     \okcelln{1}{1} & 
     \okcelln{1}{1} & 
     \okcelln{1}{1} &

     % reorderings
     \okcelln{1}{1} & 
     \okcelln{1}{1} & 
     \okcelln{1}{1} & 
     \okcelln{1}{1} &

     % eliminations
     \okcelln{1}{1} & 
     \okcelln{1}{1} & 
     \okcelln{1}{1} & 
     \okcelln{1}{1} &

     % irrelevant load elim & speculative load intro 
     \okcelln{1}{1} & 
     \okcelln{1}{1} &

     % roach motel 
     \okcelln{1}{1} & 

     % strength.
     \badcelln{0}{0} & 

     % trace preserving 
     \badcelln{0}{2} &

     % common subexpression elimination
     \badcelln{0}{1} &

     % global: register promotion, thread inlining, value range 
     \badcelln{0}{0} & 
     \badcelln{0}{0} & 
     \badcelln{0}{0} & 

     % eDRF, COH, no-UB, In-order, no-OOTA
     \okcelln{1}{1}  & 
     \okcelln{1}{1}  & 
     \okcelln{2}{2}  & 
     \badcelln{0}{2} & 
     \okcelln{2}{2} 

     \\ \hline


  \multicolumn{1}{|c|}{\clsSemDEP} & 7
     &
     % compilation 
     \okcelln{5}{5} & 
     \okcelln{4}{6} & 
     \okcelln{4}{6} & 
     \okcelln{4}{6} &  

     % reorderings
     \okcelln{4}{4} & 
     \okcelln{4}{4} & 
     \okcelln{4}{5} & 
     \okcelln{4}{4} &

     % eliminations
     \okcelln{4}{4} & 
     \okcelln{4}{4} & 
     \okcelln{3}{5} & 
     \warncelln{1}{5} &  

     % irrelevant load elim & speculative load intro  
     \okcelln{3}{3} & 
     \okcelln{2}{4} &

     % roach motel 
     \okcelln{2}{4} & 

     % strength.
     \okcelln{2}{3} &

     % trace preserving 
     \okcelln{2}{2} &

     % common subexpression elimination
     \warncelln{1}{4} &

     % global: register promotion, thread inlining, value range 
     \okcelln{1}{1}  & 
     \badcelln{0}{4} & 
     \okcelln{2}{2}  & 

     % eDRF, COH, no-UB, In-order, no-OOTA
     \okcelln{6}{6}  & 
     \okcelln{4}{6}  & 
     \okcelln{5}{7}  & 
     \badcelln{0}{7} & 
     \okcelln{7}{7} 

     \\ \hline

  \multicolumn{1}{|c|}{\clsOOTA} & 5
     &            

     % compilation 
     \okcelln{3}{3} & 
     \okcelln{3}{3} & 
     \okcelln{3}{3} & 
     \okcelln{3}{3} &

     % reorderings
     \okcelln{1}{1} & 
     \okcelln{1}{1} & 
     \okcelln{1}{1} & 
     \okcelln{1}{1} &

     % eliminations
     \okcelln{1}{1}  & 
     \okcelln{1}{1}  & 
     \okcelln{1}{1}  & 
     \badcelln{0}{1} &  

     % irrelevant load elim & speculative load intro 
     \badcelln{0}{0} & 
     \badcelln{0}{1} &

     % roach motel 
     \okcelln{1}{1} & 

     % strength.
     \okcelln{1}{1} &

     % trace preserving 
     \okcelln{1}{1} &

     % common subexpression elimination
     \badcelln{0}{1} &

     % global: register promotion, thread inlining, value range
     \badcelln{0}{0} & 
     \okcelln{1}{1}  & 
     \badcelln{0}{0} & 

     % eDRF, COH, no-UB, In-order, no-OOTA
     \badcelln{0}{4}  & 
     \okcelln{2}{4}   & 
     \okcelln{4}{5}   & 
     \badcelln{0}{5}  & 
     \badcelln{0}{5} 

     \\ \hline


\end{tabular}

\end{center}

\iftoggle{langeng}{%
  \caption{Classes of memory models and their properties}
}{%
  \caption{Классы моделей памяти и их свойства}
}

\label{table:cmp-cls}
\end{table*}

% \end{landscape}




In both \cref{table:cmp-cls} and \cref{table:cmp-mms} we order
the memory models by their weakness.  
The strongest models are located at the top rows of the tables, 
while the weakest are at the bottom.  

Columns of both tables correspond 
to the properties of the memory models.
In order to be concise, we chose a binary classification for all properties,
\ie the model is either said to satisfy a given property or not.  
We also split properties into several subgroups.

The first group is devoted to an optimality 
of compilation mappings to target hardware architectures. 
We classify a compilation scheme as either optimal or not,
in the following sense.
We chose the weakest possible access mode supported by a model
and consider the compilation scheme for memory accesses annotated by this mode. 
For memory models that treat racy non-atomic accesses
as undefined behavior, we consider the compilation mapping
for the weakest atomic access mode that model provides.
It is because the catch-fire semantics for racy non-atomics 
trivially permits the optimal compilation mapping (see~\cref{sec:background:ub}).
We say that the compilation scheme is \emph{optimal} if  
accesses annotated by the chosen mode 
can be compiled just as plain load and store instructions 
of a given hardware architecture 
(\ie without use of memory fences or other auxiliary code). 

The second group is dedicated to a soundness of various program transformations. 
The classification is also binary: a transformation is either sound or unsound 
in a given memory model (in the sense stated in~\cref{sec:background:trans}).
Again, to be concise, we do not consider all combinations 
of program transformations and memory access modes. 
Instead, we consider the weakest possible accesses 
which have fully defined semantics. 
We further split the transformations into 
global and local as in~\cref{sec:background:trans}.

The third group corresponds to reasoning 
principles guaranteed by the model. In particular, we check 
whether a model provides the external DRF guarantee (see~\cref{sec:background:drf}), 
whether it provides the coherence property (see~\cref{sec:background:coh}),
whether it has fully defined semantics for all types of accesses, 
\ie the model does not treat racy non-atomic accesses 
as undefined behavior (see~\cref{sec:background:ub}),
whether the model utilizes in-order execution 
(as opposed to speculative out-of-order executon),
and whether it forbids out of thin-air values (see~\cref{sec:background:oota}).

In \cref{table:cmp-mms} each row corresponds to 
a specific memory model, denoted by its abbreviation, 
and thus each cell describes a particular property 
of that particular model. 
We marked a cell by \cmark\xspace if the corresponding model satisfies the given property,
and we marked it by \xmark\xspace otherwise.
If the property was not studied in the research papers, 
we color the cell in gray~% 
{\protect\tikz \protect\draw[fill=colorQmark] (0,0) rectangle ++(0.35,0.35);}.

Each row of the \cref{table:cmp-cls} corresponds to a class of memory models. 
We marked a cell by \cmark\xspace if the majority of models 
in the given class satisfy the property. 
If less than the majority of models satisfy the property we mark 
the corresponding cell by \wmark\xspace.
Finally, if none of the models satisfy the property, we mark the cell by \xmark\xspace. 
Note that when counting the majority, we omit the unknowns.
Also, if a given property was not studied in the context of some class of models 
(\ie, in~\cref{table:cmp-mms} it is marked by \qmark\xspace for all models in this class)
in \cref{table:cmp-cls} we mark the corresponding cell by \xmark. 
That is, in \cref{table:cmp-cls} symbols \cmark\xspace and \wmark\xspace 
denote positive knowledge,
while~\xmark\xspace denotes negative knowledge or
an absence of information.

Besides \cref{table:cmp-cls,table:cmp-mms} which describe 
properties of the memory models, 
we also present \cref{table:features}
that provides a list of features supported by the models.
In this table each row corresponds to a particular memory model. 
Columns correspond to supported features. 
In particular, we check what types of access modes are supported:
non-atomic~(NA), relaxed~(RLX), release/acquire~(RA), sequentially-consistent~(SC); 
what types of fences are supported: release/acquire~(F-RA) 
and sequentially-consistent~(F-SC);
whether the atomic read-modify-write operations are supported~(RMW),
whether the model handles locks explicitly~(LK),
and whether it supports mixed-size accesses~(MIX). 

% Finally, the last group enumerates the list of memory access modes 
% and fences supported by the model, as well as whether the model 
% supports read-modify-write operations, locks, and mixed-size accesses.

\newpage
\onecolumn

\begin{landscape}

\begin{table*}
\scriptsize
\begin{tabular}{|c|l|c|c|c|c|c|c|c|c|c|c|c|c|c|c|c|c|c|c|c|c|c|c|c|c|c|c|c|c|c|c|c|c|c|c|} % 35
 \hline

                                                      &
 \multirow{3}{*}{Model}                               & 
 \multicolumn{ 4}{c|}{\multirow{2}{*}{Compilation}}   &
 \multicolumn{17}{c|}{Transformations}                &
 \multicolumn{ 4}{c|}{\multirow{2}{*}{Reasoning}}     &
 \multicolumn{ 9}{c|}{\multirow{2}{*}{Features}}      \\ 

 \cline{7-23}
                             &
                             &
 \multicolumn{4}{c|}{}       &
 \multicolumn{14}{c|}{Local} &
 \multicolumn{3}{c|}{Global} &
 \multicolumn{4}{c|}{}       &
 \multicolumn{9}{c|}{}       \\ 
 
 \hline
                                     &
                                     &
 \rotatebox[origin=c]{270}{x86}      & 
 \rotatebox[origin=c]{270}{Power}    & 
 \rotatebox[origin=c]{270}{ARMv7}    & 
 \rotatebox[origin=c]{270}{ARMv8}    & 
 
 \multicolumn{4}{c|}{\rotatebox[origin=c]{270}{RI}}   &
 \multicolumn{4}{c|}{\rotatebox[origin=c]{270}{RE}}   &

 \rotatebox[origin=c]{270}{ILE}    &
 \rotatebox[origin=c]{270}{SLI}    &
 \rotatebox[origin=c]{270}{RM}     &
 \rotatebox[origin=c]{270}{S}      &
 \rotatebox[origin=c]{270}{TP}     &
 \rotatebox[origin=c]{270}{CSE}    &
 \rotatebox[origin=c]{270}{RP}     &
 \rotatebox[origin=c]{270}{VR}     &
 \rotatebox[origin=c]{270}{TI}     &
 
 \rotatebox[origin=c]{270}{DRF}                                &
 \rotatebox[origin=c]{270}{COH}                                &
 \rotatebox[origin=c]{270}{no-UB}                              &
 % \rotatebox[origin=c]{270}{\makecell{$\lPO\lRF$}}              & 
 \rotatebox[origin=c]{270}{no-OOTA}                            &                              

 \rotatebox[origin=c]{270}{NA}                      &
 \rotatebox[origin=c]{270}{RLX}                     &
 \rotatebox[origin=c]{270}{RA}                      &
 \rotatebox[origin=c]{270}{SC}                      &
 \rotatebox[origin=c]{270}{F-RA}                    &
 \rotatebox[origin=c]{270}{F-SC}                    &
 \rotatebox[origin=c]{270}{RMW}                     &
 \rotatebox[origin=c]{270}{Lock}                    &
 \rotatebox[origin=c]{270}{\makecell{Mix.Sz.}}      \\ 

 \hline
  & & & & & &
 \rotatebox[origin=c]{270}{SL}       &
 \rotatebox[origin=c]{270}{SS}       &
 \rotatebox[origin=c]{270}{LL}       &
 \rotatebox[origin=c]{270}{LS}       &
 \rotatebox[origin=c]{270}{SL}       &
 \rotatebox[origin=c]{270}{SS}       &
 \rotatebox[origin=c]{270}{LL}       &
 \rotatebox[origin=c]{270}{LS}       &
 & & & & & & & & & & & & & & & & & & & & & \\
  
 \Xhline{2\arrayrulewidth}
 
 \multirow{3}{*}{\rotatebox[origin=c]{270}{\makecell{SeqCst}}}   

 & SC 
     &             
     % compilation 
     \badcell & \badcell & \badcell & \badcell & 
     % reorderings
     \badcell & \badcell & \badcell & \badcell &
     % eliminations
     \okcell & \okcell & \okcell & \okcell &
     % irrelevant load elim & speculative load intro 
     \okcell & \okcell &
     % roach motel 
     \okcell & 
     % strength.
     \unkwcell &
     % trace preserving 
     \okcell &
     % common subexpression elimination
     \badcell &
     % global: register promotion, value range, thread inlining 
     \okcell & \unkwcell & \okcell &
     % DRF, COH, no-UB, no-OOTA
     \ldrf & \okcell & \okcell & \okcell &
     % NA, RLX, RA, SC 
     \badcell & \badcell & \badcell & \okcell & 
     % F-RA, F-SC
     \badcell & \badcell & 
     % RMW, Lock
     \badcell & \badcell & 
     % Mix.sz.
     \badcell 
     \\ \cline{2-36}

 & SC-Hs 
     &
     % compilation 
     \badcell & \badcell & \badcell & \badcell & 
     % reorderings
     \badcell & \badcell & \badcell & \badcell & 
     % eliminations
     \okcell & \okcell & \okcell & \okcell &
     % irrelevant load elim & speculative load intro 
     \okcell & \okcell &
     % roach motel 
     \okcell & 
     % strength.
     \unkwcell &
     % trace preserving 
     \okcell &
     % common subexpression elimination
     \badcell &
     % global: register promotion, value range, thread inlining 
     \okcell & \unkwcell & \okcell & 
     % DRF, COH, no-UB, no-OOTA
     \ldrf & \okcell & \okcell & \okcell &
     % NA, RLX, RA, SC 
     \okcell & \badcell & \badcell & \okcell & 
     % F-RA, F-SC
     \badcell & \badcell & 
     % RMW, Lock
     \okcell & \badcell &
     % Mix.sz. 
     \badcell 
     \\ \cline{2-36}

 & DRFx
     &           
     % compilation 
     \badcell & \badcell & \badcell & \badcell & 
     % reorderings
     \badcell & \badcell & \badcell & \badcell & 
     % eliminations
     \okcell & \okcell & \okcell & \okcell &
     % irrelevant load elim & speculative load intro 
     \okcell & \badcell &
     % roach motel 
     \okcell & 
     % strength.
     \unkwcell &
     % trace preserving 
     \okcell &
     % common subexpression elimination
     \badcell &
     % global: register promotion, value range, thread inlining 
     \okcell & \unkwcell & \okcell &                                              
     % DRF, COH, no-UB, no-OOTA
     \ldrf & \okcell & \warncell & \okcell &
     % NA, RLX, RA, SC 
     \okcell & \badcell & \badcell & \okcell & 
     % F-RA, F-SC
     \badcell & \badcell & 
     % RMW, Lock
     \badcell & \badcell & 
     % Mix.sz.
     \badcell 
     \\ \Xhline{2\arrayrulewidth}

 \multirow{2}{*}{\rotatebox[origin=c]{270}{\makecell{TSO\\PSO}}}   

 & BMM
     &
     % compilation 
     \okcell & \badcell & \badcell & \badcell & 
     % reorderings
     \okcell & \badcell & \badcell & \badcell & 
     % eliminations
     \okcell & \okcell & \okcell & \badcell &  
     % irrelevant load elim & speculative load intro 
     \okcell & \okcell &
     % roach motel 
     \badcell & 
     % strength.
     \unkwcell &
     % trace preserving 
     \okcell &
     % common subexpression elimination
     \badcell &
     % global: register promotion, value range, thread inlining 
     \unkwcell & \unkwcell & \badcell &
     % DRF, COH, no-UB, no-OOTA
     \edrf & \okcell & \okcell & \okcell &
     % NA, RLX, RA, SC 
     \okcell & \badcell & \badcell & \okcell & 
     % F-RA, F-SC
     \badcell & \badcell & 
     % RMW, Lock
     \badcell & \okcell & 
     % Mix.sz.
     \badcell 
     \\ \cline{2-36}

 & RMMOA
     &
     % compilation 
     \okcell & \badcell & \badcell & \badcell & 
     % reorderings
     \okcell & \okcell & \badcell & \badcell & 
     % eliminations
     \unkwcell & \unkwcell & \unkwcell & \unkwcell &  
     % irrelevant load elim & speculative load intro 
     \unkwcell & \unkwcell &
     % roach motel 
     \unkwcell & 
     % strength.
     \unkwcell &
     % trace preserving 
     \unkwcell &
     % common subexpression elimination
     \unkwcell &
     % global: register promotion, value range, thread inlining 
     \unkwcell & \unkwcell & \unkwcell &
     % DRF, COH, no-UB, no-OOTA
     \edrf & \unkwcell & \okcell & \okcell &
     % NA, RLX, RA, SC 
     \okcell & \badcell & \badcell & \badcell & 
     % F-RA, F-SC
     \badcell & \badcell & 
     % RMW, Lock
     \badcell & \okcell & 
     % Mix.sz.
     \badcell 
     \\ \Xhline{2\arrayrulewidth}

 \multirow{5}{*}{\rotatebox[origin=c]{270}{\makecell{$\lPO\lRF$\\acyc}}}   

 & RC11
     &
     % compilation 
     \okcell & \badcell & \badcell & \badcell &  
     % reorderings
     \okcell & \okcell & \okcell & \badcell & 
     % eliminations
     \okcell & \okcell & \okcell & \badcell & 
     % irrelevant load elim & speculative load intro 
     \unkwcell & \badcell &
     % roach motel 
     \okcell & 
     % strength.
     \okcell &
     % trace preserving 
     \okcell &
     % common subexpression elimination
     \badcell &
     % global: register promotion, value range, thread inlining 
     \okcell & \unkwcell & \okcell &                                              
     % DRF, COH, no-UB, no-OOTA
     \edrf & \okcell & \warncell & \okcell &
     % NA, RLX, RA, SC 
     \okcell & \okcell & \okcell & \okcell & 
     % F-RA, F-SC
     \okcell & \okcell & 
     % RMW, Lock
     \okcell & \badcell & 
     % Mix.sz.
     \badcell 
     \\ \cline{2-36}

 & ORC11
     &
     % compilation 
     \okcell & \badcell & \badcell & \badcell &  
     % reorderings
     \okcell & \okcell & \okcell & \badcell & 
     % eliminations
     \okcell & \okcell & \okcell & \badcell & 
     % irrelevant load elim & speculative load intro 
     \unkwcell & \badcell &
     % roach motel 
     \okcell & 
     % strength.
     \okcell &
     % trace preserving 
     \okcell &
     % common subexpression elimination
     \badcell &
     % global: register promotion, value range, thread inlining 
     \okcell & \unkwcell & \okcell &
     % DRF, COH, no-UB, no-OOTA
     \edrf & \okcell & \warncell & \okcell &
     % NA, RLX, RA, SC 
     \okcell & \okcell & \okcell & \badcell & 
     % F-RA, F-SC
     \okcell & \badcell & 
     % RMW, Lock
     \okcell & \badcell & 
     % Mix.sz.
     \badcell 
     \\ \cline{2-36}

 & RAR
     & 
     % compilation 
     \okcell & \badcell & \badcell & \badcell &  
     % reorderings
     \okcell & \okcell & \okcell & \badcell & 
     % eliminations
     \okcell & \okcell & \okcell & \badcell & 
     % irrelevant load elim & speculative load intro 
     \unkwcell & \badcell &
     % roach motel 
     \okcell & 
     % strength.
     \okcell &
     % trace preserving 
     \okcell &
     % common subexpression elimination
     \badcell &
     % global: register promotion, value range, thread inlining 
     \okcell & \unkwcell & \okcell &
     % DRF, COH, no-UB, no-OOTA
     \edrf & \okcell & \okcell & \okcell &
     % NA, RLX, RA, SC 
     \badcell & \okcell & \okcell & \badcell & 
     % F-RA, F-SC
     \badcell & \badcell & 
     % RMW, Lock
     \okcell & \badcell & 
     % Mix.sz.
     \badcell 
     \\ \cline{2-36}

 & OCMM
     & 
     % compilation 
     \okcell & \badcell & \badcell & \badcell &  
     % reorderings
     \okcell & \okcell & \okcell & \badcell & 
     % eliminations
     \okcell & \okcell & \okcell & \badcell & 
     % irrelevant load elim & speculative load intro  
     \unkwcell & \unkwcell &
     % roach motel 
     \unkwcell & 
     % strength.
     \unkwcell &
     % trace preserving 
     \okcell &
     % common subexpression elimination
     \okcell &
     % global: register promotion, value range, thread inlining 
     \unkwcell & \unkwcell & \unkwcell &
     % DRF, COH, no-UB, no-OOTA
     \ldrf & \warncell & \okcell & \okcell &
     % NA, RLX, RA, SC 
     \okcell & \badcell & \badcell & \okcell & 
     % F-RA, F-SC
     \badcell & \badcell & 
     % RMW, Lock
     \okcell & \badcell & 
     % Mix.sz.
     \badcell 
     \\ \cline{2-36}

 & JAM
     & 
     % compilation 
     \okcell & \badcell & \badcell & \badcell &  
     % reorderings
     \unkwcell & \unkwcell & \unkwcell & \unkwcell &  
     % eliminations
     \unkwcell & \unkwcell & \unkwcell & \unkwcell & 
     % irrelevant load elim & speculative load intro  
     \unkwcell & \unkwcell &
     % roach motel 
     \unkwcell & 
     % strength.
     \okcell &
     % trace preserving 
     \unkwcell &
     % common subexpression elimination
     \badcell &
     % global: register promotion, value range, thread inlining 
     \unkwcell & \unkwcell & \unkwcell &
     % DRF, COH, no-UB, no-OOTA
     \edrf & \okcell & \okcell & \okcell &
     % NA, RLX, RA, SC 
     \okcell & \okcell & \okcell & \okcell & 
     % F-RA, F-SC
     \okcell & \okcell & 
     % RMW, Lock
     \okcell & \badcell & 
     % Mix.sz.
     \badcell 
     \\ \cline{2-36}

 & CRC
     &
     % compilation 
     \okcell & \badcell & \badcell & \badcell &  
     % reorderings
     \okcell & \badcell & \badcell & \badcell &  
     % eliminations
     \okcell & \okcell & \okcell & \badcell &  
     % irrelevant load elim & speculative load intro  
     \badcell & \badcell &
     % roach motel 
     \unkwcell & 
     % strength.
     \unkwcell &
     % trace preserving 
     \unkwcell &
     % common subexpression elimination
     \badcell &
     % global: register promotion, value range, thread inlining 
     \unkwcell & \unkwcell & \unkwcell &
     % DRF, COH, no-UB, no-OOTA
     \edrf & \okcell & \warncell & \okcell &
     % NA, RLX, RA, SC 
     \okcell & \badcell & \okcell & \badcell & 
     % F-RA, F-SC
     \badcell & \okcell & 
     % RMW, Lock
     \okcell & \badcell & 
     % Mix.sz.
     \badcell 
     \\ \Xhline{2\arrayrulewidth}

 \multirow{2}{*}{\rotatebox[origin=c]{270}{\makecell{$\lPPO\lRF$\\acyc}}}   

 & LKMM
     &           
     % compilation 
     \okcell & \okcell & \okcell & \okcell &  
     % reorderings
     \okcell & \okcell & \okcell & \okcell &
     % eliminations
     \unkwcell & \unkwcell & \unkwcell & \unkwcell &  
     % irrelevant load elim & speculative load intro 
     \unkwcell & \unkwcell &
     % roach motel 
     \unkwcell & 
     % strength.
     \unkwcell &
     % trace preserving 
     \badcell &
     % common subexpression elimination
     \badcell &
     % global: register promotion, value range, thread inlining 
     \unkwcell & \unkwcell & \unkwcell &
     % DRF, COH, no-UB, no-OOTA
     \edrf & \okcell & \okcell & \okcell &
     % NA, RLX, RA, SC 
     \badcell & \okcell & \okcell & \badcell & 
     % F-RA, F-SC
     \okcell & \okcell & 
     % RMW, Lock
     \okcell & \badcell & 
     % Mix.sz.
     \badcell 
     \\ \cline{2-36}

 & OHMM
     &
     % compilation 
     \unkwcell & \unkwcell & \unkwcell & \unkwcell &
     % reorderings
     \okcell & \okcell & \okcell & \okcell &
     % eliminations
     \okcell & \okcell & \okcell & \okcell &
     % irrelevant load elim & speculative load intro 
     \okcell & \okcell &
     % roach motel 
     \okcell & 
     % strength.
     \unkwcell &
     % trace preserving 
     \badcell &
     % common subexpression elimination
     \unkwcell &
     % global: register promotion, value range, thread inlining 
     \unkwcell & \unkwcell & \unkwcell &         
     % DRF, COH, no-UB, no-OOTA
     \edrf & \unkwcell & \okcell & \okcell &
     % NA, RLX, RA, SC 
     \okcell & \badcell & \badcell & \okcell & 
     % F-RA, F-SC
     \badcell & \badcell & 
     % RMW, Lock
     \badcell & \okcell & 
     % Mix.sz.
     \badcell 
     \\ \Xhline{2\arrayrulewidth}

 \multirow{7}{*}{\rotatebox[origin=c]{270}{\makecell{no-OOTA}}}   

 & JMM
     &            
     % compilation 
     & & & &
     % reorderings
     \okcell & \okcell & \okcell & \okcell &
     % eliminations
     \okcell & \okcell & \badcell & \badcell &
     % irrelevant load elim & speculative load intro  
     \okcell & \badcell &
     % roach motel 
     \badcell & 
     % strength.
     \unkwcell &
     % trace preserving 
     \okcell &
     % common subexpression elimination
     \badcell &
     % global: register promotion, value range, thread inlining 
     \unkwcell & \unkwcell & \badcell &
     % DRF, COH, no-UB, no-OOTA
     \edrf & \warncell & \okcell & \okcell &
     % NA, RLX, RA, SC 
     \okcell & \badcell & \badcell & \okcell & 
     % F-RA, F-SC
     \badcell & \badcell & 
     % RMW, Lock
     \okcell & \okcell & 
     % Mix.sz.
     \badcell 
     \\ \cline{2-36}

 & Prm
     &
     % compilation 
     \okcell & \okcell & \okcell & \okcell &  
     % reorderings
     \okcell & \okcell & \okcell & \okcell &
     % eliminations
     \okcell & \okcell & \okcell & \okcell &  
     % irrelevant load elim & speculative load intro  
     \okcell & \okcell &
     % roach motel 
     \okcell & 
     % strength.
     \okcell &
     % trace preserving 
     \okcell &
     % common subexpression elimination
     \okcell &
     % global: register promotion, value range, thread inlining 
     \okcell & \okcell & \badcell &
     % DRF, COH, no-UB, no-OOTA
     \edrf & \okcell & \okcell & \okcell &
     % NA, RLX, RA, SC 
     \okcell & \okcell & \okcell & \badcell & 
     % F-RA, F-SC
     \okcell & \okcell & 
     % RMW, Lock
     \okcell & \okcell & 
     % Mix.sz.
     \badcell 
     \\ \cline{2-36}

 & Wkmo
     &
     % compilation 
     \okcell & \okcell & \okcell & \okcell &
     % reorderings
     \okcell & \okcell & \okcell & \okcell &
     % eliminations
     \okcell & \okcell & \okcell & \okcell &  
     % irrelevant load elim & speculative load intro  
     \unkwcell & \okcell &
     % roach motel 
     \badcell & 
     % strength.
     \okcell &
     % trace preserving 
     \unkwcell &
     % common subexpression elimination
     \badcell &
     % global: register promotion, value range, thread inlining 
     \unkwcell & \unkwcell & \badcell &
     % DRF, COH, no-UB, no-OOTA
     \edrf & \okcell & \okcell & \okcell &
     % NA, RLX, RA, SC 
     \okcell & \okcell & \okcell & \okcell & 
     % F-RA, F-SC
     \okcell & \okcell & 
     % RMW, Lock
     \okcell & \badcell & 
     % Mix.sz.
     \badcell 
     \\ \cline{2-36}

 & MRD
     &
     % compilation 
     \okcell & \okcell & \okcell & \okcell &
     % reorderings
     \unkwcell & \unkwcell & \unkwcell & \unkwcell &
     % eliminations
     \unkwcell & \unkwcell & \unkwcell & \unkwcell &
     % irrelevant load elim & speculative load intro 
     \unkwcell & \unkwcell &
     % roach motel 
     \unkwcell & 
     % strength.
     \unkwcell &
     % trace preserving 
     \unkwcell &
     % common subexpression elimination
     \unkwcell &
     % global: register promotion, value range, thread inlining 
     \unkwcell & \unkwcell & \unkwcell &                                              
     % DRF, COH, no-UB, no-OOTA
     \edrf & \okcell & \okcell & \okcell &
     % NA, RLX, RA, SC 
     \badcell & \okcell & \badcell & \badcell & 
     % F-RA, F-SC
     \badcell & \badcell & 
     % RMW, Lock
     \badcell & \okcell & 
     % Mix.sz.
     \badcell 
     \\ \cline{2-36}

 & P-P/S
     &
     % compilation 
     \okcell & \okcell & \badcell & \badcell &
     % reorderings 
     \okcell & \okcell & \okcell & \okcell &
     % eliminations
     \okcell & \okcell & \okcell & \badcell &  
     % irrelevant load elim & speculative load intro  
     \okcell & \badcell &
     % roach motel 
     \okcell & 
     % strength.
     \badcell &
     % trace preserving 
     \unkwcell &
     % common subexpression elimination
     \badcell &
     % global: register promotion, value range, thread inlining 
     \unkwcell & \okcell & \badcell &
     % DRF, COH, no-UB, no-OOTA
     \unkwcell & \okcell & \warncell & \okcell &
     % NA, RLX, RA, SC 
     \okcell & \okcell & \okcell & \badcell & 
     % F-RA, F-SC
     \badcell & \badcell & 
     % RMW, Lock
     \okcell & \okcell & 
     % Mix.sz.
     \badcell 
     \\ \cline{2-36}

 & J/R
     &
     % compilation 
     \unkwcell & \badcell & \badcell & \badcell &
     % reorderings
     \unkwcell & \unkwcell & \badcell & \unkwcell &
     % eliminations
     \unkwcell & \unkwcell & \badcell & \badcell &  
     % irrelevant load elim & speculative load intro 
     \unkwcell & \unkwcell &
     % roach motel 
     \unkwcell & 
     % strength.
     \unkwcell &
     % trace preserving 
     \unkwcell &
     % common subexpression elimination
     \unkwcell &
     % global: register promotion, value range, thread inlining 
     \unkwcell & \unkwcell & \unkwcell &                                              
     % DRF, COH, no-UB, no-OOTA
     \edrf & \warncell & \okcell & \okcell &
     % NA, RLX, RA, SC 
     \badcell & \okcell & \badcell & \badcell & 
     % F-RA, F-SC
     \badcell & \badcell & 
     % RMW, Lock
     \okcell & \okcell & 
     % Mix.sz.
     \badcell 
     \\ \cline{2-36}

 & GOS
     &
     % compilation 
     \unkwcell & \unkwcell & \unkwcell & \unkwcell &
     % reorderings 
     \unkwcell & \unkwcell & \unkwcell & \unkwcell &
     % eliminations 
     \unkwcell & \unkwcell & \unkwcell & \unkwcell &
     % irrelevant load elim & speculative load intro 
     \unkwcell & \unkwcell &
     % roach motel 
     \unkwcell & 
     % strength.
     \unkwcell &
     % trace preserving 
     \unkwcell &
     % common subexpression elimination
     \unkwcell &
     % global: register promotion, value range, thread inlining 
     \unkwcell & \unkwcell & \unkwcell &                                              
     % DRF, COH, no-UB, no-OOTA
     \edrf & \unkwcell & \okcell & \okcell &
     % NA, RLX, RA, SC 
     \okcell & \badcell & \badcell & \badcell & 
     % F-RA, F-SC
     \badcell & \badcell & 
     % RMW, Lock
     \badcell & \okcell & 
     % Mix.sz.
     \badcell 
     \\ \Xhline{2\arrayrulewidth}

 \multirow{5}{*}{\rotatebox[origin=c]{270}{\makecell{OOTA}}}   

 & C11
     &            
     % compilation 
     \okcell & \okcell & \okcell & \okcell &
     % reorderings
     \okcell & \okcell & \okcell & \okcell &
     % eliminations
     \okcell & \okcell & \okcell & \badcell &  
     % irrelevant load elim & speculative load intro 
     \unkwcell & \badcell &
     % roach motel 
     \badcell & 
     % strength.
     \badcell &
     % trace preserving 
     \okcell &
     % common subexpression elimination
     \badcell &
     % global: register promotion, value range, thread inlining 
     \unkwcell & \unkwcell & \badcell &
     % DRF, COH, no-UB, no-OOTA
     \idrf & \okcell & \warncell & \badcell &
     % NA, RLX, RA, SC 
     \okcell & \okcell & \okcell & \okcell & 
     % F-RA, F-SC
     \okcell & \okcell & 
     % RMW, Lock
     \okcell & \okcell & 
     % Mix.sz.
     \okcell 
     \\ \cline{2-36}

 & JSMM
     &
     % compilation 
     \okcell & \okcell & \okcell & \okcell &
     % reorderings
     \unkwcell & \unkwcell & \unkwcell & \unkwcell &
     % eliminations
     \unkwcell & \unkwcell & \unkwcell & \unkwcell &
     % irrelevant load elim & speculative load intro 
     \unkwcell & \unkwcell &
     % roach motel 
     \unkwcell & 
     % strength.
     \unkwcell &
     % trace preserving 
     \unkwcell &
     % common subexpression elimination
     \unkwcell &
     % global: register promotion, value range, thread inlining 
     \unkwcell & \unkwcell & \unkwcell &
     % DRF, COH, no-UB, no-OOTA
     \idrf & & \okcell & \badcell &
     % NA, RLX, RA, SC 
     \okcell & \badcell & \badcell & \okcell & 
     % F-RA, F-SC
     \badcell & \badcell & 
     % RMW, Lock
     \okcell & \okcell & 
     % Mix.sz.
     \okcell 
     \\ \cline{2-36}

 & RMC
     &
     % compilation 
     \okcell & \okcell & \okcell & \okcell &
     % reorderings
     \unkwcell & \unkwcell & \unkwcell & \unkwcell &
     % eliminations
     \unkwcell & \unkwcell & \unkwcell & \unkwcell &
     % irrelevant load elim & speculative load intro 
     \unkwcell & \unkwcell &
     % roach motel 
     \unkwcell & 
     % strength.
     \unkwcell &
     % trace preserving 
     \unkwcell &
     % common subexpression elimination
     \unkwcell &
     % global: register promotion, value range, thread inlining 
     \unkwcell & \unkwcell & \unkwcell &
     % DRF, COH, no-UB, no-OOTA
     \idrf & \okcell & \okcell & \badcell &
     % NA, RLX, RA, SC 
     \badcell & \okcell & \okcell & \okcell & 
     % F-RA, F-SC
     \okcell & \okcell & 
     % RMW, Lock
     \okcell & \badcell & 
     % Mix.sz.
     \badcell 
     \\ \cline{2-36}


 & RAO
     &
     % compilation 
     \unkwcell & \unkwcell & \unkwcell & \unkwcell &
     % reorderings
     \unkwcell & \unkwcell & \unkwcell & \unkwcell &
     % eliminations
     \unkwcell & \unkwcell & \unkwcell & \unkwcell &
     % irrelevant load elim & speculative load intro 
     \unkwcell & \unkwcell &
     % roach motel 
     \unkwcell & 
     % strength.
     \unkwcell &
     % trace preserving 
     \unkwcell &
     % common subexpression elimination
     \unkwcell &
     % global: register promotion, value range, thread inlining 
     \unkwcell & \unkwcell & \unkwcell & 
     % DRF, COH, no-UB, no-OOTA
      & \warncell & \okcell & \badcell &
     % NA, RLX, RA, SC 
     \okcell & \badcell & \badcell & \okcell & 
     % F-RA, F-SC
     \badcell & \badcell & 
     % RMW, Lock
     \badcell & \badcell & 
     % Mix.sz.
     \badcell 
     \\ \cline{2-36}

 & TSC
     % compilation 
     & \unkwcell & \unkwcell & \unkwcell & \unkwcell &
     % reorderings
     \unkwcell & \unkwcell & \unkwcell & \unkwcell &
     % eliminations
     \unkwcell & \unkwcell & \unkwcell & \unkwcell &
     % irrelevant load elim & speculative load intro 
     \unkwcell & \unkwcell &
     % roach motel 
     \unkwcell & 
     % strength.
     \unkwcell &
     % trace preserving 
     \unkwcell &
     % common subexpression elimination
     \unkwcell &
     % global: register promotion, value range, thread inlining 
     \unkwcell & \unkwcell & \unkwcell & 
     % DRF, COH, no-UB, no-OOTA
     \idrf & \unkwcell & \okcell & \badcell &
     % NA, RLX, RA, SC 
     \okcell & \badcell & \badcell & \okcell & 
     % F-RA, F-SC
     \badcell & \badcell & 
     % RMW, Lock
     \badcell & \okcell & 
     % Mix.sz.
     \badcell 
     \\ \Xhline{2\arrayrulewidth}


\end{tabular}
\caption{
  % \textit{T.P.} --- trace preserving.
  % \textit{R.I.} --- reordering of independent instructions.
  % \textit{R.E.} --- redundunt load/store elimination.
  % \textit{I.L.E.} --- irrelevant load elimination.
  % \textit{S.L.I.} --- speculative load introduction.
  % \textit{S.} --- strengthening.
  % \textit{R.M.} --- roach motel reordering.
  % \textit{R.P.} --- register promotion.
  % \textit{V.R.} --- value range analysis based optimizations.
  % \textit{T.I.} --- thread inlining (sequentialization).
  % \textit{Int.} --- internal.
  % \textit{Ext.} --- external.
  % \textit{Loc.} --- local.
  % \textit{UB} --- undefined behavior.
  % \textit{OOTA} --- out-of-thin air values.
  % \textit{Mix.Sz.} --- mixed-size accesses.
}
\label{table:cmp-mms}
\end{table*}

\end{landscape}

\twocolumn



\begin{table}[t]

\newcommand{\rotateAngle}{270}
\newcommand{\lastcol}{11}

\def\arraystretch{1}
\setlength\tabcolsep{2pt}

\begin{tabular}{|c|l|c|c|c|c|c|c|c|c|c|} 

  \hline

  \multirow{2}{*}{Class}          &
  \multirow{2}{*}{Model}          &
  \multicolumn{ 9}{c|}{Features}   
  \\ 

  \cline{3-\lastcol}

                                                            &
                                                            &
  \rotatebox[origin=c]{\rotateAngle}{NA}                    &
  \rotatebox[origin=c]{\rotateAngle}{RLX}                   &
  \rotatebox[origin=c]{\rotateAngle}{RA}                    &
  \rotatebox[origin=c]{\rotateAngle}{SC}                    &
  \rotatebox[origin=c]{\rotateAngle}{F-RA}                  &
  \rotatebox[origin=c]{\rotateAngle}{F-SC}                  &
  \rotatebox[origin=c]{\rotateAngle}{RMW}                   &
  \rotatebox[origin=c]{\rotateAngle}{LK}                    &
  \rotatebox[origin=c]{\rotateAngle}{MIX}      
  \\[9pt] 

  \Xhline{2\arrayrulewidth}

  \multirow{4}{*}{\clsSC}

  & \EtESC
     & 
     % NA, RLX, RA, SC 
     \badcell & \badcell & \badcell & \okcell & 
     % F-RA, F-SC
     \badcell & \badcell & 
     % RMW, LK
     \badcell & \badcell & 
     % MIX
     \badcell 
     \\ \cline{2-\lastcol}

  & \VbD
     & 
     % NA, RLX, RA, SC 
     \okcell & \badcell & \badcell & \okcell & 
     % F-RA, F-SC
     \badcell & \badcell & 
     % RMW, LK
     \okcell & \okcell & 
     % MIX
     \badcell 
     \\ \cline{2-\lastcol}

  & \SCHs 
     &
     % NA, RLX, RA, SC 
     \okcell & \badcell & \badcell & \okcell & 
     % F-RA, F-SC
     \badcell & \badcell & 
     % RMW, LK
     \okcell & \badcell &
     % MIX 
     \badcell 
     \\ \cline{2-\lastcol}

  & \DRFx
     &           
     % NA, RLX, RA, SC 
     \okcell & \badcell & \badcell & \okcell & 
     % F-RA, F-SC
     \badcell & \badcell & 
     % RMW, LK
     \badcell & \badcell & 
     % MIX
     \badcell 
     \\ \Xhline{2\arrayrulewidth}

  \multirow{2}{*}{\clsTSO}

  & \BMM
     &
     % NA, RLX, RA, SC 
     \okcell & \badcell & \badcell & \okcell & 
     % F-RA, F-SC
     \badcell & \badcell & 
     % RMW, Lock
     \badcell & \okcell & 
     % Mix.sz.
     \badcell 
     \\ \cline{2-\lastcol}

  & \RMMOA
     &
     % NA, RLX, RA, SC 
     \okcell & \badcell & \badcell & \badcell & 
     % F-RA, F-SC
     \badcell & \badcell & 
     % RMW, Lock
     \badcell & \okcell & 
     % Mix.sz.
     \badcell 
     \\ \Xhline{2\arrayrulewidth}

  \multirow{6}{*}{\clsPO}   

  & \RCMM
     &
     % NA, RLX, RA, SC 
     \okcell & \okcell & \okcell & \okcell & 
     % F-RA, F-SC
     \okcell & \okcell & 
     % RMW, Lock
     \okcell & \badcell & 
     % Mix.sz.
     \badcell 
     \\ \cline{2-\lastcol}

  & \ORCMM
     &
     % NA, RLX, RA, SC 
     \okcell & \okcell & \okcell & \badcell & 
     % F-RA, F-SC
     \okcell & \badcell & 
     % RMW, Lock
     \okcell & \badcell & 
     % Mix.sz.
     \badcell 
     \\ \cline{2-\lastcol}

  & \RAR
     & 
     % NA, RLX, RA, SC 
     \badcell & \okcell & \okcell & \badcell & 
     % F-RA, F-SC
     \badcell & \badcell & 
     % RMW, Lock
     \okcell & \badcell & 
     % Mix.sz.
     \badcell 
     \\ \cline{2-\lastcol}

  & \CRC
     &
     % NA, RLX, RA, SC 
     \okcell & \badcell & \okcell & \badcell & 
     % F-RA, F-SC
     \badcell & \okcell & 
     % RMW, Lock
     \okcell & \badcell & 
     % Mix.sz.
     \badcell 
     \\ \cline{2-\lastcol}

  & \OCMM
     & 
     % NA, RLX, RA, SC 
     \okcell & \badcell & \badcell & \okcell & 
     % F-RA, F-SC
     \badcell & \badcell & 
     % RMW, Lock
     \okcell & \badcell & 
     % Mix.sz.
     \badcell 
     \\ \cline{2-\lastcol}

  & \JAM
     &
     % NA, RLX, RA, SC 
     \okcell & \okcell & \okcell & \okcell & 
     % F-RA, F-SC
     \okcell & \okcell & 
     % RMW, Lock
     \okcell & \badcell & 
     % Mix.sz.
     \badcell 
     \\ \Xhline{2\arrayrulewidth}

  \multirow{2}{*}{\clsSyDEP}   

  & \LKMM
     &           
     % NA, RLX, RA, SC 
     \badcell & \okcell & \okcell & \badcell & 
     % F-RA, F-SC
     \okcell & \okcell & 
     % RMW, Lock
     \okcell & \badcell & 
     % Mix.sz.
     \badcell 
     \\ \cline{2-\lastcol}

  & \OHMM
     &
     % NA, RLX, RA, SC 
     \okcell & \badcell & \badcell & \okcell & 
     % F-RA, F-SC
     \badcell & \badcell & 
     % RMW, Lock
     \badcell & \okcell & 
     % Mix.sz.
     \badcell 
     \\ \Xhline{2\arrayrulewidth}

  \multirow{7}{*}{\clsSemDEP}   

  & \JMM
     &            
     % NA, RLX, RA, SC 
     \okcell & \badcell & \badcell & \okcell & 
     % F-RA, F-SC
     \badcell & \badcell & 
     % RMW, Lock
     \okcell & \okcell & 
     % Mix.sz.
     \badcell 
     \\ \cline{2-\lastcol}

  & \PRM
     &
     % NA, RLX, RA, SC 
     \okcell & \okcell & \okcell & \badcell & 
     % F-RA, F-SC
     \okcell & \okcell & 
     % RMW, Lock
     \okcell & \okcell & 
     % Mix.sz.
     \badcell 
     \\ \cline{2-\lastcol}

  & \WMO
     &
     % NA, RLX, RA, SC 
     \okcell & \okcell & \okcell & \okcell & 
     % F-RA, F-SC
     \okcell & \okcell & 
     % RMW, Lock
     \okcell & \badcell & 
     % Mix.sz.
     \badcell 
     \\ \cline{2-\lastcol}

  & \CSRA
     &
     % NA, RLX, RA, SC 
     \okcell & \okcell & \okcell & \badcell & 
     % F-RA, F-SC
     \badcell & \badcell & 
     % RMW, Lock
     \okcell & \okcell & 
     % Mix.sz.
     \badcell 
     \\ \cline{2-\lastcol}

  & \WJES
     &
     % NA, RLX, RA, SC 
     \badcell & \okcell & \badcell & \badcell & 
     % F-RA, F-SC
     \badcell & \badcell & 
     % RMW, Lock
     \okcell & \okcell & 
     % Mix.sz.
     \badcell 
     \\ \cline{2-\lastcol}

  & \MRD
     &
     % NA, RLX, RA, SC 
     \badcell & \okcell & \badcell & \badcell & 
     % F-RA, F-SC
     \badcell & \badcell & 
     % RMW, Lock
     \badcell & \okcell & 
     % Mix.sz.
     \badcell 
     \\ \cline{2-\lastcol}

  & \GOS
     &
     % NA, RLX, RA, SC 
     \okcell & \badcell & \badcell & \badcell & 
     % F-RA, F-SC
     \badcell & \badcell & 
     % RMW, Lock
     \badcell & \okcell & 
     % Mix.sz.
     \badcell 
     \\ \Xhline{2\arrayrulewidth}

  \multirow{5}{*}{\clsOOTA}   

  & \CMM
     &            
     % NA, RLX, RA, SC 
     \okcell & \okcell & \okcell & \okcell & 
     % F-RA, F-SC
     \okcell & \okcell & 
     % RMW, Lock
     \okcell & \okcell & 
     % Mix.sz.
     \okcell 
     \\ \cline{2-\lastcol}

  & \JSMM
     &
     % NA, RLX, RA, SC 
     \okcell & \badcell & \badcell & \okcell & 
     % F-RA, F-SC
     \badcell & \badcell & 
     % RMW, Lock
     \okcell & \okcell & 
     % Mix.sz.
     \okcell 
     \\ \cline{2-\lastcol}

  & \RMC
     &
     % NA, RLX, RA, SC 
     \badcell & \okcell & \okcell & \okcell & 
     % F-RA, F-SC
     \okcell & \okcell & 
     % RMW, Lock
     \okcell & \badcell & 
     % Mix.sz.
     \badcell 
     \\ \cline{2-\lastcol}


  & \RAO
     &
     % NA, RLX, RA, SC 
     \okcell & \badcell & \badcell & \okcell & 
     % F-RA, F-SC
     \badcell & \badcell & 
     % RMW, Lock
     \badcell & \badcell & 
     % Mix.sz.
     \badcell 
     \\ \cline{2-\lastcol}

  & \TSC
     &
     % NA, RLX, RA, SC 
     \okcell & \badcell & \badcell & \okcell & 
     % F-RA, F-SC
     \badcell & \badcell & 
     % RMW, Lock
     \badcell & \okcell & 
     % Mix.sz.
     \badcell 
     \\ \Xhline{2\arrayrulewidth}

\end{tabular}

\iftoggle{langeng}{%
  \caption{Features supported by memory models}
}{%
  \caption{Поддержка примитивов моделями памяти}
}


\label{table:features}

\end{table}
