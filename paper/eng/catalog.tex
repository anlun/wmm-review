\section{Catalog of Memory Models}
\label{sec:catalog}

In this section we dive inside the 
classes of memory models given in~\cref{sec:analysis},
and present a more detailed view 
on each memory model we consider in our study. 
It contains an overview of each PL memory model,
a recap of its distinguished properties,
and references to the research papers studying this model. 
% This section can be safely skipped for the first read.

\subsection{Sequential Consistency}
\label{sec:catalog:sc}

We start with a description of several attempts 
to adopt the sequential consistency as a memory model 
for existing languages and runtimes. 
Most of the proposed solutions share similar properties, 
and thus in \cref{table:cmp-mms} we unite them 
in a single row under the name \SC. 
The only exception is \DRFx model which implements
a catch-fire semantics for racy programs 
and thus has slightly different properties. 

\paragraph{End-to-end Sequential Consistency}

Marino~\etal~\cite{Marino-al:PLDI11, Singh-al:ISCA12} 
examined the performance penalties to ensure end-to-end SC (\EtESC)
enforced by (1) a modified SC-preserving version 
of \LLVM compiler infrastructure and 
(2) a modified version of \Intel hardware. 
To mitigate the induced overhead the authors 
utilized the observation that hardware need to 
enforce \SC only for memory accesses to shared mutable variables. 
To classify memory regions as either thread-local,
shared immutable, or shared mutable they have used 
a combination of a static compiler analysis and 
a dynamic analysis powered by modified hardware. 
They evaluated their approach on a number of benchmarks
and reported performance overhead of 6.2\% on average 
and ~17\% in maximum, compared to the stock \LLVM compiler 
and \Intel hardware. 

\paragraph{Volatile-by-default}

Liu~\etal~\cite{Liu-al:OOPSLA17, Liu-al:PLDI19} studied 
sequential consistency in the context of the \Java language.  
They proposed the \emph{volatile-by-default} semantics (\VbD),
where each memory access is treated as volatile 
(\ie sequentially consistent) by default. 
The experiments showed a considerable performance penalty:
28\% slowdown on average with 81\% in maximum on x86 hardware,
and 57\% slowdown on average with 157\% in maximum on \ARMv{8} hardware.
The authors tried to mitigate performance overhead and presented  
a novel optimization technique for the language-level \SC
compatible with \emph{just-in-time} compilation. 
They propose to treat each object as thread-local speculatively 
and compile memory accesses without fences. 
If later during the execution concurrent accesses to the object  
are detected, the code is recompiled to place necessary fences.
A modified version of the \JVM which implements the technique
described above was reported to incur 37\% slowdown on average 
with 73\% in maximum on \ARMv{8} hardware.

\paragraph{SC-Haskell}

\SCHs memory model~\cite{Vollmer-al:PPoPP17}
was inspired by the similar idea of separation
between thread-local and shared mutable memory. 
To safely distinguish between the two 
the authors utilized the powerful strong type system of \Haskell. 
The consequence of this approach is that  
programmers need to follow a stricter discipline 
in order to please the type checker. 
The authors modified the GHC to preserve \SC 
and then run 1,279 benchmarks on x86-64 hardware
to measure the performance penalties.
They reported 0.4\% geometric mean slowdown,
and noticed that only 12 benchmarks experienced 
slowdown greater than 10\%.
These promising results can be partly explained 
by the fact that \Haskell encourages a 
purely functional programming style, 
which minimizes the usage of shared mutable memory.  

\paragraph{DRFx}

The \DRFx~\cite{Marino-al:PLDI10, Marino-al:TOPLAS2016} 
is another \SC preserving memory model. 
In this model a runtime system is guaranteed to raise 
an exception if a program has data-races, 
and yield only sequentially consistent outcomes otherwise.
In order to make the runtime data-race detection feasible 
in practice, the authors propose several modifications 
to existing hardware.

The authors claim that any sequentially valid optimization 
(\eg instruction reorderings or common subexpression elimination),
is sound in \DRFx model, the only limitation is that
these transformations can only be performed
withing bounds of compiler-designated program regions.
Besides that any transformation that introduces 
speculative reads or writes is unsound,
since speculative optimizations can bring
data-races into otherwise race-free programs.

Note that in \cref{table:cmp-mms} we still do not consider 
many of the transformations that claimed to be sound by the authors
as actually being sound because of our convention described in \cref{sec:comparison}.
We consider transformations sound only if they are 
sound in a fully-defined semantics. 
The \DRFx model does not meet this criterion as 
it provides catch-fire semantics.

The expected performance overhead of the model 
is reported to be 3.25\% on average
assuming an efficient implementation 
of the data-race detection in hardware. 
(compared to stock compiler and \Intel hardware). 

\subsection{Total and Partial Store Order}

In this section we consider PL memory models 
inspired by \TSO and \PSO.  

\paragraph{Buffered Memory Model}

Demange~\etal~\cite{Demange-al:POPL13} presented 
the \emph{Buffered Memory Model}, \BMM for short,
as a candidate model for the \Java language.
Their motivation, however, stemmed not from the desire 
to fully replace the Java memory model, but rather 
from a goal to build a verified version of 
Java Virtual Machine (akin to CompCertTSO 
project~\cite{Sevcik-al:JACM13} for the C language).
A more simple yet pragmatic \TSO like memory model 
was considered as a first step to achieve this goal. 

The authors proved soundness of several program transformations
(including the store/load reordering, the speculative load introduction,
and several others, see \cref{table:cmp-mms})
and the external \DRF theorem. 
They also modified existing open-source implementation of 
\JVM~\cite{Pizlo-al:ECCS10} to preserve \BMM and 
reported only~1\% average overhead 
compared to original version of the virtual machine. 
Again, the authors used only \Intel hardware in their 
experiments, and the performance penalties 
are expected to be more significant on weaker hardware.   

\paragraph{Relaxed Memory Models: an Operational Approach}

Boudol and Petri~\cite{Boudol-Petri:POPL09} proposed 
an operational approach to 
formal semantics of relaxed memory models (\RMMOA)
based on an abstract machine with a main memory 
and a hierarchical structure of store buffers 
with stores to different locations possibly 
propagating to the main memory out-of-order
(similarly to \PSO model).
The authors present a proof of the external \DRF theorem,
but do not provide an extensive study 
of the soundness of program transformations.

\subsection{Program Order Preserving}
\label{sec:catalog:porf}

In this section we describe the memory models 
that preserve program order and forbid any 
kind of speculative executions to tackle 
the problem of thin-air values. 
In particular, we consider \RCMM and 
several derivatives of this model, 
as well as the memory model of \OCaml, 
and the proposed revised model of \Java.  

\paragraph{RC11}

Lahav~\etal~\cite{Lahav-al:PLDI17} formalized 
a version of the \CMM that preserves order between load/store pairs, 
and also repairs the semantics of sequentially-consistent accesses.
The resulting model is commonly referred to as \RCMM. 

The authors proved soundness of several program transformations 
(see \cref{table:cmp-mms} for details). 
Among the unsound transformation, 
the load/store reordering is forbidden for an obvious reasons, 
the speculative load introduction is not supported 
because of the catch-fire semantics for racy programs, 
the \CSE is not supported because relaxed accesses 
enforce coherence (non-atomic accesses 
support this transformation, but their usage entails 
undefined behavior in the presence of races).

The compilation mapping to \Intel is not affected and remain optimal.
One of the possible compilation mappings 
to the \ARM and \POWER architectures 
requires to compile a relaxed load as  
a plain load followed by a spurious conditional branch.
Ou and Demsky~\cite{Ou-Demsky:OOPSLA18} have studied 
the performance penalty of this mapping on \ARMv{8} hardware.
They modified the \LLVM compiler framework 
to enforce the \RCMM memory model
by (1) adjusting the compiler optimization passes and 
(2) changing the compilation mappings.
Several compilation schemes were considered,
among them the one that uses a spurious conditional branch
as described above has demonstrated the most promising results.  
The authors measured the running time on a set of benchmarks 
implementing concurrent data-structures
(\eg locks, stacks, queues, deques, maps
from various open source libraries~\cite{CDSLib, FollyLib, JunctionLib})
and reported an overhead of 0\% on average and 6.3\% in maximum,
compared to the unmodified version of the compiler. 

\paragraph{RAR}

Doherty~\etal~\cite{Doherty-al:PPoPP19} developed an 
operational version of the \RCMM supporting 
release-acquire and relaxed accesses (\RAR). 
On top of it they built proof calculus for 
invariant-based reasoning and verified 
correctness of mutual exclusion algorithms. 

\paragraph{Operational RC11}

Dang~\etal~\cite{Dang-al:POPL19} developed yet another 
operational version of the \RCMM which they called \ORCMM. 
Their motivation was to then develop a 
new program logic and show it's soundness
with respect to the \ORCMM memory model. 
The program logic itself was utilized to 
prove correctness of several synchronization primitives 
from the standard library of the \Rust~\cite{RustBook:19}.

\paragraph{Compositional Relaxed Concurrency}

Dodds~\etal~\cite{Dodds-al:ESOP18} proposed the  
compositional relaxed concurrency semantics (\CRC) 
for the fragment of the \CMM memory model, 
including non-atomic accesses with catch-fire semantics, 
release-acquire accesses, and sequentially-consistent fences. 
Based on this semantics the authors developed 
a tool for automatic verification of program transformations
in the considered fragment of the \CMM model. 
Since the relaxed fragment was not included, 
the authors avoided problems with thin-air values. 

\paragraph{OCaml Memory Model}

Dolan~\etal~\cite{Dolan-al:PLDI18} developed a new 
memory model for the \MOCaml project. 
An important divergence of the \OCaml memory model (\OCMM)
from the \CMM-like models is that the former 
has a weaker notion of the coherence.
The choice of the weaker coherence was deliberate 
with the purpose to enable the common subexpression elimination
(see \cref{sec:analysis:coh} for details).

The authors also were the first to propose the local \DRF property (\lDRF),
a strengthening of the external \DRF (\eDRF). 
While the latter requires an absence of data-races 
for a whole program as a prerequisite, 
the former bounds the effect of races 
to a portion of the program, thus 
enabling the compositional reasoning 
about the behavior of the program. 
The authors discovered that the \lDRF property 
is not compatible with the load/store reordering.
This fact forced them to forbid this transformation
and adapt similar compilation scheme as for \RCMM. 

\paragraph{Java Access Modes}

Bender and Palsberg~\cite{Bender-Palsberg:OOPSLA19} formalized a new revision 
of the Java Memory Model~\cite{JDK9-VarHandle, JEP:193, JDK9-Modes}, 
which was developed to overcome 
the difficulties of the previous one~\cite{Manson-al:POPL05}
(see \ref{sec:catalog:jmm} for details).
The new version of the model was inspired by the \RCMM. 
It introduced a system of annotations on memory accesses, 
called ``Java Access Modes'' (hence the name of the model --- \JAM),
similar to those present in the \CMM like models.
The new model adopted the \RCMM solution to OOTA problem. 
It forbids load/store reorderings on the level of 
opaque (an analog of \CPP relaxed) or stronger accesses.
The model does not tackle the problem of 
thin-air values on the level of plain (\ie non-atomic) accesses.

\subsection{Syntactic Dependencies Preserving}
\label{sec:catalog:deprf}

Next we discuss the programming language memory models 
that track syntactic dependencies.

\paragraph{Linux Kernel Memory Model}

\LKMM~\cite{Alglave-al:ASPLOS18} has adopted 
the idea to track syntactic dependencies in order to 
forbid thin-air values. Despite this choice 
limits the list of supported trace preserving transformations,
in the context of the OS kernel development 
it can be justified by the following arguments. 
First, the Linux kernel targets 
a wide range of hardware architectures with a diverse
set of memory models. To simplify the reasoning about the code, 
it is reasonable to pick a syntactic dependency preserving
model which is conceptually close to those of hardware. 
Second, kernel developers already utilize 
various techniques to prevent certain compiler optimizations~%
\cite{Alglave-al:ASPLOS18, LK-MemBarriers, LK-RCU-Deref}.

The authors of the model have empirically tested 
soundness of compilation mappings to 
\Intel, \ARMv{7}, \ARMv{8}, and \POWER hardware. 
They also formalized the read-copy-update 
synchronization mechanism (RCU)~\cite{McKenney-RCU2007} 
extensively used in the Linux kernel development, 
and proved soundness of its implementation with respect to their model.

\paragraph{Operational Happens-Before Model}

In attempt to repair the Java Memory Model (see \cref{sec:catalog:jmm})
Zhang and Feng have proposed the 
operational happens-before model \OHMM~\cite{Zhang-Feng:FCS16}.
Their abstract machine consists of a global event buffer,
where events might be reordered before they propagate into  
a global history based memory, and a replay mechanism 
used to simulate speculative executions. 
To avoid thin-air outcomes the model tracks syntactic dependencies 
between events and forbids the reordering of dependent events. 
The authors proved the external \DRF and 
the soundness of several program transformations
(see \cref{table:cmp-mms}). 

\paragraph{Dependency Preserving Compiler}

Ou and Demsky~\cite{Ou-Demsky:OOPSLA18} studied 
the performance penalty induced by dependency preserving compiler. 
Again, they modified the \LLVM compiler infrastructure 
and run benchmarks from \SPECCPU suite on \ARMv{8} hardware. 
They have observed 3.1\% overhead on average and 17.6\% in maximum. 

\subsection{Semantic Dependencies Preserving}
\label{sec:catalog:sdeprf}

Next we discuss the memory models 
which try to tackle the thin-air values problem 
by developing a notion of semantic dependencies. 
In particular, this class includes the original Java Memory Model, 
the Promising semantics, and several models based 
on the \emph{event structures}~\cite{Winskel:86}.

\paragraph{Java Memory Model}
\label{sec:catalog:jmm}

The original version of the Java memory model \JMM~\cite{Manson-al:POPL05}
was a pioneering work in the area of programming language memory models. 
In order to forbid thin-air outcomes, the memory model used 
a notion of \emph{commit sequence}, \ie a sequence of partial execution graphs.
The model was shown to adhere the external \DRF~\cite{Huisman-Petri:CONCUR07}.
However, the model failed to justify some program transformations 
that were expected to be sound~\cite{Sevcik-Aspinall:ECOOP08} 
(\eg redundant load after load elimination, roach motel reordering, and others,
see \cref{table:cmp-mms} for details). 

\paragraph{Generative operational semantics}

Jagadeesan~\etal~\cite{Jagadeesan-al:ESOP10} attempted to fix \JMM 
and proposed the generative operational semantics 
with speculative execution (\GOS).
To avoid thin-air values they have put stratification constraints 
on speculations. The authors prove the external \DRF theorem. 
Also they verified a few program transformations 
(store/store reordering, load/load elimination, and roach motel reordering), 
but overall their study of transformations was not systematic.  

\paragraph{Promising Semantics}

Kang~\etal~\cite{Kang-al:POPL17, Lee-al:PLDI20} developed 
the Promising semantics (\PRM).
It is the most complete to this day model of the class of
semantic dependency preserving models. 
Its key ingredient is a promising and certification machinery.
During an execution, the abstract machine can 
non-deterministically \emph{promise} to perform some store,
it then has to \emph{certify} the promise is feasible. 
The certification mechanism is defined in a way that forbids thin-air values to appear.
The authors of the model have proven formally 
that \Promising semantics admits many local and global program transformations,
with a notable exception of the thread inlining
(see \cref{table:cmp-mms} for details).

Podkopaev~\etal~\cite{Podkopaev-al:ECOOP17, Podkopaev-al:POPL19} 
proved formally the soundness of standard optimal 
compilation mappings to \Intel, \ARMv{7}, \ARMv{8}, and \POWER.

The model has a fully defined semantics for plain accesses.  
Plain and relaxed accesses, however, have different semantics.
In particular, coherence is enforced only for relaxed accesses. 
This design choice, in particular, allows to support 
\CSE on the level of plain accesses. 

One of a few limitations of the \Promising semantics is that 
it does not support sequentially consistent accesses. 

\paragraph{Weakestmo}

Chakraborty and Vafeiadis~\cite{Chakraborty-Vafeiadis:CGO17, Chakraborty-Vafeiadis:POPL19}
developed a memory model based on the event structures (\WMO). 
They utilize the event structures' capability to simultaneously encode 
multiple conflicting executions in order to model speculative executions.
The model was shown to admit optimal compilation mappings~\cite{Moiseenko-al:ECOOP20},
several program transformation, and the external \DRF.
Unlike \Promising semantics it also supports 
sequentially consistent accesses.

\paragraph{A Concurrency Semantics for Relaxed Atomics}

Pichon-Pharabod and Sewell~\cite{PichonPharabod-Sewell:POPL16} 
presented the operational memory model (\CSRA) consisting of 
a memory subsystem inspired by the \POWER model 
and a thread subsystem, 
where each thread is represented as an event structure. 
At each step the abstract machine is allowed to either 
commit an event to the storage, or perform a transformation 
on one of the event structures. 
The authors have shown soundness of 
optimal compilation mappings to \Intel and \POWER, 
as well as soundness of several program transformations.
It was later revealed though that the compilation scheme
to \ARMv{7} and \ARMv{8} is not optimal~\cite{PichonPharabod:PhD18}.

\paragraph{Well-Justified Event Structures}

Jeffrey and Riely~\cite{Jeffrey-Riely:LICS16} proposed 
the memory model (\WJES) based on event structures and a notion of 
\emph{well-justification} of events inspired by the game semantics. 
Well-justification is used to prevent thin-air values 
and prove the external \DRF. The authors do not study 
the soundness of program transformations in their model. 
They show, however, a counterexample demonstrating that 
the load/load reordering is unsound. 
This fact also implies that 
the compilation mappings to \ARMv{7}, \ARMv{8}, and \POWER 
are not optimal.   

\paragraph{Modular Relaxed Dependencies}

Paviotti~\etal~\cite{Paviotti-al:ESOP20} constructed the 
denotational semantics based on the event structures (\MRD). 
They employ the event structures to capture 
semantic dependencies between memory access events, 
which are in turn used to rule out thin-air outcomes.
The authors prove the external \DRF and 
the soundness of optimal compilation mappings,
also they present a refinement relation which 
can be used to reason about validity of program transformations. 
However, they have not studied soundness of particular transformations. 

\subsection{Out of Thin-Air Values}

Finally, we discuss memory models admitting thin-air values. 

\paragraph{C11}

The most notable member of the OOTA class is the \CMM model~\cite{Batty-al:POPL11}.
The main purpose of the \CMM model was to adhere to the fundamental principle of \CPP, 
\ie to provide so-called zero-cost abstraction. 
In other words, the memory model was meant to provide 
efficient compilation mappings and support as many transformation as possible.
It was later revealed that the formal model partially fails to achive these goals.

Vafeiadis~\etal~\cite{Vafeiadis-al:POPL15} showed that several program transformation 
(load/store elimination, strengthening, roach motel reorderings, sequentialization) 
that deemed to be correct are actually unsound according to the formal model.
They proposed several local fixes to the model which 
partly repair soundness of transformations and improve 
its meta-theoretical properties. 

Batty~\etal~\cite{Batty-al:ESOP15} showed that 
the model also fails to provide the external \DRF guarantee, 
and that it is ultimately not possible to provide this guarantee
at all within the style of the \CMM formal semantics.
Only the internal \DRF can be proved for it. 

A lot of work~\cite{Batty-al:POPL11, Sarkar-al:PLDI12, Batty-al:POPL12, Batty-al:POPL16} 
was dedicated to prove soundness of optimal compilation mappings 
with respect to formal models of hardware, 
and there the results were mostly positive.
Besides that, Flur~\etal~\cite{Flur-al:POPL17} have extended 
the model to support mixed-size accesses.
Finally, Nienhuis~\etal~\cite{Nienhuis-al:OOPSLA16} presented 
a formal executable semantics in terms of an abstract machine, 
equivalent to the \CMM model. 

\paragraph{\JS Memory Model}

The \JSMM is based on the \CMM model. 
Like the latter, it also has the problem of thin-air values
and thus can only provide the internal \DRF guarantee. 
Contrary to the \CMM, the \JS model does not treat 
racy non-atomic accesses as undefined behavior. 

The main language primitive provided by the \JSMM
is \texttt{SharedArrayBuffer}, that is a linear mutable byte buffer.
Thus the model naturally supports mixed-size accesses.

\paragraph{A calculus for relaxed memory}

Crary and Sullivan~\cite{Crary-Sullivan:POPL15} proposed 
an alternative approach to the relaxed shared memory concurrency,
which they called \emph{Relaxed Memory Calculus} (\RMC).
Instead of deriving ordering constraints from annotations 
on memory accesses, they propose to directly specify 
the ordering between memory accesses in a source code. 
Their approach is highly generic and subsumes 
the traditional memory order annotations in the style of \CMM.
Their model is very weak and permits thin-air values. 
Yet the authors proved the internal \DRF theorem.

\paragraph{Relaxed Atomic + Ordering}

Saraswat~\etal~\cite{Saraswat-al:PPoPP07} presented the \RAO memory model
where relaxed behaviors are explained through transformations 
over a sequentially consistent execution.
Authors claim their model provides the external \DRF,
but at the same time permits thin-air values. 
These two properties known to be incompatible~\cite{Batty-al:ESOP15}.
We suppose that the external \DRF can be achieved in their model 
only because of the fundamental restrictions on the input programming language 
(\eg the general conditional statements are not supported~\cite{PichonPharabod-Sewell:POPL16}). 

\paragraph{A theory of speculative computation}

Boudol and Petri~\cite{Boudol-Petri:ESOP10} proposed a general 
framework to study effects of speculative execution in
shared memory setting (\TSC). 
They have also noticed that the external \DRF does not 
hold in the presence of unrestricted speculations, 
yet the internal \DRF theorem still can be proven. 
